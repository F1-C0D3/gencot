
A C function definition is translated by Gencot to a Cogent function definition. Old-style C function definitions
where the parameter types are specified by separate declarations between the parameter list and the function body
are not supported by Gencot because of the additional complexity of comment association.

The Cogent function name is generated from the C function name as described in Section~\ref{design-names}.

The Cogent function type is generated from the C function result type and from all C parameter types as described
in Section~\ref{design-types-function}. In a C
function definition the types for all parameters must be specified in the parameter list, if old-style function
definitions are ignored.

\subsection{Function Bodies}
\label{design-fundefs-body}

In C the function body consists of a compound statement. In Cogent
the function body consists of an expression. Gencot translates the C code to Cogent code as described
in Section~\ref{design-cstats}.

As described in Section~\ref{design-types-function} Gencot translates all C functions to Cogent functions with
a single parameter. If the C function has more than one parameter it is a tuple of the C parameters. To make it
possible to use the original C parameter names in the translated function body
Gencot generates a Cogent pattern for the parameter of the Cogent function which 
consists of a tuple of variables with the names generated from the C parameter names. As described in 
Section~\ref{design-names} the C parameter names are only mapped if they are uppercase, otherwise they are
translated to Cogent unmodified. Since it is very unusual
to use uppercase parameter names in C, the Cogent function will normally use the original C parameter names.

If the item property Global-State (see Section~\ref{design-types-itemprops}) has been used to introduce additional 
function parameters, the corresponding parameter names are added to the tuple after the last C parameter name.
If the item property Heap-Use has been set for the function, an additional parameter named \code{heap} or 
\code{globheap<n>} (see~\ref{design-types-itemprops}) is added to the tuple after possible Global-State parameters.
If the item property Input-Output has been set for the function, an additional parameter named \code{io} or
\code{globio<n>} (see~\ref{design-types-itemprops}) is added to the tuple after possible Global-State parameters
and the Heap-Use parameter.

If the function is variadic an additional last tuple component is added with a variable named
\code{variadicCogentParameters}, mainly to inform the developer that manual action is required.

The generated Cogent function definition has the form
\begin{verbatim}
  <name> :: (<ptype1>, ..., <ptypen>) -> <restype>
  <name> (<pname1>, ..., <pnamen>) = 
    <translated C function body>
\end{verbatim}

\subsection{Comments in Function Definitions}
\label{design-fundefs-comments}

A C function definition which is not old-style syntactically consists of a declaration with a single declarator
and the compound statement for the body.
It is treated by Gencot as a structured source code part with the declaration and the body as subparts
without any main part code. According to the structures of declarations the declaration has the single declarator as subpart
and optionally a full specifier, if present. The declarator has the parameter declarations as subparts.

\subsubsection{Function Header}

The target code part for the declaration and for its single declarator is the header of the Cogent function definition
(first two lines in the schema in the previous section). The target code part for the full specifiers with tags in
the declaration (which may be present for the result type and for each parameter) is a sequence of corresponding 
type definitions, as described for declarations in Section~\ref{design-decls-tags}, which is placed 
after the Cogent function definition. The target code part for full specifiers without tags is the generated type
expression embedded in the Cogent type for the corresponding parameter or the result.

All parameter declarations consist of a single declarator and the optional full specifier. The target code part for
a parameter declaration and its declarator is the corresponding parameter type in the Cogent function type expression.
Hence, comments associated with parameter declarations in C are moved to the parameter type expression in Cogent.

\subsubsection{Function Body}

Gencot inserts origin markers in the Cogent code generated for the function body. It also inserts them in embedded 
untranslated C code. Thus, comments and preprocessor directives will be reinserted into the code. However, since the
structure of the Cogent code may substantially differ from that of the C code, comments and directives may be misplaced 
or omitted, this must be checked and handled manually.

\subsection{Entry Wrappers}
\label{design-fundefs-wrapper}

The C function definitions are also used for generating the entry wrappers as described in Section~\ref{design-modular}.
For every definition of a C function with external linkage an entry wrapper is generated in antiquoted C.

The main task of the entry wrapper is to convert the seperate function arguments to an argument tuple which is passed
to the corresponding Cogent function. If the C function has only one argument it is passed directly. If the C 
function has no argument the unit value must be passed to the Cogent function.

The entry wrapper has the same interface (name, arguments, argument types, result type) as the original C function.
However, since the wrapper is a part of the Cogent compilation unit, the argument and result types used in the 
original function definition are not available. They must be replaced by the binary compatible types generated 
from their translation to Cogent types (see Section~\ref{design-types}). This is accomplished by using the Cogent
types in antiquoted form.

If the original C function definition has the signature
\begin{verbatim}
  <type0> fnam(<type1> arg1, ..., <typen> argn)
\end{verbatim}
the entry wrapper has the signature
\begin{verbatim}
  $ty:(<cogt0>) fnam($ty:(<cogt1>) arg1, ..., $ty:(<cogtn>) argn)
\end{verbatim}
where \code{<cogti>} is the translation of \code{<typei>} to Cogent. It would be possible to generate synthetic 
parameter names for the wrapper, however, since the original parameter names must always be specified in the 
original function definition, Gencot uses them in the wrapper signature for simplicity and better readability.

If there is more than one function argument (\code{n > 1}), the wrapper must convert them to the C implementation
of the Cogent tuple value used as argument for the Cogent function (see Section~\ref{design-types-function}).
Cogent tuples are implemented as C struct types with members named \code{"p1"}, \code{"p2"}, \ldots in the order
of the tuple components. Since tuple types are always unboxed, the C structs are allocated on the stack. Thus
the body of an entry wrapper with \code{n > 1} has the form
\begin{verbatim}
  $ty:((<cogt1>,...,<cogtn>)) arg = 
    {.p1=arg1,..., .pn=argn};
  return cogent_fnam(arg);
\end{verbatim}

If \code{n = 1} no tuple needs to be constructed and the wrapper body has the form
\begin{verbatim}
  return cogent_fnam(arg1);
\end{verbatim}

If \code{n = 0} the C implementation of the Cogent unit value must be constructed. It is a C struct with 
a single member named \code{"dummy"} of type \code{int}. The corresponding wrapper body has the form
\begin{verbatim}
  $ty:(()) arg = {.dummy=0};
  return cogent_fnam(arg);
\end{verbatim}

If the C function has no result, i.e., \code{<type0>} is \code{void}, no result is returned by the entry
wrapper. Then the last statement in its body has the form
\begin{verbatim}
  cogent_fnam(arg);
\end{verbatim}

If the C function has the Heap-Use property (see Section~\ref{design-types-itemprops}), a component of type
\code{Heap} is added to the Cogent function's argument tuple and the result is converted to a tuple where the
original result is the first component. In C the heap is accessed globally, therefore these components can 
be ignored. However, the wrapper implementation has to take care of them. Type \code{Heap} is implemented by
Gencot simply by the C type \code{int} and a \code{0} is passed as its value. The entry wrapper body for a
function with Heap-Use property has the form
\begin{verbatim}
  $ty:((<cogt1>,...,<cogtn>,Heap)) arg = 
    {.p1=arg1,..., .pn=argn, .pn+1=0};
  return cogent_fnam(arg).p1;
\end{verbatim}
if \code{n > 0} and
\begin{verbatim}
  return cogent_fnam(0).p1;
\end{verbatim}
if \code{n = 0}.

If the C function has the Input-Output property (see Section~\ref{design-types-itemprops}), a component of type
\code{SysState} is added in the same way as for Heap-Use, but after that for Heap-Use, if present. Type \code{SysState}
is also implemented by Gencot by the C type \code{int} and a \code{0} is passed as its value.

If atleast one of the C function arguments has the Add-Result property, the result is also converted to a tuple 
with the original result as its first component. In this case the last statement in the wrapper's body also
has the form
\begin{verbatim}
  return cogent_fnam(arg).p1;
\end{verbatim}
if the C function's result is not \code{void}.

If the C function has virtual parameters with Global-State properties declared in the item properties, references
to the corresponding global variables are passed to these parameters by the entry wrapper. For a parameter with 
Global-State property the corresponding variable is determined by searching all known toplevel items for an
item with a Global-State property with the same numerical argument. Since every global variable must use a 
different numerical argument, this search will result in a single toplevel item identifier. 

If the function does not access the global variable and uses the virtual parameter only to pass the variable 
reference to invoked function, it may be the case that the variable is not visible in the C source file, because
it is neither defined nor declared there. However, the name used in (antiquoted) C for the variable can always 
be constructed from the item identifier alone. No other information about the variable is needed to generate the 
entry wrapper.

Assume that a C function has two virtual parameters with Global-State properties with numerical arguments 
\code{0} and \code{3}. The global variable associated with argument \code{0} has name \code{gvar} and external linkage,
the global variable associated with argument \code{3} has name \code{table} and is defined in source file \code{scan.c}
with internal linkage. Then the entry wrapper body has the form
\begin{verbatim}
  $ty:((<cogt1>,...,<cogtn>,GlobState,GlobState3)) arg = 
    {.p1=arg1,..., .pn=argn, .pn+1=&gvar, .pn+2=&local_scan_table};
  return cogent_fnam(arg).p1;
\end{verbatim}

Together, the information required to generate the entry wrapper is the same as that for translating the
function's type to Cogent: the argument types and the result type must be translated to Cogent and the
function name must be mapped to determine the name of the Cogent function.

\subsection{Object Definitions}
\label{design-fundefs-object}

Gencot processes object declarations with block-scope (``local variables'') as described in Section~\ref{design-cstats-general}. 

A toplevel (``external'') object definition (with ``file-scope'') either has an initializer or is a ``tentative definition''
according to the C standard. Such an object definition defines a global variable. Gencot treats global variables
as described in Section~\ref{design-modular}.

If the global variable has a Global-State property (see Section~\ref{design-types-itemprops}) Gencot introduces a type 
synonym of the form \code{GlobState<i>}. The type synonym definition is generated in the translated Cogent source 
at the place where the object definition was in the C source and is marked with the corresponding origin. The 
type synonym name is determined from the numerical argument of the Global-State property, the right-hand side of
the type definition is the translation of the derived pointer type of the variable's type, translated according to
Section~\ref{design-types}.

If the global variable has the Const-Val property (see Section~\ref{design-types-itemprops}) Gencot introduces an abstract 
parameterless access function in Cogent in the form
\begin{verbatim}
  <name> : () -> <type>
\end{verbatim}
where \code{<name>} is the mapped variable name and \code{<type>} is the translation of the variable's defined type.
If the defined type is a struct, union, or array type, the translation of the derived pointer type is used instead,
with a bang operator applied to make it readonly. The access function definition is generated in the translated Cogent source 
at the place where the object definition was in the C source and is marked with the corresponding origin.

Independent from its item properties, the object definition is translated to antiquoted C and added to the file 
containing the entry wrappers. In this way all global variables are still present in the Cogent compilation unit.
The translated definition uses as type the antiquoted Cogent type which results from translating the type specified 
in the object definition. For an object with external linkage the name is the original name, so that the object can still
be referred from outside the Cogent compilation unit, and the linkage is external. For an object with internal linkage
the name is mapped as described in Section~\ref{design-names} so that it is unique in the Cogent compilation unit, and
the linkage is internal. If an initializer is present in the original definition it is transferred to the antiquoted
definition, mapping all referenced names of global variables with internal linkage.

The entry wrappers and object definitions are placed in the antiquoted C file in the same order as the original function
and object definitions. Additionally, all global variables are declared together with all external functions in a 
file included before all entry wrapper files, so that entry
wrappers defined before the variable can still pass a pointer to it to the wrapped Cogent function.

If the global variable has the Const-Val property Gencot additionally creates an implementation of the access function
in antiquoted C and puts it immediately after the variable definition. Depending on the variable's type it returns its
value or a pointer to the variable. 
