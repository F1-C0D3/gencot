
The Cogent source generated by Gencot is intended for further manual modification. Finally, it should be used as a 
replacement for the original C source. Hence, also the documentation should be transferred from the C source to
the Cogent source.

Gencot uses the following heuristics for selecting comments to be transferred: All comments at the beginning or end 
of a line and all comments on one or more full lines are transferred. Comments embedded in C code in a single line
are assumed to document issues specific to the C code and are discarded.

\subsection{Identifying and Translating Comments}

Gencot processes C block comments of the form \code{/* ... */} possibly spanning several lines, and C line comments
of the form \code{// ...} ending at the end of the same line.

Identifying C comments is rather complex, since the comment start sequences \code{/*} and \code{//} may also occur
in C code in string literals and character constants and in other comments. 

Comments are translated to Cogent comments. Every C block comment is translated to a Cogent block comment of the form
\code{\{- ... -\}}, every C line comment is translated to a Cogent line comment of the form \code{-- ...}. Only the 
start and end sequences of identified comments are translated, all other occurrences of comment start and end sequences
are left unchanged.

If a Cogent block comment end sequence \code{-\}} occurs in a C block comment, the translated Cogent block comment
will end prematurely. This will normally cause syntax errors in Cogent and must be handled manually. It is not
detected by Gencot.

\subsection{Comment Units}

Gencot assembles sequences of transferrable comments which are only separated by whitespace together to comment units
as follows. All comments starting in the same line after the last existing source code are concatenated to become 
one unit. Such units are called ``after-units''. All comments starting in a separate line with no existing source code 
or before all existing source code in that line are concatenated to become one unit. Such units are called ``before-units''. 

Additionally, all remaining comments at the end of a file after the last after-unit are concatenated to become the 
``end-unit''. At the beginning of a file there is often a schematic copyright comment. To allow for a specific treatment
a configurable number of comments at the beginning of a file are concatenated to become the ``begin-unit''. The default
number of comments in the begin-unit is 1.

As a result, every transferrable comment is either part of a comment unit and every comment unit
can be uniquely identified by its kind and by the source file line numbers where it starts and where it ends.

Heuristically, a before-unit is assumed to document the code after it, whereas an after-unit is assumed to document
the code before it. Based on this heuristics, comment units are associated to code parts. A begin-unit and an end-unit
is assumed to document the whole file and is not associated with a code part.

\subsection{Relating Comment Units to Documented Code}
\label{design-comments-relate}

Basically, Gencot translates source code parts to target code parts. Source code parts may consist of several lines,
so there may be several before- and after-units associated with them: The before-unit of the first line, the after-unit
of the last line and possibly inner units. Target code parts may also consist of several lines. The before-unit of
the first line is put before the target code part, the after-unit of the last line is put after the target code part.

If there is no inner structure in the source code part which can be mapped to an inner structure of the target code
part, there are no straightforward ways where to put the inner comment units. They could be discarded or they could be
collected and inserted at the beginning or end of the target code part. If they are collected no information is lost 
and irrelevant comments can be removed manually. However, in well structured C code inner comment units are rare,
hence Gencot discards them for simplicity and assumes, that this way no relevant information will be lost.

If the source code part has an inner structure units can be associated with subparts and transferred to subparts of the
target code part. Gencot uses the following general model for a structured source code part: It may have one or more
embedded subparts, which may be structured in a similar way. Every subpart has a first line where it begins and a last 
line where it ends. Before and after a subpart
there may be lines which contain code belonging to the surrounding part. Subparts may overlap, then the last line of 
the previous subpart is also the first line of the next subpart. Subparts may overlap with the surrounding part, then 
the first or last line of the subpart contains also code from the surrounding part.

For a structured source code part Gencot generates a target code part for the main part and a target code part for every 
subpart. The subpart targets may be embedded in the main part target or not. If they are embedded they may be reordered.

The inner comment units of a structured source code part can now be classified and associated. Every such unit is either
an inner unit of the main part, a before-unit of the first line of a subpart if that does not overlap, an inner 
unit of a subpart, or an after-unit of the last line of a subpart, if that does not overlap. The units associated with a subpart
are transferred to the generated target according to the same rules as for the main part. 

If there is no main source code before the first subpart (e.g., a declaration starting with a struct definition), the
before-group of the first line is nevertheless associated with the main part and not with the first subpart. The after-group
at the end of a part is treated in the analogous way.

Inner units of the main part may be before the first subpart, between two subparts, or after the last subpart. Following 
the same argument as for inner units of unstructured source code parts, Gencot simply discards all these inner units.

As a result, for every source code part atmost the before-unit of the first line and the after-unit of the last line 
is transferred to the target part. If the source code part is structured the same property holds for every embedded 
subpart. If no target code is generated for the main part but for subparts, the before-unit of the main part immediately
precedes the before-unit of the first subpart, if both exist, and analogously for the after-units.

Target code for a part may be generated in several separated places. If no code
is generated for the main part, it must be defined to which group of subpart targets the comments associated with the
main part is associated.

\subsection{Declaration Comments}
\label{design-comments-decl}

Since toplevel declarations are not translated to a target code part in Cogent, all comments associated with them would
be lost. However, often the API documentation of a function or global variable is associated with its declaration instead of the
definition. 

Therefore Gencot treats before-units associated with a toplevel declaration in a specific way and 
moves them to the target code part generated for the corresponding definition. There they are placed between
the comments preceding the definition and the definition itself. 

Gencot assumes, that only one declaration exists for each definition. If there are more than one declarations 
in the C code the comment associated with one of them is moved to the definition, the comments associated with
the other declarations are lost. 

Only before-units are handled this way, due to a technical problem with the C parser used. For declarations it does not provide 
the end position in a safe manner. For the intended application this is not a problem since API documentations are
usually placed before the declaration and not after it.
