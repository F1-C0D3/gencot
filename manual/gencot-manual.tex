\documentclass[a4paper]{report}
\usepackage[bookmarks]{hyperref}
\usepackage[utf8]{inputenc}
\usepackage[T1]{fontenc} % needed for italic curly braces

\newcommand{\code}[1]{\textnormal{\texttt{#1}}}

\begin{document}

\title{Gencot User Manual}
\author{Gunnar Teege}

\maketitle

\chapter{Introduction}

Gencot (GENerating COgent Toolset) is a set of tools for generating Cogent code from C code. 

Gencot is used for parsing the C sources and generating Cogent sources, 
antiquoted C sources, and auxiliary C code. It does not perform a fully automatic translation, it is
intended to be used in combination with several manual steps of pre- and post-processing. These
steps are described in this manual.

The manual assumes that you are familiar with C and Cogent and know how to work with both.

\section{Distribution}
\label{intro-dist}

The Gencot distribution consists of the following folders:
\begin{description}
\item[manual] this manual,
\item[bin] the main command scripts \code{gencot} and \code{parmod} and many auxiliary commands used by them,
\item[include] Cogent include files used by the generated code,
\item[c] C code implementing abstract types and functions used by the generated code,
\item[examples] example C programs used in this manual for introducing Gencot,
\item[src] the Haskell source code of Gencot components,
\item[doc] a comprehensive documentation of Gencot design and implementation.
\end{description}

Gencot is a command line tool. To use it make sure that you can invoke the commands \code{gencot} and
\code{parmod} (e.g., by linking them in a folder in your command path or by adding the \code{bin} folder
of the Gencot distribution to your command path.

Additionally you have to set the environment variable \code{\$GENCOT\_HOME} to the root folder of the Gencot
distribution.

We also assume that you have a working distribution of Cogent and can invoke the cogent compiler using the
command \code{cogent}.

All example folders contain a UNIX Makefile. You can either run the examples by manually typing the commands
to process them or by using \code{make} with a separate target for each step.

\section{First Encounter}
\label{intro-first}

As usual we will start with a ``Hello World'' example. Go to \code{examples/helloworld}. Ignoring the other files,
look at \code{hello.c}. It contains the C program
\begin{verbatim}
  #include <stdio.h>

  int main() {
    puts("Hello World");
  }
\end{verbatim}

\begin{description}
\item[Step 1:] (make run)

Try it: compile the program with a C compiler, name it \code{hello} and run it. It should do what you expect.

\item[Step 2:] (make cogent)

Now use Gencot to translate the program to Cogent. Enter the command
\begin{verbatim}
  gencot cfile hello.c
\end{verbatim}
It creates the file \code{hello.cogent}. Look at it. It contains
\begin{verbatim}
  cogent_main : () -> U32
  cogent_main () =
     0
     {-
         cogent_puts("Hello World");
     -}
\end{verbatim}
This is a Cogent function corresponding to the C function \code{main}. However, the function body is still C code
and put in comment. Instead, the dummy result \code{0} is used, so the file is already valid Cogent code.

The command also creates the file \code{hello-entry.ac} (Not yet implemented! Provided with the example).

Next enter the command
\begin{verbatim}
  gencot unit unit.files
\end{verbatim}
It creates three files \code{unit-externs.cogent}, \code{unit-exttypes.cogent}, and \code{unit-dvdtypes.cogent}
where the latter two are empty and can be ignored for this example. The first file contains
\begin{verbatim}
  cogent_puts : (CPtr U8)! -> U32
\end{verbatim}
which is also valid Cogent code, but uses the non-standard generic type \code{CPtr}. 

The command also creates the file \code{unit-externs.ac} (Not yet implemented! Provided with the example).

\item[Step 3:] (make edit)

Now comes the part where your manual work is demanded. You have to translate the function bodies and you have to adapt
some types. Open \code{hello.cogent} in a text editor, replace its content by
\begin{verbatim}
  cogent_main : () -> U32
  cogent_main () =
         cogent_puts("Hello World"); 0
\end{verbatim}
and save as \code{ed-hello.cogent}. In \code{unit-externs.cogent} replace the argument type by \code{String}:
\begin{verbatim}
  cogent_puts : String -> U32
\end{verbatim}
and save as \code{ed-unit-externs.cogent}.

Now you have a Cogent program which is equivalent to the original C program. It consists of several files,
which are all included by the provided file \code{unit.cogent}.

\item[Step 4:] (make cogent-c)

To see that it works, process \code{unit.cogent} by the Cogent compiler. Enter the command
\begin{verbatim}
  cogent -o gen-unit -g unit.cogent --infer-c-funcs="unit-externs.ac hello-entry.ac"
\end{verbatim}
It creates the files \code{gen-unit.c} and \code{gen-unit.h} and two \code{\_pp\_inferred.c} files. All these files
are included by the provided file \code{unit.c}, so this file wraps together the C program generated
from the Cogent program.

\item[Step 5:] (make cogent-run)

Try whether it still works. Compile \code{unit.c} with a C compiler, name it \code{cogent-hello} and run it. The result should be the same
as in Step 1.

Note that for the compilation you need to set the include path to the standard library folder of the Cogent 
distribution (\code{STDGUM} in the Makefile).

\item[Step 6:] (make clean)

Clean up all generated files. If you like you can perform the steps again.
\end{description}

\chapter{Simple C Programs}

\section{Single Source File}
\label{simple-single}

The simplest case is a C program which consists of a single \code{.c} file, such as the ``Hello World'' program
introduced in Section~\ref{intro-first}. If the C file is named \code{foo.c} the command to process it by Gencot is
\begin{verbatim}
  gencot cfile foo.c
\end{verbatim}
It creates the following files:
\begin{description}
\item[\code{foo.cogent}] The content of \code{foo.c} translated to Cogent, as far as Gencot supports a translation.
Function bodies are not translated.
\item[\code{foo-entry.ac}] (*Not yet implemented*) ``Entry wrapper'' functions for all functions defined in \code{foo.c} 
with external linkage.
\end{description}

An ``entry wrapper'' converts from the original C function API to the Cogent function API, which is usually different 
since Cogent functions always take one argument and return one value. Functions in \code{foo.cogent} are automatically
renamed so that they do not collide with the entry wrappers after translation back to C by Cogent. For a standalone
C program there is atleast an entry wrapper \code{main} for the translated main program.

Additionally, Gencot must be invoked using the command
\begin{verbatim}
  gencot unit unit.files
\end{verbatim}
where \code{unit.files} contains the name of the C source file (\code{"foo.c"}) in a single line. It creates the following
files:
\begin{description}
\item[\code{unit-externs.cogent}] Abstract definitions of all external functions used by the C program.
\item[\code{unit-exttypes.cogent}] Type definitions for all external types used by the C program.
\item[\code{unit-dvdtypes.cogent}] Type definitions for all derived types used in the C program.
\item[\code{unit-externs.ac}] (*Not yet implemented*) ``Exit wrapper'' functions for all functions defined in 
\code{unit-externs.cogent}.
\end{description}
An external function is used by the C program if it is actually invoked. Such functions must be declared in C, usually
the declarations are contained in standard include files. In the ``Hello World'' example the file \code{hello.c} includes
\code{stdio.h} where function \code{puts} is declared. Here, Gencot reads \code{stdio.h} to generate the abstract definition for
\code{gencot\_puts} and the corresponding exit wrapper. Although there are many functions declared in \code{stdio.h}, Gencot
does not generate definitions for them, since they are not invoked by the C program.

An external type is a type defined in a standard include file. Again, Gencot provides a Cogent type definition in 
\code{unit-exttypes.cogent} for only those 
types which are actually used by the C program. A derived type is a pointer type, an array type, or a function type in C. For 
some of them Gencot generates auxiliary type definitions in \code{unit-dvdtypes.cogent}. In the ``Hello World'' example 
both files are empty.

An ``exit wrapper'' is similar to an ``entry wrapper'', it converts from the Cogent function API back to the C function API.
Exit wrappers are also renamed so that they do not collide with the originally invoked C function.

All generated files are named according to the name of the file argument to the \code{gencot} command. If the file is named 
\code{foo.files} you will get \code{foo-externs.cogent}, \code{foo-exttypes.cogent} etc.

\section{Single Source File with Include Files}
\label{simple-include}

Often, a C program consists of a \code{.c} file with function definitions and an included \code{.h} file for data type 
definitions. In this case Gencot must additionally be invoked for the \code{.h} file in the form
\begin{verbatim}
  gencot hfile foo.h
\end{verbatim}
It will create the additional file \code{foo-incl.cogent} which contains the translated content of \code{foo.h}. The 
include directive in \code{foo.c} will automatically be adapted, so that \code{foo.cogent} includes \code{foo-incl.cogent}.
So the resulting Cogent program has a similar source file structure as the C program.

If there are several \code{.h} files included by the \code{.c} file gencot must be invoked for every \code{.h} file
separately.

The other files are translated as before, however, the path must be specified where to look for the included files, even
if they are in the same directory. It is specified with an option \code{-I} as for \code{cpp}. Several such options may 
be specified if included files are in different directories. Include paths may be specified relative to the current 
directory. Do not append a slash \code{/} at the end. Paths for system include files cannot be specified, the \code{cpp} 
default is used for accessing them.

Together, the commands for translating the C program are
\begin{verbatim}
  gencot hfile foo.h
  ... <other .h files> ...
  gencot -I. cfile foo.c
  gencot -I. unit unit.files
\end{verbatim}
where \code{unit.files} still contains only the name \code{foo.c} and not \code{foo.h} (because that is automatically
read together with \code{foo.c} which includes it).

\subsection{Example}

The example program in \code{examples/cards} consists of the file \code{cards.c} and the included files \code{cards.h}
and \code{rank.h}. The tranlation step (\code{make cogent}) corresponds to the commands
\begin{verbatim}
  gencot hfile cards.h
  gencot hfile rank.h
  gencot -I. cfile cards.c
  gencot -I. unit unit.files
\end{verbatim}

Note that the function declarations in the \code{.h} files are not translated to Cogent. However, the comments specified 
there are moved to the function definitions in the translated \code{.c} file. This feature depends on the order of
processing the \code{.h} files before the \code{.c} file. Otherwise, the ordering of the commands is irrelevant.

\section{Multiple Source Files}
\label{simple-multi}

A larger C program often consists of several \code{.c} files. Every \code{.c} file is a separate translation unit and
must be translated by the C compiler to a seperate \code{.o} file. The \code{.o} files are then linked together to
yield the executable program binary.



\section{Partial Translation}
\label{simple-partial}

\end{document}
 
