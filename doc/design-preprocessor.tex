
A preprocessor directive always occupies a single logical line, which may consist of several actual lines where 
intermediate line ends are backslash-escaped. No C code can be in a logical line of a preprocessor directive.
However, comments may occur before or after the directive in the same logical line. Therefore, every preprocessor 
directive may have an associated comment before-unit and after-unit, which are transferred as described in 
Section~\ref{design-comments}. Comments embedded in a preprocessor directive are discarded.

We differentiate the following preprocessor directive units:
\begin{itemize}
\item Preprocessor constant definitions
\item all other macro definitions and \code{\#undef} directives,
\item conditional directives (\code{\#if, \#ifdef, \#ifndef, \#else, \#elif, \#endif}),
\item include directives (quoted or system)
\item all other directives, like \code{\#error} and \code{\#warning}
\end{itemize}

To identify constant definitions we resolve all macro definitions as long as they are defined by another single
macro name. If the result is a C integer constant (possibly negative) or a C character constant the macro is assumed
to be a constant definition. All constant definitions are processed
as described in Section~\ref{design-const}.

For comment processing every preprocessor directive is treated as an unstructured source code part.

\subsection{Configurations}
\label{design-preprocessor-config}

Conditional directives are often used in C code to support different configurations of the code. Every configuration
is defined by a combination of preprocessor macro definitions. Using conditional directives in the code, whenever the
code is processed, only the code for one configuration is selected by the preprocessor.

In Gencot the idea is to process all configurations at the same time. This is done by removing the conditional 
directives from the code, process the code, and re-insert the conditional directives into the generated Cogent code.

Only directives which belong to the <package> are handled this way, i.e., only directives which occur in source
files belonging to the <package>. For directives in other included files, in particular in the system include files,
this would not be adequate. First, normally there is no generated target code where they could be re-inserted.
Second, configurations normally do not apply to the system include files.

However, it may be the case that Gencot cannot process two configurations at the same time, because they contain
conflicting information needed by Gencot. An example would be different definitions for the same type which
shall be translated from C to a Cogent type by Gencot.

For this reason Gencot supports a list of conditions for which the corresponding conditional directives are not 
removed and thus only one configuration is processed at the same time. Then Gencot has to be run separately for every
such configuration and the results must be merged manually.

Conditional directives which are handled this way are still re-inserted in the generated target code. This
usually results in all branches being empty but the branches which correspond to the processed configuration.
Thus the branches in the results from separate processing of different configurations can easily be merged manually
or with the help of tools like diff and patch.

Retaining Conditional directives for certain configurations in the processed code makes only sense if the corresponding
macro definitions which are tested in the directives are retained as well. Therefore also define directives can be
retained. The approach in Gencot is to specify a list of regular expressions in the format used by awk. All directives
which match one of these regular expressions are retained in the code to be interpreted before processing the code.
The list is called the ``Gencot directive retainment list'' and may be specified for every invocation of Gencot.

\subsection{Conditional Directives}

Conditional directives are used to suppress some source code according to specified conditions. Gencot aims to
carry over the same suppression to the generated code.

\subsubsection{Associating Conditional Directives to Target Code}

Conditional directives form a hierarchical block structure consisting of ``sections'' and ``groups''. A group
consists of a conditional directive followed by other code. Depending on the directive there are ``if-groups''
(directives \code{\#if, \#ifdef, \#ifndef}), ``elif-groups'' (directive \code{\#elif}), and ``else-groups''
(directive \code{\#else}). A section consists of an if-group, an optional sequence of elif-groups, an optional
else-group, and an \code{\#endif} directive. A group may contain one or more sections in the code after the
leading directive.

Basically, Gencot transfers the structure of conditional directives to the target code. Whenever a source code
part belongs to a group, the generated target code parts are put in the corresponding group. 

This only works if the source code part structure is compatible with the conditional directive structure.
In C code, theoretically, both structures need not be related. Gencot assumes the following restrictions:
Every source code part which overlaps with a section is either completely enclosed in a group or
contains the whole section. It may not span several groups or contain only a part of the section. If a
source code part is structured, contained sections may only overlap with subparts, not with code belonging
to the part itself. 

Based on this assumption, Gencot transfers conditional directives as follows. If a section is contained in an 
unstructured source code part, its directives are discarded. If a section is contained in a structured source
code part, its directives are transferred to the target code part. Toplevel sections which are not contained in
a source code part are transferred to toplevel. Generated target code parts are put in the same group which
contained the corresponding source code part.

It may be the case that for a structured source code part a subpart target must be placed separated from the
target of the structured part. An example is a struct specifier used in a member declaration. In Cogent, the 
type definition generated for the struct specifier must be on toplevel and thus separate from the generated member.
In these cases the condition directive structure must be partly duplicated at the position of the subpart target,
so that it can be placed in the corresponding group there.

Since the target code is generated without presence of the conditional directives structure, they must be 
transferred afterwards. This is done using the same markers \code{\#ORIGIN} and \code{\#ENDORIG} as for the
comments. Since every conditional directive occupies a whole line, the contents of every group consists of
a sequence of lines not overlapping with other groups on the same level. If every target code part is marked 
with the begin and end line of the corresponding source code part, the corresponding group can always be
determined from the markers.

The conditional directives are transferred literally without any changes, except discarding embedded comments. 
For every directive inserted in the target code origin markers are added, so that its associated comment before-
and after-unit will be transferred as well, if present.

\subsection{Macro Definitions}
\label{design-preprocessor-macros}

Preprocessor macros are defined in a definition, which specifies the macro name and the replacement text. Optionally,
a macro may have parameters. After the definition a macro can be used any number of times in ``macro calls''.
A macro call for a parameterless macro has the form of a single identifier (the macro name). A macro call for
a macro with parameters has the form of a C function call: the macro name is followed by actual parameter values
in parenthesis separated by commas. However, the actual parameter values need not be C expressions, they can be 
arbitrary text, thus the macro call need not be a syntactically correct C function call.

Macro calls can occur in C code or in other preprocessor directives (macro definitions, conditional directives, 
and include directives). All macro calls occuring in C code must result in valid C code after full expansion 
by the preprocessor.

\subsubsection{General Approach}

Gencot tries to preserve macros in the translated target code instead of expanding them. In general this requires
to implement two processing aspects: translating the macro definitions and translating the macro calls. Since Gencot
processes the C code separately from the preprocessor directives, macro call processing can be further distinguished
according to macro calls in C code and in preprocessor directives.

Gencot processes the C code by parsing it with a C parser. This implies that macro calls in C code must correspond
to valid C syntax, or they must be preprocessed to convert them to valid C syntax. Note that it is always possible
to do so by fully expanding the macro definition, however, then the macro calls cannot be preserved.

There are several special cases for the general approach of macro processing:
\begin{itemize}
\item if calls for a macro never occur in C code they need not be converted to valid C syntax and they need
only be processed in preprocessor directives. This is typically the case for ``flags'', i.e., macros with
an empty replacement text which are used as boolean flags in conditional directives.
\item if for a parameterless macro all calls in C code occur at positions where an identifier is expected,
the calls need not be converted to valid C syntax and can be processed in the C code. This approach applies
to the ``preprocessor defined constants'' as described in Section~\ref{design-const}. 
\item if for a macro with parameters all calls in C code are syntactically valid C function calls and
always occur at positions where a C function call is expected, the calls need not be converted to valid C 
syntax and can be processed in the C code.
\item if a macro need not be preserved by Gencot, its calls can be converted to valid C code by fully expanding
them. Then the calls are not present anymore and the calls and the definition need not be processed at all.
This approach is used for conflicting configurations as described in Section~\ref{design-preprocessor-config}.
\end{itemize}

When Gencot preserves a macro, there are several ways how to translate the macro definition and the macro calls.
An apparent way is to use again a macro in the target code. Then the macro definition is translated by translating
the replacement text and optionally also the macro name. If the macro name is translated then also all macro calls
must be translated, otherwise macro calls need only be translated if the macro has parameters and the actual 
parameter values must be translated.

How the macro replacement text is translated depends on the places where the macro is used. If it is only used 
in preprocessor directives, usually no translation is required. If it is used in C code parts which are translated to
Cogent code, the replacement text must also be translated to Cogent. If it is used in C function bodies it must
be translated in the same way as C function bodies, i.e., only the identifiers must be mapped and calls of other
macros must be processed.

A macro may also be translated to a target code construct. Then the macro definition is typically translated to 
a target code definition (such as a type definition or a function definition) and the macro calls are translated
to usages of that definition. This approach is used for the ``preprocessor defined constants'' as described in 
Section~\ref{design-const}: the macro definitions are translated to Cogent constant definitions and the macro 
calls occurring in C code are translated to the corresponding Cogent constant names. Additionally, the original
macro definitions are retained and used for all macro calls occurring in conditional directives, which are not
translated. Macro calls in the replacement text of other preprocessor constant definitions are translated to 
the corresponding Cogent constant names.

\subsubsection{Flag Translation}

A flag is a parameterless macro with an empty replacement text. Its only use is in the conditions of 
conditional preprocessor directives, hence macro calls for flags only occur in preprocessor directives.

Gencot translates flags by directly transferring them to the target code. Neither their macro definitions 
nor their macro calls are further processed by Gencot.

\subsubsection{Manual Macro Translation}

Most of the aspects of macro processing cannot be determined and handled automatically by Gencot. Therefore a
general approach is supported by Gencot where macro processing is specified manually for specific macros used in the
translated C program package.

The manual specification consists of the following parts:
\begin{itemize}
\item A specification of all macros which shall not be preserved. This specification is included in the 
Gencot directive retainment list, as described in 
Section~\ref{design-preprocessor-config}. Retained macros are processed by the C preprocessor and thus 
fully expanded before the C code is processed by Gencot.
\item A specification of all parameterless macros which shall not be processed as preprocessor defined
constants or flags. This specification consists of a list of macro names, it is called the ``Gencot manual macro list''.
For all macros in this list a manual translation must be specified. Macros with parameters are never 
processed automatically, for them a manual translation must always be specified if they shall be preserved.
\item For all manually processed macros for which macro calls may occur in C code a conversion to valid C 
code may be specified.
This specification is itself a macro definition for the same macro, where the replacement text must be valid
C code for all positions where macro calls occur. A set of such macro definitions is called a ``Gencot macro 
call conversion''. It is applied to all macro calls and the result is fed to 
the Gencot C code translation and is processed in the usual way, no further manual specification for the macro call
translation can be provided. Since the conversion is applied after all preprocessor directives have been
removed, it has no effect on macro calls in preprocessor directives. 
\item For all manually translated macros a translation of the macro definition may be specified. It has the form 
of arbitrary text marked with a specification of the position of the original macro definition in its source file.
According to this position it is inserted in the corresponding target code file. A collection of such macro 
definition translations is always specifi for a single source file and is called the ``Gencot macro translation''
for the source file.
\end{itemize}
All four parts may be specified as additional input upon every invocation of Gencot.

Macro definitions are always translated at the position where they occurred in the source file.
If the definition occurs in a file \code{x.h} it is transferred to file \code{x-types.cogent} to a corresponding position,
if it occurs in a file \code{x.c} it is transferred to file \code{x-impl.cogent} to a corresponding position.

This implies that translated macro definitions are not available in the file \code{x-globals.cogent} and in the files with
antiquoted Cogent code. If they are used there (which mainly is the case if macro calls occur in a conditional
preprocessor directive which is transferred there), a manual solution is required.

If different C compilation units use the same name for different macros, conflicts are caused in the integrated Cogent
source. These conflicts are not detected by the Cogent compiler. A renaming scheme based on the name of the file 
containing the macro definition would not be safe either, since it breaks situations where a macro is deliberately
redefined in another file. Therefore, Gencot provides no support for macro name conflicts, they must be detected and
handled manually.

\subsubsection{External Macro References}

Whenever a macro call occurs in a source file, it may reference a macro definition which is external in 
the sense of Section~\ref{design-modular-extref}. For such external references the (translated) definition 
must be made available in the Cogent compilation unit.

For all preprocessor defined constants (i.e. parameterless nonempty macros not listed in the Gencot manual 
macro list) Gencot adds the translated macro definition to the file \code{<package>-exttypes.cogent}. For
manually translated macros a separate Gencot macro translation must be specified for external macro definitions.
For them the position specification is omitted, they are simply appended to the file 
\code{<package>-exttypes.cogent}. If this is not sufficient, because macro calls already occur in 
\code{<package>-exttypes.cogent}, they must be inserted manually at the required position.

To avoid introducing additional external references, in the macro replacement text for preprocessor defined 
constants all macro calls are resolved to existing external reference names or until they are fully resolved.
Manually translated macro definitions should handle external macro calls in a similar way.

\subsection{Include Directives}

In C there are two forms of include directives: quoted includes of the form
\begin{verbatim}
  #include "x.h"
\end{verbatim}
and system includes of the form
\begin{verbatim}
  #include <x.h>
\end{verbatim}
Additionally, there can be include directives where the included file is specified by a preprocessor macro call,
they have the form
\begin{verbatim}
  #include MACROCALL
\end{verbatim}
for them it cannot be determined whether they are quoted or system includes.

Files included by system includes are assumed to be always external to the translated <package>, therefore system
include directives are discarded in the Cogent code. The information required by external references from system 
includes is always fully contained in the file \code{<package>-exttypes.cogent}.

Macro call includes are normally assumed to be quoted includes and are treated similar.

\subsubsection{Translating Quoted Include Directives}

Quoted include directives for a file \code{x.h} which belongs to the Cogent compilation unit are always translated 
to the corresponding Cogent preprocessor include directive
\begin{verbatim}
  #include "x-types.h"
\end{verbatim}
If the original include directive occurs in file \code{y.c} the translated directive is put into the file 
\code{y-impl.cogent}. If the original include directive occurs in file \code{y.h} the translated directive is put into the file
\code{y-types.h}. 

Other quoted include directives and macro call includes are transferred to the Cogent source file without modifications, if
necessary they must be processed manually.

\subsection{Other Directives}

All other preprocessor directives are discarded. Gencot displays a message for every discarded directive.

 
