The goal of this step is to create the item property declarations (see Section~\ref{impl-itemprops})
to be used in the translation from C to Cogent. This step is optional, Gencot can translate a C source 
without them. However, the assumptions it uses for the translation will in many cases lead to 
invalid Cogent code (which is syntactically correct but does not pass the semantics checks).

Instead of specifying the additional information used for the translation as annotations in the C source files,
Gencot uses the item property declarations which reside in separate files and reference the affected C 
items by name (the ``item identifiers'' described in Section~\ref{impl-itemprops-ids}. This makes it 
easier to replace the C sources by a newer version and still reuse (most of) the item properties.

Item properties could be specified fully manually by the developer, if she knows the item naming scheme
and searches the C source files for items which require properties to be declared. However, Gencot provides
support by generating default item property declarations. This has the following advantages:
\begin{itemize}
\item The developer gets a list of all items with their correct names.
\item Gencot can use simple heuristics to populate the declarations with useful default properties.
\end{itemize}

The default item property declarations are intended to be manually modified by the developer. Additionally, the 
results of a parameter modification analysis described in Section~\ref{app-parmod} are intended to be transferred
to the item property declarations.

Property declarations are used by Gencot, if present, for all items processed in the translation. These are all internal items 
which are defined in the translation set and all external items which are declared and used in the translation set.

The automatic parts of this step are executed with the help of the script command \code{items} (see 
Section~\ref{impl-all-items}).

\subsection{Grouping Item Property Declarations in Files}
\label{app-itemprops-files}

C items for which Gencot supports property declarations are types (struct/unions and named types), functions and objects (``variables'').
Items may have sub-items (function parameters and results, array elements, pointer targets). 

For type items there is always a single definition (which also defines sub-items). For functions and objects there 
can be declarations
in addition to the single definition. The definition or declaration introduces a name for the item and it specifies its type.
Afterwards, an item can be used any number of times (e.g. type items can be used in other definitions and declarations, functions
can be invoked and objects can be read and written).

In most cases Gencot uses the properties of such items only when translating the item's definition or declaration. However, there 
are also cases where item properties are required when translating the item's use. For example, as described in 
Section~\ref{design-types-typedef}, the generic type \code{MayNull} is applied to all \textit{references} of a translated type 
name, depending on the Not-Null property of the type name. Therefore, when Gencot translates a C source file it needs the 
property declarations for all items defined or declared in the source file and all included files.
Gencot expects the item properties to be used when translating file \code{x.<e>} (where \code{<e>} 
is \code{c} or \code{h}) in an auxiliary file named \code{x.<e>-itemprops} (see Section~\ref{impl-all-gencot}).

Item properties are intended to be declared manually. For this it is useful to have a single place for every item where
to expect its property declaration. For a small <package> a single common file would be feasible. To support also larger
<package>s Gencot allows grouping property declarations according to the source files where the items are defined. The 
command \code{items file} (see Section~\ref{impl-all-items}) must be executed for every single source file and it 
generates the item identifier list with the default property declarations only for the items defined in the file itself.

To group item property declarations it is recommended to store the result of \code{items file} applied to \code{<file>}
in a file \code{<file>-dfltprops}. Then manually added properties should be specified in a seperate file \code{<file>-manprops}
so that they are not overwritten when re-creating the default properties. 

When Gencot translates a single C source file it ignores all declarations and only processes item definitions and uses. 
However, when Gencot processes all C source files of the translation set together to generate the additional Cogent sources for the
Cogent compilation unit (command \code{gencot unit}, see Section~\ref{impl-all-gencot}), it processes items defined in the
translation set and also external items declared and used in the translation set. Therefore it needs property declarations 
for all these items, it expects them in the file \code{<ufile>-itemprops} where \code{<ufile>} is the unit file. 
These property declarations can be retrieved by uniting all item property declaration files for the single C sources and adding
property declarations for the external used items. 

The default item properties for the external used items are generated by command \code{items unit} (see Section~\ref{impl-all-items})
applied to a unit file \code{<ufile>}. To group item property declarations it is recommended to store the result in a file
\code{<ufile>-dfltprops} and specify manually added properties in a file \code{<ufile>-manprops}.

Since the properties in the \code{-itemprops} files may originate from several sources (default properties, manual specification, 
parmod descriptions, merging other property declarations) it is recommended to keep these sources separate and recreate
the \code{-itemprops} files from these sources whenever they are used for a translation. 

Then the file \code{<file>-itemprops} for a single C source file is generated by first modifying \code{<file>-dfltprops} 
and all \code{<f>-dfltprops} for included files \code{<f>} and the unit file according to the parameter modification descriptions 
and then adding the properties from \code{<file>-manprops} and all \code{<f>-manprops} for included files \code{<f>} and the unit 
file. All resulting property declarations must be merged. The file \code{<ufile>-itemprops} can then be generated by 
merging the \code{x.c-itemprops} files for all files \code{x.c} listed in the unit file \code{ufile}.

Gencot also supports item property declarations for a restricted form of type expressions (derived pointer types). These items
are not explicitly defined, they are used by simply denoting the type expression. For this kind of type items the properties are 
needed by Gencot for translating the item uses. 

Gencot does not provide default item property declarations for these items, they must always be specified manually. Since they 
are not related to a specific C source file or a specific translation set, they should be put in a single separate file. The 
recommended name for this file is \code{common.types-manprops}. The properties in this file must be used for the translation 
of all single C source files and for the generation of the additional Cogent sources for the compilation unit. This is 
achieved by merging the properties in this file to all generated \code{-itemprops} files.

\subsection{Generating Default Item Property Declarations}
\label{app-itemprops-default}

The first step for using item property declarations is to generate the default property declarations. For a single C source 
file or include file \code{<file>} the command
\begin{verbatim}
  items file <file>
\end{verbatim}
is used. The result should be stored in file \code{<file>-dfltprops}.

To generate the default property declarations for the external items used by the Cogent compilation unit the command
\begin{verbatim}
  items unit
\end{verbatim}
is used. The result should be stored in file \code{<uname>.unit-dfltprops} where \code{<uname>} is the unit name. 

\subsection{Transferring Parameter Modification Descriptions}
\label{app-itemprops-parmod}

If parameter modification descriptions are used as described in Section~\ref{app-parmod}, the result must be converted to item
property declarations to be used by Gencot. The result consists of a single file \code{<jfile>} in JSON format containing 
parameter modification descriptions. All entries have been confirmed and the file has been evaluated as described in 
Section~\ref{app-parmod-extern}. The file contains entries for all functions defined in the translation set, for all external 
functions used in the translation set, and maybe additional functions which have been used to follow dependencies.

Every function is an item, and every parameter is a sub-item of its function. Thus the information in the parameter modification
description can be converted to property declarations for function and parameter items. This is done using the command
\begin{verbatim}
  parmod out <jfile>
\end{verbatim}
It will create the file \code{<jfile>-itemprops}.

The file \code{<jfile>-itemprops} contains properties to be added to / removed from the default properties. 
For every file \code{<file>-dfltprops} the command 
\begin{verbatim}
  items mergeto <file>-dfltprops <jfile>-itemprops
\end{verbatim}
is used. Its output should be stored in an intermediate file \code{<file>-ip1}. The command only selects from 
\code{<jfile>-itemprops} declarations for items where items with the same toplevel item are already present in
\code{<file>-dfltprops}. In this way the collective information in \code{<jfile>-itemprops} is distributed to the different
specific property declaration files for the single C sources and the file \code{<uname>.unit-dfltprops}. Information about functions
only used for following dependencies is not transferred, it is not used by the translation to Cogent.

\subsection{Adding Manually Specified Properties}
\label{app-itemprops-manual}

It is recommended to put manually specified item properties in separate files, grouped in the same way as the default 
properties. For every file \code{<file>-dfltprops} a file \code{<file>-manprops} should be used. To create this file,
you can make a copy of \code{<file>-dfltprops}, manually remove all declared properties and then populate the empty
declarations with manually specified properties. Remaining empty declarations can be removed.

The manual properties are then merged to the default properties. This makes it possible to re-apply the manual properties 
whenever the default properties change (e.g., because the C source is modified). To avoid removing manually specified 
properties by transmitting parameter modification description results, it is recommended to merge the manual properties 
afterwards into the intermediate file \code{<file>-ip1} using the command
\begin{verbatim}
  items mergeto <file>-ip1 <file>-manprops
\end{verbatim}
and storing the result in another intermediate file \code{<file>-ip2}.

If properties shall be specified for type expression items, they should be collected in the file \code{common.types-manprops}.
No default property declarations exist for these items. Since the type expressions may be used in all C source files, the 
file should be merged into \textit{all} item property declaration files additionally to the file specific manual properties.
This must be done using the command
\begin{verbatim}
  items merge <file>-ip2 common.types-manprops
\end{verbatim}
because here the added items shall not be reduced to those already present in \code{<file>-ip2}. The result 
should be written to an intermediate file \code{<file>-ip3}. Note that this file does not contain negative properties anymore,
since they are processed and removed by \code{items mergeto} and \code{items merge}.

For convenience for the developer Gencot supports empty lines and comment lines starting with a hash sign \code{\#} in the 
\code{-manprops} files. These lines are removed by the commands \code{items merge} and \code{items mergeto}.

\subsection{Property Declarations for C Source Translations}
\label{app-itemprops-unit}

The resulting intermediate files \code{<f>-ip3} must be merged to create the files \code{<file>-itemprops} used 
by Gencot when translating C sources. For a file \code{x.h} translated by \code{gencot hfile} or \code{gencot config}
the file \code{x.h-ip3}, all \code{<f>-ip3} for (transitively) included non-system files \code{<f>}, and \code{<uname>.unit-ip3}
must be merged. For a file \code{x.c} translated by \code{gencot cfile}
the file \code{x.c-ip3}, all \code{<f>-ip3} for (transitively) included non-system files \code{<f>}, and \code{<uname>.unit-ip3}
must be merged. For a unit file \code{<uname>.unit} processed by \code{gencot unit} all the property declarations required
for all files \code{x.c} listed in \code{<uname>.unit} must be merged, which can be done by merging all corresponding
files \code{x.c-itemprops}.

Merging item property files is done by iteratively using the command
\begin{verbatim}
  items merge
\end{verbatim}
This command removes all duplicate item declarations which may originate from earlier merging operations.

