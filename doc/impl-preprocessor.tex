\subsection{Filters for Processing Steps}

Directive processing is done for the output of \code{gencot-remcomments}. All comments have been removed. 
However, there may be line directives present. 

The filter \code{gencot-selpp} selects all preprocessor directives and
copies them to the output without changes. All other lines are replaced by empty lines, so that the original
line numbers for all directives can still be determined.

The filter \code{gencot-rempp <file>} removes all preprocessor directives from its input, replacing them by empty
lines. All other lines are copied to the output without modification. If \code{<file>} is specified it must 
contain a list of regular expressions for directives which shall be retained.

How the directives are processed depends on the kind of directives (see Section~\ref{design-preprocessor}).
For every kind \code{X} from \code{const}, \code{cond}, \code{macro}, \code{incl}, Gencot provides the processing
filter \code{gencot-prcX}. Since also the way of merging the results into the Cogent code depends on the
kind, for each kind Gencot provides the merging filter \code{gencot-mrgX <file>}. It merges the content in \code{<file>} 
into the input and writes the merged code to the output. \code{<file>} must contain the output of \code{gencot-prcX}
which originated from the same file as the input.

\subsection{Separating Directives}

Gencot supports to keep some directives in the output of \code{gencot-rempp} to handle cases where
the C code of different groups in a section causes conflicts. These conditional directives are still selected
by \code{gencot-selpp} and re-inserted by \code{gencot-mrgcond}. 

\subsubsection{Filter \code{gencot-selpp}}

The filter for selecting preprocessor directives from the input for separate processing and insertion into
the generated target code is implemented as an awk script.

It detects all kinds of preprocessor directives, which always begin at the beginning of a separate line.
A directive always ends at the next newline which is not preceded by a backslash \code{\\}. All corresponding
lines are copied to the output without modifications with the exception of line directives.

Line directives in the input are expanded to the required number of empty lines
which have the same effect. This is done to simplify reading the input for all \code{gencot-prcX} filters.

Every other input line is replaced by an empty line in the output.

\subsubsection{Filter \code{gencot-rempp}}

The filter for removing preprocessor directives from its input is implemented as an awk script.
Basically, it replaces lines which are a part of a directive by empty lines. However, there are the following
exceptions:
\begin{itemize}
\item line directives are never removed, they are required to identify the position in the original source
during code processing.
\item system include directives are never removed, they are inteded to be interpreted by the language-c
preprocessor to make the corresponding information available during code processing. Since it is assumed that
all quoted include directives have already been processed in an initial step, simply all include
directives are retained.
\item directives which match a regular expression from a specified list are not removed, they are intended
to be interpreted by the language-c preprocessor to suppress information which causes conflicts during code
processing.
\end{itemize}

For conditional directives always all directives belonging to the same section are treated in the same way.
To retain them the first directive (\code{\#if}, \code{\#ifdef}, \code{\#ifndef}) must match a regular expression
in the list. For all other directives of a section (\code{\#else}, \code{\#elif}, \code{\#endif}) the 
regular expressions are ignored.

The regular expressions are specified in the argument file line by line. An example file content is
\begin{verbatim}
  ^[[:blank:]]*#[[:blank:]]*if[[:blank:]]+!?[[:blank:]]*defined\(SUPPORT_X\)
  ^[[:blank:]]*#[[:blank:]]*define[[:blank:]]+SUPPORT_X
  ^[[:blank:]]*#[[:blank:]]*undef[[:blank:]]+SUPPORT_X
\end{verbatim}
It retains all directives which define the macro \code{SUPPORT\_X} or depend on its definition.

\subsection{Processing Directives}

\subsubsection{Processing Constants Defined as Preprocessor Macros}

We provide the script \code{convert-const.csh} for automating this task. If comments should be also be converted to Cogent 
the script can be used together with the script \code{convert-comment.csh}.

\subsubsection{Processing Other Preprocessor Directives}

The line numbers for positions count actual lines. Therefore the position of a preprocessor directive is specified by its starting line and its ending line. 

\subsection{Filter \code{gencot-prcconst}}

\subsection{Merging Directive Processing Results}

\subsection{Filter \code{gencot-mrgcond}}

 
