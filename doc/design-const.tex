Often a C source file contains constant definitions of the form
\begin{verbatim}
  #define CONST1 123
\end{verbatim}
The C preprocessor substitutes every occurrence of the identifier \code{CONST1} in every C code after the definition 
by the value 123. This is a special application of the C preprocessor macro feature.

Constant names defined in this way may have arbitrary C constants as their value. Gencot only handles integer,
character, and string constants, floating constant are not supported since they are not supported by Cogent.

\subsection{Processing Direct Integer Constant Definitions}

Constant definitions of this form could be used directly in Cogent, since they are also supported by the Cogent preprocessor.
By transferring the constant definitions to the corresponding file \code{x-types.cogent} the identifiers are available
in every Cogent file including \code{x-types.cogent}. 

However, for generating proofs it should be better to use Cogent value definitions instead of having unrelated 
literals spread across the code. The Cogent value definition corresponding to the constant definition above can either 
be written in the form
\begin{verbatim}
  #define CONST1 123
  const1: U8
  const1 = CONST1
\end{verbatim}
preserving the original constant definition or directly in the shorter form
\begin{verbatim}
  const1: U8
  const1 = 123
\end{verbatim}
Since the preprocessor name \code{CONST1} may also be used in \code{\#if} directives, we use the first form. A typical pattern 
for defining a default value is
\begin{verbatim}
  #if !defined(CONST1)
  #define CONST1 123
  #endif
\end{verbatim}
This will only work if the preprocessor name is retained in the Cogent preprocessor code.

If different C compilation units use the same preprocessor name for different constants, the generated Cogent value definitions
will conflict. This will be detected and signaled by the Cogent compiler. Gencot does not apply any renaming to prevent these
conflicts.

For the Cogent value definition the type must be determined. It may either be the smallest primitive type covering the value 
or it may always be U32 and, if needed, U64. The former requires to insert upcasts whenever the value is used for a different 
type. The latter avoids the upcast in most cases, however, if the value should be used for a U8 or U16 that is not possible 
since there is no downcast in Cogent. Therefore the first approach is used.

Constant definitions are also used to define negative constants sometimes used for error codes. Typically they are used for 
type \code{int}, for example in function results. Here, the type cannot be determined in the way as for positive values, since the 
upcast does not preserve negative values. Therefore we always use type U32 for negative values, which corresponds to type 
\code{int}. This may be wrong, then a better choice must be used manually for the specific case.

Negative values are specified as negative integer literals such as -42. To be used in 
Cogent as a value of type U32 the literal must be converted to an unsigned literal using 2-complement by: 
\code{complement(42 - 1)}.
Since Cogent value definitions are translated to C by substituting the \textit{expression} for every use, it should be as 
simple as possible, such as \code{complement 41} or even \code{0xFFFFFFD6} which is \code{4294967254} in decimal notation.

As described in Section~\ref{design-names}, names for preprocessor defined constants are always mapped to a different
name for the use in Cogent. This is not strictly necessary, if a preprocessor name is lowercase. By convention, C preprocessor 
constant definitions use uppercase identifiers, thus they normally must be mapped anyways.

For comment processing, every preprocessor constant definition is treated as an unstructured source code part.

\subsection{Processing Direct Character and String Constant Definitions}

A character constant definition has the form
\begin{verbatim}
  #define CONST1 'x'
\end{verbatim}
It is translated to a Cogent value definition similar as for integer constants. As type always \code{U8}
is used, the constant is transferred literally.

A string constant definition has the form
\begin{verbatim}
  #define CONST1 "abc"
\end{verbatim}
It is translated to a Cogent value definition similar as for integer constants. As type always \code{String}
is used.

In C it is also possible to specify a string constant by a sequence of string literals, which will be concatenated.
A corresponding string constant definition has the form
\begin{verbatim}
  #define CONST1 "abc" "def"
\end{verbatim}
Since there is no string concatenation operator in Cogent, the concatenation is performed by Gencot and
a single string literal is used in the Cogent value definition.

\subsection{Processing Indirect Constant Definitions}

A constant definition may also reference a previously defined constant in the form 
\begin{verbatim}
  #define CONST2 CONST1
\end{verbatim}

In this case the Cogent constant definition uses the same type as that for \code{CONST1} and
also references the defined Cogent constant and has the form
\begin{verbatim}
  #define CONST2 CONST1
  const2: U8
  const2 = const1
\end{verbatim}

\subsection{Processing Expression Constant Definitions}

A constant definition in C may also specify its value by an expression. In this case the C preprocessor will replace
the constant upon every occurrence by the expression, every expression according to the C syntax is admissible. 

In this case Gencot also generates a Cogent value definition and transfers the expression. Gencot does not evaluate
or translate the expression, however, it maps all contained names of other preprocessor defined constants to their
Cogent form, so that they refer the corresponding Cogent value name. As type for an expression Gencot always assumes
\code{int}, i.e. \code{U32} in Cogent.

If the expression is of type \code{int} and only uses operators which also exist in Cogent, positive integer constants 
and preprocessor defined constant names, the resulting expression will be a valid Cogent expression. In all other cases
the Cogent compiler will probably detect a syntax error, these cases must be handled manually.

\subsection{External Constant References}

If the constant \code{CONST1} is an external reference in the sense of Section~\ref{design-modular-extref}, a corresponding
Cogent constant definition is generated in the file \code{<package>-exttypes.cogent}. It has the same form
\begin{verbatim}
  #define CONST1 123
  const1: U8
  const1 = CONST1
\end{verbatim}
as for a non-external reference. Thus we define the original preprocessor constant name \code{CONST1} here, although
it is already defined in the external origin file. The reason for this approach is that the define directive here 
is intended to be processed by the Cogent preprocessor. Therefore we cannot include the origin file to make the name
available, since that would also include the C code in the origin file.

If the external definition is indirect, the value used in the define directive is always resolved to the final
literal or to an existing external reference. This is done for determining the Cogent type for the constant 
and avoids introducing unnecessary intermediate constant names.

