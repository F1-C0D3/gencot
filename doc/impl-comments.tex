
In a first step all comments are removed from the C source file and are written to a separate file. The remaining 
C code is processed by Gencot. In a final step the comments are reinserted into the generated code.

Additional steps are used to move comments from declarations to definitions.

The filter \code{gencot-selcomments} selects all comments from the input, translates them to Cogent comments and writes 
them to the output.
The filter \code{gencot-remcomments} removes all comments from the input and writes the remaining code to the output.
The filter \code{gencot-mrgcomments <file>} merges the comments in \code{<file>} into the input and writes the
merged code to the output. <file> must contain the output of \code{gencot-selcomments} applied to a code X, the 
input must have been generated from the output of \code{gencot-remcomments} applied to the same code X.

\subsection{Filter \code{gencot-remcomments}}

The filter for removing comments is implemented using the C preprocessor with the option \code{-fpreprocessed}. With this
option it removes all comments, however, it also processes and removes \code{\#define} directives. To prevent this, a sed
script is used to insert an underscore \code{\_} before every \code{\#define} directive which is only preceded 
by whitespace in its line. Then it is not recognized by the preprocessor. Afterwards, a second sed script removes the underscores.

Instead of an underscore an empty block comment could have been used. This would have the advantage that the second sed script
is not required, since the empty comments are removed by the preprocessor. The disadvantage is that the empty comment
is replaced by blanks. The resulting indentation does not modify the semantics of the \code{\#define} statenments but it
looks unusual in the Cogent code.

The preprocessor also removes \code{\#undef} directives, hence they are treated in the same way.

The preprocessor preserves all information about the original source line numbers, to do so it may insert 
line directives of the form \code{\# <linenumber> <filename>}. They must be processed by all following filters. The Haskell
C parser \code{language-c} processes these line directives.

\subsection{Filter \code{gencot-selcomments}}

The filter for selecting comments is implemented as an awk script. It scans through the input for the comment start
sequences \code{/*} and \code{//} to identify comments. It translates C comments to Cogent comments in the output.
The translation is done here since the filter must identify the start and end sequences of comments, so it can 
translate them specifically. Start and end sequences which occur as part of comment content are not translated.

To keep it simple, the cases when the comment start sequences can occur in C code parts are ignored. This may lead to
additional or extended comments, which must be corrected manually. It never leads to omitted comments or missing 
comment parts. Note that \code{gencot-remcomments} always identifies comments correctly, since there comment detection
it is implemented by the C preprocessor.

To distinguish before-units and after-units, \code{gencot-selcomments} inserts a separator between them. The separator
consists of a newline followed by \code{-\}\_}. It is constructed in a way that it cannot be a part of or overlap with a 
comment and to be easy to detect when processing the output of \code{gencot-selcomments} line by line. The newline and
the \code{-\}} would end any comment. The underscore (any other character could have been used instead) distinguishes the
separator from a normal end-of-comment, since \code{gencot-selcomments} never inserts an underscore immediately after a comment. 

The separator is inserted after every unit, even if the unit is empty. The first unit in the output of \code{gencot-selcomments}
is always a before-unit.

When in an input line code is found outside of comments all this code with all embedded comments is replaced by the
separator. Only the comments before and after the code are translated to the output, if present.
Note, that the separator includes a newline, hence every source line with code outside of
comments produces two output lines. 

An after-unit consists of all comments after code in a line. The last comment is either a line comment or
it may be a block comment which may include following lines. After this last comment the after-unit ends and
a separator is inserted.

All whitespace in and between comments and before the first comment in a before-unit is preserved in the output, 
including empty lines. After a before-unit only empty lines are preserved. Whitespace around code is typically used
to align code and comments, this must be adapted manually for the generated target code. Whitespace after an after-unit
is not preserved since the last comment in an after unit in the target code is always followed by a newline.

The filter never deletes lines, hence in its output the original line numbers can still be determined by counting lines, 
if the newlines belonging to the separators are ignored.

\subsubsection{State Machine}

The implementation processes the input line by line using a finite state machine. It uses the variables \code{before} and 
\code{after} to collect block comments at the beginning and end of the current line, initially both are empty. 
The collect action appends the input from the current position up to and including the next found item in 
the specified variable. The separate action appends the separator to the specified variable. The output action 
writes the specified content to the output, replacing C comment start and end sequences by their Cogent counterpart.
The newline action advances to the beginning of the next line and clears \code{before}.

The state machine has the following states, nocode is the initial state:

\begin{description}
\item[nocode] If next is
  \begin{description}
  \item[end-of-line] output(\code{before}); newline; goto nocode
  \item[block-comment-start] collect(\code{before}); goto nocode-inblock
  \item[line-comment-start] collect(\code{before}); output(\code{before} + line-comment); newline; goto nocode
  \item[other-code] separate(\code{before}); clear(\code{after}); goto code
  \end{description}
\item[nocode-inblock] If next is
  \begin{description}
  \item[end-of-line] collect(\code{before}); output(\code{before}); newline; goto nocode-inblock
  \item[block-comment-end] collect(\code{before}); goto nocode
  \end{description}
\item[code] If next is
  \begin{description}
  \item[end-of-line] output(\code{before} + separator); newline; goto nocode
  \item[block-comment-start] append comment-start to \code{after}; goto code-inblock
  \item[line-comment-start] output(\code{before} + line-comment + separator); newline; goto nocode
  \end{description}
\item[code-inblock] If next is
  \begin{description}
  \item[end-of-line] collect(\code{after}); output(\code{before} + \code{after}); newline; goto aftercode-inblock
  \item[block-comment-end] collect(\code{after}); goto code-afterblock
  \end{description}
\item[code-afterblock] If next is
  \begin{description}
  \item[end-of-line] output(\code{before} + \code{after} + separator); newline; goto nocode
  \item[block-comment-start] collect(\code{after}); goto code-inblock
  \item[line-comment-start] collect(\code{after}); output(\code{before} + \code{after} + line-comment + separator); newline; goto nocode
  \item[other-code] clear(\code{after}); goto code
  \end{description}
\item[aftercode-inblock] If next is
  \begin{description}
  \item[end-of-line] collect(\code{before}); output(\code{before}); newline; goto aftercode-inblock
  \item[block-comment-end] collect(\code{before}); separate(\code{before}); goto nocode
  \end{description}
\end{description}

\subsection{Filter \code{gencot-mrgcomments}}

The filter for merging comments into the target code is implemented as an awk script. It consists of a function 
\code{flushbufs}, a BEGIN rule, a line rule, and an END rule. 

The BEGIN rule reads the <file> line by line and collects before- and after-units as strings in the arrays \code{before} and \code{after}. The arrays are indexed with the (original) line number of the separator between before- and after-unit.

The line rule uses a buffer for its output. It is used to process all \code{\#ORIGIN} and \code{\#ENDORIG} markers around
a nonempty line and collect the associated comment units and content. The END rule is used to flush the buffer.

For every consecutive sequence of \code{\#ORIGIN} markers (i.e., separated by a single line containing only whitespace),
the before units associated with the line numbers of all markers with 
a ``+'' sign are collected in a buffer. The following nonempty code line is put in a second buffer. 
For every consecutive sequence of \code{\#ENDORIG} markers, the after units associated with the line numbers of all 
markers with a ``+'' sign are collected in a third buffer. Whenever a code line or an \code{\#ENDORIG} marker is 
followed by a line which is no \code{\#ENDORIG} marker, the content of all three buffers is processed by the 
function \code{flushbufs}.

Normally, the buffers are output and reset. In the case that the third buffer is empty (there are no after 
units for the code) the code line is retained in a separate buffer. If in the next group the first buffer is 
empty (there are no before units for the code) the code lines of both groups are concatenated without a newline 
in between (also removing the indentation for the second line). This has the effect of completely removing all 
markers which are not used for inserting comments at their position, thus removing all unnecessary line breaks.

Markers are also completely removed, if a unit exists but contains only whitespace (such as empty lines for
formatting). This is done because the target code is already formatted during generation and inserting additional
whitespace for formatting is not useful.

In the buffer, before-units are concatenated without any separator. After-units are separated by a newline to end possibly 
trailing line comments.

\subsection{Declaration Comments}
\label{impl-comments-decl}

To safely detect C declarations and C definitions Gencot uses the language-c parser. 

Only comments associated with declarations with external linkage are transferred to their definitions. For declarations
with internal linkage the approach for transferring the comments does not work, since the declared names need not be
unique in the <package>.

Processing the decalaration comments is implemented by the following filter steps. 

\subsubsection{Filter \code{gencot-deccomments}}

The filter \code{gencot-deccomments} parses the input. For every declaration it outputs a line
\begin{verbatim}
  #DECL <name> <bline>
\end{verbatim}
where \code{<name>} is the name of the declared item, and \code{<bline>} is the original source line 
number where the declaration begins.

The filter implementation is described in Section~\ref{impl-ccomps-decls}.

\subsubsection{Filter \code{gencot-movcomments}}

The filter \code{gencot-movcomments <file>} processes the output of \code{gencot-deccomments} as input. For every
line as above, it retrieves the before-unit of \code{<bline>} from \code{<file>}
and stores it in the file \code{<name>.comment} in the
current directory. The content of \code{<file>} must be the output of \code{gencot-selcomments} applied to the same
original source from which the input of \code{gencot-deccomments} has been derived.

The filter is implemented as an awk script. It consists of a 
BEGIN rule reading \code{<file>} in the same way as \code{gencot-mrgcomments}, and a rule for lines starting with 
\code{\#DECL}. For every such line it writes the associated comment unit to the comment file. A comment
file is even written if the comment unit is empty.

\subsubsection{Filter \code{gencot-defcomments}}

For inserting the comments before the target code parts generated from a definition, Gencot uses the marker
\begin{verbatim}
  #DEF <name>
\end{verbatim}
The marker must be inserted by the filter which generates the definition target code.

The filter \code{gencot-defcomments <dir>} replaces every marker line in its input by the content of the corresponding
\code{.comment} file in directory \code{<dir>} and writes the result to its output.

It is implemented as an awk script with a single rule for all lines. If the line starts with \code{\#DEF} it is
replaced by the content of the corresponding file in the output. All other lines are copied to the output without 
modification.
