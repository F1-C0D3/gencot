The following auxiliary Gencot components exist which do not process C source code:
\begin{description}
\item[\code{parmod-proc}] processes parameter modification descriptions in JSON format (see Section~\ref{impl-parmod}).
\end{description}

\subsection{Parameter Modification Descriptions}

Processing parameter modification descriptions is implemented by the filter \code{parmod-proc}. It reads a parameter
modification descriptions from standard input as described in Section~\ref{impl-parmod-json}. The kind of processing
depends on whether an additional parameter is specified for the filter.

If no additional parameter is specified, the filter generates a textual summary of the description. It lists
all unconfirmed parameter descriptions in the form
\begin{verbatim}
  <function identifier> / <parameter identifier>
\end{verbatim}
and it lists all required invocations by their function identifiers.

This summary is intended as a support for the developer when filling the description manually. The goal is that
the summary is empty. If there are unconfirmed parameters they must be inspected and confirmed. This usually 
modifies the list of required invocations. They can be reduced by generating and adding corresponding descriptions
from other source files.

If an additional parameter is specified, it must be the file name of another parameter modification description.
In this case all required invocations which are found in this second description are added (appended) to the main 
description and the result is written to the output as described in Section~\ref{impl-parmod-json}.
