\documentclass[a4paper]{report}
\usepackage[bookmarks]{hyperref}
\usepackage[utf8]{inputenc}
\usepackage[T1]{fontenc} % needed for italic curly braces

\newcommand{\code}[1]{\textnormal{\texttt{#1}}}

\begin{document}

\title{Gencot User Manual}
\author{Gunnar Teege}

\maketitle

\chapter{Introduction}

Gencot (GENerating COgent Toolset) is a set of tools for generating Cogent code from C code. 

Gencot is used for parsing the C sources and generating Cogent sources, 
antiquoted C sources, and auxiliary C code. It does not perform a fully automatic translation, it is
intended to be used in combination with several manual steps of pre- and post-processing. These
steps are described in this manual.

The manual assumes that you are familiar with C and Cogent and know how to work with both.

\section{Distribution}
\label{intro-dist}

The Gencot distribution consists of the following folders:
\begin{description}
\item[manual] this manual,
\item[bin] the main command scripts \code{gencot} and \code{parmod} and many auxiliary commands used by them,
\item[include] Cogent include files used by the generated code,
\item[c] C code implementing abstract types and functions used by the generated code,
\item[examples] example C programs used in this manual for introducing Gencot,
\item[src] the Haskell source code of Gencot components,
\item[doc] a comprehensive documentation of Gencot design and implementation.
\end{description}

Gencot is a command line tool. To use it make sure that you can invoke the commands \code{gencot} and
\code{parmod} (e.g., by linking them in a folder in your command path or by adding the \code{bin} folder
of the Gencot distribution to your command path.

Additionally you have to set the environment variable \code{\$GENCOT\_HOME} to the root folder of the Gencot
distribution.

We also assume that you have a working distribution of Cogent and can invoke the cogent compiler using the
command \code{cogent}.

All example folders contain a UNIX Makefile. You can either run the examples by manually typing the commands
to process them or by using \code{make} with a separate target for each step.

\section{First Encounter}
\label{intro-first}

As usual we will start with a ``Hello World'' example. Go to \code{examples/helloworld}. Ignoring the other files,
look at \code{hello.c}. It contains the C program
\begin{verbatim}
  #include <stdio.h>

  int main() {
    puts("Hello World");
  }
\end{verbatim}

\begin{description}
\item[Step 1:] (make run)

Try it: compile the program with a C compiler, name it \code{hello} and run it. It should do what you expect.

\item[Step 2:] (make cogent)

Now use Gencot to translate the program to Cogent. Enter the command
\begin{verbatim}
  gencot cfile hello.c
\end{verbatim}
It creates the file \code{hello.cogent}. Look at it. It contains
\begin{verbatim}
  cogent_main : () -> U32
  cogent_main () =
     0
     {-
         cogent_puts("Hello World");
     -}
\end{verbatim}
This is a Cogent function corresponding to the C function \code{main}. However, the function body is still C code
and put in comment. Instead, the dummy result \code{0} is used, so the file is already valid Cogent code.

The command also creates the file \code{hello-entry.ac} (Not yet implemented! Provided with the example).

Next enter the command
\begin{verbatim}
  gencot unit hello.unit
\end{verbatim}
It creates three files \code{hello-externs.cogent}, \code{hello-exttypes.cogent}, and \code{hello-dvdtypes.cogent}
where the latter two are empty and can be ignored for this example. The first file contains
\begin{verbatim}
  cogent_puts : (CPtr U8)! -> U32
\end{verbatim}
which is also valid Cogent code, but uses the non-standard generic type \code{CPtr}. 

The command also creates the file \code{hello-externs.ac} (Not yet implemented! Provided with the example).

\item[Step 3:] (make edit)

Now comes the part where your manual work is demanded. You have to translate the function bodies and you have to adapt
some types. Open \code{hello.cogent} in a text editor, replace its content by
\begin{verbatim}
  cogent_main : () -> U32
  cogent_main () =
         cogent_puts("Hello World"); 0
\end{verbatim}
and save as \code{ed-hello.cogent}. In \code{hello-externs.cogent} replace the argument type by \code{String}:
\begin{verbatim}
  cogent_puts : String -> U32
\end{verbatim}
and save as \code{ed-hello-externs.cogent}.

Now you have a Cogent program which is equivalent to the original C program. It consists of several files,
which are all included by the provided file \code{unit.cogent}.

\item[Step 4:] (make cogent-c)

To see that it works, process \code{unit.cogent} by the Cogent compiler. Enter the command
\begin{verbatim}
  cogent -g unit.cogent --infer-c-funcs="ed-hello-externs.ac hello-entry.ac"
\end{verbatim}
It creates the files \code{unit.c} and \code{unit.h} and two \code{\_pp\_inferred.c} files. All these files
are included by the provided file \code{cogent-hello.c}, so this file wraps together the C program generated
from the Cogent program.

\item[Step 5:] (make cogent-run)

Try whether it still works. Compile \code{cogent-hello.c} with a C compiler and run it. The result should be the same
as in Step 1.

Note that for the compilation you need to set the include path to the standard library folder of the Cogent 
distribution (\code{STDGUM} in the Makefile).

\item[Step 6:] (make clean)

Clean up all generated files. If you like you can perform the steps again.
\end{description}

\chapter{Simple C Programs}

\section{Single Source File}
\label{simple-single}

\section{Single Source File with Include Files}
\label{simple-include}

\section{Multiple Source Files}
\label{simple-multi}

\section{Partial Translation}
\label{simple-partial}

\end{document}
 
