\subsection{Mapping Names from C to Cogent}
\label{design-names}

Names used in the C code shall be translated to similar names in the Cogent code, since they usually are descriptive for the
programmer. Ideally, the same names would be used. However, this is not possible, since Cogent differentiates between 
uppercase and lowercase names and uses them for different purposes. Therefore, atleast the names in the ``wrong'' case
need to be mapped.

Additionally, when the Cogent compiler translates a Cogent program to C code, it transfers the names without changes to
the names for the corresponding C items. We will see below, that often these generated C items are needed additionally to
the original C item which has been translated to Cogent by Gencot. If Gencot uses the same name in Cogent, this would cause
a name conflict in the code generated by the Cogent compiler.

For this reason, Gencot uses name mapping schemas mapping all kinds of names which can cause such a conflict to a different, but 
similar name in Cogent. Generally, this is done by substituting a prefix of the name.

Often, a <package> uses one or more specific prefixes for its names, at least for names with external linkage. In this case
Gencot should be able to substitute these prefixes by other prefixes specific for the Cogent translation of the <package>.
Therefore, the Gencot name mapping is configurable. For every <package> a set of prefix mappings can be provided which is
used by Gencot. Two separate mappings are provided depending on whether the Cogent name must be uppercase or lowecase, so 
that the target prefixes can be specified in the correct case.

If a name must be mapped by Gencot which has neither of the prefixes in the provided mapping, it is mapped 
by prepending the prefix \code{cogent\_} or \code{Cogent\_}, depending on the target case.

\subsubsection{Name Kinds in C}

In C code the tags used for struct, union and enum declarations constitute an own namespace separate from the ``regular''
identifiers. These tags are mapped to Cogent type names by Gencot and could cause name conflicts with regular identifiers
mapped to Cogent type names. To avoid these conflicts Gencot maps tags by prepending the prefixes \code{Struct\_}, 
\code{Union\_}, or \code{Enum\_}, respectively, after the mapping described above. Since tags are always translated to Cogent 
type names, which must be uppercase, only one case variant is required.

Member names of C structs or unions are translated to Cogent record field names. Both in C and Cogent the scope of these
names is restricted to the surrounding structure. Therefore, Gencot normally does not map these names and uses them unmodified
in Cogent. However, since Cogent field names must be lowercase, Gencot applies the normal mapping for lowercase target 
names to all uppercase member names (which in practice are unusual in C). 

C function parameter names are translated to Cogent variable names bound in the Cogent function body expression. Hence, both
in C and Cogent the scope of these names is restricted to the function body. They are treated by Gencot in the same way as 
member names and are only mapped if they are uppercase in C, which is very unusual in practice.

The remaining names in C are type names, tags, function names, enum constant names, and names for global and local variables.
Additionally, there may be C constant names defined by preprocessor macro directives.
Local variables only occur in C function bodies which are not translated by Gencot. The other names are always mapped by
Gencot, irrespective whether they have the correct case or not. The reasons are explained in Section~\ref{design-mdular}
below.

\subsubsection{Names with internal linkage}

In C a name may have external or internal linkage. A name with internal linkage is local to the compilation unit in which it
is defined, a name with external linkage denotes the same item in all compilation units. Since the result of Gencot's 
translation is always a Cogent program which is translated to a single compilation unit by the Cogent compiler, names 
with internal linkage could cause conflicts when they origin in different C compilation units.

To avoid these conflicts, Gencot uses a name mapping scheme for names with internal linkage which is based on the 
compilation unit's file name. Names with internal linkage are mapped by substituting a prefix by the prefix \code{local\_x\_}
where \code{x} is the basename of the file which contains the definition, which is usually a file \code{x.c}. The default
is to substitute the empty prefixe, i.e., prepend the target prefix. The mapping can be configured by specifying prefixes
to be substituted. This is motivated by the practice to sometimes also use a common prefix for names with internal linkage
which can be removed in this way.

Name conflicts could also occur for type names and tags defined in a \code{.h} file. This would be the case if different
C compilation units include individual \code{.h} files which use the same identifier for different purposes. However, most
C packages avoid this to make include files more robust. Gencot assumes that all identifiers defined in a \code{.h} file
are unique in the <package> and does not apply a file-specific renaming scheme. If a <package> does not satisfy this assumption
Gencot will generate several Cogent type definitions with the same name, which will be detected and signaled by the Cogent 
compiler and must be handled manually.

 
