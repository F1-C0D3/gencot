The following auxiliary Gencot components exist which do not process C source code:
\begin{description}
\item[\code{parmod-proc}] processes parameter modification descriptions in JSON format (see Section~\ref{impl-parmod}).
\end{description}

\subsection{Parameter Modification Descriptions}

Processing parameter modification descriptions is implemented by the filter \code{parmod-proc}. It reads a parameter
modification description from standard input as described in Section~\ref{impl-parmod-json}. The first command line 
argument acts as a command how to process the parameter modification description. The filter implements the following
commands:
\begin{description}

\item[\code{check}]
Verify the structure of the parameter modification description 
according to Section~\ref{impl-parmod-struct} and lists all errors found.

\item[\code{funids}]
List the function identifiers of all functions described in the parameter modification description.

\item[\code{unconfirmed}] 
List all unconfirmed parameter descriptions in the form
\begin{verbatim}
  <function identifier> / <parameter identifier>
\end{verbatim}

\item[\code{required}]
List all required invocations by their function identifiers.

\item[\code{sort}]
Takes an additional file name as command line argument. The file contains a list of function identifiers.
The input is sorted according to the order in which the function identifiers appear in the list. This command
is intended to be applied after the \code{merge} command to (re)establish a certain ordering.

\item[\code{filter}]
Takes an additional file name as command line argument. The file contains a list of function identifiers.
The input is filtered and only function descriptions for the listed identifiers are retained in the output.

\item[\code{merge}]
Takes an additional file name as command line argument. The file contains a parameter modification description 
in JSON format. Both descriptions are merged. If both contain a description for the same function identifier,
the description with less unconfirmed parameter descriptions is used in the output.

\item[\code{eval}]
The parameter modification description must be structurally correct, contain no
unconfirmed parameter descriptions and contain no required dependencies. Then all dependencies are eliminated 
as described in Section~\ref{impl-parmod-eval}.
The resulting parameter modification description contains no parameter dependencies and
no invocation descriptions. It is intended to be read by the filters which translate C function types and function 
definitions.
\end{description}

All lists mentioned above are structured as a sequence of text lines.

If the result is a parameter modification description in JSON format is written to the output as described in 
Section~\ref{impl-parmod-json}.

The first three commands are intended as a support for the developer when filling the description manually. The goal is that
for all three the output is empty. If there are unconfirmed parameters they must be inspected and confirmed. This usually 
modifies the list of required invocations. They can be reduced by generating and merging corresponding descriptions
from other source files.

