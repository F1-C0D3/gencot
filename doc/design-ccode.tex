
After comments and preprocessor directives have been removed from a C source file, it is parsed and
the C language constructs are processed to yield Cogent language constructs. 

\subsection{Including Files for C Code Processing}

When Gencot processes the C code in a source file, it may need access to information in files included by
the source file. An example is a type definition for a type name used in the source file. Hence for
C code processing Gencot always reads the source file together with all included files. Since in the source
file all comments and preprocessor directives have been  removed, they must also be removed in the included files
which belong to the same <package>. Gencot assumes these are the files included by a quoted include directive.

There are two possible approaches how this can be done.

The first approach is to use the C preprocessor which is invoked by language-c before parsing. It processes
include directives as usual, hence it would be sufficient to leave the include directives in the source code
when removing the other preprocessor directives. However, this would include the original \code{.h} files and
\textit{process} all preprocessor directives there, instead of removing them. The directives have to be 
removed from all included files in <package> in a separate step, then the include directives have to be modified to include
the results of that step instead of the original files.

The second approach is to process all quoted include directives before the other directives are removed, 
resulting for every source
file in a single file containing all included information and all other preprocessor directives. Then the directives
are removed from this file with the exception of the system include directives. When the result is fed to the 
language-c parser its preprocessor will expand the system includes as usual, thus providing the complete 
information needed for processing the C code.

Gencot uses the second approach, since this way it can process every source file independently from previous steps for 
other source files and it needs no intermediate files which must be added to the include file path of the language-c
preprocessor.

In the included original files the comments are still present and must be removed as well. This could be done 
by the language-c preprocessor immediately before parsing. However, it is easier to remove the preprocessor
directives when the comments are not present anymore. Therefore, Gencot removes the comments immediately after
processing the quoted include directives.

\subsection{Processing the C Code}



 
