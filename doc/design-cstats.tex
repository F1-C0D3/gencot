In C statements and expressions occur in the bodies of function definitions and in initializers of object definitions. Gencot translates
both to corresponding Cogent constructs, as described in Section~\ref{design-fundefs}. In both cases the C statements and expressions
must be translated to Cogent expressions. This means a transformation from imperative constructs with possible side effects in a global
state to purely functional constructs where all effects are explicitly represented in the construct itself. It also means a transformation
from arbitrary treatment of pointers to the Cogent uniqueness type system where pointers are never shared or discarded.

The first issue is handled by Gencot. However, since it is complex to detect and translate all side effects, it uses a semiautomatic approach.
It tries to cover most of the standard cases but leaves some cases to be manually translated by the programmer. In such cases the generated 
Cogent code contains a dummy expression which documents the reason for not translating it automatically, together with the corresponding C code as
a comment. So the generated Cogent code is always syntactically correct and can be parsed by the Cogent compiler, although it may not reflect
the complete original C code.

The second issue is not handled by Gencot. If the C code involves sharing or discarding of pointers the translation will result in Cogent
code which violates the uniqueness type assumptions. This will be detected by the Cogent compiler in its type checking phase and must be
handled manually, either by modifying and retranslating the C code, or by modifying the Cogent code.

\subsection{General}
\label{design-cstats-general}

The basic building blocks of C function bodies are declarations and statements. Here we only cover declarations of 
local variables. Such a declaration specifies a name for the local variable and optionally an initial value.

A C statement causes modifications in its context. Cogent, as a functional language, does not support this concept,
in particular, it does not support modifying the value of a variable. Cogent only supports expressions which functionally
depend on their input, and it supports introducing immutable variables by binding them to a value. 
The main task of translating function bodies is to translate C statements to Cogent expressions.

The main idea of the translation is to translate a C variable which is modified to a sequence of bindings of Cogent
variables to the values stored in the C variable over time. Everytime when a C statement modifies a variable, a new
variable is introduced in Cogent and bound to the new value. A major difference of both approaches is that in Cogent 
the old variable and its value are still available after the modification. To prevent this, we use the same name for
both variables. Then the new variable ``shadows'' the old one in its scope, making the old value unaccessible there.
For example the C code
\begin{verbatim}
  int i = 0; i = i+1; ...
\end{verbatim}
is translated to
\begin{verbatim}
  let i = 0 in let i = i+1 in ...
\end{verbatim}
where the single C variable \code{i} is translated to two Cogent variables which are both named \code{i}.

\subsubsection{Statements}

A C statement in a function can modify several of the following:
\begin{itemize}
\item function parameters
\item local variables
\item global variables
\end{itemize}
All of them are specified by an identifier which is unique at the position of the statement. The modification
may replace the value as a whole or only modify one or more parts in the case of a structured value.

As a first step modifications of global variables must be eliminated by passing all such variables as additional 
parameters to the function and returning them as additional component in the result. Then a modification of a
global variable becomes a modification of a function parameter and only the first two cases remain.

Global variables can be turned to parameters or access functions using the Global-State and Const-Val properties
(see Section~\ref{design-types-itemprops}). If neither property has been specified for a global variable, Gencot
does not translate accesses to it and inserts a dummy expression instead.

After this step the effect of the modification caused by a C statement can be described by a set of identifiers 
of all modified parameters and local variables together with the new values for them. Syntactically this corresponds
to a Cogent \textit{binding} of the form
\begin{verbatim}
  (id1,...,idn) = expr
\end{verbatim}
where \code{expr} is a Cogent expression for the tuple of new values for the identifiers.

\subsubsection{Pointers}

Cogent treats C pointers in a special way as values of ``linear type'' and guarantees that no memory is shared
among different values of these types. More general, all values which may contain pointers (such as a struct with
some pointer members) have this property. All other values are of ``nonlinear type'' and never have common parts 
in C.

Gencot assumes that no sharing occurs among the parameters and local variables in the function. This implies that 
a C statement can only modify parameters and variables for which the identifiers occur literally in the statement
source code text. Thus it is possible to determine the effect of the modification caused by a C statement
syntactically from the statement.

If the assumption is not true the Cogent Compiler will detect this when it typechecks the translated code. See
Section~\ref{app-transfunction-pointers} for proposals how to handle such cases manually.

\subsubsection{Variable Declarations}

An initializer \code{init} in a variable declaration \code{t v = init;} is either an expression or an initializer for
a struct or an array. The C declaration can be rewritten as
\begin{verbatim}
  t v;
  v = init;
\end{verbatim}
with a separate statement for initializing the variable. If \code{init} is an expression this is valid C code. Otherwise,
if \code{init} is an initializer for a struct or array, the statement is not valid in C, but Gencot will translate it 
according to the intended semantics. For a struct type Cogent provides
corresponding expressions for unboxed records. For other types Gencot provides its initialization functions 
described in Section~\ref{design-operations-init} and~\ref{design-operations-array},
or the initialization can be done by several applications of operation \code{set} (see Section~\ref{design-operations-parts}).

For a declaration without an initializer Gencot inserts an assignment with a default value. If the variable has a regular 
type, the Gencot operation \code{defaultVal} (see Section~\ref{design-operations-default}) is used. If the variable has 
linear type (is a pointer or directly contains pointers) the contained pointers are translated to \code{MayNull} types 
(see Section~\ref{design-types-pointer}) and are initialized to value \code{null()} (see Section~\ref{design-operations-null}).
If a Not-Null property (see Section~\ref{design-types-itemprops}) is used to avoid the \code{MayNull} type Gencot 
will not translate the declaration if it has no initializer and will generate a dummy expression instead.

Together, this way all declarations are replaced by statements and the declarations need not be translated to Cogent, 
since in Cogent a new variable is introduced whenever the C variable is modified by a statement. Therefore, only the C statements 
need to be translated to Cogent (including pseudo assignment statements corresponding to record and array initializers).

\subsection{Expressions}
\label{design-cstats-expr}

C statements usually have C expressions as syntactic parts. For translating C statements the contained C expressions must be 
translated. C expressions have a value but may also cause modifications as side effects. Especially, in C an assignment
is syntactically an expression. C expressions are translated depending on whether they have side effects or not.

\subsubsection{Expressions without Side Effects}

C expressions without side effects are literals, variable references, member accesses of the form \code{s.m} or 
\code{s->m}, index
expressions of the form \code{a[e]}, applications of binary operators of the form \code{e1 op e2}, 
applications of the unary operators \code{+,-,!,~} of the form \code{op e}, and function 
call expressions of the form \code{f(e1,...,en)}, if all subexpressions
\code{s,a,e,f,e1,...,en} have no side effects and function \code{f} does not modify its parameter values. 

These expressions are translated to Cogent expressions in a straightforward way with the same or a similar syntax.
A member access \code{s->m} is translated as \code{s.m}. An index expression \code{a[e]} is translated as function 
call \code{getArr(a,e)}. Note, that some C operators have a different form in Cogent:
\begin{verbatim}
  C  Cogent
  !=   /=
  ^    .^.
  &    .&.
  |    .|.
  !    not
  ~    complement
\end{verbatim}

Member accesses of the form \code{s->m} (or written \code{(*s).m}) and index expressions \code{a[e]} can only be 
translated in this way if the
container value \code{s} or \code{a}, respectively, is translated to a readonly value in Cogent. If it is a parameter,
variable, or record field, it may have been defined as readonly. Otherwise, it can be made readonly in the expression's
context by applying the bang operator \code{!} to it at the end of the context:
\begin{verbatim}
  ... s.m ...  !s
\end{verbatim}
The resulting expressions are also of readonly type. For expressions of nonlinear type this is irrelevant, for expressions
of linear type it implies that they cannot be modified. If they are used in C by modifying them, they must be translated
in a different way. For example in the C code fragment
\begin{verbatim}
  *(s->p) = 5
\end{verbatim}
the expression \code{s->p} denotes a pointer which is modified, therefore it cannot be translated to \code{s.m} where 
s is readonly.

For resultig expressions of linear types, using the bang operator \code{!} also means that the readonly result cannot
escape from the banged context, it can only be used inside the context.

Translation of expressions using the address operator \code{\&} are described in Section~\ref{app-transfunction-addrop}.

Dereferencing (application of the indirection operator \code{*} to) a pointer is translated depending on the type of value referenced by 
the pointer. If it points to a function, \code{*p} is translated as \code{fromFunPtr(p)} 
(See Section~\ref{design-operations-function}). 

If it points to a primitive type, an enum type, or again a pointer type, 
\code{*p} is translated as \code{getPtr(p)} (See Section~\ref{design-operations-pointer}). In this case \code{p} 
must be readonly as above, and the result is readonly. If it should be modified it must be translated differently.

Otherwise it points to a type which is mapped to Cogent as a record or abstract type. Then \code{*p} is translated as \code{p}. 

Expressions using the C operators \code{sizeof} or \code{\_Alignof} cannot be translated to Cogent. Usually, an abstract
function implemented in C is required here.

\subsubsection{Expressions with Side Effects}

We translate an expression with side effects to a Cogent binding of the form \code{pattern = expr}. Here, the
\code{pattern} is a tuple of variables \code{(v,v1,...,vn)}. The variable \code{v} is a
new variable which is not already bound in the context of the expression, it is used to bind the result value of the
expression. The other variables are the identifiers of all parameters and local 
variables modified by the expression. Since we presume that the C expression has side effects, there is at least one
such variable.
The \code{expr} is a Cogent expression for a corresponding tuple consisting of the result value of the C expression and the new values
of the identifiers modified by the C expression.

Expressions with side effects are applications of the increment and decrement operators \code{++,-{}-}, assignments,
and invocations of functions which modify one or more parameter values.

Applications of increment and decrement prefix operators must be rewritten in C using assignments. Assignments using assignment
operators other than \code{=} must be rewritten in C using the \code{=} operator. After these steps the only 
remaining expressions with side effects are assignments using \code{=}, increment and decrement postfix operators, 
and invocations of functions which modify parameter values.

If for such an expression all subexpressions are without side effects, they are translated as follows. 

The translation of an assignment expression of the form \code{lhs = e} depends on the form of \code{lhs}. If it is
a single identifier \code{v1} (name of a parameter or local variable), it is translated as
\begin{verbatim}
  (v,v1) = let v = expr in (v,v)
\end{verbatim}
where \code{expr} is the translation of \code{e}.

Note that this code is illegal in Cogent, if \code{expr} has a linear type, since it uses the result value twice. However, this
is a natural property of C code, where an assignment is an expression and the assigned value can be used in the context. For example,
the C code fragment \code{f(p = q)} assigns the value of \code{q} to \code{p} and also passes it as argument to function \code{f}.

There are several ways how to cope with this situation. In the simplest case, the outer variable \code{v} is never used in its
scope, then the double use of the value can be eliminated by simplifying the Cogent binding to the form
\begin{verbatim}
  v1 = expr
\end{verbatim}
This is typically the case if the assignment is used as a simple statement, where its result value is discarded. Most assignments
in C are used in this way.

Otherwise it depends on how the variable \code{v} is used in its scope. In some cases it may be possible to replace its use by 
using \code{v1} instead, then it can be eliminated in the same way as above. If that is not possible, the C program uses true
sharing of pointers, then the code cannot be translated to Cogent and must be translated using abstract functions.

If \code{lhs} is a logical chain of \code{n} member access, index, and dereferencing expressions starting with identifier \code{v1} 
(name of a parameter or local variable), the most general translation is
\begin{verbatim}
  (v,v1) =
    let v = expr
    in (v,fst(modify1(v1,(modify2,(...(modifyn-1,(set,v))...)))))
\end{verbatim}
where \code{expr} is the translation of \code{e} and the sequence of \code{modifyi} functions is determined by the chain of access
expressions and \code{set} is an 
instance of the operation \code{set} (see Section~\ref{design-operations-parts}). Again, the result of \code{expr} is used twice,
the same considerations as above apply if it has a linear type.

For example the C assignment expression
\begin{verbatim}
  s->a[i]->x = 5
\end{verbatim}
is translated according to this rule to the Cogent binding
\begin{verbatim}
  (v,s) = let v = 5
    in (v,fst(modifyFld_a(s,(modifyArr,(i,setFld_x,v)))))
\end{verbatim}
where \code{modifyFld\_a} and \code{setFld\_x} are abstract modification functions for the fields \code{a} and \code{x},
respectively, as described in Section~\ref{design-operations-record}. 

This translation is correct, but it is inefficient because \code{modifyFld\_a} will copy the whole array \code{a} only to access
one element of it by applying \code{modifyArr} to the unboxed array. Using \code{modref} (see Section~\ref{design-operations-modify}) 
this can be improved to
\begin{verbatim}
  (v,s) = let v = 5
    in (v,fst(modrefFld_a(s,(modifyArr,(i,setFld_x,v)))))
\end{verbatim}
here \code{modrefFld\_a} retrieves only the pointer to \code{a} and applies \code{modifyArr} to the boxed array. If the array element
would be a larger data structure it could be further improved by using \code{modrefArr} instead of \code{modifyArr}, here it does
not pay because the array element is already a single pointer.

No other cases for \code{lhs} are valid in a C assignment.

If \code{lhs} logically starts with a chain of member accesses \code{v1->m1->...->mn...} an alternative translation using the Cogent
take and put operations is the binding
\begin{verbatim}
  (v,v1) = 
    let v1{m1=m1{m2=...mn-1{mn}...}}
    and (v,mn) = expr
    in (v,v1{m1=m1{m2=...mn-1{mn=mn}...}})
\end{verbatim}
where \code{(v,mn) = expr} is the binding to which \code{mn... = e} is translated, if \code{mn} is assumed to be a local variable. 
For example, a corresponding translation of \code{*(s->m1->m2) = 5} is
\begin{verbatim}
  (v,s) = 
    let s{m1=m1{m2}}
    and (v,m2) = let v = 5 in (v,setPtr(m2,v))
    in (v,s{m1=m1{m2=m2}})
\end{verbatim}
This approach avoids the need to manually define and implement the functions \code{modifyFld\_mi}, however it prevents the improvement
described above by using \code{modref}.

An application of an increment/decrement postfix operator \code{ss} where \code{s} is \code{+} or \code{-} has the form
\code{lhs ss}. If \code{lhs} is a single identifier \code{v1} this identifier must have a numerical type and the expression 
is translated to the binding
\begin{verbatim}
  (v,v1) = (v1,v1 s 1)
\end{verbatim}
Using \code{v1} twice is always possible here since it has nonlinear type. As an example, \code{i++} is translated to
\begin{verbatim}
  (v,i) = (i,i+1)
\end{verbatim}

If \code{lhs} is a logical chain of \code{n} member access, index, and dereferencing expressions starting with identifier \code{v1}
this identifier must have linear type and the most general translation is the binding
\begin{verbatim}
  (v,v1) = 
    let v = tlhs !v1
    in (v, fst(modify1(v1,(modify2,(...(modifyn-1,(set,v s 1))...))))
\end{verbatim}
where \code{tlhs} is the translation of \code{lhs} when \code{v1} is readonly and \code{modifyi} and \code{set} are as 
for the assignment. Here \code{v1} is needed twice, first for retrieving the old
numerical value \code{v} and afterwards to set it to the incremented/decremented value. Since \code{v1} is linear the old
value must be retrieved in a readonly context and the modification must be done seperately. The double use of \code{v}
is always possible since it has numeraical type.

Again, this translation may be improved by using \code{modref} instead of \code{modify} where appropriate.

If \code{lhs} logically starts with a chain of member accesses \code{v1->m1->...->mn...} an alternative translation using the Cogent
take and put operations is the binding
\begin{verbatim}
  (v,v1) =
    let v = tlhs !v1
    and v1{m1=m1{m2=...mn-1{mn}...}} 
    in (v, v1{m1=m1{m2=...mn-1{mn=expr}...}})
\end{verbatim}
where \code{expr} is as above for the assignment using \code{(v s 1)} as the new value.

For example, a corresponding translation of \code{(*(s->m1->m2))++} is
\begin{verbatim}
  (v,s) =
    let v = getPtr(s.m1.m2) !s
    and s{m1=m1{m2}}
    in (v,s{m1=m1{m2=setPtr(m2,v+1)}})
\end{verbatim}

A C function which modifies parameter values is translated by Gencot to a Cogent function returning the tuple \code{(y,p1,...,pn)}
of the original function result \code{y} and the modified parameter values \code{p1,...,pn} (which must be pointers). 
A C function call \code{f(...)} is
translated depending on the form of the actual arguments passed to \code{f} for the modified parameters.

If all such arguments are identifiers (names of parameters and local variables) the function call is translated to
the binding
\begin{verbatim}
  (v,v1,...,vn) = f(...)
\end{verbatim}
where \code{v1,...,vn} are the identifiers passed to the parameters modified by \code{f} in the order returned by \code{f}.

If some of the arguments are member access chains of the form \code{vi->mi1->...->miki} they must be translated using the 
Cogent take and put operations as above to a binding of the form
\begin{verbatim}
  (v,v1,...,vn) = 
  let v1{m11=...{m1k1}...} = v1
  and ... 
  and vn{mn1=...{mnkn}...} = vn
  and (v,m1k1,...,mnkn)=f(...)
  in (v,v1{m11=...{m1k1=m1k1}...},
       ...
        vn{mn1=...{mnkn=mnkn}...})
\end{verbatim}
where in the arguments of \code{f} the chains are replaced by their last member name \code{miki}.

For example, the function call \code{f(5,s->m,t->n,z)} where \code{f} modifies its second and third parameter, is translated
to the binding
\begin{verbatim}
  (v,s,t) = 
  let s{m} = s
  and t{n} = t
  and (v,m,n)=f(5,m,n,z)
  in (v,s{m=m},t{n=n})
\end{verbatim}

If function \code{f} modifies only one parameter \code{p} of Cogent type \code{P} its standard type generated by
Gencot can be changed to the form of a modification function
\begin{verbatim}
  f: ModFun P (...) Res
\end{verbatim}
where \code{(...)} is the tuple of types of the other parameters and \code{Res} is the result type. Then for an arbitrary
chain of member access, index, and dereferencing expressions used as actual argument for \code{p} the function call 
can be translated to a binding of the form
\begin{verbatim}
  (v,v1) = let (v1,v) = 
    modify1(v1,(modify2,(...(modifyn,(f,(a1,...,an)))...)))
    in (v,v1)
\end{verbatim}
where the sequence of \code{modifyi} functions is determined by the chain of access expressions. Note
that the order of the variables must be exchanged since the original result of \code{f} is the second component in 
the inner tuple pattern due to the way \code{ModFun} is defined.

For example, the function call \code{f(5,s->a[i]->x,z)} can be translated by first modifying the translation of 
\code{f} so that it takes as parameters instead of the tuple \code{(a,b,c)} the pair \code{(b,(a,c))}. Then a translation
for the function call is
\begin{verbatim}
  (v,s) = let (s,v) = 
    modifyFld_a(s,(modifyArr,(i,modifyFld_x,(f,(5,z)))))
    in (v,s)
\end{verbatim}
If index \code{i} is invalid, according to the definition of \code{modifyArr} the function call \code{f(5,s->a[0]->x,z)} 
is executed instead.

Again, if the chain logically starts with member accesses, that part of the chain can be translated using take and put operations.

In all other cases the function \code{f} must be modified so that it takes the starting identifiers of the chains as
arguments instead of the chains, then it can be translated as in the first case where all actual arguments are identifiers.
Note that different translations of the function to Cogent may be required for translating different function calls.

\subsubsection{Nested Expressions with Side Effects}

If an expression contains subexpressions with side effects, these must be translated separately. 

Let \code{e1,...,en} be the expressions with side effects directly contained in the expression \code{e} and
let \code{p1 = expr1, ..., pn = exprn} their translations to Cogent bindings. Since in C the order of evaluation 
of the \code{e1,...,en} is undefined, we can only translate \code{e} if the subexpressions
modify pairwise different sets of identifiers, i.e., the \code{p1,...,pn} are pairwise disjunct tuples.
Let \code{x1,...,xn} be the
first variables of the patterns \code{p1,...,pn} and let \code{w1,...,wm} be the union of all other variables
in \code{p1,...,pn} in some arbitrary order. Let \code{e'} be \code{e} with every \code{ei}
substituted by \code{xi}. Then \code{e'} contains no nested expressions with side effects and can be translated
to Cogent according to the previous sections.

If \code{e'} has no side effects, let \code{expr} be its translation to a Cogent expression. Then the translation 
of \code{e} is the binding
\begin{verbatim}
  (v,w1,...,wm) = 
    let p1 = expr1 and ...
    and pn = exprn 
    in (expr,w1,...,wm)
\end{verbatim}
For example the expression \code{a[i++]} is translated according to this rule to
\begin{verbatim}
  (v,i) = 
    let (v,i) = (i,i+1)
    in (getArr(a,v),i)
\end{verbatim}

If \code{e'} has side effects, let \code{(v,v1,...,vk) = expr} be its translation to a Cogent binding. Then
the translation of \code{e} is the binding
\begin{verbatim}
  (v,v1,...,vk,w1,...,wm) = 
    let p1 = expr1 and ...
    and pn = exprn 
    and (v,v1,...,vk) = expr
    in (v,v1,...,vk,w1,...,wm)
\end{verbatim}
For example the expression \code{a[i++] = 5} is translated according to this rule to
\begin{verbatim}
  (v,a,i) = 
    let (v,i) = (i,i+1)
    and (w,a) = let w = 5 in (w,setArr(a,v,w))
    in (w,a,i)
\end{verbatim}
which can be simplified in Cogent to the binding
\begin{verbatim}
  (v,a,i) = (5,setArr(a,i,5),i+1)
\end{verbatim}

\subsubsection{Readonly Access and Modification of the Same Value}

If a C expression uses and modifies parts of a linear value at the same time, a special translation approach is required.
An example is the expression \code{r->sum = r->n1 + r->n2}. According to the rules above, translating the right hand side
requires to make \code{r} readonly by applying the bang operator \code{!} which then prevents translating the expression
as a whole, modifying \code{r}. There are three cases how to deal with this situation.

If all used parts of the value are nonlinear, they can be retrieved in a separate step using \code{!}, bound to Cogent 
variables and then used in the modification step. The translation of an expression \code{e} then has the general form
\begin{verbatim}
  (v,v1,...,vn) = 
    let (w1,...,wm) = ... !r1 ... !rk
    in expr
\end{verbatim}
where \code{w1,...,wm} are auxiliary Cogent variables for binding the used values, \code{r1,...,rk} are all linear values
from which parts are used in \code{e} and \code{(v,v1,...,vn) = expr} is the 
normal translation of \code{e} with all used parts replaced by the corresponding \code{wi}.

We used this approach for translating applications of postfix increment/decrement operations to complex expression,
such as \code{(*(s->m1->m2))++}.

If the used values are themselves linear, such as in \code{r->p = f(r->p)} where \code{p} is a pointer, this approach
cannot be used since in Cogent the binding \code{w = r.p !r} is illegal, \code{r.p} has a readonly linear type and
is not allowed to escape from the banged context so that it can be bound to \code{w}.

If the used linear value is replaced by the modification, as in the example, it is possible to use the take and put
operations. In a first step the used values are taken from the linear containers \code{r1,...,rk}, in the modification
step they are put back in. The translation of \code{e} then has the general form
\begin{verbatim}
  (v,v1,...,vn) = 
    let r1{... = w1 ...} ...
    and rk{... = wk ...}
    in expr
\end{verbatim}
where \code{expr} contains the necessary put operations for all \code{r1,...,rk}.

It may also be the case that the modification cannot be implemented by a combination of take and put operations, either 
because data types like arrays and pointers are involved, which are not translated to Cogent records, or because
the modification causes sharing or discarding linear values. In both cases the modification cannot be implemented in
Cogent, it must be translated by introducing an abstract function \code{fexpr} which implements the expression \code{e}
as a whole. Then the translation has the form
\begin{verbatim}
  (v,v1,...,vn) = 
    fexpr(v1,...,vn,x1,...,xm)
\end{verbatim}
where \code{x1,...,xm} are additional nonlinear values used by the expression.

\subsubsection{Comma Operator}

In C the expression \code{e1,e2} first evaluates \code{e1}, discarding its result and then evaluates \code{e2} for which
the result is the result of the expression as a whole. Thus, the comma operator only makes sense if \code{e1} has side
effects. Let \code{(v,v1,...,vn) = expr1} be the translation of \code{e1} to a Cogent binding.

If \code{e2} has no side effects and is translated to the Coent expression \code{expr2}, the expression \code{e1,e2}
is translated to the Cogent binding
\begin{verbatim}
  (v,v1,...,vn) = 
    let (_,v1,...,vn) = expr1
    in (expr2,v1,...,vn) 
\end{verbatim}
Note that the translation is only valid if the result value of \code{e1} is not linear, since it is discarded. If 
\code{expr1} is a tuple, this can be simplified by omitting the first component. This avoids the discarding and 
may even remove a double use of a linear value in \code{expr2}.

If \code{e2} has side effects let \code{(w,w1,...,wm) = expr2} be the translations of \code{e2}. Then the translation of 
\code{e1,e2} is 
\begin{verbatim}
  (v,u1,...,uk) = 
    let (_,v1,...,vn) = expr1
    and (w,w1,...,wm) = expr2
    in (w,u1,...,uk) 
\end{verbatim}
where \code{u1,...,uk} is the union of \code{v1,...,vn} and \code{w1,...,wm}. 

\subsubsection{Conditional Expression}

A conditional expression \code{e0 ? e1 : e2} is translated as follows. Let \code{(x,x1,...,xn) = expr0}, 
\code{(y,y1,...,ym) = expr1}, \code{(z,z1,...,zp) = expr2} be the translations of 
\code{e0}, \code{e1}, \code{e2}, respectively.
If \code{e0} evaluates to a value of numerical type the translation of the conditional expression is 
\begin{verbatim}
  (v,v1,...,vk) = 
    let (x,x1,...,xn) = expr0
    and (v,u1,...,uq) =
      if x /= 0 then 
        let (y,y1,...,ym) = expr1 
        in (y,u1,...,uq)
      else
        let (z,z1,...,zp) = expr2
        in (z,u1,...,uq)
    in (v,v1,...,vk) 
\end{verbatim}
where \code{u1,...,uq} is the union of \code{y1,...,ym} and \code{z1,...,zp} in some order, and 
\code{v1,...,vk} is the union of \code{y1,...,ym} and \code{x1,...,xn} in some order.

If the outermost operator of \code{e0} is a ``logical'' operator (one of \code{<, <=, >, >=, ==, !=, \&\&, ||})
it can be translated in a way that the first component in the result of \code{expr0} is of type Bool.
Then, instead of \code{if x /= 0 then} the condition is written as \code{if x then}.

In C the value used for testing the condition may be of any scalar type, i.e., a numerical or pointer type.
In case of a pointer type the condition corresponds to testing the pointer for being not \code{NULL}.
The translated expression \code{expr0} will return a value of linear type as first component. Since it
may be null (otherwise the conditional expression could be replaced by \code{e1}) it must be translated
so that \code{expr0} returns a value of type \code{MayNull a} (see Section~\ref{design-operations-null})
as its first component. Since the use as condition in Cogent would discard the value, it must be readonly,
i.e., of type \code{(MayNull a)!}. Then, instead of \code{if x /= 0 then} the condition must be written as
\begin{verbatim}
      if not isNull(x) then 
\end{verbatim}
using function \code{isNull} as described in Section~\ref{design-operations-null}. If the first component
of the result of \code{expr0} is not already of readonly type it must be made readonly in the condition:
\begin{verbatim}
      if not isNull(x) !x then 
\end{verbatim}

Often in C, after testing x successfully for not being \code{NULL}, the pointer \code{x} is used in 
\code{e1} by dereferencing it. In Cogent this is not possible, since a value of type \code{MayNull a} cannot
be dereferenced. Instead, a value of type \code{a} is required. This is made accessible by the functions
\code{notNull} and \code{roNotNull} (see Section~\ref{design-operations-null}). Then the translation of the
conditional expression has the form
\begin{verbatim}
  (v,v1,...,vk) = 
    let (x,x1,...,xn) = expr0
    and (v,u1,...,uq) =
      notNull(x)
      | Some x -> 
        let (y,y1,...,ym) = expr1 
        in (y,u1,...,uq)
      | None ->
        let (z,z1,...,zp) = expr2
        in (z,u1,...,uq)
    in (v,v1,...,vk) 
\end{verbatim}
Here the \code{x} introduced by matching \code{Some x} is of type \code{a} and can be used in \code{expr1}
to access and modify it. If \code{expr1} only reads \code{x} and should not consume it, it must be introduced
as readonly by replacing the line \code{notNull(x)} by
\begin{verbatim}
      roNotNull(x) !x
\end{verbatim}
Then the \code{x} introduced by matching \code{Some x} is of type \code{a!}. Function \code{roNotNull} must also
be used when the outer \code{x} is already of the readonly type \code{(MayNull a)!}.

In the following patterns we only show the form \code{if x /= 0 then}, it must be replaced as needed.

Usually, the condition \code{e0} has no side effects, the the translation can be simplified to
\begin{verbatim}
  (v,u1,...,uq) = 
    if expr0 /= 0 then 
      let (y,y1,...,ym) = expr1 
      in (y,u1,...,uq)
    else
      let (z,z1,...,zp) = expr2
      in (z,u1,...,uq)
\end{verbatim}

If also \code{e1} and \code{e2} have no side effects the translation can be further simplified to
the Cogent expression
\begin{verbatim}
  if expr0 /= 0 then expr1 else expr2
\end{verbatim}

As an example the C expression \code{i == 0? a = 5 : b++} is translated to
\begin{verbatim}
  (v,a,b) = 
    if i == 0 then 
      let a = 5
      in (a,a,b)
    else
      let (v,b) = (b,b+1)
      in (v,a,b)
\end{verbatim}

For these translations the way how the Cogent compiler is used is relevant. When it translates the Cogent code back to 
C it translates an expression of the form
\begin{verbatim}
  let v = if a then b else c
  in rest
\end{verbatim}
to a C statement of the form
\begin{verbatim}
  if a { v = b; rest}
  else { v = c; rest}
\end{verbatim}
duplicating the translated code for \code{rest} (which is called ``a-normal form'' in Cogent). 
The same happens for all bindings and expressions after
the binding of \code{v} in the context of the \code{let} expression. If several such bindings to a 
conditional expressions are used this leads to an exponential growth in size of the C code.

To prevent this the Cogent compiler must be used with the flag \code{--fnormalisation=knf}
which translates to ``k-normal form''
\begin{verbatim}
  if a { v = b}
  else { v = c}
  rest
\end{verbatim}

Alternatively, the conditional expression can be wrapped in a lambda expression in the form
\begin{verbatim}
  let v = (\(x1,...,xn) => if a then b else c) 
    (x1,...,xn)
  in rest
\end{verbatim}
where \code{x1,...,xn} are all Cogent variables used in the expressions \code{a, b, c}.
Cogent translates this code to C without duplicating \code{rest}, even if a-normal form is used.

\subsection{Statements}
\label{design-cstats-stat}

A C statement has no result value, it is only used for its side effects. Therefore we basically translate every
statement to a Cogent binding of the form \code{pattern = expr} where pattern is a tuple of all identifiers
modified by the statement.

However, a C statement can also alter the control flow in its environment, which is the case for 
\code{return} statements, \code{goto} statements etc. We treat this by adding a 
component to the \code{pattern}, so that the translation becomes
\begin{verbatim}
  (c,v1,...,vn) = expr
\end{verbatim}
The variable \code{c} is a
new variable which is not already bound in the context of the expression, it is used to bind the information about 
control flow modification by the statement. The other variables are the identifiers of all parameters and local 
variables modified by the statement. Since a statement need not modify variables, the pattern may also be
a single variable \code{c}. In the following descriptions we always use a pattern of the form \code{(c,v1,...,vn)}
with the meaning that for \code{n=0} it is the single variable \code{c}.

The value bound to \code{c} is a tuple of type 
\begin{verbatim}
  (Bool, Bool, Option T)
\end{verbatim}
where \code{T} is the original result type of the surrounding function. In a value \code{(cc,cb,res)} the component
\code{cc} is true if the statement contains a \code{return}, \code{break}, or \code{continue} statement outside
of a \code{switch} or loop statement. The component \code{cb} is true if the statement contains a \code{return}
or \code{break} statement outside of a \code{switch} or loop statement. The component \code{res} is the value 
\code{None} if the statement contains no \code{return} statement, otherwise it is the value \code{Some v} where
\code{v} is the value to be returned by the function.

The translation of \code{goto} statement and labels is not supported, this must be done manually.

\subsubsection{Simple Statements}

A simple C statement consists of a C expression, where the result is discarded. Therefore it makes only sense
if the C expression has side effects. Simple statements cannot modify the control flow, they always bind the 
variable \code{c} to the value \code{(False,False,None)} which we abbreviate as \code{ffn} in the following.

The binding in the translation of a simple C statement \code{e;} is mainly the binding resulting from the translation 
of the expression \code{e}, as described in Section~\ref{design-cstats-expr}. Let its translation
be \code{(v,v1,...,vn) = expr}.
Then the binding for the statement \code{e;} is
\begin{verbatim}
  (c,v1,...,vn) = 
    let (v,v1,...,vn) = expr
    in (ffn,v1,...,vn)
\end{verbatim}

The result \code{v} of \code{e} is discarded, the same considerations apply here as described for expressions using
the comma operator in Section~\ref{design-cstats-expr}. 

If \code{expr} is a tuple 
\code{(e0,e1,...,en)} the binding can further be simplified to
\begin{verbatim}
  (c,v1,...,vn) = (ffn,e1,...,en)
\end{verbatim}

As an example, the translation of the simple statement \code{i++;} is
\begin{verbatim}
  (c,i) = let (v,i) = (i,i+1) in (ffn,i)
\end{verbatim}
which can be simplified to
\begin{verbatim}
  (c,i) = (ffn,i+1)
\end{verbatim}

The empty C statement \code{;} is translated as
\begin{verbatim}
  c = ffn
\end{verbatim}

\subsubsection{Jump Statements}

A jump statement is a \code{return} statement, \code{break} statement, or \code{continue} statement.

A C return statement has the form \code{return;} or \code{return e;}. It always ends the control flow in the surrounding
function body. The first form can only be used in functions returning \code{void}, these are translated to Cogent as returning 
\code{()} or a tuple with \code{()} as first component.

The translation of a return statement of the form \code{return;} is the binding
\begin{verbatim}
  c = (True,True,Some ())
\end{verbatim}
If \code{e} has no
side effects, the translation of statement \code{return e;} is
\begin{verbatim}
  c = (True,True,Some expr)
\end{verbatim}
where \code{expr} is
the translation of \code{e}. 

Otherwise, let \code{(v,v1,...,vn) = expr} be the translation of \code{e}. Then 
the translation of \code{return e;} is
\begin{verbatim}
  (c,v1,...,vn) = 
    let (v,v1,...,vn) = expr
    in ((True,True,Some v), v1,...,vn)
\end{verbatim}
which can be simplified if \code{expr} is the tuple \code{(e0,e1,...,en)} to
\begin{verbatim}
  (c,v1,...,vn) = ((True,True,Some e0),e1,...,en)
\end{verbatim}

A C \code{break} statement has the form \code{break;}. It ends the next surrounding \code{switch} or loop statement,
otherwise it may not be used. It is translated to
\begin{verbatim}
  c = (True,True,None)
\end{verbatim}

A C \code{continue} statement has the form \code{continue;}. It ends the body of the next surrounding loop statement,
otherwise it may not be used. It is translated to
\begin{verbatim}
  c = (True,False,None)
\end{verbatim}

\subsubsection{Conditional Statements}

A conditional C statement has the form \code{if (e) s1} or \code{if (e) s1 else s2} where \code{e} is an expression
and \code{s1} and \code{s2} are statements. Let \code{(c1,v1,...,vn) = expr1} be the translation 
of \code{s1} and \code{(c2,w1,...,wm) = expr2} be the translation of \code{s2}.
If \code{e} has no side effects and translates to the Cogent expression \code{expr} we translate
the second form of the conditional statement to the binding
\begin{verbatim}
  (c,u1,...,uk) =
    if expr /= 0 then 
      let (c1,v1,...,vn) = expr1
      in (c1,u1,...,uk)
    else
      let (c2,w1,...,wm) = expr2
      in (c2,u1,...,uk)
\end{verbatim}
where \code{u1,...,uk} is the union of \code{v1,...,vn} and \code{w1,...,wm} in some order. 
Note that this closely corresponds to the translation 
of the conditional expression in Section~\ref{design-cstats-expr}. In particular, the condition test 
\code{if expr /= 0 then} must be replaced by \code{if expr then} or \code{if not isNull(expr) then} or 
the forms with \code{notNull} or \code{roNotNull} as needed.

The first form of the conditional statement is translated as
\begin{verbatim}
  (c,v1,...vn) =
    if expr /= 0 then expr1
    else ((False,False,None),v1,...vn)
\end{verbatim}

If expression \code{e} has side effects the translation is extended as for the conditional expression.

As an example the translation of the conditional statement \code{if (i==0) return a; else a++;} is the binding
\begin{verbatim}
  (c,a) =
    if i==0 then 
      let c = (True,True,Some a)
      in (c,a)
    else
      let (c,a) = ((False,False,None),a+1)
      in (c,a)
\end{verbatim}
which can be simplified to
\begin{verbatim}
  (c,a) =
    if i==0 then ((True,True,Some a),a)
    else ((False,False,None),a+1)
\end{verbatim}

\subsubsection{Compound Statements}

A compound statement is a block of the form \code{\{ s1 ... sn \}} where every \code{si} is a statement or a declaration.
Declarations of local variables are treated as described in Section~\ref{design-cstats-general}: if they contain an
initializer they are translated as a statement, otherwise they are omitted. This reduces the translation of a compound
statement to the translation of a sequence of statements.

We provide the translation for the simplest case \code{\{ s1 s2 \}} where \code{s1} and \code{s2} are statements.
The general case can be translated by rewriting the compound statement as a sequence of nested blocks, associating
from the right.

Let \code{(c,v1,...,vn) = expr1} be the translation of statement \code{s1} and \code{(c,w1,...,wm) = expr2} 
be the translation of statement \code{s2}. Then the translation of
\code{\{ s1 s2 \}} is
\begin{verbatim}
  (c,u1,...,uk) =
    let ((cc,cb,res),v1,...,vn) = expr1
    in if cc then ((cc,cb,res),u1,...,uk)
    else let (c,w1,...,wm) = expr2
    in (c,u1,...,uk)
\end{verbatim}
where \code{u1,...,uk} is the union of \code{v1,...,vn} and \code{w1,...,wm} in some order without the local variables 
declared in the block.

The values of local variables declared in the block are discarded. If such variables have linear type, they must
be allocated on the heap and disposed at the end of the block, as described in Section~\ref{app-transfunction-addrop}.

As an example, the compound statement \code{\{ if (i==0) return a; else a++; b = a; \}} is translated to the binding
\begin{verbatim}
  (c,a,b) =
    let ((cc,cb,res),a) = 
      if i==0 then ((True,True,Some a),a)
      else ((False,False,None),a+1)
    in if cc then ((cc,cb,res),a,b)
    else let (c,b) = ((False,False,None),a)
    in (c,a,b)
\end{verbatim}

If \code{s1} contains no jump statement the variable \code{cc} is bound to \code{False} and the translation
can be simplified to the binding
\begin{verbatim}
  (c,u1,...,uk) =
    let (c,v1,...,vn) = expr1
    and (c,w1,...,wm) = expr2
    in (c,u1,...,uk)
\end{verbatim}

As an example, the compound statement \code{\{ int i = 0; a[i++] = 5; b = i + 3; \}} is translated to the 
simplified binding
\begin{verbatim}
  (c,a,b) =
    let (c,i) = (ffn,0)
    and (c,i,a) = (ffn,i+1,setArr(a,i,5))
    and (c,b) = (ffn,i+3)
    in (c,a,b)
\end{verbatim}
which can be further simplified to
\begin{verbatim}
  (c,a,b) =
    let i = 0
    and (i,a) = (i+1,setArr(a,i,5))
    and b = i+3
    in (ffn,a,b)
\end{verbatim}

A compound statement used as the body of a C function is translated to a Cogent expression instead of a Cogent binding.
If the C function has result type \code{void} and modifies its parameters \code{pm1,...,pmn}, its translation is 
a Cogent function returning the tuple \code{((),pm1,...,pmk)}. If the translation of the compound statement
used as function body is the binding \code{(c,v1,...,vn) = expr}, the body is translated to the expression
\begin{verbatim}
  let (c,v1,...,vn) = expr
  in ((),pm1,...,pmn)
\end{verbatim}
If the function modifies no parameters
it returns the unit value \code{()}. Note that if the C function is correct, \code{c} contains the value \code{Some ()}.

Here all local variables and the unmodified parameters are discarded. If local variables have linear type, they must
be allocated on the heap and disposed at the end of the function, as described in Section~\ref{app-transfunction-addrop}.
If an unmodified parameter has linear type, it must be disposed at the end of the function.

If the C function returns a value its translation is a Cogent function returning the tuple \code{(v,pm1,...,pmk)}
where \code{v} is the original result value of the C function. Then the body is translated to the expression
\begin{verbatim}
  let ((_,_,res),v1,...,vn) = expr
  in res | None -> (defaultVal(),pm1,...,pmn)
         | Some v -> (v,pm1,...,pmn)
\end{verbatim}
where \code{defaultVal()} is as defined in Section~\ref{design-operations-default}. If the type of the original function
result \code{v} is not regular (in C: contains a pointer) \code{defaultVal} cannot be used, in this case a different
expression for the \code{None} alternative must be specified manually. Note that in this case the \code{None} alternative
is never used, if the C program is correct. Nevertheless, for Cogent to be correct some result value must be specified 
also for this case, it is irrelevant which value is used.

\subsubsection{Switch Statements}

A \code{switch} statement in C specifies a ``control expression'' of integer type and a body statement. Control jumps to a 
\code{case} statement in the body marked with the value of the control expression or to a \code{default} statement. If neither
is present in the body it is not executed.

The \code{case} and \code{default} statements may occur anywhere in the body other than in a nested \code{switch} statement,
in particular they may occur in nested conditional statements and in loop statements. In these cases the jump is past the 
condition or loop initialization. The resulting behavior is clearly defined in the C standard, however, it is difficult to 
be transferred to a functional specification and usually involves code copying. The jump targets may also occur in a nested 
compound statement so that the jump may be past some declarations into an inner scope, this is also difficult to transfer.

However, there is a simple form of \code{switch} statements which can be transferred in a rather straightforward way. It is
the case where the body is a compound statement and all \code{case} and \code{default} statements occur directly in the 
sequence of contained statements. These \code{switch} statements have the form
\begin{verbatim}
  switch (e) {
    ...
  case c1: s11 ... s1k1
  ...
  case cn: sn1 ... snkn
  default: s01 ... s0k0
  }
\end{verbatim}
where the \code{default} statement may have any other position or may be omitted. The C standard requires that all \code{ci}
constants are pairwise distinct. In practical applications most \code{switch} statements are of this form. 

If there are declarations among the \code{sij} or before the first \code{case} statement the \code{switch} statement may cause
a jump past the declaration which means the declared object exists but is not initialized, which is also difficult to transfer.
Therefore we assume that there are no declarations in the body.

Then the \code{switch} statement of the form shown above can be rewritten to the following statement
\begin{verbatim}
  { t v = e;
    if (v == c1) {s11 ... s1k1}
    ...
    if (v == c1 || ... || v == cn) {sn1 ... snkn}
    {s01 ... s0k0}
  }
\end{verbatim}
and can be translated as described in the corresponding sections. Here \code{v} is an otherwise unused variable and \code{t} is
the type of \code{e}. It is needed even if \code{e} is a single variable, because this variable could be modified in some cases
but in the following case conditions the original value must be used.

Note that the additional disjunctions and the omitted condition for the default part cover the feature of ``fall through'' 
in a \code{switch} statement: the execution of following cases is only prevented by an explicit \code{break} or \code{return}
statement. If the \code{default} statement is placed before other \code{case} statements it also needs a condition which prevents
its execution for the cases following it.

Since \code{break} statements which are not enclosed by a nested loop or \code{switch} statement are consumed by the \code{switch}
statement and not propagated to the context, the control flow tuple must be reset by the switch statement. If it is bound to
\code{(cc,cb,res)} by the body, it must be rebound by the \code{switch} statement to
\begin{verbatim}
  (res /= None || (cc && not cb),res /= None, res)
\end{verbatim}
which will only preserve the effect of \code{return} and \code{continue} statements. A \code{continue} statement does not 
affect the \code{switch} statement but may be present and affect a surrounding loop.

As an example, the \code{switch} statement
\begin{verbatim}
  switch (i) {
    case 1: return 0;
    case 2: i++;
    case 3: return i+10;
    default: return i;
  }
\end{verbatim}
can be rewritten to
\begin{verbatim}
  {
    if (v==1) return 0;
    if (v==1 || v==2) i++;
    if (v==1 || v==2 || v==3) return i+10;
    return i;
  }
\end{verbatim}
which is translated to
\begin{verbatim}
  (c,i) = let v = i in
    let (cc,cb,res) = 
      if v==1 then (True,True,Some 0)
              else (False,False,None)
    in if cc then ((cc,cb,res),i) else 
      let ((cc,cb,res),i) = 
        if v==1 || v==2 then ((False,False,None),i+1)
                        else ((False,False,None),i)
      in if cc then ((cc,cb,res),i) else 
        let c = 
          let (cc,cb,res) = 
            if v==1 || v==2 || v==3 
            then (True,True,Some i+10)
            else (False,False,None)
          in if cc then (cc,cb,res)
                   else (True,True,Some i)
        in (c,i)
\end{verbatim}
which could be simplified to
\begin{verbatim}
  (c,i) = 
    ((True,True,
      if i==1 then Some 0
      else if i==2 then Some 13
      else if i==3 then Some 13
      else Some i),
     if i==2 then 3 else i)
\end{verbatim}

\subsubsection{For Loops}

A \code{for} loop has the form \code{for (ee1; ee2; ee3) s} with optional expressions \code{eei} and a statement \code{s}. 
The first expression \code{ee1} can also be a declaration. The \code{for} loop can always be transformed to a \code{while}
loop of the form
\begin{verbatim}
  ee1;
  while (ee2) { s ee3; }
\end{verbatim}
therefore \code{ee2} must evaluate to a scalar value, as for the condition in a conditional expression or statement.

Since \code{while} loops are not directly supported by Cogent and need additional work for proofs when translated by using
an abstract function, we do not use this transformation in general, instead, for certain forms of \code{for} loops we
translate them using functions from the Cogent standard library. This is possible, whenever a \code{for} loop is a 
``counted'' loop of the form
\begin{verbatim}
  for ( w=e1; w<e2; w=e3) s
\end{verbatim}
with a counting variable \code{w} of an unsigned integer type. The only free variable in \code{e3} must be \code{w} and
\code{w} must not occur free in \code{e2}. The \code{e1, e2, e3} must not have side effects.

If a \code{for} loop is not of this form it may be possible to rewrite it in C to that form. Specifically, this can be done 
if it is possible to rewrite the expressions \code{eei} as follows:
\begin{verbatim}
  ee1: ee1', w=e1
  ee2: w<e2 && ee2'
  ee3: ee3', w=e3
\end{verbatim}
Then the loop \code{for (ee1; ee2; ee3) s} can be rewritten as the counted loop
\begin{verbatim}
  ee1';
  for (w=e1; w<e2; w=e3) {
    if (!ee2') break;
    s ee3'
  }
\end{verbatim}
In other cases, especially if some of the \code{eei} are empty, it may still be possible to rewrite the loop as counted by
moving parts from before the loop or from the loop body into the \code{ei}.

Let \code{expri} be the translation of the \code{ei} and let \code{(c,v1,...,vn) = expr} be the translation of 
statement \code{s}. Let \code{w1,...,wm} be the variables occurring free in \code{expr} other than \code{w} and the \code{vi}.
Then the translation of the counted loop 
\begin{verbatim}
  for (w=e1; w<e2; w=e3) s
\end{verbatim}
is
\begin{verbatim}
  (c, v1,...,vn) = 
  let ((v1,...,vn),lr) = 
    seq32_stepf #{
      frm = expr1, to = expr2, 
      stepf = \w => expr3, 
      acc = (v1,...,vn), obsv = (w1,...,wm), 
      f = \#{acc = (v1,...,vn), obsv = (w1,...,wm), 
             idx = w} => 
        let ((cc,cb,res),v1,...,vn) = expr
        in ((v1,...,vn),
            if cb then Break (res) else Iterate ())
      }
  in lr | Iterate () -> ((False,False,None), v1,...,vn)
        | Break (None) -> ((False,False,None), v1,...,vn)
        | Break (res) -> ((True,True,res), v1,...,vn)
\end{verbatim}

The function \code{seq32\_stepf} is defined in the Cogent standard library in \code{gum/common/common.cogent}.

The variables \code{w,w1,...,wm,v1,...,vn} either occur free in the loop body or are modified in the loop body.
In both cases they must be declared in C before the counted loop, so they are defined in Cogent and can be used
when \code{acc} and \code{obsv} are set. The \code{w1,...,wm} must not be of linear type. Since they are used
in the loop body, depending on how often the body is executed they could either be used several times or not at all.
If a \code{wi} is of linear type the loop cannot be translated in this way and must be handled manually.

Since all \code{w1,...,wm} have a regular or readonly type the tuple \code{(w1,...,wm)} is a valid value for 
both \code{obsv} fields which have the readonly type \code{obsv!}.

The translation correctly handles all \code{return}, \code{break}, and \code{continue} statements which may occur 
in the body. 

Note that the translation is only possible if the type of the counting variable \code{w} can be represented in 
Cogent by the type \code{U32}.

Since \code{seq32\_stepf} is an abstract function the Cogent compiler does not generate a refinement proof for it.
The refinement proof for \code{seq32\_stepf} can only be successful if the loop terminates. This depends on the 
expression \code{e3}. As usual, if it strictly counts the variable \code{w} up, and some additional assumptions 
about the modulo calculation for \code{U32} are valid, termination can be proved.

If the expression \code{w=e3} can be rewritten as \code{w+=e3'} where \code{w} does not occur free in \code{e3'}
the counted loop can be translated using 
function \code{seq32} instead of \code{seq32\_stepf}. The translation has the same form as above, with the line 
setting \code{stepf} replaced by
\begin{verbatim}
      step = expr3', 
\end{verbatim}
where \code{expr3'} is the translation of \code{e3'}. Here, arbitrary other variables may occur free in \code{expr3'}.

In this case the counting variable \code{w} may also have a type which can be represented in Cogent by the type \code{U64}.
Then the translation has the same form but uses the function \code{seq64} instead of \code{seq32}.

If the expression \code{w=e3} can be rewritten as \code{w-=e3'} where \code{w} does not occur free in \code{e3'}
and the expression \code{ee2} can be rewritten as \code{w>e2 \&\& ee2'}
the counted loop can be translated as for \code{w+=e3'} using the function \code{seq32\_rev} instead of \code{seq32}.

If the translation can be done using \code{seq32} and the body does not contain any \code{break} or \code{return} statement 
the translation can be simplified using the function \code{seq32\_simpl} in the form
\begin{verbatim}
  (c, v1,...,vn) = 
  let w = expr1
  and (v1,...,vn,w1,...,wm,w) = 
    seq32_simple #{
      frm = w, to = expr2, step = expr3', 
      acc = (v1,...,vn,w1,...,wm,w), 
      f = \(v1,...,vn,w1,...,wm,w) => 
        let (_,v1,...,vn) = expr
        in (v1,...,vn,w1,...,wm,w+expr3')
      }
  in ((False,False,None), v1,...,vn)
\end{verbatim}
Note that \code{seq32\_simple} does not pass the counting variable \code{w} to the body, so we must add it to \code{acc}
and count it explicitly in \code{f}.
If the counting variable \code{w} is not used in the body (does not occur free in \code{expr}) this can further be simplified to
\begin{verbatim}
  (c, v1,...,vn) = 
  let (v1,...,vn,w1,...,wm) = 
    seq32_simple #{
      frm = expr1, to = expr2, step = expr3', 
      acc = (v1,...,vn,w1,...,wm), 
      f = \(v1,...,vn,w1,...,wm) => 
        let (_,v1,...,vn) = expr
        in (v1,...,vn,w1,...,wm)
      }
  in ((False,False,None), v1,...,vn)
\end{verbatim}

As an example, the counted loop
\begin{verbatim}
  for (i=0; i<size; i++) { 
    sum += a[i];
  }
\end{verbatim}
is translated to
\begin{verbatim}
  (c, sum) = 
  let i = 0
  and (sum,a,i) = 
    seq32_simple #{
      frm = i, to = size, step = 1, 
      acc = (sum,a,i), 
      f = \(sum,a,i) => 
        let sum = sum + getArr(a,i)
        in (sum,a,i+1)
      }
  in ((False,False,None), sum)
\end{verbatim}
where \code{a} must have a readonly Gencot array type. 

For the counted loop
\begin{verbatim}
  for (i=0; i<size; i++) { 
    if (a[i] == trg) return i;
  }
\end{verbatim}
the body translates to 
\begin{verbatim}
  c = if getArr(a,i) == trg 
      then (True,True,Some i)
      else (False,False,None)
\end{verbatim}
where again \code{a} must be of a readonly Gencot array type. The translation of the loop is
\begin{verbatim}
  c = 
  let ((),lr) = 
    seq32 #{
      frm = 0, to = size, step = 1, 
      acc = (), obsv = (a,trg), 
      f = \#{acc = (), obsv = (a,trg), idx = i} => 
        let (cc,cb,res) =
          if getArr(a,i) == trg 
          then (True,True,Some i)
          else (False,False,None)
        in ((),if cb then Break (res) else Iterate ())
      }
  in lr | Iterate () -> (False,False,None)
        | Break (None) -> (False,False,None)
        | Break (res) -> (True,True,res)
\end{verbatim}
which can be simplified to
\begin{verbatim}
  c = 
  let ((),lr) = 
    seq32 #{
      frm = 0, to = size, step = 1, 
      acc = (), obsv = (a,trg), 
      f = \#{obsv = (a,trg), idx = i} => 
        ((),if getArr(a,i) == trg 
            then Break (Some i) 
            else Iterate ())
      }
  in lr | Iterate () -> (False,False,None)
        | Break (None) -> (False,False,None)
        | Break (res) -> (True,True,res)
\end{verbatim}

Since end of 2021 there is a new abstract function \code{repeat} for loops in Cogent for which a general refinement proof 
has been developed. It is intended to be added to the Cogent standard library. It is even more general than \code{seq32\_stepf} 
and can be used to translate all \code{for} loops for which an upper limit of the number of iterations can be calculated.
This includes the terminating counted \code{for} loops described above.

A general \code{for} loop \code{for (ee1; ee2; ee3) s} where \code{ee2} has no side effects is first rewritten to 
\begin{verbatim}
  ee1;
  for (; ee2; ee3) s
\end{verbatim}
by moving the initialization expression or declaration before the loop.

Let \code{expr2} be the translation of \code{ee2} and let \code{(c,v1,...,vn) = expr} be the translation of 
the statement sequence \code{\{s ee3\}}. Let \code{w1,...,wm} be the variables occurring free in \code{expr} other than 
\code{c} and the \code{vi}.
Then the translation of the remaining loop 
\begin{verbatim}
  for (; ee2; ee3) s
\end{verbatim}
is
\begin{verbatim}
  (c, v1,...,vn) =
  let ((_,_,res),v1,...,vn) = repeat #{
    n = exprmax,
    stop = \#{acc = ((_,cb,_),v1,...,vn), 
              obsv = (w1,...,wm)} => cb || not expr2,
    step = \#{acc = (_,v1,...,vn), 
              obsv = (w1,...,wm)} => expr
    acc = ((False,False,None),v1,...,vn), obsv = (w1,...,wm)
    }
  in ((res /= None,res /= None,res),v1,...,vn)
\end{verbatim}
where \code{exprmax} is an expression for calculating the upper limit of the number of iterations. The \code{repeat}
function is defined in \code{C} using a \code{for} loop of the form \code{for (i=0; i<n; i++)}, so it iterates atmost
\code{n} times. If the loop has the form \code{for (w=e1;w<e2;w+=e3') s} where \code{w} does not occur free in \code{e1},
\code{e2}, and \code{e3'} the expression \code{exprmax} can be constructed by translating the \code{C} expression
\begin{verbatim}
  (e2 - e1) / e3'
\end{verbatim}
In other cases the expression for the upper limit must be determined manually.

The \code{repeat} function passes an ``accumulator'' \code{acc} and an ``observed object'' \code{obsv} through all
iterations, were \code{acc} may be modified and \code{obsv} not. The \code{repeat} function takes as arguments the 
initial values of \code{acc} and \code{obsv} and two functions \code{stop} and \code{step} which both have \code{acc}
and \code{obsv} as arguments. The \code{stop} function is used before each iteration to determine whether a next iteration
should be executed, it returns a boolean value whether to stop.  The \code{step} function covers the effect of one loop iteration and returns the 
new \code{acc} value. In the translation the \code{acc} value includes the control variable \code{c}. The \code{stop}
function inspects the value of its \code{cb} component to detect whether a \code{break} or \code{return} statement
has been executed in the body and terminates the loop in that case.

Note that this translation may not be fully equivalent to the original C code. If the expression \code{e3'} evaluates 
to a value greater 1, the increment of \code{w} may cause an overflow of the word arithmetics, even if \code{w<e2}
holds before. Then the loop will not terminate in that step and possibly never. The translation described above will
instead terminate safely after \code{n} iterations. Since this situation is normally a programming error in C
this is not viewed as a problem.

\subsubsection{While Loops}

A \code{while} loop has the form \code{while (e) s} or \code{do s while (e);}. Since Cogent does not support recursive 
function definitions these loops cannot be translated in a natural way to Cogent.

To translate a \code{while} loop it may be rewritten in C as a counted \code{for} loop which can then be translated 
as described in the previous section. If that is not possible the loop must be put in an abstract function which is 
defined in C using the original \code{while} loop.

Alternatively, if an upper limit of the number of iterations can be determined, the loop can be translated using the 
abstract \code{repeat} function described above as follows.

Let \code{expr1} be the translation of \code{e} and let \code{(c,v1,...,vn) = expr2} be the translation of 
statement \code{s}. Let \code{w1,...,wm} be the variables occurring free in \code{expr1} and \code{expr2} other than 
\code{c} and the \code{vi}.
Then the translation of the loop \code{while (e) s} is
\begin{verbatim}
  (c, v1,...,vn) = 
  let ((_,cb,_),v1,...,vn) = repeat #{
    n = exprmax,
    stop = \#{acc = ((_,cb,_),v1,...,vn), 
              obsv = (w1,...,wm)} => cb || not expr1,
    step = \#{acc = (_,v1,...,vn), 
              obsv = (w1,...,wm)} => expr2
    acc = (c,v1,...,vn), obsv = (w1,...,wm)
    }
  in ((res /= None,res /= None,res),v1,...,vn)
\end{verbatim}
where \code{exprmax} is an expression which calculates an upper limit for the number of loop iterations.

\subsection{Dummy Expressions}
\label{design-cstats-dummy}

Dummy expressions are generated when Gencot cannot translate a C declaration, statement, or expression. It consist of 
a Cogent expression of the form
\begin{verbatim}
  gencotDummy "... reason for not translating ..."
  {- ... untranslated C code ... -}
\end{verbatim}
(see Section~\ref{design-operations-dummy}) with the C code contained as a comment.

The C code is intended for the programmer who has to deal with it manually. Therefore it should
be readable, even if it is a larger C fragment. Therefore Gencot provides the following features for it:
\begin{itemize}
\item generate origin markers so that comments and preprocessor directives are inserted,
\item map C names to Cogent names so that they can be related to the surrounding Cogent code.
\end{itemize}

\subsubsection{Origin Markers}

To preserve comments embedded in the embedded C code it is also considered as a structured source code part
and origin markers are inserted around its subparts. Its 
subpart structure corresponds to the syntactic structure of the C AST. Since in the target code only identifiers 
are substituted, the target code
structure is the same as that of the source code. The structure is only used for identifying and re-inserting
the transferrable comments and preprocessor directives. Note that this works only if the conditional directive 
structure is compatible with the syntactic structure, i.e., a group must always contain a complete syntactical
unit such as a statement, expression or declaration, which is the usual case in C code in practice.

An alternative approach would be to treat all nonempty source code lines as subparts of a function body, resulting
in a flat sequence structure of single lines. The advantage is that it is always compatible to the conditional 
directive structure and
all comment units would be transferred. However, generating the corresponding origin markers in an abstract syntax
tree is much more complex than generating them for syntactical units for which the origin information is present
in the syntax tree. Since the Gencot implementation generates the
target code as an abstract syntax tree, the syntactical statement structure is preferred. 

\subsubsection{Mapping Names}

To relate the names in the C code to the Cogent code, Gencot substitutes names occurring free in the C code. These may
be names with global scope (for types, functions, tags, global variables, enum constants or preprocessor constants)
or names of parameters and local variables. For all names with global scope Gencot has generated a Cogent definition using a mapped name.
These names are substituted in the C code of the function body by the corresponding mapped names so that the 
mapping need not be done manually by the programmer.

Names with local scope (either local variables or types/tags) are also mapped by Gencot, both in their (local)
definition and whenever they are used. 

The mapping does not include the mapping of derived types. Type derivation in C is done in declarators which refer
a common type specification in a declaration. In Cogent there is no similar concept, every declarator must be 
translated to a separate declaration. This is not done for embedded C code. As result, an embedded
local declaration may have the form
\begin{verbatim}
  Struct_s1 a, *b, c[5];
\end{verbatim}
although the Cogent types for \code{a, b, c} would be \code{\#Struct\_s1}, \code{Struct\_s1}, and 
\code{\#(CArr5 Struct\_s1)}, respectively. This translation for the derived types must be done manually.

Global variables referenced in a function body are treated in a specific way. If the global variable has a 
Global-State property (see Section~\ref{design-types-itemprops}) and a virtual parameter item with the same
property has been declared for the function, the identifier of the global variable is replaced by the 
mapped parameter name with the dereference operator applied. This corresponds to the intended purpose of the 
Global-State properties: instead of accessing global variables directly from within a function, a pointer is 
passed as parameter and the variable is accessed by dereferencing this pointer.

If a global variable has the Const-Val property (see Section~\ref{design-types-itemprops}) its access is
replaced by an invocation of the corresponding parameterless access function. If the global variable has 
neither of these properties declared, Gencot simply uses its mapped name, this case must be handled manually.

