
Here we define rules how to map common C types to binary compatible Cogent types. Since the usefulness of a mapping
also depends on the way how values of the type are processed in the C program, the rules here are only defaults
which are used, if no better mapping is suggested by the way of value processing.

\subsubsection{Numerical Types}

The Cogent primitive types are mapped to C types in \code{cogent/lib/cogent-defns.h} which is included by the Cogent compiler
in every generated C file with \code{\#include <cogent-defns.h>}. The mappings are: 
\begin{verbatim}
  U8 -> unsigned char
  U16 -> unsigned short 
  U32 -> unsigned int
  U64 -> unsigned long long
  Bool -> struct bool_t { unsigned char boolean }
  String -> char*
\end{verbatim}
The inverse mapping can directly be used for the unsigned C types. For the corresponding signed C types to be binary
compatible, the same mapping is used. Differences only occur when negative values are actually used, this must be handled by using specific functions for numerical operations in Cogent.

The C99 standard defines the ``exact-width integer types'', such as \code{uint32\_t}, provided by \code{\#include <stdint.h>}. They can safely be mapped to the corresponding primitive Cogent types. 

Type \code{size\_t} is defined by the C99 standard to be the type of all values returned by the \code{sizeof} operator.
On 64 Bit architectures this is equivalent to type \code{long long} and corresponds to the Cogent type \code{U64}.

Together we have the following mappings:
\begin{verbatim}
char, unsigned char, uint8_t -> U8
short, unsigned short, uint16_t -> U16
int, unsigned int, uint32_t -> U32
long long, unsigned long long, uint64_t -> U64
size_t -> U64
\end{verbatim}

Note that the mappings for \code{int}, \code{unsigned int} and \code{size\_t} are architecture dependent. 
Here we ignore this dependency and always use the mapping for 64 Bit architecture. An architecture independent 
approach would be possible using a conditional mapping with the help of preprocessor \code{\#if} directives.

\subsubsection{Structure and Union Types}
\label{design-types-struct}

A C structure type of the form \code{struct \{ ... \}} is equivalent to a Cogent unboxed record type \code{\#\{ ... \}}.
The Cogent compiler translates the unboxed record type to the C struct and maps all fields in the same order.
If every C field type is mapped to a binary compatible Cogent field type both types are binary compatible as a whole.

For a \code{const} qualified C structure type the fields may not be modified. This is equivalent to the behavior
of an unboxed record in Cogent, the same mapping is used here.

A C structure may contain bit-fields where the number of bits used for storing the field is explicitly specified.
Gencot maps every consecutive sequence of bit-fields to a single Cogent field with a primitive Cogent type.
The Cogent type is determined by the sum of the bits of the bit-fields in the sequence. It is the smallest 
type chosen from \code{U8, U16, U32, U64} which is large enough to hold this number of bits. 
***--> test whether this is correct.
The name of the
Cogent field is \code{cogent\_bitfield}<n> where <n> is the number of the bit-field sequence in the C structure.
Gencot does not generate Cogent code for accessing the single bit-fields. If needed this must be done manually in Cogent.
However, Gencot adds comments after the Cogent bitfield showing the original C bit-field declarations.

A C union type of the form \code{union \{ ... \}} is not binary compatible to any type generated by the Cogent compiler.
The semantic equivalent would be a Cogent variant type. However, the Cogent compiler translates every variant type
to a \code{struct} with a field for an \code{enum} covering the variants, and one field for every variant. Even if a variant
is empty (has no additional fields), in the C \code{struct} it is present with type \code{unit\_t} which
has the size of an \code{int}. Therefore Gencot maps every union type to an abstract Cogent type.

Since an abstract Cogent type is always linear, Gencot maps a \code{const} qualified C union type to the corresponding
readonly type.

Together we have the mapping rules:
\begin{verbatim}
  struct s, const struct s -> unboxed record
  union s -> abstract type S
  const union s -> abstract type S!
\end{verbatim}

\subsubsection{Pointer Types}

In general, a C pointer type \code{t*} is the kind of types targeted by Cogent linear types. The linear type 
allows the Cogent compiler to statically guarantee that pointer values will neither be duplicated nor 
discarded by Cogent code, it will always be passed through. 

If a pointer points to a C \code{struct} there is additional support for field access available in Cogent by 
mapping the pointer to a Cogent boxed record type. For all other pointer types the Cogent type must be abstract, 
then the pointer is opaque in Cogent code, it can only be passed around but no operations can be performed 
directly in Cogent. All processing must be implemented externally by an abstract data type.

Since a C \code{void*} pointer type is also opaque in C, it corresponds directly to a Cogent abstract type.

For a C pointer there are two cases of readonly types. A ``pointer to const'' type of the form \code{const t*}
means that the data structure pointed to cannot be modified, whereas the pointer itself can be replaced 
(if, e.g., it is stored in a variable of that type). A ``constant pointer'' type of the form \code{t* const}
instead means that the pointer itself cannot be modified, whereas the data structure pointed to can. In Cogent 
no difference is made between the pointer and its target, both together are always immutable, which corresponds 
to the combination of both C cases. However, for linear types, Cogent internally supports modification of the
data structure using \code{put} and \code{take}. The Cogent readonly types prevent this, therefore they correspond
to the first case in C. The second case is always respected by Cogent, if a pointer should be replaced, this 
must be implemented by an adequate processing approach in Cogent. 

Together we have the basic mapping rules:
\begin{verbatim}
  struct s *, struct s * const -> boxed record
  void *, void * const -> abstract type
  t *, t * const -> abstract type T
  const t *, const t* const -> type T!
\end{verbatim}

In C a \code{t*} pointer can be assigned to a \code{const t*} pointer, but not vice versa. This corresponds to
the Cogent property that a linear value may be made readonly, but not the other way round.

\subsubsection{Array Types}

The implementation of a C array type \code{t[n]} is equivalent to that of the pointer type \code{t*}. 
Whereas the type \code{t*} has the meaning of a pointer to a single \code{t} instance, type \code{t[n]}
has the meaning of a pointer to a consecutive sequence of n \code{t} instances. 

Basically, Cogent does not support accessing elements by an index value of an array represented by a pointer. 
This is an important security feature since the index value is computed at runtime and cannot be statically 
compared to the array length by the compiler. Therefore, a C array type can only be mapped to an abstract type 
in Cogent, which prevents accessing its elements in Cogent code. Element access must be implemented externally 
with the help of abstract functions.

The Cogent standard library contains three abstract data types for arrays (\code{Wordarray, Array, UArray}. 
However, they cannot be used as a binary compatible replacement for C arrays, because they are implemented by 
pointers to a \code{struct} containing the array length together with the pointer to the array elements. 
Only if the C array pointer is contained in such a struct, it is possible to use the abstract data types. 
In existing C code the array length is often present somewhere at runtime, but not in a single \code{struct}
directly before the array pointer.

As of December 2018 there is an experimental Cogent array type written \code{T[n]}. It is binary compatible 
with the C array type \code{t[n]}. It is not linear, however it only supports read access to the array elements, 
the element values cannot be replaced. Thus it can be used as replacement for a pure abstract type, if the array 
is never modified and if it does not contain any pointers (directly or indirectly). If it is modified, replacing
elements can be implemented externally with the help of abstract functions.

An array of type \code{const t []} cannot be modified, hence it is fully supported by the corresponding 
Cogent array type.

The primitive Cogent type \code{String} is mapped to C type \code{char*}. It is used to pass the usual 
null terminated C strings through Cogent code. The characters in the string cannot be accessed in Cogent, 
neither for replacing them nor for reading them. Thus, if the characters are not accessed, Cogent type
\code{String} is a useful mapping for all kinds of C character arrays.

In C, arrays of type \code{t[n]} can also be accessed through a Pointer of type \code{t*} using pointer
arithmetics. In this case type \code{t*} can be mapped to Cogent in the same way as C type \code{t[n]}.

In C the incomplete type \code{t[]} can be used in certain places. It may be completed statically, e.g. 
when initialized. Then the number of elements is statically known and the type can be mapped like \code{t[n]}.
If the number of elements is not statically known (this is also often the case when type \code{t*} 
is used for an array) the type cannot be mapped to a Cogent array, it must be mapped to an abstract type in
the same way as type \code{t*}.

The Cogent array type \code{T[1]} can be used as an alternative mapping for arbitrary pointer types which 
only point to a single element, if it does not contain pointers. Then it becomes possible to differentiate 
between the pointer and the pointed value in Cogent code. Since the array cannot be modified, this fully 
supports only the functionality of type \code{const t*}. Semantically, in Cogent there should be no difference 
to the non-linear type T, however, for binary compatibility the difference becomes relevant.

Together we have the following mapping rules for C arrays with element type \code{el}. Here, C type \code{el*} 
is a pointer type used to access an array.
\begin{verbatim}
  char[n], char[], unsigned char[n], unsigned char[], 
     const char[n], const char[],
     const unsigned char[n], const unsigned char[] -> String
  char*, unsigned char*, const char*, const unsigned char* 
     -> String, if used to access a C string
  const el[n], const el[], const el* 
     -> El[n], if n can be statically determined
  const el[], const el* 
     -> abstract type, if array size cannot be statically determined
  el[n], el[], el* -> abstract type
  const t* -> T[1], if t contains no pointers
\end{verbatim}

\subsubsection{Enumeration Types}
\label{design-types-enum}

A C enumeration type of the form \code{enum e} is a subset of type \code{int} and declares enumeration 
constants which have type \code{int}. According to the C99 standard, an enumeration type may be implemented
by type \code{char} or any integer type large enough to hold all its enumeration constants.

A natural mapping for C enumeration types would be Cogent variant types. However, the C implementation
of a Cogent variant type is never binary compatible with an integer type (see above). 

Therefore C enumeration types must be mapped to a primitive integer type in Cogent. Depending on the C
implementation, this may always be type \code{U32} or it may depend on the value of the last enumeration
constant and be either \code{U8}, \code{U16}, \code{U32}, or maybe even \code{U64}. Under Linux, both cc
and gcc always use type \code{int}, independent of the value of the last enumeration constant. 
Therefore we always map enumeration types to Cogent type \code{U32}.

The enumeration constants must be mapped to Cogent constant definitions of the corresponding type. In 
C the value for an enumeration constant may be explicitly specified, this can easily be mapped to
the Cogent constant definitions.

The rule for mapping enumeration types is
\begin{verbatim}
  enum e -> U32
\end{verbatim}

An enumeration declaration of the form \code{enum e \{C1, C2, C3=5, C4\}} is translated as
\begin{verbatim}
  cogent_C1: U32
  cogent_C1 = 0
  cogent_C2: U32
  cogent_C2 = 1
  cogent_C3: U32
  cogent_C3 = 5
  cogent_C4: U32
  cogent_C4 = 6
\end{verbatim}
Note that the C constant names are mapped to Cogent names as described in Section~\ref{design-names}.

\subsubsection{Function Types}
\label{design-types-function}

C function types of the form \code{t (...)} are used in C only for declaring or defining functions. In all other
places they are either not allowed or automatically adjusted to the corresponding function pointer type
of the form \code{t (*)(...)}. 

In Cogent the distinction between function types and function pointer types does not exist. 
A Cogent function type of the form \code{T1 -> T2} is used both when
defining functions and when binding functions to variables. If used in a function definition, it is mapped by
the Cogent compiler to the corresponding C function type, when used in other places it is mapped to the 
corresponding C function pointer type.

Binary compatibility is only relevant when a function is stored, then it is always a function pointer. All function
pointers are of the same size, hence a C function pointer type can be mapped to an arbitrary Cogent function type.
Of course, to be useful the types of the parameters and result should be mapped as well. In Cogent every function
has only one parameter. To be mapped to Cogent, the parameters of a C function with more than one parameter must
be aggregated in a tuple or in a record. A C function type \code{t (void)} which has no parameters is mapped
to the Cogent function type \code{() -> T} with a parameter of unit type.

The difference between using a tuple or record for the function parameters is that the fields in a 
record are named, in a tuple they are not. In 
a C function definition the parameters may be omitted, otherwise they are specified with names in a prototype.
In C function types the names of some or all parameters may be omitted, specifying only the parameter type.

It would be tempting to map C function types to Cogent functions with a record as parameter, whenever parameter 
names are available in C, and use a tuple as parameter otherwise. However, in C it is possible to assign a 
pointer to a function which has been defined 
with parameter names to a variable where the type does not provide parameter names such as in 
\begin{verbatim}
  int add (int x, int y) {...}
  int (*fun)(int,int);
  fun = &add;
\end{verbatim}
This case would result in Cogent code with incompatible function types.

For this reason we always use a tuple as parameter type in Cogent. Cogent tuple types are equivalent, if they
have the same number of fields and the fields have equivalent types. To preserve the C parameter names in 
a function definition, the parameter is matched with a tuple pattern containing variables of these
names as fields.

C function types where the parameters are omitted, such as in \code{t ()} or where a variable number of
parameters is specified such as in \code{t (...)} cannot be mapped to a Cogent function type in this way. 
They can only be mapped using an abstract type as parameter type. This can again lead to incompatible 
Cogent types if a function pointer is assigned where parameters have been specified, these cases must 
be treated manually in specific ways. Gencot maps these function types to a Cogent function type with an abstract
parameter type.

Together the rules for mapping function types are
\begin{verbatim}
  t(t1, ..., tn), t (*)(t1, ..., tn) 
    -> (T1, ..., Tn) -> T
  t(void), t (*)(void) 
    -> () -> T
  t(), t(*)(), t(t1,...,tn,...), t (*)(t1,...,tn,...) 
    -> P -> T, where P is abstract
\end{verbatim}

 
