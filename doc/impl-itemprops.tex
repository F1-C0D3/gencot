As described in Section~\ref{design-types-itemprops} Gencot supports declaring properties for typed 
items in the C program. The corresponding declarations are specified in textual form as a sequence
of text lines where each line may declare several properties for a single item. The item is
named by a unique identifier, every property is encoded by a short identifier.

\subsection{Item Identifiers}
\label{impl-itemprops-ids}

Item identifiers are text strings which contain no whitespace. 

Global individual items are functions or objects.
They can always be identified by their name. However, for items with internal linkage the name is only unique
per compilation unit, whereas the item declarations span the whole Cogent compilation unit which may consist of several C 
compilation units. Therefore, items with 
external linkage are identified by their plain name (which is an unstructured C identifier), items with
internal linkage are identified in the form
\begin{verbatim}
    <source file name> : <item name>
\end{verbatim}
If the source file name has extension \code{".c"} the extension is omitted, otherwise (typically in the case of \code{".h"})
it is not omitted, resulting in identifiers such as \code{"app.h:myitem"}.

As source file name only the base name is used. C packages with source files of the same name in different directories are
not supported by the current Gencot version.

Local individual items are variables (``objects'') locally defined in a function body. They are identified in the form
\begin{verbatim}
    ? <item name>
\end{verbatim}
which is not unique. The current version of Gencot does not try to identify local variables in a unique way, since the 
language-c parser does not directly provide the name of the surrounding function for them. Even if that name would be
used for disambiguation, this would not help for the case of different variables with the same name in the same function.
Therefore in current Gencot version it is not feasible to declare item properties for individual local variables. 

Collective items are identified by certain type expressions. For a typedef name the identifier has the form
\begin{verbatim}
    typedef | <typedef name>
\end{verbatim}
For a tag name it has one of the forms
\begin{verbatim}
    struct | <tag name>
    union  | <tag name>
\end{verbatim}
For a derived pointer type where the base type can be used as a collective item it has the form
\begin{verbatim}
    <base item identifier> *
\end{verbatim}

Sub-items of an item are identified by appending a sub-item specification to the item's identifier.
The members of a struct or union are specified in the form
\begin{verbatim}
    <item identifier> . <member name>
\end{verbatim}
The elements of an array are collectively specified in the form
\begin{verbatim}
    <item identifier> / []
\end{verbatim}
The data referenced by a pointer is specified in the form
\begin{verbatim}
    <item identifier> / *
\end{verbatim}
The result of a function is specified in the form
\begin{verbatim}
    <item identifier> / ()
\end{verbatim}
The parameters of a function are specified in either of the forms
\begin{verbatim}
    <item identifier> / <parameter name>
    <item identifier> / <position number>
\end{verbatim}
where the position number of the first parameter is 1.

As an example, if \code{f\_t} is a typedef name for a function type with a parameter \code{par} 
which is a pointer to an array of function pointers, the results of the function pointers in the
array can be identified by 
\begin{verbatim}
    typedef|f_t/par/*/[]/*/()
\end{verbatim}

\subsection{Property Identifiers}
\label{impl-itemprops-property}

Each property is identified by a two-letter code as follows:
\begin{description}
\item[Read-Only:] ro
\item[Not-Null:] nn
\item[No-String:] ns
\item[Heap-Use:] hu
\item[Add-Result:] ar
\item[Modification-Function:] mf
\end{description}

\subsection{Item Property Declarations}
\label{impl-itemprops-decl}

An item property declaration consists of a single text line which starts with an item identifier, followed
by an arbitrary number of property identifiers in arbitrary order, separated by space sequences.
For better readability a colon may be appended to the item identifier.

As an example, to declare properties for the parameters \code{p1, par2, param3} of a function pointer \code{myfunp}
and the Heap-Use property for the function itself, four declarations are needed:
\begin{verbatim}
    myfunp/*/p1:     nn ro
    myfunp/*/par2:   ar mf
    myfunp/*/param3: ns
    myfunp/*:        hu
\end{verbatim}

\subsection{Internal Representation}
\label{impl-itemprops-internal}

When used the textual form is converted to an internal Haskell representation of type 
\begin{verbatim}
  type ItemProperties = Map String [String]
\end{verbatim}
defined in module \code{Gencot.Util.Properties}. It maps item property identifiers to the list of identifiers of declared properties.
If several declarations for the same item are present in the textual form they are combined by uniting the 
declared properties. Repeated property specifications are removed, the order of the properties in the list is
arbitrary.

The \code{ItemProperties} map is added as component to the user state of the \code{FTrav} monad. The monadic actions
\begin{verbatim}
  getProperties :: String -> FTrav [String]
  hasProperty :: String -> String -> FTrav Bool
\end{verbatim}
are defined in \code{Gencot.Traversal} for accessing the item properties.
