If a function parameter of linear type is modified by a function it can be returned as part of the function result, 
as described in Section~\ref{design-types-function}. This is done by declaring the item property Add-Result 
(see Section~\ref{design-types-itemprops}) for the parameter. Gencot automatically declares this property for all 
parameters of linear type and provides specific support to suppress its effect by either adding the Read-Only 
property or by omitting the Add-Result property.

The Read-Only property can only be declared, if the parameter value is not modified by the C function implementation.
A parameter is modified, when a part of it is changed, which is referenced through a pointer. Only then the 
modification is shared with the caller and persists after the function call.

The Add-Result property must be omitted if a parameter is actually discarded by a function. This is only the case
for the external C standard function \code{free} or for parameters which are directly or indirectly passed to \code{free}.

\subsection{Detecting Parameter Modifications}

A parameter may be modified by an assignment to a referenced part. Such an assignment
is easy to detect, if it directly involves the parameter name. However, the assignment can also be indirect, where
the parameter or part of it is first assigned to a local variable or another structure and then the assignment
is applied to this variable or structure. This implies that a full data flow analysis would be necessary
to detect all parameter modifications.

Gencot uses a simpler semiautomatic approach. It detects all direct parameter modifications which involve the 
parameter name automatically. It then generates a JSON description which lists all functions with their parameters
and the information about the detected modification cases. The developer has to check this description and
manually add additional cases of other parameter modifications. The resulting description is then used by
Gencot to declare the Read-Only property for all unmodified parameters.

Parameter modifications are only relevant if the parameter is a pointer or if a pointer can be reached by the parameter
and if at least one such pointer is not declared to have a \code{const} qualified referenced type. Gencot detects
this information from the C type and adds it to the JSON description, so that the developer can easily identify
the relevant cases where to look for modifications.

In the case of a variadic function, the number of its parameters cannot be determined from the function definition.
Here Gencot uses the invocation with the most arguments to determine the number of parameter descriptions added
to the JSON description.

A parameter may be modified by a C function locally, but it can 
also be modified by passing it or a part of it as parameter to an invoked function which modifies its corresponding
parameter. Gencot supports these dependent parameter modifications, by detecting dependencies on parameter
modifications by invoked functions and adding them to the JSON description. Again, only dependencies which
involve the parameter name are detected automatically, other dependencies must be added by the developer.

Before Gencot uses the JSON descriptions it calculates the transitive closure of all dependencies and uses it to
determine the parameter modification information. Thus the developer has only to look at every function locally,
it is not necessary to take into account the effects of invoked functions.

Gencot generates the JSON descriptions from function definitions, so that it can analyse the function body for
detecting parameter modifications and invocations of other functions. If for a function no definition is available 
(which is the case for function pointers and for functions defined outside the C <package>), Gencot only generates 
a description template which must be filled by the developer.

\subsection{Required Invocations}

To support the incremental translation of single C source files, Gencot determines the parameter modification
description for single C source files. It processes all function definitions in the file by analysing their 
function bodies. However, for an invoked function the definition may not be available, it may be defined externally
in another C source file. This situation must be handled manually: the developer has to process additional
C source files where the invoked functions are defined, to add their descriptions.

Gencot specifically selects only the relevant invocations, which are required because a parameter modification depends on it.
If a parameter is already modified locally, additional dependencies are ignored and the corresponding invocations
are not selected. In this way Gencot keeps the JSON descriptions minimal.

An invoked function may also be specified using a function pointer. In this case the possible function bodies cannot 
be determined from the program. Again, it is left to the developer to determine, whether such invocations may modify
their parameters. Gencot supports this by also adding parameter description templates for function pointers defined in the 
source file, to be filled by the developer. 

In contrast to functions, function pointers can also be defined locally in a function\footnote{The current version of
Gencot does not support C extensions for nested function definitions.} (as local variable or as parameter). Gencot
automatically adds description templates for all invoked local function pointers.

\subsection{External Invocations}

Finally, a required invocation may be external to the package which is processed by Gencot, such as an invocation of
a function in the C standard library. When all remaining required invocations are external in this sense, Gencot
can be told to ``close'' the JSON description. Then it uses the declarations of all required invocations to generate
a description template for each of them. Since no bodies are available, the developer must fill in the information
for all these external functions (and function pointers).

Invoked function pointers may also be members of a struct or union type. For them no definition exists, they are also 
handled by Gencot when closing a JSON description: for all required invocations of a struct or union member a 
description template is added which is identified by the member name and the item identifier of the struct or union type.
 
\subsection{Described Entities}

Gencot uses the information about parameter modifications to declare Read-Only properties or omit Add-Result properties
for the parameters. 

C function types are only used when functions are declared or defined. Function definitions are translated to
Cogent for all functions defined in the Cogent compilation unit. Function declarations are translated to Cogent
for all functions invoked but not defined in the Cogent compilation unit. In both cases Gencot automatically 
determines the corresponding definitions or declarations and generates the JSON descriptions for them, to be
filled by the developer and read when declaring item properties for parameters.

C function pointer types can be used as type for several entities: for global and local variables, for members of struct
and union types, for parameters and the result of declared and defined functions, and as base type or parameter 
types in derived types.

As described in Section~\ref{design-fundefs-body}, local variable declarations are not translated by Gencot, hence
their types need not be mapped and no JSON description is provided if it is a function pointer type. 
For all other items (in the sense of Section~\ref{design-types-itemprops}) which have a function pointer type Gencot 
provides a JSON description. The
description is provided independent of whether the function has parameters of linear type or not. If not, it is
not needed by Gencot for translating the item's type, however it may be needed as required invocation for
calculating information for parameters depending on it.

If a typedef name is defined in C for a function type or a type derived from a function type, the parameters are collective items
and properties can only be declared for them collectively. In this case Gencot does not provide a JSON description for the individual
item(s) to which the typedef name applies, but instead a JSON description for the typedef name as a collective item. This 
is only done if the function type occurs directly in the typedef name's definition, if it is specified by another typedef
name the JSON description is only provided for that other name.

If a type name is defined external to the Cogent compilation unit and is only used indirectly, it is resolved and not 
used in the Cogent code. As described for items in Section~\ref{design-types-itemprops}, then collective sub-items of the
type are replaced by individual sub-items of the items declared to have that type. Gencot treats parameter modification 
descriptions in the same way and uses different individual descriptions, if function items use a common function type name which 
is resolved because it is external.
