
Since the \code{language-c} parser does not support parsing preprocessor directives and C comments, the general approach
is to remove both from the source file, process them separately, and re-insert them into the generated files.

For re-inserting it must be possible to relate comments and preprocessor directives to the generated target code parts. As 
described in Sections~\ref{design-comments} and~\ref{design-preprocessor}, comments and preprocessor directives are associated
to the C source code via line numbers. Whenever a target code part is generated, it is annotated with the line numbers of its
corresponding source code part. Based on these line number annotations the comments and preprocessor directives can be
positioned at the correct places.

The line number annotations are markers of one of the following forms, each in a single separate line:
\begin{verbatim}
  #ORIGIN <bline>
  #ENDORIG <aline>
\end{verbatim}
where \code{<bline>} and \code{<aline>} are line numbers.

An \code{\#ORIGIN} marker specifies that the next code line starts a target code part which was generated from a source code
part starting in line \code{<bline>}. An \code{\#ENDORIG} marker specifies that the previous code line ends a target code part
which was generated from a source code part ending in line \code{<aline>}. Thus, by surrounding a target code part with an 
\code{\#ORIGIN} and \code{\#ENDORIG} marker the position and extension of the corresponding source code part can be derived.

In the case of a structured source code part the origin marker pairs are nested, if the target code part generated from a subpart 
is nested in the target code part generated from the main part. If there is no code generated for the main part, the 
\code{\#ORIGIN} marker for the first subpart immediately follows the \code{\#ORIGIN} marker for the main part and the 
\code{\#ENDORIG} marker for the last subpart is immediately followed by the \code{\#ENDORIG} marker for the main part.

If no target code is generated from a source code part, the origin markers are not present. This implies, that an 
\code{\#ORIGIN} marker is never immediately followed by an \code{\#ENDORIG} marker.

It may be the case that several source code parts follow each other on the same line, but the corresponding target code
parts are positioned on different lines. Or from a single source code part several target code parts on different lines 
are generated. In both cases there are several origin markers with the same line number. Conditional preprocessor directives
associated with that line must be duplicated to all these target code parts. For comments, however, duplication is not
adequate, they should only be associated to one of the target code parts. This is implemented by appending an additional 
``+'' sign to an origin marker, as in 
\begin{verbatim}
  #ORIGIN <bline> +
  #ENDORIG <aline> +
\end{verbatim}
Comments are only associated with markers where the ``+'' sign is present, all other markers are ignored. In this way,
the target code generation can decide where to associate comments, if a position is not unique.

