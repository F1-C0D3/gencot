The goal of this step is to translate a single C source file which is part of the translation set to Cogent.
This is done using Gencot to translate the file to Cogent with embedded partially translated C code.
Afterwards the embedded C code must be translated manually, as described in Section~\ref{app-transfunction}.

The translated file is either a \code{.c} file which can be seperately compiled by the C compiler as a compilation unit,
or it is a \code{.h} file which is a part of one or more compilation units by being included by other files. In both cases
the file can include \code{.h} files, both in the package and standard C include files such as \code{stdio.h}.

To be translated by Gencot the C source file must first be prepared for being read by Gencot as described in
Section~\ref{app-prep}. Then the parameter modification descriptions for all functions and function types defined 
in the source file should be created and evaluated as described in Section~\ref{app-parmod-defined}. Afterwards,
the default item properties must be generated as described in Section~\ref{app-itemprops}. They may be manually
modified, if necessary.

\subsection{Translating Normal C Sources}
\label{app-transauto-normal}

A \code{.c} file is translated using the command
\begin{verbatim}
  gencot [options] cfile file.c
\end{verbatim}
where \code{file.c} is the file to be translated. The result of the translation is stored as \code{file.cogent} 
in the current directory.

An \code{.h} file is translated using the command
\begin{verbatim}
  gencot [options] hfile file.h
\end{verbatim}
where \code{file.h} is the file to be translated. The result of the translation is 
stored as \code{file-incl.cogent} in the current directory.

\subsection{Translating Configuration Files}
\label{app-transauto-config}

Special support is provided for translating configuration files. A configuration file is a \code{.h} file mainly
containing preprocessor directives for defining flags and macros, some of which are deactivated by being ``commented
out'', i.e., they are preceded by \code{//}. If such a file is translated using the command
\begin{verbatim}
  gencot [options] config file.h
\end{verbatim}
it is translated like a normal \code{.h} file, but before, all \code{//} comment starts at line beginnings are 
removed, and afterwards the corresponding Cogent comment starts \code{-\relax-} are re-inserted before the definitions.

