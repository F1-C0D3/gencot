\documentclass[a4paper]{report}
\usepackage[bookmarks]{hyperref}
\usepackage[utf8]{inputenc}
\usepackage[T1]{fontenc} % needed for italic curly braces

\newcommand{\code}[1]{\textnormal{\texttt{#1}}}

\begin{document}

\title{HoBit Report on Gencot Development}
\author{Gunnar Teege}

\maketitle

\chapter{Introduction}

Gencot (GENerating COgent Toolset) is a set of tools for generating Cogent code from C code. It is developed by UniBw 
as part of the HoBit project conducted with Hensoldt Cyber. In the project it is used for reimplementing
the mbedTLS library in Cogent as sembedTLS. However, Gencot is not specific for mbedTLS and should be applicable
to other C code as well.

Gencot is used for parsing the C sources and generating templates for the required Cogent sources, 
antiquoted Cogent sources, and auxiliary C code. 

Gencot is not intended to perform an automatic translation, it prepares the manual translation by 
generating templates and perfoming some mechanic steps.

Roughly, Gencot supports the following tasks:
\begin{itemize}
\item translate C preprocessor constant definitions and enum constants to Cogent object definitions,
\item generate function invocation entry and exit wrappers,
\item generate Cogent abstract function definitions for invoked exit wrappers,
\item translate C type definitions to default Cogent type definitions,
\item generate C type mappings for abstract Cogent types referring to existing C types,
\item generate Cogent function definition skeletons for all C function definitions,
\item rename constants, functions, and types to satisfy Cogent syntax requirements and avoid collisions,
\item convert C comments to Cogent comments and insert them at useful places in the Cogent source files,
\item generate the main files \code{<package>.cogent} and \code{<package>.ac} for Cogent compilation.
\end{itemize}

To do this, Gencot processes in the C sources most comments and preprocessor directives, all declarations (whether
on toplevel or embedded in a context), and all function definitions. It does not process C statements other than
for processing embedded declarations.

\chapter{Design}

\section{General Context}

We assume that there is a C application <package> which consists of C source files \code{.c} and \code{.h}. The 
\code{.h} files are included by \code{.c} files and other \code{.h} files. There may be included \code{.h} which
are not part of <package>, such as standard C includes, all of them must be accessible. Every \code{.c} file
is a separate compilation unit. There may be a \code{main} function but Gencot provides no specific support for it.

From the C sources of <package> Gencot generates Cogent source files \code{.cogent} and antiquoted Cogent source
files \code{.ac} as a basis for a manual translation from C to Cogent. All function definition bodies have to be
translated manually, for the rest a default translation is provided by Gencot.

Gencot supports an incremental translation, where some parts of <package> are already translated to Cogent and
some parts consist of the original C implementation, together resulting in a runnable system.

\subsection{Mapping Names from C to Cogent}
\label{design-names}

Names used in the C code shall be translated to similar names in the Cogent code, since they usually are descriptive for the
programmer. Ideally, the same names would be used. However, this is not possible, since Cogent differentiates between 
uppercase and lowercase names and uses them for different purposes. Therefore, atleast the names in the ``wrong'' case
need to be mapped.

Additionally, when the Cogent compiler translates a Cogent program to C code, it transfers the names without changes to
the names for the corresponding C items. To distinguish these names from the names in the original C code Gencot uses 
name mapping schemas which support mapping all kinds of names to a different name in Cogent. Generally, this is done by 
substituting a prefix of the name.

Often, a <package> uses one or more specific prefixes for its names, at least for names with external linkage. In this case
Gencot should be able to substitute these prefixes by other prefixes specific for the Cogent translation of the <package>.
Therefore, the Gencot name mapping is configurable. For every <package> a set of prefix mappings can be provided which is
used by Gencot. Two separate mappings are provided depending on whether the Cogent name must be uppercase or lowecase, so 
that the target prefixes can be specified in the correct case.

If a name must be mapped by Gencot which has neither of the prefixes in the provided mapping, it is mapped 
by prepending the prefix \code{cogent\_} or \code{Cogent\_}, depending on the target case.

\subsubsection{Name Kinds in C}

In C code the tags used for struct, union and enum declarations constitute an own namespace separate from the ``regular''
identifiers. These tags are mapped to Cogent type names by Gencot and could cause name conflicts with regular identifiers
mapped to Cogent type names. To avoid these conflicts Gencot maps tags by prepending the prefixes \code{Struct\_}, 
\code{Union\_}, or \code{Enum\_}, respectively, after the mapping described above. Since tags are always translated to Cogent 
type names, which must be uppercase, only one case variant is required.

Member names of C structs or unions are translated to Cogent record field names. Both in C and Cogent the scope of these
names is restricted to the surrounding structure. Therefore, Gencot normally does not map these names and uses them unmodified
in Cogent. However, since Cogent field names must be lowercase, Gencot applies the normal mapping for lowercase target 
names to all uppercase member names (which in practice are unusual in C). Moreover, Cogent field names must not begin
with an underscore, so these names are mapped as well, by prepending \code{cogent\_} (which results in two consecutive 
underscores).

C function parameter names are translated to Cogent variable names bound in the Cogent function body expression. Hence, both
in C and Cogent the scope of these names is restricted to the function body. They are treated by Gencot in the same way as 
member names and are only mapped if they are uppercase in C, which is very unusual in practice.

The remaining names in C are type names, function names, enum constant names, and names for global and local variables.
Additionally, there may be C constant names defined by preprocessor macro directives.
Local variables only occur in C function bodies which are not translated by Gencot. The other names are always mapped by
Gencot, irrespective whether they have the correct case or not.

\subsubsection{Names with internal linkage}

In C a name may have external or internal linkage. A name with internal linkage is local to the compilation unit in which it
is defined, a name with external linkage denotes the same item in all compilation units. Since the result of Gencot's 
translation is always a Cogent program which is translated to a single compilation unit by the Cogent compiler, names 
with internal linkage could cause conflicts when they origin in different C compilation units.

To avoid these conflicts, Gencot uses a name mapping scheme for names with internal linkage which is based on the 
compilation unit's file name. Names with internal linkage are mapped by substituting a prefix by the prefix \code{local\_x\_}
where \code{x} is the basename of the file which contains the definition, which is usually a file \code{x.c}. A name with
internal linkage can also be defined in an included file, but this is rarely done, because it introduces a separate object
in every compilation unit which includes the file. Gencot uses the same prefix in both cases, because if two files \code{x.c}
and \code{x.h} exist, the file \code{x.c} usually includes \code{x.h} and no different objects with the same name can be defined 
in both files. 

The default
is to substitute the empty prefix, i.e., prepend the target prefix. The mapping can be configured by specifying prefixes
to be substituted. This is motivated by the C programming practice to sometimes also use a common prefix for names 
with internal linkage which can be removed in this way.

Name conflicts could also occur for type names and tags (which have no linkage) defined in a \code{.h} file. 
This would be the case if different
C compilation units include individual \code{.h} files which use the same identifier for different purposes. However, most
C packages avoid this to make include files more robust. Gencot assumes that all identifiers without linkage 
are unique in the <package> and does not apply a file-specific renaming scheme. If a <package> does not satisfy this assumption
Gencot will generate several Cogent type definitions with the same name, which will be detected and signaled by the Cogent 
compiler and must be handled manually.

\subsubsection{Introducing Type Names}

There are cases where in Cogent a type name must be introduced for an unnamed C type (directly specified by a C type 
expression). Then the Cogent type name cannot be generated by mapping the C type name.

Unnamed C types are tagless struct/union/enum types and all derived types, i.e., array types, pointer types and 
function types. Basically, an unnamed C type could be mapped to a corresponding Cogent type expression. However,
this is not always possible or feasible.

Tagless enum types are always mapped to a primitive type in Cogent.

A tagless C struct could be mapped to a corresponding Cogent record type expression. However, the tagless struct
can be used in several declarators and several different types can be derived from it. In this case the Cogent record
expression would occur syntactically in several places, which is semantically correct, but may not be feasible for
large C structs. Therefore, Gencot introduces a Cogent type name for every tagless C struct and union.

Tagless structs and unions syntactically occur at only a single place in the source. The unique name is derived from 
that place, using the name of the corresponding source file and the line number where the struct/union begins
in that file (this is the line where the struct or enum keyword occurs).
The generated names have the forms
\begin{verbatim}
  <kind><lnr>_x_h
  <kind><lnr>_x_c
\end{verbatim}
where the suffix is constructed from the name \code{x.h} or \code{x.c} of the source file. \code{<kind>} is one of
\code{Struct} or \code{Union}, and \code{<lnr>} is the line number in the source file.

Derived C types are pointer types, function types, and array types. They always depend on a base type which in the C program 
must be defined before the base type is used.
In Cogent a similar dependency can only be expressed ba a generic type: the Cogent compiler takes care that in the generated C code
each instance
of a generic type is introduced after all its type arguments have been defined. Therefore Gencot translates every C derived
types to a generic Cogent type with its base type as its (single) type argument. 

For C pointer types a fixed set of generic types is used by Gencot (see Section~\ref{design-types-pointer}). Function types 
are translated to Cogent function type expressions. Only for array types generic type names are introduced which are specific
for the translated C program.

To be able to process every source file independently from all other source files, Gencot uses a schema which generates
a unique generic type name for every C array type. Derived types may syntactically occur at many places in a C program, so
it is not feasible to generate the name from a position in the source file.

For C array types its size (the number of elements) may be part of the derived type specification. Gencot uses two seperate generic
type names for every size for which an array type is used in the C program. If the size is specified by a literal the names
have the form
\begin{verbatim}
  CArr<size>
  UArr<size>
\end{verbatim}
where \code{<size>} is the literal size specification. If the size is specified by a single identifier the names have the form
\begin{verbatim}
  CArrX<size>X
  UArrX<size>X
\end{verbatim}
where \code{X} is a letter not occurring in the identifier.
In all other cases (also if no size is specified) the predefined names
\begin{verbatim}
  CArrXX
  UArrXX
\end{verbatim}
are used which may lead to name conflicts in Cogent and must be handled manually. 

Note that the generated Cogent type names could still cause conflicts with mapped type names. These conflicts can be
avoided if no configured mapping prefix starts with one of the \code{<kind>} strings
or the strings \code{"CArr"}, \code{"UArr"} used for mapping C array types, or any other predefined type used by Gencot.

\subsection{Modularization and Interfacing to C Parts}
\label{design-mdular}

Every C compilation unit produces
a set of global variables and a set of defined functions. Data of the same type may be used in
different compilation units, e.g. by passing it as parameter to an invoked function. In this case type compatibility in C is
only guaranteed by including the \code{.h} file with the type definition in both compilation units. In the compiled
units no type information is present any more. 

This organisation makes it possible to use different \code{.h} files in different compilation units. Even the type definitions
in the included files may be different, as long as they are binary compatible, i.e., have the same memory layout.

We exploit this organisation for an incremental translation from C to Cogent as follows. At every stage we replace some 
C compilation units by Cogent sources. All C data types used both in C units and in Cogent units are mapped to binary compatible
Cogent types. Compiling the Cogent sources again produces C code which together with the remaining C units are linked to
the target program. The C code resulting from Cogent compilation is completely separated from the code of the remaining C units,
common include files are only used for types which are abstract in Cogent, i.e., have no Cogent definition.

All interfacing between C compilation units is done by name. All names of C objects with external linkage can be referred
from other compilation units. This is possible for functions and for global variables. Interfacing from and to Cogent works
in the same way. 

\subsubsection{Interfacing to Functions}

Cogent functions always take a single parameter, the same is true for the C functions generated by the Cogent compiler. Hence
for interfacing from or to an arbitrary C function, wrapper functions are needed which convert between arbitrary many parameters
and a single structured parameter. These wrapper functions are implemented in C. 

The ``entry wrapper'' for invoking a Cogent function 
from C has the same name as the original C function, so it can be invoked transparently. Thus the Cogent implementation of
the function must have a different name so that it does not conflict with the name of the wrapper. This is guaranteed by the 
Gencot renaming scheme as described in Section~\ref{design-names}.

The Cogent implementation of a C function generated by Gencot is never polymorphic. This implies that the Cogent compiler
will always translate it to a single C function of the same name.

The ``exit wrapper'' for invoking a C function from Cogent invokes the C function by its original name, hence the wrapper
must have a different name. We use the same renaming scheme for these wrappers as for the defined Cogent functions.
This implies that every exit wrapper
can be replaced by a Cogent implementation without modifying the invocations in existing Cogent code. Note that for every
function either the exit wrapper or the Cogent implementation must be present, but not both, since they have the same name.

Note that if the C function has only one parameter, a wrapper is not required. For consistency reasons we generate and use
the wrappers also for these functions.

Cogent translates all function definitions to C definitions with internal linkage. To make them accessible the entry wrappers
must have external linkage. They are defined in an antiquoted Cogent (.ac) file which includes the complete code generated from 
Cogent, there all functions translated from Cogent are accessible from the entry wrappers. The exit wrappers are only invoked 
from code generated from Cogent. They are defined with internal linkage in an included antiquoted Cogent file.

\subsubsection{Interfacing to Global Variables}

Accessing an existing global C variable from Cogent is not possible in a direct way, since there are no ``abstract constants'' in 
Cogent. Access may either be implemented with the help of abstract functions which are implemented externally in additional C code and
access the global variable from there. Or it may be implemented by passing a pointer as (part of) a ``system state'' to the
Cogent function which performs the access.

Accessing a Cogent object definition from C is not possible, since the Cogent compiler does not generate a definition for them, it
simply substitutes all uses of the object name by the corresponding value. Hence, all global variable definitions need to remain 
in C code to be accessible there.

Since the way how global state is treated in a Cogent program is crucial for proving program properties, Gencot does not 
provide any automatic support for accessing global C variables from Cogent, this must always be implemented manually.

\subsubsection{Cogent Compilation Unit}

As of December 2018, Cogent does not support modularization by using separate compilation units. A Cogent program may be distributed
across several source files, however, these must be integrated on the source level by including them in a single compilation unit.
It would be possible to interface between several Cogent compilation units in the same way as we interface from C units to Cogent
units, however this will probably result in problems when generating proofs. 

Therefore Gencot always generates a single Cogent compilation unit for the <package>. 
At every intermediate stage of the incremental translation the package consists of one Cogent compilation unit 
together with all remaining original C compilation units and optionally additional C compilation units (e.g., for implementing 
Cogent abstract data types).

Conflicts for names with internal linkage originating in different C compilation units are avoided by Gencot's name mapping scheme
as described in Section~\ref{design-names}.

 

\subsection{Cogent Source File Structure}
\label{design-files}

Although the Cogent source is not structured on the level of compilation units, Gencot intends to reflect the structure of 
the C program at the level of Cogent source files. 

Note, that there are four kinds of include statements available in Cogent source files. One is the \code{include} statement which
is part of the Cogent language. When it is used to include the same file several times in the same Cogent compilation unit,
the file content is automatically inserted only once. However, the Cogent preprocessor is executed separately for every file included 
with this \code{include} statement, thus preprocessor macros defined in an included file are not available in all other files. For 
this reason it cannot be used to reflect the file structure of a C program.

The second kind is the Cogent preprocessor \code{\#include} directive, it works like the C preprocessor \code{\#include} directive
and is used by Gencot to integrate the separate Cogent source files. 
The third kind is the preprocessor \code{\#include} directive 
which can be used in antiquoted Cogent files where the Cogent \code{include} statement is not available. This is only possible 
if the included content is also an antiquoted Cogent file. The fourth kind
is the \code{\#include} directive of the C preprocessor which can be used in antiquoted Cogent files in the form 
\code{\$esc:(\#include ...)}. It is only executed when the C code generated by the Cogent compiler is processed by the C compiler.
Hence it can be used to include normal C code.

Gencot assumes the usual C source structure: Every \code{.c} file contains definitions with internal or external linkage.
Every \code{.h}
file contains preprocessor constant definitions, type definitions and function declarations. The constants and type definitions 
are usually mainly those which are needed for the function declarations. Every \code{.c} file includes the \code{.h} file which
declares the functions which are defined by the \code{.c} file to access the constants and type definitions. Additionally it may
include other \code{.h} files to be able to invoke the functions declared there. A \code{.h} file may include other \code{.h} files
to reuse their constants and type definitions in its own definitions and declarations.

\subsubsection{Cogent Source Files}

In Cogent a function which is defined may not be declared as an abstract function elsewhere in the program. If the types and constants,
needed for defining a set of functions, should be moved to a separate file, like in C, this file must not contain the 
function declarations for the defined functions. Declarations for functions defined in Cogent are not needed at all, since the Cogent 
source is a single compilation unit and functions can be invoked at any place in a Cogent program, independently whether their definition 
is statically before or after this place.

Therefore we map every C source file \code{x.c} to a Cogent source file \code{x-impl.cogent} containing definitions of the same 
functions. We map every C include file \code{x.h} to a Cogent source file \code{x-types.cogent} 
containing the corresponding constant and type definitions, but omitting any function declarations. The include relations among 
\code{.c} and \code{.h} files are directly transferred to \code{-impl.cogent} and \code{-types.cogent} using the Cogent preprocessor 
\code{\#include} directive. 

Although it is named \code{x-types.cogent}, the file also contains Cogent value definitions generated from C preprocessor
constant definitions and from enumeration constants (see below). It would be possible to put the value definitions in a 
separate file. However, then for other preprocessor macro definitions it would not be clear where to put them, since they could
be used both in constant and type definitions. They cannot be moved to a common file included by both at the beginning,
since their position relative to the places where the macros are used is relevant.

This file mapping implies that for every translated \code{.c} file all directly or indirectly included \code{.h} files must be 
translated as well.
Alternatively, instead of using a Cogent type definition for every C type in an included \code{.h} file, a Cogent abstract type
can be used. In this way further included \code{.h} files may become unnecessary and need not be translated. However, this
must be decided and realized manually. Gencot always generates default Cogent type definitions and the include directives for all
\code{-types.cogent} files.

\subsubsection{External Name References}

For external name references Gencot generates the information required for Cogent. 
All generated type and constant definitions are put in the file \code{<package>-exttypes.cogent}.

Additionally, for all origin files used by at least one external reference, an include directive is put in the file
\code{<package>-extincludes.c}, to make the information available on the C level.

\subsubsection{Wrapper Definition Files}

The entry wrappers for the functions defined with external linkage in \code{x.c} are implemented in antiquoted Cogent code and
put in the file \code{x-entry.ac}. 

The exit wrappers for invoking C functions from Cogent are only created for the actual
external references in a processing step for the whole <package>. They are implemented in antiquoted Cogent
and put in the file \code{<package>-exit.ac}.

\subsubsection{Abstract Functions as Interface to C Functions}

If a Cogent function in \code{x-impl.cogent} invokes a function which is externally referenced and not defined in another
file \code{y-impl.cogent}, this function must be declared as an abstract function in Cogent. These abstract function declarations
are only created for the actual
external references in a processing step for the whole <package>. They are put in the file \code{<package>-exit.cogent}.

The implementation for these functions is provided by the exit wrappers in \code{<package>-exit.ac}.

\subsubsection{Abstract Types as Interface to C Types}

A C type can be used in Cogent in two possible ways. Either it is defined as a Cogent abstract type, or it is defined by providing
a Cogent type expression as definition. In the second case the Cogent compiler will generate a C type definition with the
same name. The Cogent type expression must be chosen in a way, that this C type definition is binary compatible to the
original C type definition, i.e., it has the same memory layout. Since Gencot always maps a C type name to a different Cogent
type name, both C type names do not conflict.

For an abstract type the Cogent compiler does not generate any definition, it is intended to directly refer to the original
C type. Since the Cogent type name is different from the C type name, or has been generated if the C type has no name,
Gencot generates a C type definition mapping the Cogent type name to the C type name.
However, this definition may only be present for abstract types, for the other types it would conflict with the C type
definition generated by the Cogent compiler. 

Abstract type definitions referencing an existing C type may be generated in the files \code{x-types.cogent}, 
\code{x-impl.cogent}, and \code{<package>-exttypes.cogent}. 

For file \code{x-types.cogent} we put the corresponding 
type mapping definitions in file \code{x-abstypes.h}. To make the information required for the referenced original 
C type definitions available in the Cogent compilation unit the file \code{x.h} must be 
included there as well. Note that this is 
possible without conflicts, since the type names generated by the Cogent compiler for non-abstract types are always
mapped and thus different from all types in \code{x.h} or included files.

For file \code{x-impl.cogent} we put the corresponding
type mapping definitions in the file \code{x-abstypes.c}. However, to make the information required for the 
referenced original C type definition
available, it is not possible to include \code{x.c}, since the C function definitions would conflict with their
entry wrappers in the Cogent compilation unit. Instead, the file \code{x-globals.c} is used, which is described in the
next section.

For file \code{<package>-exttypes.cogent} we put the corresponding
type mapping definitions in the file \code{<package>-exttypes.c}. The information required for the referenced original C type
definitions is always available, since all origin files for external references are included in the Cogent compilation unit.

\subsubsection{Global Variables}

In C a compilation unit can define global variables. Gencot does not generate an access interface to these variables
from Cogent code. However, the variables must still be present in a compilation unit, since they may be accessed
from other C compilation units (if they have external linkage). 

Gencot assumes that global variables are only defined in \code{.c} files. For every file \code{x.c} Gencot generates
the file \code{x-globals.c} containing all toplevel object definitions with external linkage in \code{x.c}. For 
these definitions, some type and constant definitions may be required, so they must also be added to \code{x-globals.c}.
Since the required types may be defined in included \code{.h} files, these files must be included in \code{x-globals.c}.
Instead of tracking, what is required for the global variable definitions, Gencot simply generates \code{x-globals.c}
from \code{x.c} by removing all function definitions and all object definitions with internal linkage. Note, that
this approach also makes all type definitions available which are needed by \code{x-abstypes.c}.

Toplevel object definitions with internal linkage cannot be accessed from other C compilation units. They cannot be
accessed from Cogent code either, hence they are useless, they must be replaced manually by a Cogent solution for
managing the corresponding global state. 

However, to inform the Cogent programmer about the global variables defined in \code{x.c} and their types, Gencot 
generates corresponding Cogent value definitions for all toplevel object definitions with internal or external linkage. 
For each of them the initializer is transferred unmodified from C, no Cogent expression for the defined value is 
generated. Either the initializer is manually converted to a Cogent expression, or the value definition is replaced
by another solution. 

All Cogent value definitions for global variables in \code{x.c} are put in the file \code{x-globals.cogent}. Since it
is only intended as an information for the Cogent programmer it is \textit{not} included automatically by any generated
Cogent source file. For external references to global variables no information is generated.

\subsubsection{Abstract Data Types}

There may also be cases of C types where no corresponding Cogent type can be defined, in this case it must be mapped to an 
abstract data type T in Cogent, consisting of an abstract type together with abstract functions. Both are put in 
the file \code{abstract/T.cogent} which must be included manually by all \code{x-types.cogent} where it is used. The types and 
functions of T must be implemented in additional C code. In contrast to the abstract functions defined in \code{<package>-exit.cogent},
there are no existing C files where these functions are implemented. The implementations are provided as antiquoted Cogent 
code in the file \code{abstract/T.ac}. If T is generic, the additional file \code{abstract/T.ah} is required for 
implementing the types, otherwise they are implemented in \code{abstract/T.h}. 

Gencot does not provide any support for using abstract data types, they must be managed manually according to the following
proposed schema. All related files should be stored in the subdirectory \code{abstract}.
An abstract data type T is defined in the following files:
\begin{description}
\item[\code{T.ac}] Antiquoted Cogent definitions of all functions of T. 
\item[\code{T.ah}] Antiquoted Cogent definition for T if T is generic.
\item[\code{T.h}] Antiquoted Cogent definitions of all non-generic types of T.
\end{description}
Using the flag \code{--infer-c-types} the Cogent compiler generates from \code{T.ah} files \code{T\_t1...tn.h} for all 
instantiations of T with type arguments t1...tn used in the Cogent code.

\subsubsection{File Summary}

Summarizing, Gencot uses the following kinds of Cogent source files for existing C source files \code{x.c} and \code{x.h}:
\begin{description}
\item[\code{x-impl.cogent}] Implementation of all functions defined in \code{x.c}. For each file \code{y.h} included by
  \code{x.c} the file \code{y-types.cogent} is included.
\item[\code{x-globals.cogent}] Value definitions for all objects defined in \code{x.c}. No files are included, the file is not
  included by any other file.
\item[\code{x-types.cogent}] Constant and type definitions for all constants and types defined in \code{x.h}. 
  If possible, for every C type definition a binary compatible Cogent type 
  definition is generated by Gencot. Otherwise an abstract type definition is used. Includes
  all \code{y-types.cogent} for which \code{x.h} includes \code{y.h}.
\item[\code{x-entry.ac}] Antiquoted Cogent definitions of entry wrapper functions for all function definitions with external linkage
  defined in \code{x.c}.
\item[\code{x-abstypes.h}] C definitions for abstract Cogent types defined in \code{x-types.cogent} used to reference existing C types.
\item[\code{x-abstypes.c}] C definitions for abstract Cogent types defined in \code{x-impl.cogent} used to reference existing C types.
\item[\code{x-globals.c}] Content of \code{x.c} with all function definitions removed.
\end{description}

For the Cogent compilation unit the following common files are used:
\begin{description}
\item[\code{<package>-exttypes.cogent}] Type and constant definitions for all external type and constant references.
\item[\code{<package>-exit.cogent}] Abstract function definitions for all external function references.
\item[\code{<package>-exit.ac}] Exit wrapper definitions for all external function references.
\item[\code{<package>-exttypes.c}] C type definitions for abstract types defined in \code{<package>-exttypes.cogent}.
\item[\code{<package>-extincludes.c}] Include directives for the origin files of all external references.
\end{description}

\subsubsection{Main Files}

To put everything together we use the files \code{<package>.cogent} and \code{<package>.ac}. The former includes all 
existing \code{x-impl.cogent} files and the files \code{<package>-exttypes.cogent} and \code{<package>-exit.cogent}.
It is the file processed by the Cogent compiler which translates it to files \code{<package>.c} 
and \code{<package>.h} where \code{<package>.c} includes \code{<package>.h}. 

The file \code{<package>.ac} includes all existing files 
\code{x-entry.ac}, and the files \code{<package>-exit.ac} and \code{<package>.c} and is processed by the Cogent compiler through the 
\code{--infer-c-funcs} flag. The resulting file is \code{<package>\_pp\_inferred.c} which is the C compilation unit for 
all parts of <package> already translated to Cogent. All existing files \code{x-abstypes.h} and \code{x-abstypes.c} 
and the files \code{<package>-exttypes.c} and \code{<package>-extincludes.c} are 
\code{\$esc}-included in \code{<package>.ac}, thus the corresponding normal includes for them are present in 
\code{<package>\_pp\_inferred.c}.
For all existing files \code{x-impl.cogent} the corresponding file \code{x-globals.c} is \code{\$esc}-included in 
\code{<package>.ac}, to make all global variables with external linkage and all type definitions in \code{x.c} 
a part of \code{<package>\_pp\_inferred.c}.

Every abstract type T yields an additional separate C compilation unit \code{T\_pp\_inferred.c}. 

The content of \code{abstract/T.h} and all \code{abstract/T\_t1...tn.h} is required in the compilation unit for T and in 
that for \code{<package>.c}. The Cogent compiler automatically generates includes for all \code{abstract/T\_t1...tn.h} in 
\code{<package>.h}, 
thus they are available in \code{<package>\_pp\_inferred.c}. By manually \code{\$esc}-including \code{<package>.h} in every 
\code{abstract/T.ac} they are made available there as well. In the same way \code{abstract/T.h} can be \code{\$esc}-included
in \code{abstract/T.ac}. To make it available in the \code{<package>.c} unit Gencot also \code{\$esc}-includes all 
existing \code{abstract/T.h} files in \code{<package>.ac}.

 


\section{Processing Comments}
\label{design-comments}

The Cogent source generated by Gencot is intended for further manual modification. Finally, it should be used as a 
replacement for the original C source. Hence, also the documentation should be transferred from the C source to
the Cogent source.

Gencot uses the following heuristics for selecting comments to be transferred: All comments at the beginning or end 
of a line and all comments on one or more full lines are transferred. Comments embedded in C code in a single line
are assumed to document issues specific to the C code and are discarded.

\subsection{Identifying and Translating Comments}

Gencot processes C block comments of the form \code{/* ... */} possibly spanning several lines, and C line comments
of the form \code{// ...} ending at the end of the same line.

Identifying C comments is rather complex, since the comment start sequences \code{/*} and \code{//} may also occur
in C code in string literals and character constants and in other comments. 

Comments are translated to Cogent comments. Every C block comment is translated to a Cogent block comment of the form
\code{\{- ... -\}}, every C line comment is translated to a Cogent line comment of the form \code{-- ...}. Only the 
start and end sequences of identified comments are translated, all other occurrences of comment start and end sequences
are left unchanged.

If a Cogent block comment end sequence \code{-\}} occurs in a C block comment, the translated Cogent block comment
will end prematurely. This will normally cause syntax errors in Cogent and must be handled manually. It is not
detected by Gencot.

\subsection{Comment Units}

Gencot assembles sequences of transferrable comments which are only separated by whitespace together to comment units
as follows. All comments starting in the same line after the last existing source code are concatenated to become 
one unit. Such units are called ``after-units''. All comments starting in a separate line with no existing source code 
or before all existing source code in that line are concatenated to become one unit. Such units are called ``before-units''. 

Additionally, all remaining comments at the end of a file after the last after-unit are concatenated to become the 
``end-unit''. At the beginning of a file there is often a schematic copyright comment. To allow for a specific treatment
a configurable number of comments at the beginning of a file are concatenated to become the ``begin-unit''. The default
number of comments in the begin-unit is 1.

As a result, every transferrable comment is either part of a comment unit and every comment unit
can be uniquely identified by its kind and by the source file line numbers where it starts and where it ends.

Heuristically, a before-unit is assumed to document the code after it, whereas an after-unit is assumed to document
the code before it. Based on this heuristics, comment units are associated to code parts. A begin-unit and an end-unit
is assumed to document the whole file and is not associated with a code part.

\subsection{Relating Comment Units to Documented Code}
\label{design-comments-relate}

Basically, Gencot translates source code parts to target code parts. Source code parts may consist of several lines,
so there may be several before- and after-units associated with them: The before-unit of the first line, the after-unit
of the last line and possibly inner units. Target code parts may also consist of several lines. The before-unit of
the first line is put before the target code part, the after-unit of the last line is put after the target code part.

If there is no inner structure in the source code part which can be mapped to an inner structure of the target code
part, there are no straightforward ways where to put the inner comment units. They could be discarded or they could be
collected and inserted at the beginning or end of the target code part. If they are collected no information is lost 
and irrelevant comments can be removed manually. However, in well structured C code inner comment units are rare,
hence Gencot discards them for simplicity and assumes, that this way no relevant information will be lost.

If the source code part has an inner structure units can be associated with subparts and transferred to subparts of the
target code part. Gencot uses the following general model for a structured source code part: It may have one or more
embedded subparts, which may be structured in a similar way. Every subpart has a first line where it begins and a last 
line where it ends. Before and after a subpart
there may be lines which contain code belonging to the surrounding part. Subparts may overlap, then the last line of 
the previous subpart is also the first line of the next subpart. Subparts may overlap with the surrounding part, then 
the first or last line of the subpart contains also code from the surrounding part.

For a structured source code part Gencot generates a target code part for the main part and a target code part for every 
subpart. The subpart targets may be embedded in the main part target or not. If they are embedded they may be reordered.

The inner comment units of a structured source code part can now be classified and associated. Every such unit is either
an inner unit of the main part, a before-unit of the first line of a subpart if that does not overlap, an inner 
unit of a subpart, or an after-unit of the last line of a subpart, if that does not overlap. The units associated with a subpart
are transferred to the generated target according to the same rules as for the main part. 

If there is no main source code before the first subpart (e.g., a declaration starting with a struct definition), the
before-group of the first line is nevertheless associated with the main part and not with the first subpart. The after-group
at the end of a part is treated in the analogous way.

Inner units of the main part may be before the first subpart, between two subparts, or after the last subpart. Following 
the same argument as for inner units of unstructured source code parts, Gencot simply discards all these inner units.

As a result, for every source code part atmost the before-unit of the first line and the after-unit of the last line 
is transferred to the target part. If the source code part is structured the same property holds for every embedded 
subpart. If no target code is generated for the main part but for subparts, the before-unit of the main part immediately
precedes the before-unit of the first subpart, if both exist, and analogously for the after-units.

Target code for a part may be generated in several separated places. If no code
is generated for the main part, it must be defined to which group of subpart targets the comments associated with the
main part is associated.

\subsection{Declaration Comments}
\label{design-comments-decl}

Since toplevel declarations are not translated to a target code part in Cogent, all comments associated with them would
be lost. However, often the API documentation of a function or global variable is associated with its declaration instead of the
definition. 

Therefore Gencot treats before-units associated with a toplevel declaration in a specific way and 
moves them to the target code part generated for the corresponding definition. There they are placed between
the comments preceding the definition and the definition itself. 

Gencot assumes, that only one declaration exists for each definition. If there are more than one declarations 
in the C code the comment associated with one of them is moved to the definition, the comments associated with
the other declarations are lost. 

Only before-units are handled this way, due to a technical problem with the C parser used. For declarations it does not provide 
the end position in a safe manner. For the intended application this is not a problem since API documentations are
usually placed before the declaration and not after it.


\section{Processing Constants Defined as Preprocessor Macros}
\label{design-const}
Often a C source file contains constant definitions of the form
\begin{verbatim}
  #define CONST1 123
\end{verbatim}
The C preprocessor substitutes every occurrence of the identifier \code{CONST1} in every C code after the definition 
by the value 123. This is a special application of the C preprocessor macro feature.

Constant names defined in this way may have arbitrary C constants as their value. Gencot only handles integer,
character, and string constants, floating constant are not supported since they are not supported by Cogent.

\subsection{Processing Direct Integer Constant Definitions}

Constant definitions of this form could be used directly in Cogent, since they are also supported by the Cogent preprocessor.
By transferring the constant definitions to the corresponding file \code{x-types.cogent} the identifiers are available
in every Cogent file including \code{x-types.cogent}. 

However, for generating proofs it should be better to use Cogent value definitions instead of having unrelated 
literals spread across the code. The Cogent value definition corresponding to the constant definition above can either 
be written in the form
\begin{verbatim}
  #define CONST1 123
  const1: U8
  const1 = CONST1
\end{verbatim}
preserving the original constant definition or directly in the shorter form
\begin{verbatim}
  const1: U8
  const1 = 123
\end{verbatim}
Since the preprocessor name \code{CONST1} may also be used in \code{\#if} directives, we use the first form. A typical pattern 
for defining a default value is
\begin{verbatim}
  #if !defined(CONST1)
  #define CONST1 123
  #endif
\end{verbatim}
This will only work if the preprocessor name is retained in the Cogent preprocessor code.

If different C compilation units use the same preprocessor name for different constants, the generated Cogent value definitions
will conflict. This will be detected and signaled by the Cogent compiler. Gencot does not apply any renaming to prevent these
conflicts.

For the Cogent value definition the type must be determined. It may either be the smallest primitive type covering the value 
or it may always be U32 and, if needed, U64. The former requires to insert upcasts whenever the value is used for a different 
type. The latter avoids the upcast in most cases, however, if the value should be used for a U8 or U16 that is not possible 
since there is no downcast in Cogent. Therefore the first approach is used.

Constant definitions are also used to define negative constants sometimes used for error codes. Typically they are used for 
type \code{int}, for example in function results. Here, the type cannot be determined in the way as for positive values, since the 
upcast does not preserve negative values. Therefore we always use type U32 for negative values, which corresponds to type 
\code{int}. This may be wrong, then a better choice must be used manually for the specific case.

Negative values are specified as negative integer literals such as -42. To be used in 
Cogent as a value of type U32 the literal must be converted to an unsigned literal using 2-complement by: 
\code{complement(42 - 1)}.
Since Cogent value definitions are translated to C by substituting the \textit{expression} for every use, it should be as 
simple as possible, such as \code{complement 41} or even \code{0xFFFFFFD6} which is \code{4294967254} in decimal notation.

As described in Section~\ref{design-names}, names for preprocessor defined constants are always mapped to a different
name for the use in Cogent. This is not strictly necessary, if a preprocessor name is lowercase. By convention, C preprocessor 
constant definitions use uppercase identifiers, thus they normally must be mapped anyways.

For comment processing, every preprocessor constant definition is treated as an unstructured source code part.

\subsection{Processing Direct Character and String Constant Definitions}

A character constant definition has the form
\begin{verbatim}
  #define CONST1 'x'
\end{verbatim}
It is translated to a Cogent value definition similar as for integer constants. As type always \code{U8}
is used, the constant is transferred literally.

A string constant definition has the form
\begin{verbatim}
  #define CONST1 "abc"
\end{verbatim}
It is translated to a Cogent value definition similar as for integer constants. As type always \code{String}
is used.

In C it is also possible to specify a string constant by a sequence of string literals, which will be concatenated.
A corresponding string constant definition has the form
\begin{verbatim}
  #define CONST1 "abc" "def"
\end{verbatim}
Since there is no string concatenation operator in Cogent, the concatenation is performed by Gencot and
a single string literal is used in the Cogent value definition.

\subsection{Processing Indirect Constant Definitions}

A constant definition may also reference a previously defined constant in the form 
\begin{verbatim}
  #define CONST2 CONST1
\end{verbatim}

In this case the Cogent constant definition uses the same type as that for \code{CONST1} and
also references the defined Cogent constant and has the form
\begin{verbatim}
  #define CONST2 CONST1
  const2: U8
  const2 = const1
\end{verbatim}

\subsection{Processing Expression Constant Definitions}

A constant definition in C may also specify its value by an expression. In this case the C preprocessor will replace
the constant upon every occurrence by the expression, every expression according to the C syntax is admissible. 

In this case Gencot also generates a Cogent value definition and transfers the expression. Gencot does not evaluate
or translate the expression, however, it maps all contained names of other preprocessor defined constants to their
Cogent form, so that they refer the corresponding Cogent value name. As type for an expression Gencot always assumes
\code{int}, i.e. \code{U32} in Cogent.

If the expression is of type \code{int} and only uses operators which also exist in Cogent, positive integer constants 
and preprocessor defined constant names, the resulting expression will be a valid Cogent expression. In all other cases
the Cogent compiler will probably detect a syntax error, these cases must be handled manually.

\subsection{External Constant References}

If the constant \code{CONST1} is an external reference in the sense of Section~\ref{design-modular-extref}, a corresponding
Cogent constant definition is generated in the file \code{<package>-exttypes.cogent}. It has the same form
\begin{verbatim}
  #define CONST1 123
  const1: U8
  const1 = CONST1
\end{verbatim}
as for a non-external reference. Thus we define the original preprocessor constant name \code{CONST1} here, although
it is already defined in the external origin file. The reason for this approach is that the define directive here 
is intended to be processed by the Cogent preprocessor. Therefore we cannot include the origin file to make the name
available, since that would also include the C code in the origin file.

If the external definition is indirect, the value used in the define directive is always resolved to the final
literal or to an existing external reference. This is done for determining the Cogent type for the constant 
and avoids introducing unnecessary intermediate constant names.



\section{Processing Other Preprocessor Directives}
\label{design-preprocessor}

A preprocessor directive always occupies a single logical line, which may consist of several actual lines where 
intermediate line ends are backslash-escaped. No C code can be in a logical line of a preprocessor directive.
However, comments may occur before or after the directive in the same logical line. Therefore, every preprocessor 
directive may have an associated comment before-unit and after-unit, which are transferred as described in 
Section~\ref{design-comments}. Comments embedded in a preprocessor directive are discarded.

We differentiate the following preprocessor directive units:
\begin{itemize}
\item Preprocessor constant definitions
\item all other macro definitions and \code{\#undef} directives,
\item conditional directives (\code{\#if, \#ifdef, \#ifndef, \#else, \#elif, \#endif}),
\item include directives (quoted or system)
\item all other directives, like \code{\#error} and \code{\#warning}
\end{itemize}

To identify constant definitions we resolve all macro definitions as long as they are defined by another single
macro name. If the result is a C integer constant (possibly negative) or a C character constant the macro is assumed
to be a constant definition. All constant definitions are processed
as described in Section~\ref{design-const}.

For comment processing every preprocessor directive is treated as an unstructured source code part.

\subsection{Configurations}
\label{design-preprocessor-config}

Conditional directives are often used in C code to support different configurations of the code. Every configuration
is defined by a combination of preprocessor macro definitions. Using conditional directives in the code, whenever the
code is processed, only the code for one configuration is selected by the preprocessor.

In Gencot the idea is to process all configurations at the same time. This is done by removing the conditional 
directives from the code, process the code, and re-insert the conditional directives into the generated Cogent code.

Only directives which belong to the <package> are handled this way, i.e., only directives which occur in source
files belonging to the <package>. For directives in other included files, in particular in the system include files,
this would not be adequate. First, normally there is no generated target code where they could be re-inserted.
Second, configurations normally do not apply to the system include files.

However, it may be the case that Gencot cannot process two configurations at the same time, because they contain
conflicting information needed by Gencot. An example would be different definitions for the same type which
shall be translated from C to a Cogent type by Gencot.

For this reason Gencot supports a list of conditions for which the corresponding conditional directives are not 
removed and thus only one configuration is processed at the same time. Then Gencot has to be run separately for every
such configuration and the results must be merged manually.

Conditional directives which are handled this way are still re-inserted in the generated target code. This
usually results in all branches being empty but the branches which correspond to the processed configuration.
Thus the branches in the results from separate processing of different configurations can easily be merged manually
or with the help of tools like diff and patch.

Retaining Conditional directives for certain configurations in the processed code makes only sense if the corresponding
macro definitions which are tested in the directives are retained as well. Therefore also define directives can be
retained. The approach in Gencot is to specify a list of regular expressions in the format used by awk. All directives
which match one of these regular expressions are retained in the code to be interpreted before processing the code.
The list is called the ``Gencot directive retainment list'' and may be specified for every invocation of Gencot.

\subsection{Conditional Directives}

Conditional directives are used to suppress some source code according to specified conditions. Gencot aims to
carry over the same suppression to the generated code.

\subsubsection{Associating Conditional Directives to Target Code}

Conditional directives form a hierarchical block structure consisting of ``sections'' and ``groups''. A group
consists of a conditional directive followed by other code. Depending on the directive there are ``if-groups''
(directives \code{\#if, \#ifdef, \#ifndef}), ``elif-groups'' (directive \code{\#elif}), and ``else-groups''
(directive \code{\#else}). A section consists of an if-group, an optional sequence of elif-groups, an optional
else-group, and an \code{\#endif} directive. A group may contain one or more sections in the code after the
leading directive.

Basically, Gencot transfers the structure of conditional directives to the target code. Whenever a source code
part belongs to a group, the generated target code parts are put in the corresponding group. 

This only works if the source code part structure is compatible with the conditional directive structure.
In C code, theoretically, both structures need not be related. Gencot assumes the following restrictions:
Every source code part which overlaps with a section is either completely enclosed in a group or
contains the whole section. It may not span several groups or contain only a part of the section. If a
source code part is structured, contained sections may only overlap with subparts, not with code belonging
to the part itself. 

Based on this assumption, Gencot transfers conditional directives as follows. If a section is contained in an 
unstructured source code part, its directives are discarded. If a section is contained in a structured source
code part, its directives are transferred to the target code part. Toplevel sections which are not contained in
a source code part are transferred to toplevel. Generated target code parts are put in the same group which
contained the corresponding source code part.

It may be the case that for a structured source code part a subpart target must be placed separated from the
target of the structured part. An example is a struct specifier used in a member declaration. In Cogent, the 
type definition generated for the struct specifier must be on toplevel and thus separate from the generated member.
In these cases the condition directive structure must be partly duplicated at the position of the subpart target,
so that it can be placed in the corresponding group there.

Since the target code is generated without presence of the conditional directives structure, they must be 
transferred afterwards. This is done using the same markers \code{\#ORIGIN} and \code{\#ENDORIG} as for the
comments. Since every conditional directive occupies a whole line, the contents of every group consists of
a sequence of lines not overlapping with other groups on the same level. If every target code part is marked 
with the begin and end line of the corresponding source code part, the corresponding group can always be
determined from the markers.

The conditional directives are transferred literally without any changes, except discarding embedded comments. 
For every directive inserted in the target code origin markers are added, so that its associated comment before-
and after-unit will be transferred as well, if present.

\subsection{Macro Definitions}
\label{design-preprocessor-macros}

Preprocessor macros are defined in a definition, which specifies the macro name and the replacement text. Optionally,
a macro may have parameters. After the definition a macro can be used any number of times in ``macro calls''.
A macro call for a parameterless macro has the form of a single identifier (the macro name). A macro call for
a macro with parameters has the form of a C function call: the macro name is followed by actual parameter values
in parenthesis separated by commas. However, the actual parameter values need not be C expressions, they can be 
arbitrary text, thus the macro call need not be a syntactically correct C function call.

Macro calls can occur in C code or in other preprocessor directives (macro definitions, conditional directives, 
and include directives). All macro calls occuring in C code must result in valid C code after full expansion 
by the preprocessor.

\subsubsection{General Approach}

Gencot tries to preserve macros in the translated target code instead of expanding them. In general this requires
to implement two processing aspects: translating the macro definitions and translating the macro calls. Since Gencot
processes the C code separately from the preprocessor directives, macro call processing can be further distinguished
according to macro calls in C code and in preprocessor directives.

Gencot processes the C code by parsing it with a C parser. This implies that macro calls in C code must correspond
to valid C syntax, or they must be preprocessed to convert them to valid C syntax. Note that it is always possible
to do so by fully expanding the macro definition, however, then the macro calls cannot be preserved.

There are several special cases for the general approach of macro processing:
\begin{itemize}
\item if calls for a macro never occur in C code they need not be converted to valid C syntax and they need
only be processed in preprocessor directives. This is typically the case for ``flags'', i.e., macros with
an empty replacement text which are used as boolean flags in conditional directives.
\item if for a parameterless macro all calls in C code occur at positions where an identifier is expected,
the calls need not be converted to valid C syntax and can be processed in the C code. This approach applies
to the ``preprocessor defined constants'' as described in Section~\ref{design-const}. 
\item if for a macro with parameters all calls in C code are syntactically valid C function calls and
always occur at positions where a C function call is expected, the calls need not be converted to valid C 
syntax and can be processed in the C code.
\item if a macro need not be preserved by Gencot, its calls can be converted to valid C code by fully expanding
them. Then the calls are not present anymore and the calls and the definition need not be processed at all.
This approach is used for conflicting configurations as described in Section~\ref{design-preprocessor-config}.
\end{itemize}

When Gencot preserves a macro, there are several ways how to translate the macro definition and the macro calls.
An apparent way is to use again a macro in the target code. Then the macro definition is translated by translating
the replacement text and optionally also the macro name. If the macro name is translated then also all macro calls
must be translated, otherwise macro calls need only be translated if the macro has parameters and the actual 
parameter values must be translated.

How the macro replacement text is translated depends on the places where the macro is used. If it is only used 
in preprocessor directives, usually no translation is required. If it is used in C code parts which are translated to
Cogent code, the replacement text must also be translated to Cogent. If it is used in C function bodies it must
be translated in the same way as C function bodies, i.e., only the identifiers must be mapped and calls of other
macros must be processed.

A macro may also be translated to a target code construct. Then the macro definition is typically translated to 
a target code definition (such as a type definition or a function definition) and the macro calls are translated
to usages of that definition. This approach is used for the ``preprocessor defined constants'' as described in 
Section~\ref{design-const}: the macro definitions are translated to Cogent constant definitions and the macro 
calls occurring in C code are translated to the corresponding Cogent constant names. Additionally, the original
macro definitions are retained and used for all macro calls occurring in conditional directives, which are not
translated. Macro calls in the replacement text of other preprocessor constant definitions are translated to 
the corresponding Cogent constant names.

\subsubsection{Flag Translation}

A flag is a parameterless macro with an empty replacement text. Its only use is in the conditions of 
conditional preprocessor directives, hence macro calls for flags only occur in preprocessor directives.

Gencot translates flags by directly transferring them to the target code. Neither their macro definitions 
nor their macro calls are further processed by Gencot.

\subsubsection{Manual Macro Translation}

Most of the aspects of macro processing cannot be determined and handled automatically by Gencot. Therefore a
general approach is supported by Gencot where macro processing is specified manually for specific macros used in the
translated C program package.

The manual specification consists of the following parts:
\begin{itemize}
\item A specification of all macros which shall not be preserved. This specification is included in the 
Gencot directive retainment list, as described in 
Section~\ref{design-preprocessor-config}. Retained macros are processed by the C preprocessor and thus 
fully expanded before the C code is processed by Gencot.
\item A specification of all parameterless macros which shall not be processed as preprocessor defined
constants or flags. This specification consists of a list of macro names, it is called the ``Gencot manual macro list''.
For all macros in this list a manual translation must be specified. Macros with parameters are never 
processed automatically, for them a manual translation must always be specified if they shall be preserved.
\item For all manually processed macros for which macro calls may occur in C code a conversion to valid C 
code may be specified.
This specification is itself a macro definition for the same macro, where the replacement text must be valid
C code for all positions where macro calls occur. A set of such macro definitions is called a ``Gencot macro 
call conversion''. It is applied to all macro calls and the result is fed to 
the Gencot C code translation and is processed in the usual way, no further manual specification for the macro call
translation can be provided. Since the conversion is applied after all preprocessor directives have been
removed, it has no effect on macro calls in preprocessor directives. 
\item For all manually translated macros a translation of the macro definition may be specified. It has the form 
of arbitrary text marked with a specification of the position of the original macro definition in its source file.
According to this position it is inserted in the corresponding target code file. A collection of such macro 
definition translations is always specifi for a single source file and is called the ``Gencot macro translation''
for the source file.
\end{itemize}
All four parts may be specified as additional input upon every invocation of Gencot.

Macro definitions are always translated at the position where they occurred in the source file.
If the definition occurs in a file \code{x.h} it is transferred to file \code{x-types.cogent} to a corresponding position,
if it occurs in a file \code{x.c} it is transferred to file \code{x-impl.cogent} to a corresponding position.

This implies that translated macro definitions are not available in the file \code{x-globals.cogent} and in the files with
antiquoted Cogent code. If they are used there (which mainly is the case if macro calls occur in a conditional
preprocessor directive which is transferred there), a manual solution is required.

If different C compilation units use the same name for different macros, conflicts are caused in the integrated Cogent
source. These conflicts are not detected by the Cogent compiler. A renaming scheme based on the name of the file 
containing the macro definition would not be safe either, since it breaks situations where a macro is deliberately
redefined in another file. Therefore, Gencot provides no support for macro name conflicts, they must be detected and
handled manually.

\subsubsection{External Macro References}

Whenever a macro call occurs in a source file, it may reference a macro definition which is external in 
the sense of Section~\ref{design-modular-extref}. For such external references the (translated) definition 
must be made available in the Cogent compilation unit.

For all preprocessor defined constants (i.e. parameterless nonempty macros not listed in the Gencot manual 
macro list) Gencot adds the translated macro definition to the file \code{<package>-exttypes.cogent}. For
manually translated macros a separate Gencot macro translation must be specified for external macro definitions.
For them the position specification is omitted, they are simply appended to the file 
\code{<package>-exttypes.cogent}. If this is not sufficient, because macro calls already occur in 
\code{<package>-exttypes.cogent}, they must be inserted manually at the required position.

To avoid introducing additional external references, in the macro replacement text for preprocessor defined 
constants all macro calls are resolved to existing external reference names or until they are fully resolved.
Manually translated macro definitions should handle external macro calls in a similar way.

\subsection{Include Directives}

In C there are two forms of include directives: quoted includes of the form
\begin{verbatim}
  #include "x.h"
\end{verbatim}
and system includes of the form
\begin{verbatim}
  #include <x.h>
\end{verbatim}
Files included by system includes are assumed to be always external to the translated <package>, therefore system
include directives are discarded in the Cogent code. The information required by external references from system 
includes is always fully contained in the file \code{<package>-exttypes.cogent}.

\subsubsection{Translating Quoted Include Directives}

Quoted include directives for a file \code{x.h} which belongs to the Cogent compilation unit are always translated 
to the corresponding Cogent preprocessor include directive
\begin{verbatim}
  #include "x-types.h"
\end{verbatim}
If the original include directive occurs in file \code{y.c} the translated directive is put into the file 
\code{y-impl.cogent}. If the original include directive occurs in file \code{y.h} the translated directive is put into the file
\code{y-types.h}. 

Otherwise the quoted include includes an external file and is discarded in the Cogent source file for the same reason
why the system includes are discarded.

\subsection{Other Directives}

All other preprocessor directives are discarded. Gencot displays a message for every discarded directive.

 


\section{Parsing and Processing C Code}
\label{design-ccode}

After comments and preprocessor directives have been removed from a C source file, it is parsed and
the C language constructs are processed to yield Cogent language constructs. 

\subsection{Including Files for C Code Processing}

When Gencot processes the C code in a source file, it may need access to information in files included by
the source file. An example is a type definition for a type name used in the source file. Hence for
C code processing Gencot always reads the source file together with all included files. Since in the source
file all comments and preprocessor directives have been  removed, they must also be removed in the included files
which belong to the same <package>. Gencot assumes these are the files included by a quoted include directive.

There are two possible approaches how this can be done.

The first approach is to use the C preprocessor which is invoked by language-c before parsing. It processes
include directives as usual, hence it would be sufficient to leave the include directives in the source code
when removing the other preprocessor directives. However, this would include the original \code{.h} files and
\textit{process} all preprocessor directives there, instead of removing them. The directives have to be 
removed from all included files in <package> in a separate step, then the include directives have to be modified to include
the results of that step instead of the original files.

The second approach is to process all quoted include directives before the other directives are removed, 
resulting for every source
file in a single file containing all included information and all other preprocessor directives. Then the directives
are removed from this file with the exception of the system include directives. When the result is fed to the 
language-c parser its preprocessor will expand the system includes as usual, thus providing the complete 
information needed for processing the C code.

Gencot uses the second approach, since this way it can process every source file independently from previous steps for 
other source files and it needs no intermediate files which must be added to the include file path of the language-c
preprocessor.

In the included original files the comments are still present and must be removed as well. This could be done 
by the language-c preprocessor immediately before parsing. However, it is easier to remove the preprocessor
directives when the comments are not present anymore. Therefore, Gencot removes the comments immediately after
processing the quoted include directives.

\subsection{Processing the C Code}



 


\section{Mapping C Datatypes to Cogent Types}
\label{design-types}

Here we define rules how to map common C types to binary compatible Cogent types. Since the usefulness of a mapping
also depends on the way how values of the type are processed in the C program, the resulting types may require manual 
modification.

\subsection{Numerical Types}
\label{design-types-prim}

The Cogent primitive types are mapped to C types in \code{cogent/lib/cogent-defns.h} which is included by the Cogent compiler
in every generated C file with \code{\#include <cogent-defns.h>}. The mappings are: 
\begin{verbatim}
  U8 -> unsigned char
  U16 -> unsigned short 
  U32 -> unsigned int
  U64 -> unsigned long long
  Bool -> struct bool_t { unsigned char boolean }
  String -> char*
\end{verbatim}
The inverse mapping can directly be used for the unsigned C types. For the corresponding signed C types to be binary
compatible, the same mapping is used. Differences only occur when negative values are actually used, this must be handled by using specific functions for numerical operations in Cogent.

In C all primitive types are numeric and are mapped by Gencot to a primitive type in Cogent. Note that in C the representation 
of numeric types may depend on the C version and target system architecture. However, the main goal of Gencot is only to generate
Cogent types which are, after translation to C, binary compatible with the original C types. Hence it is sufficient for the numerical 
types to simply invert the Mapping used by the Cogent compiler.

Together we have the following mappings:
\begin{verbatim}
char, unsigned/signed char -> U8
short, unsigned/signed short -> U16
int, unsigned/signed int -> U32
long int, unsigned/signed long int -> U64
long long int, unsigned/signed long long int -> U64
\end{verbatim}

The only mapping not determined by the Cogent compiler mapping is that for \code{long int}. For the gcc C version 
it depends on the architecture and is either the same as \code{int} (on 32 bit systems) or \code{lon long int}
(on 64 bit systems). Gencot assumes a 64 bit system and maps it like \code{long long int}.

\subsection{Enumeration Types}
\label{design-types-enum}

A C enumeration type of the form \code{enum e} is a subset of type \code{int} and declares enumeration 
constants which have type \code{int}. According to the C99 standard, an enumeration type may be implemented
by type \code{char} or any integer type large enough to hold all its enumeration constants.

A natural mapping for C enumeration types would be Cogent variant types. However, the C implementation
of a Cogent variant type is never binary compatible with an integer type (see below). 

Therefore C enumeration types must be mapped to a primitive integer type in Cogent. Depending on the C
implementation, this may always be type \code{U32} or it may depend on the value of the last enumeration
constant and be either \code{U8}, \code{U16}, \code{U32}, or maybe even \code{U64}. Under Linux, both cc
and gcc always use type \code{int}, independent of the value of the last enumeration constant. 
Therefore Gencot always maps enumeration types to Cogent type \code{U32}.

If an enumeration type has a tag, Gencot preserves the tag information for the programmer and uses
a type name of the form \code{Enum\_tag}, as described in Section~\ref{design-names}. For tagless enums
no type names are introduced, they are directly mapped to type \code{U32}.

The rules for mapping enumeration types are
\begin{verbatim}
  enum { ... } -> U32
  enum e { ... } -> Enum_e
  enum e -> Enum_e
\end{verbatim}

The enumeration constants must be mapped to Cogent constant definitions of the corresponding type. In 
C the value for an enumeration constant may be explicitly specified, this can easily be mapped to
the Cogent constant definitions.

An enumeration declaration of the form \code{enum e \{C1, C2, C3=5, C4\}} is translated as
\begin{verbatim}
  cogent_C1: U32
  cogent_C1 = 0
  cogent_C2: U32
  cogent_C2 = 1
  cogent_C3: U32
  cogent_C3 = 5
  cogent_C4: U32
  cogent_C4 = 6
\end{verbatim}
Note that the C constant names are mapped to Cogent names as described in Section~\ref{design-names}.

\subsection{Structure and Union Types}
\label{design-types-struct}

A C structure type of the form \code{struct \{ ... \}} is equivalent to a Cogent unboxed record type \code{\#\{ ... \}}.
The Cogent compiler translates the unboxed record type to the C struct and maps all fields in the same order.
If every C field type is mapped to a binary compatible Cogent field type both types are binary compatible as a whole.

*** This is not true! Record fields are reordered by the Cogent compiler. Dargent must be used to get binary compatible types.

\subsubsection{Mapping Struct and Union Types}
A C structure may contain bit-fields where the number of bits used for storing the field is explicitly specified.
Gencot maps every consecutive sequence of bit-fields to a single Cogent field with a primitive Cogent type.
The Cogent type is determined by the sum of the bits of the bit-fields in the sequence. It is the smallest 
type chosen from \code{U8, U16, U32, U64} which is large enough to hold this number of bits. 
***--> test whether this is correct.
The name of the
Cogent field is \code{cogent\_bitfield}<n> where <n> is the number of the bit-field sequence in the C structure.
Gencot does not generate Cogent code for accessing the single bit-fields. If needed this must be done manually in Cogent.
However, Gencot adds comments after the Cogent bitfield showing the original C bit-field declarations.

A C union type of the form \code{union \{ ... \}} is not binary compatible to any type generated by the Cogent compiler.
The semantic equivalent would be a Cogent variant type. However, the Cogent compiler translates every variant type
to a \code{struct} with a field for an \code{enum} covering the variants, and one field for every variant. Even if a variant
is empty (has no additional fields), in the C \code{struct} it is present with type \code{unit\_t} which
has the size of an \code{int}. Therefore Gencot maps every union type to an unboxed abstract Cogent type.

Another semantic equivalent would be a Cogent record type, where always all fields but one are taken. However,
this type is not binary compatible either, it is translated to a normal struct where every member has another
offset. Even if the translated type in C is manually changed to a union, the Cogent take and put operations cannot
be used since they respect the field offsets. 

Together we have the mapping rules:
\begin{verbatim}
  struct s -> #Struct_s
  union s -> #Union_s
\end{verbatim}
where \code{Struct\_s} is the Cogent name of a record type corresponding to \code{struct s} and \code{Union\_s} is the
name of the abstract type introduced for \code{union s}.

As explained in Section~\ref{design-names}, Gencot always introduces a Cogent type name for each struct and union,
even if no tag is present in C. Since the tag name is either defined to name a Cogent record or an abstract type,
it is always linear and names a boxed type which corresponds to a pointer. Hence, the type name generated for a struct
is always used to refer to the type ``pointer to struct'', the struct type itself is translated to the type name 
with the unbox operator applied. The same holds for union types.

\subsection{Array Types}
\label{design-types-array}

A C array type \code{t[n]} has the semantics of a consecutive sequence of n instances of type \code{t}. A value
of type \code{t[n]} is a pointer to the first element and therefore compatible to type \code{t*}.

Basically, Cogent does not support accessing elements by an index value in an array. 
This is an important security feature since the index value is computed at runtime and cannot be statically 
compared to the array length by the compiler. Therefore, a C array type can only be mapped to an abstract type 
in Cogent, which prevents accessing its elements in Cogent code. Element access must be implemented externally 
with the help of abstract functions.

The Cogent standard library includes three abstract data types for arrays (\code{Wordarray, Array, UArray}). 
However, they cannot be used as a binary compatible replacement for C arrays, because they are implemented by 
pointers to a \code{struct} containing the array length together with the pointer to the array elements. 
Only if the C array pointer is contained in such a struct, it is possible to use the abstract data types. 
In existing C code the array length is often present somewhere at runtime, but not in a single \code{struct}
directly before the array pointer.

As of December 2018 there is an experimental Cogent array type written \code{T[n]}. It is binary compatible 
with the C array type \code{t[n]}. It is not linear, however it only supports read access to the array elements, 
the element values cannot be replaced. Thus it can be used as replacement for a pure abstract type, if the array 
is never modified and if it does not contain any pointers (directly or indirectly). If it is modified, replacing
elements can be implemented externally with the help of abstract functions.

In C the incomplete type \code{t[]} can be used in certain places. It may be completed statically, e.g. 
when initialized. Then the number of elements is statically known and the type can be mapped like \code{t[n]}.
If the number of elements is not statically known the type cannot be mapped to a Cogent array, it must be mapped 
to an abstract type.

Since the Cogent array type is still under development, the current version of Gencot does not use it for
mapping C array types. Instead, all C array types are mapped to Cogent with the help of generated names for 
abstract types.

As a C derived type, every array type depends on its base type, which is the element type. It must be defined in
the C source before the array type can be used. The Cogent compiler can only respect this ordering requirement if
it knows that the abstract type to which the array type is mapped depends on the type to which the element type
is mapped. The only way to make this known to the Cogent compiler is to use a generic type for the array with
a single type parameter for the element type. 

When a C array type is used for a field in a record, after translation from Cogent to C a type
must be used which includes the array size. To be able to specify a corresponding C typedef for the 
Cogent abstract type name, array types with different size specifications must be mapped to different 
Cogent abstract type names. This is the reason why the size specification is encoded into the type name,
as described in Section~\ref{design-names}. So a different type name is required for every array size 
occurring in the C program.

\subsubsection{Mapping Array Types}

A C array type \code{t[n]} has two slightly different meanings. When it is used for allocating space in memory,
it is used to determine the required space as \code{n * sizeof(t)}. When it is used as declared type for an 
identifier, it means that the identifier names a (constant) value of type \code{t*}, since C arrays are 
always represented by a pointer to the first element. In Cogent both cases must be supported. However, since 
the first case corresponds to a nonlinear (unboxed) type and the second case corresponds to a linear type, 
different Cogent types must be used. If the types are directly implemented by \code{t*} and \code{t[n]}, the 
linear type is not the derived pointer type of the nonlinear type, then the difference cannot be implemented 
by using a single type name with the unbox operator applied or not.

If the nonlinear case is implemented as a struct with the array as its only member, the
linear case can be implemented by a pointer to this struct which corresponds in Cogent to the same type
with the unbox operator omitted. The pointer to the struct is binary compatible with the pointer to its
only member which is binary compatible with the pointer to the first array element. 
This solution depends on the property, that a C struct with a single array as member has the same memory
layout as the array itself. If the C implementation adds padding after the array (it
cannot add it before the array according to the C specification), another solution must be used.

Together, this way the C array type is mapped for literal size specifications using a single generic abstract Cogent type
\begin{verbatim}
  type UArr<size> el
\end{verbatim}
whith an antiquoted C type definition 
\begin{verbatim}
typedef struct $id:(UArr<size> el) {
	$ty:el arr[<size>];
} $id:(UArr<size> el);
\end{verbatim}

As described in Section~\ref{design-operations-create} Gencot provides for every mapped linear type a separate type
for ``empty'' values. For a record type \code{R} the type \code{R take (..)} is used. However, the 
\code{take} operator can normally not be applied to an abstract type. There are three possible approaches for mapped 
array types.

The first approach uses the Cogent compiler flag \code{--flax-take-put} which allows the \code{take} operator to
be applied to abstract types (including abstract generic types). However, as of October 2019 the Cogent type checker 
is instable for such types and cannot process all applications. Therefore this approach is not used by Gencot.

The second approach uses a wrapper struct also in Cogent:
\begin{verbatim}
  type CArr<size> el = {arr: #(UArr<size> el)}
  type UArr<size> el
\end{verbatim}
This makes it possible to use the Cogent take type operator for modelling empty-value array types.
The drawback is that two different Cogent type names are required and that arrays are wrapped twice as struct.

The double wrapping could be avoided if type \code{UArr<size>} is directly implemented by the array type
in C as in
\begin{verbatim}
  typedef $ty:el $id:(UArr<size> el)[<size>];
\end{verbatim}
However, there are two problems with this approach. First, the Cogent compiler does not support antiquoted
type definitions for generating this form (the type parameter \code{\$ty:el} is only known in a struct type
with the generic type as tag name). Second, Cogent translates take and put operations for a record field 
to an assignment in C. Therefore, if field \code{arr} in a value of type \code{CArr<size> El} is put or taken, 
the C code generated by Cogent will be wrong, since arrays cannot be assigned in C. 

The third approach uses a second generic abstract type 
\begin{verbatim}
  type UEArr<size> el
\end{verbatim}
for every \code{<size>} to represent the empty-value array type. It is defined in C in the same way as type \code{UArr<size>}.
This also avoids the double wrapping. It does not use the \code{take} operator to construct the empty-value type
for array types, so there is no relation between both which is known to Cogent. However, the only situation where this
relation would be useful is for type variables, but the \code{take} operator cannot be applied to type variables since
it is not possible in Cogent to restrict a type variable to record types. 

The drawback of the third approach is that the empty-value 
type is constructed in different ways for mapped arrays and for mapped structs. This makes it impossible to define 
a single common macro \code{EVT} for this type construction in Section~\ref{design-operations-create}. This further 
implies that no single macro is possible if it uses \code{EVT}, and, in particular, since the type for pointers to
an unboxed array is the boxed array type (see Section~\ref{design-types-pointer}) this case must be distinguished 
in Macros which apply \code{EVT} to a constructed pointer type.

Therefore the decision in Gencot is to use the second approach. For every array \code{<size>} used in the C 
program the two generic Cogent types \code{CArr<size>} and \code{UArr<size>} are used where the latter is
abstract and is defined in antiquoted C as described above.

For constructing an array type from a given size and element type Gencot provides the preprocessor macro
\begin{verbatim}
  CARR(<size>,<ek>,el)
\end{verbatim}
defined in \code{include/gencot/CArray.cogent}. The parameter \code{<ek>} can be used to specify an unbox
operator for the element type, it may be \code{U} or empty. The form \code{CARR(<size>,U,el)} is equivalent
to \code{CARR(<size>,,\#el)}. The main reason for this parameter is compatibility to the macro \code{CPTR}
defined in Section~\ref{design-types-pointer}, so that it is possible to construct an array type and the 
pointer-to-element type from the same input. Both macros are mainly intended to be used for defining other
macros.

The unboxed type \code{\#(Carr<size> El)} is binary compatible with the C array type and can be used for
the nonlinear case. It is a usual Cogent unboxed record and can be used freely. 
If another Cogent record has a field of an unboxed array type, the take and put operations can be applied
in the usual way, assigning the array content as a whole.

As described in Section~\ref{design-names} the form \code{CArrX<size>X}
is used if the size is specified by an identifier and the single predefined generic type \code{CArrXX} is used for all other
size specifications and for array types without size specification. In the latter case no antiquoted C type definition
for the corresponding type \code{UArrXX} is provided by Gencot, C type definitions for instances must be provided 
manually. The types \code{CArrXX} and \code{UArrXX} are defined in \code{include/gencot/CArray.cogent}.

If the element type \code{El} is again an array type, the meaning in C is a multidimensional array.
This case is mapped to Cogent in the straightforward way, using an unboxed inner array type as element type
for the outer array type. For example, the C type \code{int[2][7]} is mapped to the Cogent type
\begin{verbatim}
  CArr2 #(CArr7 U32)
\end{verbatim}
which in C is a doubly wrapped array of doubly wrapped elements which should nevertheless be binary compatible 
to \code{int[2][7]}. Using the macro \code{CARR} it can be specified as \code{CARR(2,U,CARR(7,,U32))}.

Whether a C array type is mapped to the boxed or unboxed form depends on its usage context.
A C array type cannot be used as result type of a function. Thus, the remaining possible uses of a C array type are
\begin{itemize}
\item as type of a global variable. This is translated using an access function which returns the variable
value as a pointer of the boxed type \code{CArr<size> El}.
\item as type of a function parameter. In C this is ``adjusted'' to the pointer-to-element type. Since this
is binary compatible with type \code{CArr<size> El}, this type is used as parameter type in Cogent. Alternatively 
the adjusted type could be mapped, resulting in type \code{CPtr El} or \code{El} (in case of elements of unboxed 
record or array type). The type \code{CArr<size> El} is preferred here by Gencot since it preserves more information.
\item as type of a member in a struct type. Here the size of the array is relevant, therefore the unboxed type
\code{\#(CArr<size> El)} is used as type for the corresponding Cogent record field.
\item as element type of another array, resulting in a multidimensional array. Again the size of the array 
is relevant here, the unboxed type \code{\#(CArr<size> El)} is used as element type.
\item as base type of a derived pointer type. Here the binary compatible type is the boxed type \code{CArr<size> El}
since it corresponds to a pointer to the array.
\item as defining type for a typedef name. Here the type to be used depends on the context where the typedef
name is used. As described in Section~\ref{design-types-typedef}, Gencot defines the mapped 
typedef name always as an alias for the boxed type \code{CArr<size> El}. Then applying the unbox operator to the typedef name 
is equivalent to applying it to \code{CArr<size> El}.
\end{itemize}

Together we have the following mapping rules for C arrays with element type \code{el} and size 
specification \code{<size>}. Depending on the context of the C array type, additionally the unbox operator must be applied.
\begin{verbatim}
  el[<literal size>] -> CArr<literal size> El
  el[<size id>] -> CArrX<size id>X El
  el[<complex size>] -> CArrXX El
  el[] -> CArrXX El
  el[*] -> CArrXX El
\end{verbatim}
where El is the result of mapping the element type \code{el} to a Cogent type. If the size specification
is too complex (not a literal or single identifier) it is omitted and the type must be handled manually.

\subsection{Function Types}
\label{design-types-function}

C function types of the form \code{t (...)} are used in C only for declaring or defining functions or
when a typedef name is defined for a function type. In all other
places they are either not allowed or automatically adjusted to the corresponding function pointer type
of the form \code{t (*)(...)}. 

\subsubsection{Mapping Function Parameters}

In Cogent every function
has only one parameter. To be mapped to Cogent, the parameters of a C function with more than one parameter must
be aggregated in a tuple or in a record. A C function type \code{t (void)} which has no parameters is mapped
to the Cogent function type \code{() -> T} with a parameter of unit type.

The difference between using a tuple or record for the function parameters is that the fields in a 
record are named, in a tuple they are not. In 
a C function definition the parameters may be omitted, otherwise they are specified with names in a prototype.
In C function types the names of some or all parameters may be omitted, specifying only the parameter type.

It would be tempting to map C function types to Cogent functions with a record as parameter, whenever parameter 
names are available in C, and use a tuple as parameter otherwise. However, in C it is possible to assign a 
pointer to a function which has been defined 
with parameter names to a variable where the type does not provide parameter names such as in 
\begin{verbatim}
  int add (int x, int y) {...}
  int (*fun)(int,int);
  fun = &add;
\end{verbatim}
This case would result in Cogent code with incompatible function types.

For this reason we always use a tuple as parameter type in Cogent. Cogent tuple types are equivalent, if they
have the same number of fields in the same order and the fields have equivalent types. To preserve the C parameter names in 
a function definition, the parameter is matched with a tuple pattern containing variables of these
names as fields.

C function types where a variable number of
parameters is specified such as in \code{t (...)} (``variadic function type'') must 
be treated manually in specific ways. Gencot maps variadic function types
with an additional last parameter type \code{VariadicCogentParameters}. This pseudo type is intended 
to inform the developer that manual action is required. The bang operator is applied as a hint
that no modifications are returned. For a function pointer, the corresponding Cogent type has
the form \code{\#(CFun ((...,VariadicCogentParameters!) -> <result type>))}.

C function types where the parameters are omitted, such as in \code{t ()} (``incomplete function type'') 
cannot be mapped to a Cogent function type in this way. 
They can only be mapped using an abstract type as parameter type. This can again lead to incompatible 
Cogent types if a function pointer is assigned where parameters have been specified, these cases must 
be treated manually in specific ways. Note that incomplete function types cannot be used for function
definitions, only for function pointers and for declarations of external functions. 
Gencot does not translate declarations of functions with incomplete types, these must be added manually.
This behavior of Gencot is also exploited to handle manually translated macros, as described in 
Section~\ref{impl-ccomps-externs}.

Together the rules for mapping function types are
\begin{verbatim}
  t(t1, ..., tn) -> (T1, ..., Tn) -> T
  t(void) -> () -> T
  t(t1,...,tn,...) -> (T1, ..., Tn, Variadic_Cogent_Parameters!) -> T
\end{verbatim}

\subsubsection{Linear and Readonly Parameter Types}

Like every other type, the type of a function parameter may be readonly because this property can be derived
from the C type information, as described in Section~\ref{design-types-readonly}. However, in a C program 
often a parameter is actually never modified, although this property cannot be derived from the C type information.
Since this is an important property for working with linear types in Cogent, Gencot tries to capture these
cases as well and make the parameter type readonly.

For a parameter with linear type, the function can only be defined in Cogent if the parameter is not discarded,
i.e. it must be part of the result. Gencot assumes the most simple handling of this case, where the result
is a tuple consisting of the original result of the C function together with a component for every parameter
of linear type. To avoid these extra result components, Gencot tries to make all parameters with linear type
readonly and add only the remaining parameters to the result type.

A parameter of linear type may not be discarded in Cogent, but it may be passed to an abstract function which discards
it. In this case the parameter must not be returned, although it does not have readonly type.

Gencot determines the information about parameter modification and parameter discarding in a semi-automatic way
as described in Section~\ref{design-parmod}. It uses this information to make parameter types readonly or
add parameters to the function result, according to the following rules:
\begin{itemize}
\item If the C type of a parameter is not linear according to Section~\ref{design-types-readonly}, or if it
is discarded according to Section~\ref{design-parmod}, it is translated as described before.
\item Otherwise, if the C type of the parameter is readonly according to Section~\ref{design-types-readonly}, or
is not modified according to Section~\ref{design-parmod} its translated type is made readonly by applying the
bang operator \code{!}. For a single such parameter of type \code{r} the translation rule becomes
\begin{verbatim}
  t(t1, ... r, ... tn) -> (T1, ... R!, ... Tn) -> T
\end{verbatim}
\item Otherwise, if the parameter is already returned as result of the function, it is translated as described before.
\item Otherwise, the function result is changed to a tuple and the parameter is added as component to that tuple.
For a single such parameter of type \code{l} the translation rule becomes
\begin{verbatim}
  t(t1, ... l, ... tn) -> (T1, ... L, ... Tn) -> (T,L)
\end{verbatim}
\end{itemize}

If the result is modified to a tuple, the first component is the original function result and the remaining components
are the parameters of linear type in their order as they occur in the parameter tuple.

\subsection{Pointer Types}
\label{design-types-pointer}

In general, a C pointer type \code{t*} is the kind of types targeted by Cogent linear types. The linear type 
allows the Cogent compiler to statically guarantee that pointer values will neither be duplicated nor 
discarded by Cogent code, it will always be passed through. 

If a pointer points to a C \code{struct} there is additional support for field access available in Cogent by 
mapping the pointer to a Cogent boxed record type. For all other pointer types this support can be employed by
mapping the type to a Cogent record with a single field of the type referenced by the pointer.

In C, a pointer-to-array type is not the type of pointers to the array address, instead its values are array addresses.
The difference from the array type is only that when applying the index operator \code{[]}, the whole array is
selected instead of only the first element. In Cogent the corresponding type is a Cogent record with a single
field of the unboxed mapped array type. Note, that this is equivalent to the mapping of the array type itself, as defined
in Section~\ref{design-types-array}, so it is used by Gencot.

As described for array types in Section~\ref{design-types-array}, Cogent must know that a pointer type depends on 
its base type, which is the type of the referenced value. Therefore pointer types are mapped to generic types which
have the base type as single type parameter. Since there is no additional information to be encoded for a pointer
type other than its base type, a single generic type is sufficient for all pointer types. As described in 
Section~\ref{design-names} the type name \code{CPtr} is used for this purpose.

\subsubsection{Mapping Pointer Types}

A pointer type \code{t*} to a struct is mapped to the corresponding boxed type, 
that means, it is mapped like the struct type \code{t}, but the unbox operator is omitted.

A pointer type \code{t*} where \code{t} is a union type is mapped in the same way to the corresponding
boxed type, omitting the unbox operator from the mapped type \code{t}. Thus no support for accessing
the referenced union is provided. The reason is, that access to the union as a whole is mostly useless
and further access to the union mebers cannot be provided. For consistency, Gencot treats union types
in the same way as struct types.

A pointer type \code{t*} where \code{t} is a primitive type, an enum type, 
or again a pointer type is mapped to a boxed record with a single field \code{cont} of the type \code{T} to which \code{t} is 
mapped. For every such type the generic type \code{CPtr T} name is used. This makes the 
Cogent program slightly more readable. 

The resulting generic type can be completely defined in Cogent as
\begin{verbatim}
  type CPtr ref = { cont: ref }
\end{verbatim}
it is predefined in file \code{include/gencot/CPointer.cogent}.

Values of such types are binary compatible to the C pointer type and they can be dereferenced with the help of
the Cogent take and put operations, thus supporting the full functionality of the C pointer.

A pointer type \code{t*} where \code{t} is an array type is mapped to the mapped (boxed) array type 
\begin{verbatim}
  CArr<size> El
\end{verbatim}
This approach is similar as for struct and union types: the pointer type is mapped to the plain boxed type,
the base type is mapped to the corresponding unboxed type. The difference in case of array pointers is, that also the base
type is often ``adjusted'' to the boxed type, because in C a pointer is used instead. Other than in C, however, Gencot
represents adjusted arrays by a pointer to the array as a whole, instead of as a pointer to the first element, because
this preserves more information about the array.

Finally, a pointer to \code{void} is mapped to the abstract type
\begin{verbatim}
  CVoidPtr
\end{verbatim}
defined in \code{include/gencot/CPointer.cogent}.
It is intended as a placeholder. Here the type of the referenced data is unknown. In C it is typically used 
as a generic pointer type which is cast to specific types of referenced values. This cannot be transferred to
Cogent, to be used, the type must be replaced manually. No C type definition is provided for it by Gencot.

Together we have the mapping rules for pointer types:
\begin{verbatim}
  void * -> CVoidPtr
  struct s * -> Struct_s = { ... }
  union s * -> Union_s
  (*el)[...] -> CArr... El
  otherwise: t * -> CPtr T
\end{verbatim}
where \code{Struct\_s} and \code{Union\_s} are the names introduced for the struct or union types and \code{El} 
and \code{T} are the Cogent types to which \code{el} and \code{t} are mapped, respectively.

To make the construction of a pointer type for a given Cogent type more generic (in particular for its use 
in other preprocessor macros), Gencot provides in \code{include/gencot/CPointer.cogent} the preprocessor macro
\begin{verbatim}
  CPTR(<knd>,<type>)
\end{verbatim}
where \code{<type>} is a Cogent type and \code{<knd>} is \code{U} or empty. \code{CPTR(U,T)} expands to \code{T} and
must be used if the \code{<type>} is an unboxed type of the form \code{\#Struct\_s}, \code{\#Union\_s}, or
\code{\#(Carr<size> El)}. \code{CPTR(,T)} expands to \code{(CPtr T)} and must be used in all other cases.

\subsubsection{Mapping Function Pointers}

In Cogent the distinction between function types and function pointer types does not exist. 
A Cogent function type of the form \code{T1 -> T2} is used both when
defining functions and when binding functions to variables. If used in a function definition, it is mapped by
the Cogent compiler to the corresponding C function type, as described in Section~\ref{design-types-function}.

In other places, however, Cogent does not translate its function types to C function pointers. Instead, it uses 
a C enumeration type where every known function has an associated enumeration constant. Whenever a 
function is bound to a variable, passed as a parameter or is invoked through a function pointer, it is 
represented by this enumeration constant in C, i.e., by an integer value.
For function invocation Cogent generates dispatcher functions which receive the integer value as an argument
and invoke the corresponding C function. 

Binary compatibility is only relevant when a function is stored, then it is always represented by the enumeration
constant in C generated from Cogent. Thus, a C function pointer type cannot be mapped by Gencot to a Cogent function type,
since this will not be binary compatible. Instead, it must be mapped to a Cogent abstract type together with 
abstract functions which translate between the abstract type and the Cogent function type (needed when invoking 
the function in Cogent).

Together, Gencot treats C function types and C function pointer types in completely different ways. It maps
C function types to Cogent function types and it maps C function pointer types to Cogent abstract types.

These abstract types treat C function pointers in Cogent as fully opaque values. The only operation which can 
be applied to them is to translate them to the enumeration value used in Cogent. This is done by comparing the
actual pointer values for equality. Since this can be done in C independent from the pointer's type, a single 
common C pointer type such as \code{void *} is sufficient. Accordingly, a single abstract type in Cogent is
sufficient to represent all C function pointers. In particular, no dependencies on the parameter or result types 
need to be known by Cogent to position their definitions before the first use of the type in C. 

However, for the developer the information about the parameter and result types are useful. Moreover, if different
Cogent types are used for different C function pointer types, the Cogent type checker can be used to find
mistakes. Since the full information about the parameter and result types is contained in the Cogent function types
Gencot uses generic abstract types with a function type as single type argument. This can be thought of as an 
``annotated'' or ``wrapped'' function type. 

Gencot uses the following two generic abstract types for representing binary compatible C function pointers:
\begin{verbatim}
  type CFunPtr funtype
  type CFunInc restype
\end{verbatim}
defined in \code{include/gencot/CPointer.cogent}.

The first type is used for complete function pointer types where the parameter types are specified or it is specified that
the function has no parameters using the keyword \code{void}. The type argument must be a Cogent function type, 
although this constraint cannot be specified or checked in Cogent. The second type is used for incomplete C function 
pointer types, where only the result type
is specified but no parameter types. Here only the result type is used as type argument, it may be an arbitrary 
Cogent type, only restricted by the C rules that it may not be a function or array type. Since however C array types
are mapped by wrapping them in a record, the corresponding unboxed types of the form \code{\#(CArr<size> El)}
\textit{can} be used as function result types, although Gencot will never do so when translating a valid C program.

Although the C function pointer is a pointer, the pointer target value (the machine code implementing the function) 
normally cannot be modified. Hence, semantically a function pointer type does not correspond to a linear type
in Cogent, it could be represented by a readonly type or by an unboxed type. Gencot uses an unboxed type
since values of readonly types cannot escape from banged contexts. Gencot automatically applies the 
unbox operator in the form \code{\#(CFunPtr (X->Y))} whenever it uses one of the two generic function pointer types.

For a complete C function pointer type its base type is translated to a Cogent function type as described in
Section~\ref{design-types-function}. This includes the cases for variadic function types and the cases where
parameter types are converted to readonly types or cause additional components in the result type for returning 
modified parameters.

Together the rules for mapping a function pointer type are
\begin{verbatim}
  t (*)( ... ) -> #(CFunPtr (( ... ) -> T))
  t (*)() -> #(CFunInc T)
\end{verbatim}
where \code{( ... ) -> T} is the mapping of type \code{t( ... )} according to Section~\ref{design-types-function}
and \code{T} is the mapping of \code{t}, extended by the types of returned modified parameters.

For example, the C function pointer type \code{int (*)(int, int[10])} is mapped to the abstract type
\begin{verbatim}
  #(CFunPtr ((U32,CArr10 U32) -> U32))
\end{verbatim}

Note that the macro call \code{CPTR(,T)} expands to \code{(CPtr T)} also if \code{T} is a Cogent function type.
This corresponds to a pointer to the function enumeration value used by Cogent and is not related to the C
function pointer type.

\subsection{Defined Type Names}
\label{design-types-typedef}

In C a typedef can be used to define a name for every possible type. In principle, it would be possible to
map a typedef name by resolving it to its type and then mapping this type as described above. However, the
typedef name often bears information for the programmer, hence the goal for Gencot is to preserve this information
and map the typedef name to the corresponding Cogent type name which is defined by translating the typedef
to a Cogent type definition.

Moreover, when a C type is derived from a typedef name, Gencot also maps the derived type using the mapped typedef name
as type argument in the corresponding parameterized type. Since the mapped typedef name is defined to be a synonym 
for the mapped type definition, the resulting parameterized Cogent type is equivalent to the mapping of the derived 
type with the resolved typedef name as base type.

An exception from this rule are typedef names for struct, union, and array types. Mapping pointer types derived 
from such typedef names with the help of \code{CPtr} would result in the following situation:
\begin{verbatim}
  typedef struct s snam
  mapping: struct s -> #Struct_s
  mapping: struct s * -> Struct_s
  mapping: snam -> Cogent_snam
  mapping: snam * -> CPtr Cogent_snam
\end{verbatim}
where \code{Struct\_s} is the name of the Cogent record type corresponding to \code{struct s}. The problem here
is that for the mapped typedef name the pointer corresponds to a pointer to pointer to struct which is not binary 
compatible with the mapped struct pointer.

Therefore Gencot treats every typedef name resolving to a struct, union, or array type as if 
it would resolve to the corresponding pointer type. The plain name is mapped with the unbox operator 
applied (for arrays depending on the usage context), the pointer type derived from it is mapped without 
unbox operator applied. 

For an array type derived from a typedef name for a struct, union, or array, the unbox operator is 
applied to the mapped typedef name if the element type is the unboxed type.

For a derived function type a typedef name for a struct or union type can occur as parameter or result
type, if the type is unboxed the unbox operator is applied to the mapped typename. A typedef name for 
an array cannot occur as result type if the type is unboxed. If it occurs as parameter type it is 
adjusted by C and the mapped typedef name is always used without unbox operator.

A typedef name resolving to a function type is also treated as if it would resolve to the corresponding 
function pointer type. The pointer type derived from it is mapped to the mapped typedef name with the
unbox operator applied (since it is a function pointer type). This implies that the plain name cannot
be mapped. However, that is no restriction, since a C typedef name for a function type can only be used
for constructing the corresponding C function pointer type, either explicitly or implicitly by adjustment.
In particular, it cannot be used for defining a function of that type, since a definition must always 
include the parameter names.

For array types derived from a typedef name for a function type no special rules apply. 

The resulting mapping rules are for function type names:
\begin{verbatim}
  tn -> #TN
  tn* -> #TN
\end{verbatim}
for names of function pointer types:
\begin{verbatim}
  tn -> #TN
\end{verbatim}
for names of a struct or union type:
\begin{verbatim}
  tn -> #TN
  tn* -> TN
\end{verbatim}
for names of an array type:
\begin{verbatim}
  tn -> TN, #TN
  tn* -> TN
\end{verbatim}
depending on its usage context for \code{tn}, 
and for all other type names:
\begin{verbatim}
  tn -> TN
\end{verbatim}
where \code{TN} is the name mapping of \code{tn}.

This implies, that also the Cogent type definitions generated from a C typedef have to be modified, if
the target type is a struct, union, or array type. In this case Gencot translates the typedef 
to a Cogent type definition which defines the mapped typedef name as a synonym 
for the corresponding boxed type in Cogent.

If the target type is a function type Gencot translates typedef to a Cogent type definition which
defines the mapped typedef name as a synonym for the translated function pointer type. Here, the unbox
operator can be applied either to the righthand side of the type definition or to every occurrence
of the mapped typedef name or both. Gencot applies it to both to make it apparent that in both cases
the type is not linear.

If the target type in a typedef is another typedef name Gencot resolves it to the final target type
before applying the rules above.

\subsection{Linear and Readonly Types}
\label{design-types-readonly}

C types can be qualified as \code{const}. This means, the values of the type are immutable and could be stored in 
a readonly memory. A variable declared with a readonly type is initialized with a value and cannot be modified 
afterwards. The immutability of an aggregate type also implies that values cannot be modified by modifying parts: 
for a struct the fields cannot be modified and for an array the elements cannot be modified. This behavior corresponds 
to the behavior of all primitive and unboxed types in Cogent. 

If a C type is not qualified as \code{const}, stored values of the type may be modified. This may have non-local
effects if the stored value is shared (part of several other values). In Cogent, values of primitive and unboxed types
cannot be shared (only copies can be part of other values). Therefore a modification of the C value always corresponds to
replacing the value bound to a variable in Cogent. This can be represented by binding the new value to a variable
of the same name which will shadow the previous binding. Together, this means that a \code{const} qualifier is
irrelevant whenever a C type is translated to a primitive or unboxed type in Cogent.

The situation is different for C pointer types which are translated to linear types in Cogent. Values of linear types
may be modified using put and take operations in Cogent, but they are restricted in their use. Put and take operations
correspond to modifications of the value referenced by the pointer. Thus, a \code{const} qualification of the pointer
type is still irrelevant for them, however, a \code{const} qualification of the pointer's \textit{base type} means
that put and take operations are not possible. This case is supported by Cogent as readonly types, which are not
restricted in their use in the same way as linear types.

Note, however, that Cogent does not separate between pointers and their referenced values: the referenced value
is treated as part of the linear value. If the referenced value itself contains references, the values referenced
by them are also treated as part of the overall value. This implies, that a readonly type in Cogent corresponds to
a C pointer type with \code{const} qualified base type where all components with a pointer type recursively have
the same property.

It further implies, that a C type also corresponds to a linear type in Cogent, if it directly or indirectly 
\textit{contains} pointers where the base type is not \code{const} qualified. This may be the case for struct or union
types (members may have such pointer types) or for array types (the elements may have such a pointer type).

An exception are C pointers to functions. It is assumed that the function code cannot be modified, hence a C pointer 
to function is treated like a primitive type in Cogent.

Gencot tests every C type for being a pointer or containing a pointer. If this is the case, the translated Cogent 
type is known to be linear. The C type is then further tested whether all pointers have a \code{const} qualified
base type. If this is the case, the type is translated to a Cogent readonly type,
by applying the bang operator \code{!} to the type after translating it as described in Section~\ref{design-types-pointer}.

In particular, the readonly property is valid for all pointer types where the base type is \code{const} qualified 
and contains no pointers, such as \code{const char*}.



\section{Processing C Declarations}
\label{design-decls}

A C declaration consists of zero or more declarators, preceded by information applying to all declarators together.
Gencot translates every declarator to a separate Cogent definition, duplicating the common information as needed.
The Cogent definitions are generated in the same order as the declarators.

A C declaration may either be a \code{typedef} or an object declaration. A typedef can only occur on toplevel or 
in function bodies in C.
For every declarator in a toplevel typedef Gencot generates a Cogent type definition at the corresponding position. Hence
all these Cogent type definitions are on toplevel, as required in Cogent. Typedefs in function bodies are not
processed by Gencot, as described in Section~\ref{design-fundefs}.

A C object declaration may occur 
\begin{itemize}
\item on toplevel (called an ``external declaration'' in C),
\item in a struct or union specifier for declaring members,
\item in a parameter list of a function type for declaring a parameter,
\item in a compound statement for declaring local variables.
\end{itemize}

External declarations are simply discarded by Gencot. In Cogent there is no corresponding concept, it is not needed
since the scope of a toplevel Cogent definition is always the whole program. 

Compound statements in C only occur 
in the body of a function definition, which is not translated by Gencot 
(see Section~\ref{design-fundefs}). Thus, declarations embedded in a body are not processed by Gencot.

Union specifiers are always translated to abstract types by Gencot, hence declarations for union members are
never processed by Gencot.

The remaining cases are struct member declarations and function parameter declarations. 
For every declarator in an object declaration, Gencot generates a Cogent record field definition, if the C declaration
declares struct members, or it generates a tuple field definition, if the C declaration declares a function parameter.

\subsection{Target Code for struct/union/enum Specifiers}
\label{design-decls-tags}

Additionally, whenever a struct-or-union-specifier or enum-specifier occurring in the C declaration has a body and
a tag, a Cogent type definition is generated for the corresponding type, since it may be referred in C by its tag from
other places. A C declaration may contain atmost one 
struct-or-union-specifier or enum-specifier directly. Here we call such a specifier the ``full specifier'' of 
the declaration, if it has a body. 

Since Cogent type definitions must be on toplevel, Gencot defers it to the next possible toplevel position after the
target code generated from the context of the struct/union/enum declaration. If the context is a typedef, it is placed
immediately after the corresponding Cogent type definition. If the typedef contains several full specifiers (which
may be nested), all corresponding Cogent type definitions are positioned on toplevel in the order of the beginnings
of the full specifiers in C (which corresponds to a depth-first traversal of all full specifiers).

If the context is a member declaration in a struct-or-union-specifier, the Cogent type definition is placed after that
generated for its context. 

If the context is a parameter declaration it may either be embedded in a function definition or in a declarator of another
declaration. Function definitions in C always occur on toplevel, the Cogent type definitions for all struct/union/enum
declarations in the parameter list are placed after the target code for the function definition (which may be unusual for 
manually written Cogent code, but it is easier to generate for Gencot). In all other cases the Cogent type definitions
for struct/union/enum declarations in a parameter list are treated in the same way as if they directly occur in the surrounding
declaration.

Note, that a struct/union/enum tag declared in a parameter list has only ``prototype scope'' or ``block scope'' which
ends after the function type or definition. Gencot nevertheless generates a toplevel type definition for it, since the
tag may be used several times in the parameter list or in the corresponding body of a function definition. Note that 
this may introduce name conflicts, if the same tag is declared in different parameter lists. Since declaring tags in a 
parameter list is very unusual in C, Gencot does not try to solve these conflicts, they will be detected by the Cogent
compiler and must be handled manually.

A full specifier without a tag can only be used at the place where it statically occurs in the C code, however, it
may be used in several declarators. Therefore Gencot also generates a toplevel type definition for it, with an 
introduced type name as described in Section~\ref{design-names}.

\subsection{Relating Comments}

A declaration is treated as a structured source code part. The subparts are the full specifier, if present, and all
declarators. Every declarator includes the terminating comma or semicolon, thus there is no main part code between or after
the declarators. The specifiers may consist of a single full specifier, then there is no main part code at all.

The target code part generated for a declaration consists of the sequence of target code parts generated for the declarators,
and of the sequence of target code parts generated for the full specifier, if present. No target code is generated for the 
main part itself. In both sequences the subparts are 
positioned consecutively, but the two sequences may be separated by other code, since the second sequence consists of 
Cogent type definitions which must always be on toplevel. 

According to the rules defined in Sectiom~\ref{design-comments-relate}, the before-unit of the declaration is put before
the target of the first subpart, which is that for the full specifier, if present, otherwise it is the target for the
first declarator. In the first case the comments will be moved to the type definition for the full specifier. The rationale
is that often a comment describing the struct/union/enum declaration is put before the declaration which contains it.

The after-unit of the declaration is always put behind the target of the last declaration.

A declarator may derive a function type specifying a parameter-type-list. If that list is not \code{void}, the 
declarator is a structured source code part with the parameter-declarations as embedded subparts. Every
parameter-declaration includes the separating comma after it, if another parameter-declaration follows,
thus there is no main part code between the parameter-declarations. The parentheses around the parameter-type-list
belong to the main part, thus a comment is only associated with a parameter if it occurs inside the parentheses.

In all other cases a declarator is an unstructured
source code part.

\subsection{Typedef Declarations}
\label{design-decls-typedefs}

For a C typedef declaration Gencot generates a separate toplevel Cogent type definition for every declarator.

For every declarator a C type is determined from the declaration specifiers together with the derivation specified
in the declarator. As described in Section~\ref{design-types}, either a Cogent type expression is determined from this C type,
or the Cogent type is decided to be abstract. 

The defined type name is generated from the C type name according to the mapping described in Section~\ref{design-names}.
Type names used in the C type specification are mapped to Cogent type names in the Cogent type expression in the same way.

\subsection{Object Declarations}

C object declarations are processed if they declare struct members or function parameters.

For such a C object declaration Gencot generates a separate Cogent field 
definition for every declarator. This is a named record field definition if the declaration is embedded in the
body of a struct-or-union-specifier, it is an unnamed tuple field definition if the declaration is embedded in the
parameter-type-list of a function type. In the first case declarators with function type are not allowed, in the
second case they are adjusted to function pointer type. In both cases the Cogent field type is determined from the 
declarator's C type as described in Section~\ref{design-types}. 

In the case of a named record field the Cogent
field name is determined from the name in the C declarator as described in Section~\ref{design-names}. In the 
case of an unnamed tuple field a name specified in the C parameter declaration is always discarded.

\subsection{Struct or Union Specifiers}

For a full specifier with a tag Gencot generates a Cogent type definition. The name of the defined type is generated
from the tag as described in Section~\ref{design-names}. For a union specifier the type is abstract, no defining type
expression is generated. For a struct specifier a (boxed) Cogent record type expression is generated, which has a field
for every declared struct member which is not a bitfield. Bitfield members are aggregated as described in 
Section~\ref{design-types-struct}. 

A specifier without a body must always have a tag and is used in C to reference the full specifier with the same tag.
Gencot translates it to the Cogent type name defined in the type definition for the full specifier.

Note that the Cogent type defined for the full specifier corresponds to the C type of a pointer to the struct or union, 
whereas the unboxed Cogent type corresponds to the C struct or union itself. This is adapted by Gencot when translating
the C specifier embedded in a context to the corresponding Cogent type reference.

\subsection{Enum Specifiers}

For a full enum specifier with a tag Gencot generates a Cogent type definition immediately followed by Cogent object
definitions for all enum constants. The name of the defined type is generated
from the tag as described in Section~\ref{design-names}. The defining Cogent type is always \code{U32}, as described in
Section~\ref{design-types-enum}. 

A specifier without a body must always have a tag and is used in C to reference the full specifier with the same tag.
Gencot translates it to the Cogent type name defined in the type definition for the full specifier.

 


\section{Processing C Function Definitions}
\label{design-fundefs}

A C function definition is translated by Gencot to a Cogent function definition. Old-style C function definitions
where the parameter types are specified by separate declarations between the parameter list and the function body
are not supported by Gencot because of the additional complexity of comment association.

The Cogent function name is generated from the C function name as described in Section~\ref{design-names}.

The Cogent function type is generated from the C function result type and from all C parameter types as described
in Section~\ref{design-types-function}. In a C
function definition the types for all parameters must be specified in the parameter list, if old-style function
definitions are ignored.

\subsection{Function Bodies}
\label{design-fundefs-body}

In C the function body consists of a compound statement which is specified in imperative programming style. In Cogent
the function body consists of an expression which is specified in functional programming style with additional 
restrictions which are crucial for proving properties of the Cogent program. Therefore Gencot does not try
to translate function bodies, this must be done by a human programmer.

It would be possible, however, to translate C declarations embedded in the body. These may be type definitions
and definitions for local variables. However, there are no good choices for the generated target code. Type
definitions cannot be local in an expression in Cogent, they must be moved to the toplevel where they may cause
conflicts. Local variable definitions could be translated to Cogent variable bindings in let-expressions, however,
C assignments cannot be translated for them. Also, the resulting mixture of C code and Cogent code is expected 
to be quite confusing to the programmer who has to do the manual translation. Therefore, no declarations in 
function bodies are processed by Gencot.

The only processing done for function bodies is the substitution of names occurring free in the body. These may
be names with global scope (for types, functions, tags, global variables, enum constants or preprocessor constants)
or parameter names. For all names with global scope Gencot has generated a Cogent definition using a mapped name.
These names are substituted in the C code of the function body by the corresponding mapped names so that the 
mapping need not be done manually by the programmer.

This does not include the mapping of derived types. Type derivation in C is done in declarators which refer
a common type specification in a declaration. In Cogent there is no similar concept, every declarator must be 
translated to a separate declaration. This is not done due to the reasons described above. As result, a translated
local declaration may have the form
\begin{verbatim}
  Struct_s1 a, *b, c[5];
\end{verbatim}
although the Cogent types for \code{a, b, c} would be \code{\#Struct\_s1}, \code{Struct\_s1}, and 
\code{\#CarrayOf \#Struct\_s1}, respectively. This translation for the derived types must be done manually.

Additionally, the function parameter names usually occur free in the function body. To make them apparent to
the programmer, Gencot generates a Cogent pattern for the (single) parameter of the Cogent function which 
consists of a tuple of variables with the names generated from the C parameter names. As described in 
Section~\ref{design-names} the C parameter names are only mapped if they are uppercase, otherwise they are
translated to Cogent unmodified. If they are mapped they are substituted in the body. Since it is very unusual
to use uppercase parameter names in C, the Cogent function will normally use the original C parameter names.

If the C function body is directly written into the Cogent source file, it cannot be syntax checked by the 
Cogent compiler. Therefore it is wrapped as a Cogent comment by adding a \code{-} sign immediately after the
opening brace and before the closing brace.

To yield a correct Cogent function, a dummy result expression is generated. Gencot generates the expression
depending on the result type, as described in Section~\ref{design-types}. 

The dummy result expression always ignores all parameter values. This will cause an error message by the
Cogent compiler if any parameter type is linear, since then the parameter may not be discarded. Therefore
Gencot always makes all parameter types readonly by applying the bang operator to the parameter type
(which may be the tuple type of all parameter types if the C function has more than one parameter). If 
not all parameters are in fact read only, this must be corrected when the function body is translated
manually.

The generated Cogent function definition has the form
\begin{verbatim}
<name> :: (<ptype1>, ..., <ptypen>)! -> <restype>
<name> (<pname1>, ..., <pnamen>) = <dummy result>
{- <compound statement> -}
\end{verbatim}
where the \code{<compound statement>} is plain C code with substituted names.

\subsection{Comments in Function Definitions}

A C function definition which is not old-style syntactically consists of a declaration with a single declarator
and the compound statement for the body.
It is treated by Gencot as a structured source code part with the declaration and the body as subparts
without any main part code. According to the structures of declarations the declaration has the single declarator as subpart
and optionally a full specifier, if present. The declarator has the parameter declarations as subparts.

\subsubsection{Function Header}

The target code part for the declaration and for its single declarator is the header of the Cogent function definition
(first two lines in the schema in the previous section). The target code part for the full specifiers with tags in
the declaration (which may be present for the result type and for each parameter) is a sequence of corresponding 
type definitions, as described for declarations in Section~\ref{design-decls-tags}, which is placed 
after the Cogent function definition. The target code part for full specifiers without tags is the generated type
expression embedded in the Cogent type for the corresponding parameter or the result.

All parameter declarations consist of a single declarator and the optional full specifier. The target code part for
a parameter declaration and its declarator is the corresponding parameter type in the Cogent function type expression.
Hence, comments associated with parameter declarations in C are moved to the parameter type expression in Cogent.

\subsubsection{Function Body}

To preserve comments embedded in the C function body it is also considered as a structured source code part. Its 
subpart structure corresponds to the syntactic structure of the C AST. Since in the target code only identifiers 
are substituted, the target code
structure is the same as that of the source code. The structure is only used for identifying and re-inserting
the transferrable comments and preprocessor directives. Note that this works only if the conditional directive 
structure is compatible with the syntactic structure, i.e., a group must always contain a complete syntactical
unit such as a statement, expression or declaration, which is the usual case in C code in practice.

An alternative approach would be to treat all nonempty source code lines as subparts of a function body, resulting
in a flat sequence structure of single lines. The advantage is that it is always compatible to the conditional 
directive structure and
all comment units would be transferred. However, generating the corresponding origin markers in an abstract syntax
tree is much more complex than generating them for syntactical units for which the origin information is present
in the syntax tree. Since the Gencot implementation generates the
target code as an abstract syntax tree, the syntactical statement structure is preferred. 


\section{Function Parameter Modifications}
\label{design-parmod}
Gencot uses the information, whether a parameter value may be modified by a function, when it translates
the function, as described in Section~\ref{design-types-function}.

A parameter is modified, when a part of it is changed, which is referenced through a pointer. Only then the 
modification is shared with the caller and persists after the function call.

\subsection{Detecting Parameter Modifications}

A parameter may be modified by an assignment to a referenced part. Such an assignment
is easy to detect, if it directly involves the parameter name. However, the assignment can also be indirect, where
the parameter or part of it is first assigned to a local variable or another structure and then the assignment
is applied to this variable or structure. This implies that a full data flow analysis would be necessary
to detect all parameter modifications.

Gencot uses a simpler semiautomatic approach. It detects all direct parameter modifications which involve the 
parameter name automatically. It then generates a JSON description which lists all functions with their parameters
and the information about the detected modification cases. The developer has to check this description and
manually add additional cases of other parameter modifications. The resulting description is then read by
Gencot and used during translation.

Gencot also treats the special cases, where a parameter is discarded (by directly or indirectly invoking the C standard
function \code{free}) or is modified but already returned as the (single) result of the C function. These cases 
must also be detected by the developer.

Parameter modifications are only relevant if the parameter is a pointer or if a pointer can be reached by the parameter
and if at least one such pointer is not declared to have a \code{const} qualified referenced type. Gencot detects
this information from the C type and adds it to the JSON description, so that the developer can easily identify
the relevant cases where to look for modifications.

In the case of a variadic function, the number of its parameters cannot be determined from the function definition.
Here Gencot uses the invocation with the most arguments to determine the number of parameter descriptions added
to the JSON description.

A parameter may be modified by a C function locally, but it can 
also be modified by passing it or a part of it as parameter to an invoked function which modifies its corresponding
parameter. Gencot supports these dependent parameter modifications, by detecting dependencies on parameter
modifications by invoked functions and adding them to the JSON description. Again, only dependencies which
involve the parameter name are detected automatically, other dependencies must be added by the developer.

When Gencot reads the JSON template it calculates the transitive closure of all dependencies and uses it to
determine the parameter modification information. Thus the developer has only to look at every function locally,
it is not necessary to take into account the effects of invoked functions.

\subsection{Required Invocations}

To support the incremental translation of single C source files, Gencot determines the parameter modification
description for single C source files. It processes all function definitions in the file by analysing their 
function bodies. However, for an invoked function the definition may not be available, it may be defined externally
in another C source file. This situation must be handled manually: the developer has to process additional
C source files where the invoked functions are defined, to add their descriptions.

Gencot specifically selects only the relevant invocations, which are required because a parameter modification depends on it.
If a parameter is already modified locally, additional dependencies are ignored and the corresponding invocations
are not selected. In this way Gencot keeps the JSON descriptions minimal.

An invoked function may also be specified using a function pointer. In this case the possible function bodies cannot 
be determined from the program. Again, it is left to the developer to determine, whether such invocations may modify
their parameters. Gencot supports this by also adding parameter description templates for function pointers defined in the 
source file, to be filled by the developer. 

In contrast to functions, function pointers can also be defined locally in a function\footnote{The current version of
Gencot does not support C extensions for nested function definitions.} (as local variable or as parameter). Gencot
automatically adds description templates for all invoked local function pointers.

\subsection{External Invocations}

Finally, a required invocation may be external to the package which is processed by Gencot, such as an invocation of
a function in the C standard library. When all remaining required invocations are external in this sense, Gencot
can be told to ``close'' the JSON description. Then it uses the declarations of all required invocations to generate
a description template for each of them. Since no bodies are available, the developer must fill in the information
for all these external functions (and function pointers).

Invoked function pointers may also be members of a struct or union type. For them no definition exists, they are also 
handled by Gencot when closing a JSON description: for all required invocations of a struct or union member a 
description template is added which is identified by the member name and the tag of the struct or union type. Invocations
of function pointer members in anonymous struct or union types are ignored by the current Gencot version.
 
\subsection{Described Entities}

Gencot uses the information about parameter modifications when it maps a C function type or function pointer type
to Cogent, as described in Section~\ref{design-types-function}. 

C function types are only used when functions are declared or defined. Function definitions are translated to
Cogent for all functions defined in the Cogent compilation unit. Function declarations are translated to Cogent
for all functions invoked but not defined in the Cogent compilation unit. In both cases Gencot automatically 
determines the corresponding definitions or declarations and generates the JSON descriptions for them, to be
filled by the developer and read when translating the C code to Cogent.

C function pointer types can be used as type for several entities: for global and local variables, for members of struct
and union types, for parameters and the result of declared and defined functions, and as base type or parameter 
types in derived types.

As described in Section~\ref{design-fundefs-body}, local variable declarations are not translated by Gencot, hence
their types need not be mapped and no JSON description is provided if it is a function pointer type. In all other 
cases, however, the declaration is translated to Cogent and the type must be mapped. For every C source file (also 
\code{.h} files), the current version of Gencot provides JSON descriptions of all function pointer types which are 
used
\begin{itemize}
\item for a global variable defined in the file (including those with internal linkage, see Section~\ref{design-files}),
\item for a member of a tagged struct type defined in the file,
\item for a parameter of a function defined in the file,
\item as element type of an array type used for one of the kinds of entities specified above.
\end{itemize}
These are the cases where it is easy to create a unique identifier for the function pointer entities, which can be
understood by the developer and used to associate the description to the entity when translating it to Cogent. The
description is provided independent of whether the entity has parameters of linear type or not. If not, it is
not needed by Gencot for translating the entity definition, however it may be needed as required invocation for
calculating information for parameters depending on it.

In all other cases, such as a parameter type of another function pointer type, or a member of a tagless struct,
Gencot does not provide or use
JSON descriptions when mapping the type. For parameters of linear type it always assumes that they may be modified
and are not discarded and maps the type to a corresponding Cogent type name.

\subsection{Function Pointer Type Names}
\label{design-parmod-typedef}

In C, in the definition of an entity of function pointer type, the type may be specified directly as a derived
type, or it may be specified by a typedef name. In the latter case, several entities may share a common
function pointer type which is specified in the type name definition.

The parameter modification descriptions are needed for all single function pointer entities, since these are
used in invocations and may be required for calculating dependencies for parameters of other functions. If 
several entities use a common typedef name, they must agree on the parameter modification description and
the common description must be known to Gencot when translating the type name definition.

The situation can be even more complex if first a typedef name for a function type is defined and then several
different typedef names for function pointer types or function pointer array types are derived from it. All
these typedef names and the corresponding entities must agree on the parameter modification description.

However, entities using a common typedef name may be defined in different C source files and even in different
C compilation units, some of which belong to the Cogent compilation unit and others are external to it, but may
be added later. This makes it impossible to guarantee that the descriptions for all entities are available
and agree, if an incremental translation to Cogent shall be supported.

For this reason Gencot does not relate entities with a common typedef name and does not test whether their
descriptions agree, they are processed completely independently. For translating the type name definition,
Gencot uses another parameter modification description, which is also independent from those of the entities.
It is the task of the developer to take care that all these descriptions agree. If they do not, this will 
cause inconsistent uses of the Cogent type to which the common type name is mapped in the generated Cogent code.

Note that, in contrast to the entities, the defined type may directly be a function type. As described in 
Section~\ref{design-types-typedef} such a type definition is not translated to a definition for a Cogent function
type, instead it is translated in the same way as for a function pointer type, mapping to a generated Cogent
abstract type name. This makes it possible to translate derived pointer types and pointer array types to 
Cogent using the mapped type name only, the parameter modification description is only needed for translating
the original function type definition. For example the C type name definitions
\begin{verbatim}
  typedef int fun(int,char*);
  typedef fun *pfun;
\end{verbatim}
where the second parameter may be modified, are translated to the Cogent type definitions
\begin{verbatim}
  type Cogent_fun = #F_'U32'LP_U8'_U32
  type Cogent_pfun = #Cogent_fun
\end{verbatim}
so that the parameter modification description is only needed for the first type definition translation.

The additional parameter modification descriptions are associated with the common type name of the definition
which directly contains the derived function type. For every C source file (also 
\code{.h} files) Gencot provides JSON descriptions for all such type name definition in the file, where the defined
type is a function type, a function pointer type, or an array type with a function pointer type
as element type. The
description is provided independent of whether the function type has parameters of linear type or not. If not,
the description is not used at all by Gencot, but it may be useful as additional information for the developer.

More complex types involving a derived function type are ignored. As for the corresponding
entities Gencot assumes for their translation that all parameters of linear type may be modified.



\chapter{Implementation}

Gencot is implemented by a collection of unix shell scripts using the unix tools \code{sed}, \code{awk}, and the 
C preprocessor \code{cpp} and by Haskell programs using the C parser \code{language-c}. 

Many steps are implemented as Unix filters, reading from standard input and writing to standard output. A filter
may read additional files when it merges information from several steps. The filters
can be used manually or they can be combined in scripts or makefiles. Gencot provides some predefined scripts
for filter combinations.

Steps which process a set of input files where the number of files is not restricted are implemented in the usual
way by taking the list of file names as arguments.

\section{Origin Positions}
\label{impl-origin}

Since the \code{language-c} parser does not support parsing preprocessor directives and C comments, the general approach
is to remove both from the source file, process them separately, and re-insert them into the generated files.

For re-inserting it must be possible to relate comments and preprocessor directives to the generated target code parts. As 
described in Sections~\ref{design-comments} and~\ref{design-preprocessor}, comments and preprocessor directives are associated
to the C source code via line numbers. Whenever a target code part is generated, it is annotated with the line numbers of its
corresponding source code part. Based on these line number annotations the comments and preprocessor directives can be
positioned at the correct places.

The line number annotations are markers of one of the following forms, each in a single separate line:
\begin{verbatim}
  #ORIGIN <bline>
  #ENDORIG <aline>
\end{verbatim}
where \code{<bline>} and \code{<aline>} are line numbers.

An \code{\#ORIGIN} marker specifies that the next code line starts a target code part which was generated from a source code
part starting in line \code{<bline>}. An \code{\#ENDORIG} marker specifies that the previous code line ends a target code part
which was generated from a source code part ending in line \code{<aline>}. Thus, by surrounding a target code part with an 
\code{\#ORIGIN} and \code{\#ENDORIG} marker the position and extension of the corresponding source code part can be derived.

In the case of a structured source code part the origin marker pairs are nested, if the target code part generated from a subpart 
is nested in the target code part generated from the main part. If there is no code generated for the main part, the 
\code{\#ORIGIN} marker for the first subpart immediately follows the \code{\#ORIGIN} marker for the main part and the 
\code{\#ENDORIG} marker for the last subpart is immediately followed by the \code{\#ENDORIG} marker for the main part.

If no target code is generated from a source code part, the origin markers are not present. This implies, that an 
\code{\#ORIGIN} marker is never immediately followed by an \code{\#ENDORIG} marker.

It may be the case that several source code parts follow each other on the same line, but the corresponding target code
parts are positioned on different lines. Or from a single source code part several target code parts on different lines 
are generated. In both cases there are several origin markers with the same line number. Conditional preprocessor directives
associated with that line must be duplicated to all these target code parts. For comments, however, duplication is not
adequate, they should only be associated to one of the target code parts. This is implemented by appending an additional 
``+'' sign to an origin marker, as in 
\begin{verbatim}
  #ORIGIN <bline> +
  #ENDORIG <aline> +
\end{verbatim}
Comments are only associated with markers where the ``+'' sign is present, all other markers are ignored. In this way,
the target code generation can decide where to associate comments, if a position is not unique.

Gencot uses the filter \code{gencot-reporigs} for removing repeated origin markers from the generated target code, as 
described in Section~\ref{impl-ccode-origin}.


\section{Parameter Modification Descriptions}
\label{impl-parmod}
As described in Section~\ref{design-parmod} Gencot uses a parameter modification description in JSON format 
to be filled in collaboration with the developer to determine which function parameters may be modified during
a function call. 

\subsection{Description Structure}
\label{impl-parmod-struct}

This
description is structured as follows. It is a list of JSON objects, where each object is an entry which describes a function using
the following attributes:
\begin{verbatim}
    { "f_name" : <string>
    , "f_comments" : <string>
    , "f_def_loc" : <string>
    , "f_heap" : <string>
    , "f_num_params" : <int> or <string>
    , "f_result" : <string>
       <parameter descriptions>
    , "f_num_invocations" : <int>
    , "f_invocations" : <list of invocation descriptions>
    }
\end{verbatim}

The attribute \code{f\_name} specifies a unique identifier for the function, as described in Section~\ref{impl-parmod-ids}.
The attribute \code{f\_comments} is not used by Gencot, it can be used by the developer to add arbitrary textual
descriptions to the function entry. The attribute \code{f\_def\_loc} specifies the name of the C source file containing
the definition of the function or function pointer (or its declaration, when the function entry has been 
generated for closing the JSON description). 

The attribute \code{f\_num\_params} specifies the number of parameters. In the case of a variadic function or a function
with incomplete type (which may be the case if the entry has been generated from a declaration), it is specified
as \code{"variadic"} or \code{"unknown"}, respectively. The attribute \code{f\_result} specifies the identifier of
the parameter which is returned as function result (typically after modifying it). If the result is not one of the 
function parameters, the attribute is not present. This attribute is never generated by Gencot, it must be added manually
by the developer. 

The attribute \code{f\_heap} specifies whether the function uses the heap. Possible values are \code{"yes"}, \code{"no"},
\code{"depends"}, or \code{"?depends"}. The last value is generated by Gencot if heap usage cannot be decided locally.
The value \code{"depends"} is never generated by Gencot, it must be set by the developer to confirm that heap usage 
must be determined transitively.

All known parameters of the function are described in the \code{<parameter descriptions>}. 
Every parameter description consists of a single attribute where the parameter identifier (see Section~\ref{impl-parmod-ids})
is the attribute name. The value is one of the following strings:
\begin{description}
\item[\code{"nonlinear"}] According to its type the parameter is not a pointer and its value does not contain pointers directly or
indirectly.
\item[\code{"readonly"}] The parameter is not \code{"nonlinear"} but according to its type all pointers in the parameter have a \code{const} qualified referenced type.
\item[\code{"yes"}] The parameter is neither \code{"nonlinear"} nor \code{"readonly"} and it is directly modified by the function.
\item[\code{"discarded"}] The parameter is neither \code{"nonlinear"} nor \code{"readonly"} and it is directly discarded 
(``freed'') by the function. 
\item[\code{"depends"}] The parameter is neither \code{"nonlinear"} nor \code{"readonly"}, it is neither modified 
or discarded directly, but it may be modified by an invoked function. 
\item[\code{"no"}] None of the previous cases applies to the parameter.
\item[\code{"?"}] The parameter is neither \code{"nonlinear"} nor \code{"readonly"}, but the remaining properties are unconfirmed.
\item[\code{"?depends"}] The parameter is neither \code{"nonlinear"} nor \code{"readonly"} and it may be modified by an invoked function,
but the remaining properties are unconfirmed.
\end{description}

Gencot only generates parameter descriptions with the values \code{"nonlinear"}, \code{"readonly"}, \code{"yes"}, \code{"?"}, and \code{"?depends"}.
The first two can be safely determined from the C type. If Gencot finds a direct modification, it sets the description to 
\code{"yes"}. Otherwise, if it finds a dependency on an invocation, it sets the description to \code{"?depends"}. Otherwise
it sets it to \code{"?"}.

The task for the developer is to check all unconfirmed parameter descriptions by inspecting the source code. If a local modification
is found, the value must be changed to \code{"yes"}. Otherwise, if the description was \code{"?depends"} it must be confirmed 
by changing it to \code{"depends"}. Otherwise, if a dependency is found, the value \code{"?"} must be changed to \code{"depends"},
otherwise it must be set to \code{"no"}.

Discarding a parameter is normally only possible by invoking the function ``free'' in the C standard library. No other function
can directly discard one of its parameters. However, a parameter may be discarded by invoking an external function which indirectly
invokes free. The task for the developer is to identify all these cases where an external function (where the entry has been
generated during closing the JSON description) discards one of its parameters and set its parameter description to \code{"discarded"}.

It may be the case that a C function modifies a value referenced by a parameter and returns the parameter as its result, such as 
the C standard function \code{memcpy}. In this case the parameter need not be added to the Cogent result tuple, since it already
is a part of it. To inform Gencot about this case the developer has to add a \code{f\_result} attribute to the function
description. As an example, the description for the C function \code{memcpy} should be
\begin{verbatim}
    { "f_name" : "memcpy"
    , "f_comments" : ""
    , "f_def_loc" : "string.h"
    , "f_heap" : "no"
    , "f_num_params" :3
    , "f_result" :"1-__dest"
    , "1-__dest" :"yes" 
    , "2-__src" :"readonly" 
    , "3-__n" :"nonlinear" 
    }
\end{verbatim}
since it always returns its first parameter as result.

The attribute \code{f\_num\_invocations} specifies the number of function invocations found in the body of the described function.
If the same function is invoked several times, it is only counted once. The attribute \code{f\_invocations} specifies a list
of JSON objects where each object describes an invocation. If the function entry describes a function pointer or an external
function, no body is available, so no invocations can be found and both attributes are omitted. If no parameter depends on
any invocation, the second attribute (invocation list) is omitted, only the number of invocations is specified.

An entry in the list of invocations describes an invocation using the following attributes:
\begin{verbatim}
    { "name" : <string> 
    , "heap_depends" : <string>
    , "num_params" : <int> or <string>
      <argument descriptions>
    }
\end{verbatim}

The attribute \code{"name"} specifies the function identifier of the invoked function. 

The attribute \code{"heap\_depends"} specifies whether the heap usage may depend on the invocation of this function.
Possible values are \code{yes}, \code{no}, and \code{?}. The last value is the default generated by Gencot, it must be
confirmed to \code{yes} or \code{no} by the developer. This attribute is only used if the attribute \code{"f\_heap"}
for the invoking function has the value \code{depends}. Otherwise all \code{"heap\_depends"} attributes at the 
invocations are ignored.

The attribute \code{"num\_params"}
specifies the number of parameters according to the type of the invoked function, in the same way as the attribute
\code{"f\_num\_params"}. If an invoked function has no parameters, no entry for it is added to the invocation list.

The \code{<argument descriptions>} describe all known arguments of invocations of the function. When Gencot creates an
invocation description, it inserts argument descriptions according to the maximal number of arguments found in an 
invocation of this function. Thus, also for invocations of incompletely defined or variadic functions, an argument
description is present for every actual argument used in an invocation.

Every argument description consists of a single attribute where the attribute name is the parameter identifier of the 
parameter corresponding to the argument. The value is one of the strings \code{"nonlinear"} or \code{"readonly"}
(according to the type of the parameter of the invoked function), or a list of parameter identifiers.

If the value is \code{"nonlinear"} or \code{"readonly"} parameter dependencies on this argument are irrelevant, since
the invoked function cannot modify or discard it. Otherwise, the list specifies parameters of the \textit{invoking}
function for which the modification or discarding depends on whether the invoked function modifies or discards this
argument. A parameter in the list is only effective, if its description is specified as \code{depends}. In all other
cases the parameter is ignored when it appears in lists of invocation arguments.

The task for the developer is for all unconfirmed parameter descriptions of the invoking function to check whether
there are (additional) dependencies on arguments of invoked functions and add these dependencies to the argument 
descriptions.

\subsection{Identifiers for Functions and Parameters}
\label{impl-parmod-ids}

In the JSON description unique identifiers are needed for all described functions, so that they can be referenced by
invocations. Gencot uses the item identifiers as defined in Section~\ref{impl-itemprops-ids} as function identifiers.
An item identifier is a function identifier, if the item has a derived function type.

In the JSON description unique identifiers are also needed for all described function parameters, so that they can 
be referenced by argument descriptions. Since these references are always local in a function description, parameter
identifiers need only be unique per function. Therefore the C parameter name is usually sufficient as id in the 
JSON description.

However, the parameter name is not always available: If the function has an incomplete type no parameter names are
specified, if the function is variadic, only the names of the non-variadic parameters are specified. Therefore
Gencot always uses the position number as parameter id, where the first parameter has position 1. To make the JSON
description more readable for the developer, Gencot appends the parameter name whenever it is available. Together,
a parameter id is a string with one of the forms
\begin{verbatim}
    <pos>
    <pos>-<name>
\end{verbatim}
where \code{<pos>} is the position number and \code{<name>} is the declared parameter name.

Since all parameter identifiers are strings they can be used as JSON attribute names and since they always
start with a digit they can be recognized and do not conflict with other JSON attribute names. 

When Gencot reads and processes a JSON parameter modification description it removes the optional name from 
all parameter identifiers and uses only the position. Hence parameter identifiers are considered equal if they
begin with the same position.

\subsection{Example}

The following example illustrates the format of the parameter modification descriptions. It consists of two
function descriptions, one for a defined function with internal linkage and one for a function pointer parameter
invoked by that function.
\begin{verbatim}
[
    { "f_name" :"app:somefun" 
    , "f_comments" :"" 
    , "f_def_loc" :"app.c" 
    , "f_heap" :"depends"
    , "f_num_params" :5
    , "1-f_sort" :"nonlinear" 
    , "2-desc" :"readonly" 
    , "3-input" :"?depends" 
    , "4-size" :"nonlinear" 
    , "5-result" :"yes" 
    , "f_num_invocations" :2
    , "f_invocations" :
        [ 
            { "name" :"memcpy" 
            , "heap_depends" :"no"
            , "num_params" :3
            , "1-__dest" :[ "3-input" ]
            , "2-__src" :"readonly" 
            , "3-__n" :"nonlinear" 
            } 
        , 
            { "name" :"app:somefun/f_sort/*" 
            , "heap_depends" :"yes"
            , "num_params" :3
            , "1-arr" :[ "3-input" ]
            , "2-h" :[]
            , "3-len" :"nonlinear" 
            } 
        ] 
    } 
, 
    { "f_name" :"app:somefun/f_sort/*"
    , "f_comments" :"" 
    , "f_def_loc" :"app.c"
    , "f_heap" : "no"
    , "f_num_params" :3
    , "1-arr" :"?" 
    , "2-h" :"?" 
    , "3-len" :"nonlinear" 
    } 
]
\end{verbatim}
The description is not closed, since it does not contain an entry for the invocation \code{memcpy}. This
invocation is required, since parameter \code{input} of \code{somefun} depends on its first argument
which is neither nonlinear nor readonly. However, it is not required for deciding heap use, since that
is specified as not depending on \code{memcpy}. If the description of parameter \code{input} is changed
to \code{"yes"}, the description is closed, since now the invocation \code{memcpy} is not required
anymore.

\subsection{Reading and Writing Json}
\label{impl-parmod-json}

Input and output of JSON data is done using the package \code{Text.Json}. There is another package \code{Data.Aeson} for
Json processing which supports a much more flexible conversion between JSON and Haskell types, but this is not
needed here.

The package \code{Text.Json} uses the type \code{JSValue} to represent an arbitrary JSON value. It provides the functions
\code{encode} and \code{decode} to convert between \code{JSValue} and the JSON string representation. Depending
on the kind of actual value, a \code{JSValue} can be converted from and to corresponding Haskell types using
the functions \code{readJSON} and \code{showJSON}. A JSON object is converted to the type \code{JSObject a} where
\code{a} is the type of attribute values, usually this is again \code{JSValue}. A \code{JSObject a} can be further
processed by converting it from and to an attribute-value list of type \code{[(String,a)]} using the functions
\code{fromJSobject} and \code{toJSObject}.

The Gencot parameter modification description is represented as a list of Json objects of type \code{[JSObject JSValue]}.
The same representatiom is used for the contained invocation description lists. For internal processing the objects
are converted to attribute-value lists. All processing of JSON data is directly performed on these lists.

A parameter modification description is read by applying the \code{decode} function to the input to yield a
list of \code{JSObject JSValue}. It is output by applying the \code{encode} function to such a list. 

Additionally,
the resulting string representation is formatted using the general prettyprint package \code{Text.Pretty.Simple}.
The function \code{pStringNoColor} is used since it does not insert control sequences for colored represenation
and produces a plainly formatted text representation. Its result is of type \code{Text} and must be converted to a
string using the function \code{unpack} from package \code{Data.Text}.

\subsection{Evaluating a Description}
\label{impl-parmod-eval}

A parameter modification description is complete, if all paremeter descriptions are confirmed and the description 
is ``closed'', i.e., has no required invocations. This implies that whenever a parameter is dependent on an invocation argument,
there is a function description present for the invoked function where the corresponding parameter is described.
This allows to eliminate all parameter dependencies by following them until an independent parameter description is found.
This process is called ``evaluation'' of the parameter modification description.

In the same way, whenever the heap use depends on an invoked function, there is a function description present for the 
invoked function where the heap use is either specified directly or again depends on further invocations.

The result of evaluation is a simplified parameter modification description where the value \code{"depends"} does not 
occur anymore as parameter description or heap use specification. Additionally, all information about function invocations 
is removed, since it is no more needed.

Evaluation only terminates, if there are no cyclic parameter dependencies. This property is checked by Gencot. If there
are cyclic parameter dependencies in the C code they must be eliminated manually by the developer by removing enough dependencies 
to break all cycles.

A parameter may have several dependencies, which may result in different description values. This cannot be the case for
the values \code{"nonlinear"} and \code{"readonly"}: If the parameter of the invoked function has nonlinear type, this
also holds for the passed parameter, no dependency can exist in this case. If the parameter of the invoked function has
a readonly type, the passed parameter can still have a linear type which is not readonly. But it cannot be modified by the
function invocation, hence no dependency can exist in this case either.

Thus, after evaluation, a parameter may have any subset of the values \code{"yes"}, \code{"discarded"}, and \code{"no"}, meaning 
that it may be modified by some invocations, discarded by some invocations, and not modified by others. This is reduced to
a single value as follows. The value \code{"no"} is only used if none of the other two is present. If both \code{"yes"}
and \code{"discarded"} are present, Gencot cannot decide whether the parameter is always discarded (perhaps after modification),
or always modified (perhaps after discarding and reallocating it). In this case it always assumes a modification and
uses the single value \code{"yes"}. If this is not correct it must be handled manually by the developer.

The reason for this treatment is that before evaluation Gencot gives a local modification a higher priority than a dependency,
to reduce the number of required invocations. This may hide a dependency which results in discarding the parameter. To
be consistent, Gencot also prefers modifications resulting from dependencies after evaluation over discarding. 

Since in evaluated parameter modification descriptions all parameter descriptions are still confirmed, the only values
possible for a parameter description are \code{"nonlinear"}, \code{"readonly"}, \code{"yes"}, \code{"discarded"}, and \code{"no"}.
This information is used for declaring the properties Read-Only and Add-Result for the function parameters.

\subsection{Haskell Modules}
\label{impl-parmod-modules}

Parameter modification descriptions are implemented in the following three Haskell modules.

Module \code{Gencot.Json.Parmod} defines the following types for representing parameter modification descriptions:
\begin{verbatim}
  type Parmod = JSObject JSValue
  type Parmods = [Parmod]
\end{verbatim}
They are used for constructing and processing the JSON representation. Type \code{Parmod} represents a single
function description as a JSON object with arbitrary JSON values. Type \code{Parmods} represents a full parameter modification
description consisting of a sequence of function descriptions.

Module \code{Gencot.Json.Translate} defines the translation from C source code to parameter modification descriptions
in JSON format. The monadic action
\begin{verbatim}
  transGlobals :: [DeclEvent] -> CTrav Parmods
\end{verbatim}
translates a sequence of global declarations and definitions (see Section~\ref{impl-ccode-read}) to a sequence of
function descriptions. It translates all definitions and declarations of functions and of objects with function pointer
type or function pointer array type. It translates all type name definitions for a function type, a function pointer type, 
or a function pointer array type.
For every definition of a compound struct or union type it translates all members with function pointer type or function 
pointer array type.
All other \code{DeclEvent}s are ignored.

Module \code{Gencot.Json.Process} defines functions for reading and processing parameter modification descriptions in JSON format.
The main functions are 
\begin{verbatim}
  showRemainingPars :: Parmods -> [String]
  getRequired :: Parmods -> [String]
  filterParmods :: Parmods -> [String] -> Parmods
  mergeParmods :: Parmods -> Parmods -> Parmods
  sortParmods :: Parmods -> [String] -> Parmods
  addParsFromInvokes :: Parmods -> Parmods
  evaluateParmods :: Parmods -> Parmods
  convertParmods :: Parmods -> ItemProperties
\end{verbatim}
The first function retrieves the list of all unconfirmed parameters in the form
\begin{verbatim}
  <function identifier> / <parameter identifier>
\end{verbatim}
The function \code{getRequired} retrieves the function identifiers of all invocations on which at least one parameter depends.
The function \code{filterParmods} restricts a parameter modification description to the descriptions of all function
where the function identifier belongs to the string list. 
The function \code{mergeParmods} merges two parameter modification description. If a function is described in both, the description
with less unconfirmed parameter descriptions is used. If the same number of parameter descriptions are confirmed 
always the description in the first sequence is selected. The function \code{sortParmods} sorts the function descriptions
in a parameter modification description according to the order specified by a list of function identifiers. The parameter
modification description is reduced to the functions specified in the list. If a function occurs in the list more than once, 
the position of its first occurrence is used for the ordering.
The function \code{evaluateParmods} evaluates a parameter modification description as described in 
Section~\ref{impl-parmod-eval}. 

The function \code{convertParmods} converts the JSON representation to an item
property map (see Section~\ref{impl-itemprops-internal}) with declarations of the properties Read-Only, Add-Result, and Heap-Use.
The parameter modification description must have been evaluated as described in Section~\ref{impl-parmod-eval}. The 
resulting single parameter description values are translated to properties as follows: the description \code{"yes"}
is translated to the Add-Result property, the descriptions \code{"readonly"} and \code{"no"} are translated to the 
Read-Only Property. For the remaining descriptions \code{"nonlinear"} and \code{"discarded"} and the \code{f\_result}
attribute no property is declared. The Heap-Use property is declared, if \code{"f\_heap"} has the value \code{"yes"}.


\section{Comments}
\label{impl-comments}

In a first step all comments are removed from the C source file and are written to a separate file. The remaining 
C code is processed by Gencot. In a final step the comments are reinserted into the generated code.

Additional steps are used to move comments from declarations to definitions.

The filter \code{gencot-selcomments} selects all comments from the input, translates them to Cogent comments and writes 
them to the output.
The filter \code{gencot-remcomments} removes all comments from the input and writes the remaining code to the output.
The filter \code{gencot-mrgcomments <file>} merges the comments in \code{<file>} into the input and writes the
merged code to the output. <file> must contain the output of \code{gencot-selcomments} applied to a code X, the 
input must have been generated from the output of \code{gencot-remcomments} applied to the same code X.

\subsection{Filter \code{gencot-remcomments}}

The filter for removing comments is implemented using the C preprocessor with the option \code{-fpreprocessed}. With this
option it removes all comments, however, it also processes and removes \code{\#define} directives. To prevent this, a sed
script is used to insert an underscore \code{\_} before every \code{\#define} directive which is only preceded 
by whitespace in its line. Then it is not recognized by the preprocessor. Afterwards, a second sed script removes the underscores.

Instead of an underscore an empty block comment could have been used. This would have the advantage that the second sed script
is not required, since the empty comments are removed by the preprocessor. The disadvantage is that the empty comment
is replaced by blanks. The resulting indentation does not modify the semantics of the \code{\#define} statenments but it
looks unusual in the Cogent code.

The preprocessor also removes \code{\#undef} directives, hence they are treated in the same way.

The preprocessor preserves all information about the original source line numbers, to do so it may insert 
line directives of the form \code{\# <linenumber> <filename>}. They must be processed by all following filters. The Haskell
C parser \code{language-c} processes these line directives.

\subsection{Filter \code{gencot-selcomments}}

The filter for selecting comments is implemented as an awk script. It scans through the input for the comment start
sequences \code{/*} and \code{//} to identify comments. It translates C comments to Cogent comments in the output.
The translation is done here since the filter must identify the start and end sequences of comments, so it can 
translate them specifically. Start and end sequences which occur as part of comment content are not translated.

To keep it simple, the cases when the comment start sequences can occur in C code parts are ignored. This may lead to
additional or extended comments, which must be corrected manually. It never leads to omitted comments or missing 
comment parts. Note that \code{gencot-remcomments} always identifies comments correctly, since there comment detection
it is implemented by the C preprocessor.

To distinguish before-units and after-units, \code{gencot-selcomments} inserts a separator between them. The separator
consists of a newline followed by \code{-\}\_}. It is constructed in a way that it cannot be a part of or overlap with a 
comment and to be easy to detect when processing the output of \code{gencot-selcomments} line by line. The newline and
the \code{-\}} would end any comment. The underscore (any other character could have been used instead) distinguishes the
separator from a normal end-of-comment, since \code{gencot-selcomments} never inserts an underscore immediately after a comment. 

The separator is inserted after every unit, even if the unit is empty. The first unit in the output of \code{gencot-selcomments}
is always a before-unit.

When in an input line code is found outside of comments all this code with all embedded comments is replaced by the
separator. Only the comments before and after the code are translated to the output, if present.
Note, that the separator includes a newline, hence every source line with code outside of
comments produces two output lines. 

An after-unit consists of all comments after code in a line. The last comment is either a line comment or
it may be a block comment which may include following lines. After this last comment the after-unit ends and
a separator is inserted.

All whitespace in and between comments and before the first comment in a before-unit is preserved in the output, 
including empty lines. After a before-unit only empty lines are preserved. Whitespace around code is typically used
to align code and comments, this must be adapted manually for the generated target code. Whitespace after an after-unit
is not preserved since the last comment in an after unit in the target code is always followed by a newline.

The filter never deletes lines, hence in its output the original line numbers can still be determined by counting lines, 
if the newlines belonging to the separators are ignored.

\subsubsection{State Machine}

The implementation processes the input line by line using a finite state machine. It uses the variables \code{before} and 
\code{after} to collect block comments at the beginning and end of the current line, initially both are empty. 
The collect action appends the input from the current position up to and including the next found item in 
the specified variable. The separate action appends the separator to the specified variable. The output action 
writes the specified content to the output, replacing C comment start and end sequences by their Cogent counterpart.
The newline action advances to the beginning of the next line and clears \code{before}.

The state machine has the following states, nocode is the initial state:

\begin{description}
\item[nocode] If next is
  \begin{description}
  \item[end-of-line] output(\code{before}); newline; goto nocode
  \item[block-comment-start] collect(\code{before}); goto nocode-inblock
  \item[line-comment-start] collect(\code{before}); output(\code{before} + line-comment); newline; goto nocode
  \item[other-code] separate(\code{before}); clear(\code{after}); goto code
  \end{description}
\item[nocode-inblock] If next is
  \begin{description}
  \item[end-of-line] collect(\code{before}); output(\code{before}); newline; goto nocode-inblock
  \item[block-comment-end] collect(\code{before}); goto nocode
  \end{description}
\item[code] If next is
  \begin{description}
  \item[end-of-line] output(\code{before} + separator); newline; goto nocode
  \item[block-comment-start] append comment-start to \code{after}; goto code-inblock
  \item[line-comment-start] output(\code{before} + line-comment + separator); newline; goto nocode
  \end{description}
\item[code-inblock] If next is
  \begin{description}
  \item[end-of-line] collect(\code{after}); output(\code{before} + \code{after}); newline; goto aftercode-inblock
  \item[block-comment-end] collect(\code{after}); goto code-afterblock
  \end{description}
\item[code-afterblock] If next is
  \begin{description}
  \item[end-of-line] output(\code{before} + \code{after} + separator); newline; goto nocode
  \item[block-comment-start] collect(\code{after}); goto code-inblock
  \item[line-comment-start] collect(\code{after}); output(\code{before} + \code{after} + line-comment + separator); newline; goto nocode
  \item[other-code] clear(\code{after}); goto code
  \end{description}
\item[aftercode-inblock] If next is
  \begin{description}
  \item[end-of-line] collect(\code{before}); output(\code{before}); newline; goto aftercode-inblock
  \item[block-comment-end] collect(\code{before}); separate(\code{before}); goto nocode
  \end{description}
\end{description}

\subsection{Filter \code{gencot-mrgcomments}}

The filter for merging comments into the target code is implemented as an awk script. It consists of a BEGIN rule, a line rule, and
an END rule. The BEGIN rule reads the <file> line by line and collects before- and after-units as strings in the arrays \code{before} and \code{after}. The arrays are indexed with the (original) line number of the separator between before- and after-unit.

The line rule uses a buffer for its output. It is used to process all \code{\#ORIGIN} and \code{\#ENDORIG} markers belonging to
a line and collect the associated comment units and content. The END rule is used to flush the buffer.

For every consecutive sequence of \code{\#ORIGIN} markers, the before units associated with the line numbers of all markers with 
a ``+'' sign are collected in a buffer. Every single code line is put in a second buffer. For every consecutive sequence of
\code{\#ENDORIG} markers, the after units associated with the line numbers of all markers with 
a ``+'' sign are collected in a third buffer. Whenever a code line or an \code{\#ENDORIG} marker is followed by a line which 
is no \code{\#ENDORIG} marker, the content of all three buffers is output and the buffers are reset.

In the buffer, before-units are concatenated without any separator. After-units are separated by a newline to end possibly 
trailing line comments.





\subsection{Declaration Comments}

To safely detect C declarations and C definitions Gencot uses the language-c parser. 

Only comments associated with declarations with external linkage are transferred to their definitions. For declarations
with internal linkage the approach for transferring the comments does not work, since the declared names need not be
unique in the <package>.

Processing the decalaration comments is implemented by the following filter steps. 

\subsubsection{Filter \code{gencot-deccomments}}

The filter \code{gencot-deccomments} parses the input. For every declaration it outputs a line
\begin{verbatim}
  #DECL <name> <bline> <aline>
\end{verbatim}
where \code{<name>} is the name of the declared item, \code{<bline>} is the original source line 
number where the declaration begins and \code{<aline>} is
the original source line number where the declaration ends. (In many cases the declaration will be
a single line and \code{<bline>} and \code{<aline>} will be the same.)

\subsubsection{Filter \code{gencot-movcomments}}

The filter \code{gencot-movcomments <file>} processes the output of \code{gencot-deccomments} as input. For every
lines as above, it retrieves the before-unit of \code{<bline>} and the after-unit of \code{<aline>} from \code{<file>}
and stores them in the files \code{before-<name>.comment} and \code{after-<name>.comment} in the
current directory. The content of \code{<file>} must be the output of \code{gencot-selcomments} applied to the same
original source from which the input of \code{gencot-deccomments} has been derived.

The filter is implemented as an awk script. It consists of a 
BEGIN rule reading \code{<file>} in the same way as \code{gencot-mrgcomments}, and a rule for lines starting with 
\code{\#DECL}. For every such line it writes the associated comment units to the comment files. A comment
file is even written if the comment unit is empty.

\subsubsection{Filter \code{gencot-defcomments}}

For inserting the comments around the target code parts generated from a definition, Gencot uses the markers
\begin{verbatim}
  #DEF before-<name>
  #DEF after-<name>
\end{verbatim}
The markers must be inserted by the filter which generates the definition target code.

The filter \code{gencot-defcomments <dir>} replaces every marker line in its input by the content of the corresponding
\code{.comment} file in directory \code{<dir>} and writes the result to its output.

It is implemented as an awk script with a single rule for all lines. If the line starts with \code{\#DEF} it is
replaced by the content of the corresponding file in the output. All other lines are copied to the output without 
modification.

 


\section{Preprocessor Directives}
\label{impl-preprocessor}
\subsection{Filters for Processing Steps}

Directive processing is done for the output of \code{gencot-remcomments}. All comments have been removed. 
However, there may be line directives present. 

The filter \code{gencot-selpp} selects all preprocessor directives and
copies them to the output without changes. All other lines are replaced by empty lines, so that the original
line numbers for all directives can still be determined. 

The filter \code{gencot-selppconst <file>} selects from the result of \code{gencot-selpp} all macro definitions
which shall be processed as preprocessor defined constant. If \code{<file>} is specified it must 
contain the Gencot manual macro list (see Section~\ref{design-preprocessor-macros}).

The filter \code{gencot-rempp <file>} removes all preprocessor directives from its input, replacing them by empty
lines. All other lines are copied to the output without modification. If \code{<file>} is specified it must 
contain the Gencot directive retainment list (see Section~\ref{design-preprocessor-config}) of regular 
expressions for directives which shall be retained.

The filter \code{gencot-unline} is a utility filter. It expects line directives in its input. It removes all
included content (detected by the file specification in the line directives) and expands line directives
in the content from \code{<stdin>} to sequences of empty lines.

How the directives are processed depends on the kind of directives (see Section~\ref{design-preprocessor}).
Gencot provides the processing filters \code{gencot-prcppflags <file>}, \code{gencot-prcppconst} and 
\code{gencot-prcppincl <file>}. 
Conditional directives are not processed, they are inserted without changes. Other macro
definitions are processed manually, the result is simply inserted in the target code. Flags are not translated,
but they must be selected and Origin markers must be added.

Conditional directives are merged into the target code in a different way, therefore Gencot provides 
the specific merging filter \code{gencot-mrgppcond <file>} for them. It also separates the conditional
directives from other directives and adds Origin markers. All other directives are merged into the 
target code by filter \code{gencot-mrgpp <file>}. Both filters merge the content in \code{<file>} 
into the input and write the merged code to the output. \code{<file>} must contain the directives to be merged,
for \code{gencot-mrgpp} they must have been marked with Origin markers.

Constant definitions must be preprocessed to map constants names in the replacement bodies (see below). This
is done by filter \code{gencot-preppconst}.

Configuration files typically consist of preprocessor flag definitions, where some of them are commented out.
Gencot supports translating configuration files by providing two additional filters. The filter \code{gencot-preconfig}
uncomments all commented macro definitions. The filter \code{gencot-postconfig <file>} re-comments the 
generated target code, it takes the original configuration file as its argument.

\subsection{Separating Directives}

The Gencot directive retainment list is used to retain some directives in the output of \code{gencot-rempp}.
These directives are still selected by \code{gencot-selpp} and re-inserted by \code{gencot-mrgcond}, if they
are not suppressed during directive processing. For conditional directives it is intended to always re-insert
them. Preprocessor defined constants are never suppressed. Other macros can be suppressed by not specifying
a manual macro definition translation.

\subsubsection{Filter \code{gencot-selpp}}

The filter for selecting preprocessor directives from the input for separate processing and insertion into
the generated target code is implemented as an awk script.

It detects all kinds of preprocessor directives (including line directives), which always begin at the 
beginning of a separate line.
A directive always ends at the next newline which is not preceded by a backslash \code{\\}. All corresponding
lines are copied to the output without modifications.

Copied directives are normalized in the following sense: All whitespace before and after the leading hash
sign are removed (for line directives a single blank is retained after the hash sign). This is done to 
simplify further directive processing.

Every other input line is replaced by an empty line in the output.

\subsubsection{Filter \code{gencot-selppconst}}

The filter for selecting constant definitions is implemented as an awk script. It is intended to be applied 
to the output of \code{gencot-selpp}, hence it expects normalized preprocessor directives in its input.

Roughly, preprocessor constant definitions are macro definitions without macro parameters but with a 
nonempty replacement body (i.e., not a flag). However, parameterless macros can be used for defining
and inserting all other kinds of C code fragments, such as statements or declarations. So, not all
parameterless macro definitions can be translated to Cogent value definitions.

Parameterless macro definitions can be recognized syntactically, since for a macro with parameters 
an opening parenthesis must follow the macro name without separating whitespace. 

Parameterless macro definitions to be translated to Cogent value definitions can in general not be 
recognized without parsing the replacement body as C code, and even that would not be safe since 
it can consist of arbitrary fragments with other macro calls embedded. Hence Gencot supports the Gencot
manual macro list (see Section~\ref{design-preprocessor-macros})
for cases which cannot be recognized automatically. Since usually these complex cases
are seldom, the Gencot manual macro list consists of a file specifying names of parameterless macros which
should \textit{not} be processed as a preprocessor defined constant. The name of this file is 
passed to the filter \code{gencot-selppconst} as an additional argument.

The filter selects and transfers all constant definitions to its output without modifications. Line directives 
are also transferred. All lines belonging to other directives are replaced by empty lines.

\subsubsection{Filter \code{gencot-rempp}}

The filter for removing preprocessor directives from its input is implemented as an awk script.
Basically, it replaces lines which are a part of a directive by empty lines. However, there are the following
exceptions:
\begin{itemize}
\item line directives are never removed, they are required to identify the position in the original source
during code processing.
\item system include directives are never removed, they are intended to be interpreted by the language-c
preprocessor to make the corresponding information available during code processing. It is assumed that
all quoted include directives have already been processed in an initial step, however, there may also be 
include directives where the file is specified by a macro call, which should normally not be retained.
Therefore system include directives are retained and all other include directives are removed.
\item directives which match a regular expression from the Gencot directive retainment list are not removed, 
they are intended to be interpreted by the language-c preprocessor to suppress information which causes 
conflicts during code processing or for other reasons.
\end{itemize}

For conditional directives always all directives belonging to the same section are treated in the same way.
To retain them the first directive (\code{\#if}, \code{\#ifdef}, \code{\#ifndef}) must match a regular expression
in the list. For all other directives of a section (\code{\#else}, \code{\#elif}, \code{\#endif}) the 
regular expressions are ignored.

The regular expressions are specified in the argument file line by line. Before a directive is matched with 
a regular expression, it is normalized by removing all whitespace around the leading hash sign, hence the
regular expressions can be written without considering such whitespace. An example file content is
\begin{verbatim}
  ^#if[[:blank:]]+!?[[:blank:]]*defined\(SUPPORT_X\)
  ^#define[[:blank:]]+SUPPORT_X
  ^#undef[[:blank:]]+SUPPORT_X
\end{verbatim}
It retains all directives which define the macro \code{SUPPORT\_X} or depend on its definition.

\subsubsection{Filter \code{gencot-unline}}

This filter is implemented as an awk script.

Line directives in the input are expanded to the required number of empty lines
which have the same effect. This is done to simplify reading the input for all subsequent filters.

All content which does not origin in file \code{<stdin>} is removed together with its line directives.

\subsection{Processing Directives}

\subsubsection{Processing Constants Defined as Preprocessor Macros}

The replacement body of a preprocessor constant definition may reference the names of other preprocessor
constant definitions. They must be replaced by the mapped Cogent name which corresponds to the name of
the Cogent constant. In general, that would also require at least a lexical analysis of the replacement
bodies to find the preprocessor constant names. Gencot uses a simpler approach and implements this as follows.

A file is generated which contains for every preprocessor constant definition a macro definition 
which maps the original name to the Cogent name. This file is prepended to the original file where all
constant definitions are masked in a way that they are not recognized by the C preprocessor.
The resulting file is processed by the C preprocessor, this will substitute all constant names in the replacement 
bodies by their mapped form. Together, these steps are implemented by the preprocessing filter 
\code{gencot-preppconst}.

Since constant names used in replacement bodies may have been defined in included files, the filter
\code{gencot-preppconst} is intended to be applied to the file after all include directives have been 
expanded, as described for C code in Section~\ref{impl-ccode-include}. In the input line directives
are expected which specify the origin file for all its content. They are transferred to the output.

The output of \code{gencot-preppconst} is then processed by
filter \code{gencot-prcppconst} to translate the constant definitions to Cogent value definitions.

\paragraph{Filter \code{gencot-preppconst}}

The preprocessing filter is implemented as an awk script. It is intended to be applied to the result
of \code{gencot-selppconst}. 

The selected macro definitions in the input are processed twice. In the first phase for every macro 
definition of the form
\begin{verbatim}
  #define CONST1 replacement-body
\end{verbatim}
a definition of the following form is generated:
\begin{verbatim}
  #define CONST1 cogent_CONST1
\end{verbatim}
where \code{cogent\_CONST1} is the result of mapping the name \code{CONST1} to a Cogent variable name
according to Section~\ref{design-names}.

In the second phase a masked definition of the following form is generated:
\begin{verbatim}
  _#define XCONST1 replacement-body
\end{verbatim}
All backslashes marking continuation lines are duplicated, because the C preprocessor removes them.

The sequence of all definitions from the first phase is prepended to the sequence of definitions from the second 
phase. Then the resulting file is piped through the C preprocessor
which applies the definitions from the first phase to the replacement-bodies in the definitions from the
second phase.

The definitions from the first phase are preceded by a line directive for the dummy file name \code{<mappings>}
so that their lines do not count to the content of \code{<stdin>}.

\paragraph{Filter \code{gencot-prcppconst}}

The processing filter is implemented as an awk script. It is intended to be applied to the result of 
\code{gencot-preppconst}. It contains the masked definitions to be processed, both from the original source file
and from all included files. The definitions from included files are not translated to Cogent value
definitions, but they are needed to determine the type and value of externally defined constants.

The type and value needed for the Cogent value definition are determined from the replacement body as follows.

If the body is in the definition line and does not contain any whitespace it may be a C literal. First, 
enclosing parentheses are removed to also recognize replacement bodies of the form \code{(127)}. Then 
the following cases are recognized.
\begin{itemize}
\item if the body starts with a decimal digit and only contains digits and letters it is assumed to be an 
integer literal in decimal, octal (leading zero), or hexadecimal (leading \code{0x}) notation. Its type
is determined to be \code{U8, U16, U32,} or \code{U64} depending on the integer value.
\item if the body starts with a minus sign, followed by an integer literal as above, its type is
determined as \code{U32} and the unsigned value is constructed by calculating the 2-complement.
\item if the body is enclosed in single quotes it is assumed to be a character constant. Its type is
determined as \code{U8}.
\item if the body is enclosed in double quotes it is assumed to be a string constant, its type is determined
as \code{String}.
\item if the body is a single identifier it is resolved using the previous constant definitions. If successful,
the type is determined from that definition. As value the identifier is used.
\end{itemize}

If the body is in the definition line and contains whitespace, it is cecked whether it consists of a sequence
of words which are either string literals or constant names which resolves to type string. In this case it 
is assumed that the value is a sequence of string literals to be concatenated. Its type is determined as 
\code{String}, its value is constructed by concatenating all string values.

In all other cases (also for all bodies using continuation lines) Gencot assumes an expression of type int
and determines the type as \code{U32}. As value the unmodified C expression is used.

Cogent value definitions are only generated from constant definitions belonging to the content of \code{<stdin>}
Every generated Cogent value definition is wrapped in \code{\#ORIGIN} markers according to the line offset
of the original constant definition in \code{<stdin>}.

\subsubsection{Processing Flags}

Processing flags consists only in selecting the flag definitions from all preprocessor directives and adding
\code{\#ORIGIN} markers.

A flag definition is a parameterless macro definition with an empty replacement text. The macro name must 
be followed by optional whitespace and a newline which is not masked by a backslash. Thus every flag definition
occupies exactly one line.

The processing filter \code{gencot-prcppflags} is implemented as an awk script. It is intended to be applied 
to the result
of \code{gencot-selpp} after applying \code{gencot-unline}. The latter removes all directives not contained
in the current source file and expands line directives. 

The processing filter wraps every selected flag definition line in a pair of \code{\#ORIGIN} and \code{\#ENDORIG} 
markers for the corresponding line number.

The Gencot
manual macro list (see Section~\ref{design-preprocessor-macros})
also applies to flags. For flags named in this list their definitions are removed. The name of the list file is 
passed to the filter \code{gencot-prcppflags} as an additional argument.

\subsubsection{Processing Other Preprocessor Macros}

All other preprocessor macros must be processed manually by providing a Gencot macro call conversion and a
Gencot macro translation (see Section~\ref{design-preprocessor-macros}). Each is provided by a manually
prepared file.

The Gencot macro call conversion file is passed as additional input to the language-c preprocessor when
processing the result of \code{gencot-rempp}. The preprocessor prepends the file to its main input thus 
the contained macro definitions are applied to all macro calls for converting them to valid C syntax.

The Gencot macro translation file is passed as <file> argument to a separate execution of \code{gencot-mrgpp}
before the execution of \code{gencot-mrgppcond}. It must contain \code{\#ORIGIN} and \code{\#ENDORIG}
markers which have been manually inserted according to the origin line range of the translated macro
definition. Note that, although \code{gencot-mrgpp} only processes \code{\#ORIGIN} markers, the \code{\#ENDORIG}
markers are required for subsequent steps such as inserting conditional directives and comments.

\subsubsection{Processing Include Directives}

The preprocessing filter \code{gencot-prcppincl} is implemented as an awk script. It is intended to be applied to the result
of \code{gencot-selpp} after applying \code{gencot-unline}. The latter removes all directives not contained
in the current source file and expands line directives. 

The argument file contains the Gencot include omission list (see Section~\ref{design-preprocessor-incl}).

\subsection{Merging Directive Processing Results}

When the conditional directives are merged into the target code, the other 
directives must have already been merged in, since the conditional directives are inserted depending on the content 
of groups and their positions. Therefore, first all other directives must be merged using the filter \code{gencot-mrgpp},
then the conditional directives must be merged using the filter \code{gencot-mrgppcond}.

\subsubsection{Filter \code{gencot-mrgppcond}}

The filter for merging the conditional directives into the target code is implemented as an awk script. As argument
it takes the name of a file containing the directives to be merged. Since conditional directives need not be
processed it is intended that this file contains the output of filter \code{gencot-selpp}, i.e., all directives
selected from the source, processed by \code{gencot-unline} to remove included input and line directives. 
\code{gencot-mrgppcond} selects only the conditional directives and merges them into
the filter input, additionally generating origin markers for every merged directive. 

The filter input must contain the generated Cogent target code and the other preprocessor directives. 
All content in the input must have been marked by origin markers. 

In its BEGIN rule the filter reads the
conditional directives from the argument \code{<file>} and associates them with their line numbers, building the
list of all sections, ordered according to the line number of their first directive (\code{\#if, \#ifdef, \#ifndef}).
Every section is represented by its list of directives.

While processing the input the program maintains a stack of active sections and for every section in the stack the 
active group. A section is active if some of its directives have been output but not all. The active group is that
corresponding to the last directive which has been output for the section, i.e. it is the group which will contain
all target code which is currently output.

The filter only uses the \code{\#ORIGIN} markers to position conditional directives. Gencot assumes that for every 
\code{\#ENDORIG} marker a previous \code{\#ORIGIN} marker exists for the same line number. Since a condition group
always contains a sequence of complete lines, the information about the origin lines in the input is fully specified
by the \code{\#ORIGIN} markers, the \code{\#ENDORIG} markers are not relevant for placing the conditional directives.

For every \code{\#ORIGIN} marker in the input the following steps are performed:
\begin{itemize}
\item if the line belongs to a group of an active section which is after the active group, all directives of the active
section are output until the group containing the line is reached, this group is the new active group of the section.
\item if for an active section the line does not belong to the active group or any of its following groups, 
the \code{\#endif} directive for the 
section is output and the section becomes inactive (is removed from the stack of active sections).
\item if the line belongs to a group of a section which is (after the previous step) not active, the section is set 
active (put on the stack) and all its directives are output until the group containing the line is reached.
\item finally the \code{\#ORIGIN} directive and all following lines which are not an \code{\#ORIGIN} marker
are output without any changes.
\end{itemize}

Note that due to the semantics of the \code{\#elif} and \code{\#else} directives, for every group in a section all
directives of preceding groups are relevant and must be output when the active group changes, even if this produces 
empty groups in between.

Due to the nesting structure, when newly active sections are pushed on the stack in the order of the position of their
first directive, the stack reflects the section nesting and the sections will be inactivated in the reverse order and
can be removed from the stack accordingly.

If an \code{\#ORIGIN} marker indicates, that the next line belongs to a group \textit{before} the active group, the
steps described above imply that the current section is ended and restarted from the beginning.

If the line before an inserted conditional directive contains an \code{\#ENDORIG} marker the newline after it belongs
to the marker. If the marker is not used to insert a comment it will be removed completely and the conditional
directive will be appended to the previous code line. In this case an additional newline is inserted before 
the conditional directive.

\subsubsection{Filter \code{gencot-mrgpp}}

The filter for merging other directives into the target code is implemented as an awk script. It can be used to 
merge arbitrary code with origin markers into target code with origin markers, if every code part should be
merged in exactly once.

The filter only interpretes \code{\#ORIGIN} markers. Whenever an \code{\#ORIGIN} marker is reached in the target
code, all content is merged in up to the first \code{\#ORIGIN} marker with a line number not before that in the target
code.

If the line before an inserted code contains an \code{\#ENDORIG} marker an additional newline is inserted before 
the code for the same reason as for conditional directives.

\subsection{Special Processing for Configuration Files}

The filter \code{gencot-preconfig} simply removes all \code{//} comment starts before a preprocessor directive.
It must be applied before \code{gencot-remcomments} so that the commented directives are still present. Gencot 
assumes that commented directives have no continuation lines.

The filter \code{gencot-postconfig <file>} reads the original file as its argument and determines from it the
line numbers of all commented directives. As its input it expects a target code file with origin markers.
According to the origin markers it inserts a \code{-{}-} comment start at the beginning of every line which
originated from a line with a commented directive. 

Although the Cogent preprocessor does not remove comments (neither C style nor Cogent style), the comment start
marker prevents it to process a following directive. The commented target code is later discarded by the Cogent 
parser.


\section{Parsing and Processing C Code}
\label{impl-ccode}

Parsing and processing C code in Gencot is always implemented in Haskell, to be able to use an existing
C parser. There are at least two choices for a C parser in Haskell:
\begin{itemize}
\item the package ``language-c'' by Benedikt Huber and others,
\item the package ``language-c-quote'' by Geoffrey Mainland and others.
\end{itemize}

The Cogent compiler uses the package language-c-quote for outputting the generated C code and for parsing the antiquoted
C source files. The reason is its support for quasiquotation (embedding C code in Haskell code) and antiquotation
(embedding Haskell code in the embedded C code). The antiquotation support is used for parsing the antiquoted C sources.

Gencot performs three tasks related to C code:
\begin{itemize}
\item read the original C code to be translated,
\item generate antiquoted C code for the function wrapper implementations,
\item output normal C code for the C function bodies as placeholder in the generated Cogent function definitions.
\end{itemize}

The first task is supported by both packages: a C parser reads the source text and creates an internal abstract syntax tree (AST).
Every package uses its own data structures for representing the AST. However, the language-c package provides an additional
``analysis'' module which processes the rather complicated syntax of C declarations and returns a ``symbol map'' mapping
every globally declared identifier to its declaration or definition. Since Gencot generates a single Cogent definition for
every single globally declared identifier, this is the ideal starting point for Gencot. For this reason Gencot uses
the language-c parser for the first task.

The second task is only supported by the package language-c-quote, therefore it is used by Gencot. 

The third task is supported by both packages, since both have a prettyprint function for outputting their AST. Since the 
function bodies have been read from the input and are output with only minor modifications, it is easiest to use
the language-c prettyprinter, since language-c has been used for parsing and the body is already represented by its 
AST data structures. However, the language-c prettyprinter cannot be extended to generate the ORIGIN markers, therefore
the AST is translated to the language-c-quote AST and the corresponding prettyprinter is used for the third task (see 
Section~\ref{impl-ccode-expr}).

Note that in both packages the main module is named \code{Language.C}. If both packages are exposed to the ghc Haskell
compiler, a package-qualified import must be used in the Haskell program, which must be enabled by a language pragma:
\begin{verbatim}
  {-# LANGUAGE PackageImports #-}
  ...
  import "language-c" Language.C
\end{verbatim}

\subsection{Including Files}
\label{impl-ccode-include}

The filter \code{gencot-include <dirlist>} processes all quoted include directives and replaces them (transitively) by the 
content of the included file. Line directives are inserted at the begin and end of an included file, so that
for all code in the output the original source file name and line number can be determined. The \code{<dirlist>}
specifies the directories to search for included files.

\subsubsection{Filter \code{gencot-include}}

The filter for expanding the include directives is implemented as an awk script, heavily inspired by the ``igawk''
example program in the gawk infofile, edition 4.2, in Section 11.3.9.

As argument it expects a directory list specified with ``:'' as separator. The list corresponds
to directories specified with the \code{-I} cpp option, it is used for searching included files.
All directories for searching included files must be specified in the arguments, there are no defaults.

Similar to cpp, a file included by a quoted directive is first searched in the directory of the including file. 
If not found there, the argument directory list is searched.

Since the input of \code{gencot-include} is read from standard input it is not associated with a directory. Hence
if files are included from the same directory, that directory must also be specified explicitly in an argument directory
list.

\subsubsection{Generating Line Directives}

Line directives are inserted into the output as follows.

If the first line of the input is a line directive, it is copied to the output. Otherwise 
the line directive
\begin{verbatim}
  # 1 "<stdin>"
\end{verbatim}
is prepended to the output.

If after a generated line directive with file name \code{\"fff\"} the input line \code{NNN} contains the 
directive 
\begin{verbatim}
  #include "filepath"
\end{verbatim}
the directive is replaced in the output by the lines 
\begin{verbatim}
  # 1 "dir/filepath" 1
  <content of file filepath>
  # NNN+1 "fff" 2
\end{verbatim}

The \code{\"dir/\"} prefix in the line directives for included files is determined as follows. 
If the included file has been found in the 
directory of its includer, the directory pathname is constructed from \code{\"fff\"} by taking the pathname 
up to and including the last ``/'' (if present, otherwise the prefix is empty).
If the included file has been found in a directory from the argument directory list
the directory pathname is used as specified in the list.

\subsubsection{Multiple Includes}

The C preprocessor does not prevent a file from being included multiple times. Usually, C include files use
an ifdef directive around all content to prevent multiple includes. The \code{gencot-include} filter does
not interprete ifdef directives, instead, it simply prevents multiple includes for all files independent 
from their contents, only based on their full file pathnames. To mimic the behavior of cpp, if a file is 
not include due to repeated include, the corresponding line directives are nevertheless generated in the form
\begin{verbatim}
  # 1 "dir/filepath" 1
  # NNN+1 "fff" 2
\end{verbatim}

\subsection{Preprocessing}
\label{impl-ccode-preproc}

The language-c parser supports an integrated invocation of an external preprocessor, the default is to use
the gcc preprocessor. However, the integrated invocation always reads the C code from a file (and checks
its file name extension) and not from standard input.

To implement C code processing as a filter, Gencot does not use the integrated preprocessor,
it invokes the preprocessor as an additional separate step. For consistency reasons it is wrapped in
the minimal filter script \code{gencot-cpp}. 

The preprocessor step only has the following purpose:
\begin{itemize}
\item process all system include directives by including the file contents,
\item process retained conditional directives to prevent conflicts in the C code.
\end{itemize}
All other preprocessing has already been done by previous steps.

\subsection{Reading the Input}
\label{impl-ccode-read}

\subsubsection{Parsing}

To apply the language-c parser to the standard input we invoke it using function \code{parseC}. It needs an \code{InputStream}
and an initial \code{Position} as arguments. 

The language-c parser defines \code{InputStream} to be the standard type \code{Data.ByteString}. To get the 
standard input as a \code{ByteString} the function \code{ByteString.getContents} can be used. 

The language-c parser uses type \code{Position} to describe a character position in a named file. It provides
the function \code{initPos} to create an initial position at the beginning of a file, taking a \code{FilePath}
as argument, which is a \code{String} containing the file name. Since Gencot and the C preprocessor create
line directives with the file name \code{<stdin>} for the standard input, this string is the correct argument
for \code{initPos}. 

The result of \code{parseC} is of type \code{(Either ParseError CTranslUnit)}. Hence it should be checked whether
an error occurred during parsing. If not, the value of type \code{CTranslUnit} is the abstract syntax tree for
the parsed C code.

Both \code{parseC} and \code{initPos} are exported by module \code{Language.C}. The function \code{ByteString.getContents}
is exported by the module \code{Data.Bytestring}. Hence to use the parser we need the following imports:
\begin{verbatim}
  import Data.ByteString (getContents)
  import "language-c" Language.C (parseC,initPos)
\end{verbatim}

Then the abstract syntax tree can be bound to variable \code{ast} using
\begin{verbatim}
  do
    input_stream <- Data.ByteString.getContents
    ast <- either (error . show) return $ parseC input_stream (initPos "<stdin>")
\end{verbatim}

\subsubsection{Analysis}

Although it is not complete and only processes toplevel declarations (including typedefs), and object definitions, the
language-c analysis module is very
useful for implementing Gencot translation. Function definition bodies are not covered by analysis, but they are
not covered by Gencot either.

The main result of the analysis module is the symbol table. Since at the end of traversing a correct C AST the toplevel
scope is reached, the symbol table only contains all globally defined identifiers. From this symbol table 
a map is created containing all toplevel declarations and object definitions, mapping the identifiers
to their semantics, which is mainly its declared type. Whereas in the abstract syntax tree there may be several declarators
in a declaration, declaring identifiers with different types derived from a common type, the map maps every identifier
to its fully derived type. 

Also, tags for structs, unions and enums are contained in the map. In C their definitions can be embedded in other declarations.
The analysis module collects all these possibly embedded declarations in the symbol table. The map also gives for
every defined type name its definition.

Together, the information in the map is much more appropriate for creating Cogent code, where all type definitions are on
toplevel. Therefore, Gencot uses the map resulting from the analysis step as starting point for its translation. 
Additionally, Gencot uses the symbol table built by the analysis module during its own processing to access the
types of globally defined identifiers and for managing local declarations when traversing function bodies, as described in
Section~\ref{impl-ccode-trav}.

Additionally, the analysis module provides a callback handler which is invoked for every declaration entered into the symbol 
table (with the exception of tag forward declarations and enumerator declarations). The callback handler can accumulate results 
in a user state which can be retrieved after analysis together with the
semantics map. Since the callback handler is also invoked for all local declarations it is useful when all declarations
shall be processed in some form.

To use the analysis module, the following import is needed:
\begin{verbatim}
  import Language.C.Analysis
\end{verbatim}

Then, if the abstract syntax tree has been bound to variable \code{ast}, it can be analysed by
\begin{verbatim}
  (table,state) <- either (error . show) return $ 
    runTrav uinit (withExtDeclHandler (analyseAST ast >> getDefTable) uhandler)
\end{verbatim}
which binds the resulting symbol table to variable \code{table} and the resulting state to \code{ustate}. \code{runTrav}
returns a result of type \code{Either [CError] (DefTable, TravState s)}, where \code{DefTable}
is the type of the symbol table and \code{s} is the type of the user state. The error list in the first alternative contains 
fatal errors which made the analysis fail. The state in the second alternative contains warnings about semantic inconsistencies, 
such as unknown identifiers, and it contains the user state. \code{uinit} is the initial user state and \code{uhandler}
is the callback handler of type
\begin{verbatim}
  DeclEvent -> Trav s ()
\end{verbatim}
It returns a monadic action without result.

The semantics map is created from the symbol table by the function \code{globalDefs}, its type is \code{GlobalDecls}.

On this basis, Gencot implements the following functions in the module \code{Gencot.Input} as utility for parsing and analysis:
\begin{verbatim}
  readFromInput :: s -> (DeclEvent -> Trav s ()) -> IO (DefTable, s)
  readFromFile :: FilePath -> s -> (DeclEvent -> Trav s ()) -> IO (DefTable, s)
\end{verbatim}
The first one takes as arguments an initial user state and a callback handler. It reads C code from standard input, parses
and analyses it and returns the symbol table and the user state accumulated by the callback handler. The second function
takes a file name as additional argument and does the same reading from the file.

All Gencot filters which read C code use one of these two functions.

\subsubsection{Source Code Origin}

The language-c parser adds information about the source code origin to the AST. For every syntactic construct represented
in the AST it includes the start origin of the first input token and the start origin and length of the last input token.
The start origin of a token is represented by the type \code{Position} and includes the original source file name and 
line number, affected by line directives if present in the input. It also includes the absolute character offset in the 
input stream. The latter can be used to determine the ordering of constructs which have been placed in the same line.
The type \code{Position} is declared as instance of class \code{ORD} by comparing the character offset, hence it can 
easily be used for comparing and sorting.

The origin information about the first and last token is contained in the type \code{NodeInfo}. All types for representing
a syntactic construct in the AST are parameterized with a type parameter. In the actual AST types this parameter is always 
substituted by the type \code{NodeInfo}. 

The analysis module carries the origin information over to its results, by including a \code{NodeInfo} in most of its
result structures. This information can be used to
\begin{itemize}
\item determine the origin file for a declared identifier,
\item filter declarations according to the source file containing them,
\item sort declarations according to the position of their first token in the source,
\item translate identifiers to file specific names to avoid conflicts.
\end{itemize}

For the last case the true name of the processed file is required, however, the parsed input is read from a pipe where
the name is always given as \code{<stdin>}. The true name is passed to the Haskell program as an additional 
argument, as described in Section~\ref{impl-comps-filters}. Since there is no easy way to replace the file name in
all \code{NodeInfo} values in the semantic map, Gencot adds the name to the monadic state used for processing
(see Section~\ref{impl-ccode-trav}).

\subsubsection{Preparing for Processing}

The main task for Gencot is to translate all declarations or definitions which are contained in a single source file, where
nested declarations are translated to a sequence of toplevel Cogent definitions. This is achieved by parsing and analysing
the content of the file and all included files, filtering the resulting set of declarations according to the source file name
\code{<stdin>}, removing all declarations which are not translated to Cogent, and sorting the remaining ones in a list. 
Translating every list entry to Cogent yields the resulting Cogent definitions in the correct ordering.

The type \code{GlobalDecls} consists of three separate maps, one for tag definitions, one for type definitions,
and one for all other declarations and definitions. Every map uses its own type for its range values, however, 
there is the wrapper type \code{DeclEvent} which has a variant for each of them. 

The language-c analysis module provides a filtering function for its resulting map of type \code{GlobalDecls}. The filter 
predicate is defined for values of type \code{DeclEvent}. If the map has been bound to the variable \code{gmap}
it can be filtered by
\begin{verbatim}
  filterGlobalDecls globalsFilter gmap
\end{verbatim}
where \code{globalsFilter} is the filter predicate.

Gencot uses a filter which reduces the declarations to those contained directly in the input file, removing all
content from included files. Since the input file is always associated with the name \code{<stdin>} in the \code{NodeInfo}
values, a corresponding filter function is
\begin{verbatim}
  (maybe False ((==) "<stdin>") . fileOfNode)
\end{verbatim}
Additionally, for a specific Gencot component, the declarations are reduced to those which are processed by the component. 

Every map range value, and hence every \code{DeclEvent} value contains the identifier which is mapped to it, 
hence the full information required for translating the definitions is contained in the range values. 
Gencot wraps every range value as a \code{DeclEvent}, and puts them in a common list for all three maps. This
is done by the function
\begin{verbatim}
  listGlobals :: GlobalDecls -> [DeclEvent]
\end{verbatim}

Finally, the declarations in the list are sorted according to the offset position of their first tokens, using the
compare function
\begin{verbatim}
  compEvent :: DeclEvent -> DeclEvent -> Ordering
  compEvent ci1 ci2 = compare (posOf ci1) (posOf ci2)
\end{verbatim}

Together, the list for processing the code is prepared from the symbol table \code{table} by
\begin{verbatim}
  sortBy compEvent $ listGlobals $ filterGlobalDecls globalsFilter $ globalDefs table
\end{verbatim}

All this preprocessing is implemented in module \code{Gencot.Input}. It provides the function
\begin{verbatim}
  getDeclEvents :: GlobalDecls -> (DeclEvent -> Bool) -> [DeclEvent]
\end{verbatim}
It performs the preprocessing and returns the list of \code{DeclEvent}s to be processed.
As its second argument it expects a predicate for filtering the content of \code{<stdin>} to the
\code{DeclEvent}s to be processed by the specific Gencot component.

\subsection{Dummy Declarations for Preprocessor Macros}
\label{impl-ccode-dummydecl}

As described in Section~\ref{design-preprocessor-macros} macro calls in C code must either be syntactically correct
C code or they must be converted to syntactically correct C code. Due to the language-c analysis step this is not 
sufficient. The analysis step checks for additional properties. In particular, it requires that every identifier 
is either declared or defined.

Thus for every identifier which is part of a converted macro call a corresponding declaration must be added to the 
C code. They are called ``dummy declarations'' since they are only used for making the analysis step happy. 

For all preprocessor defined constants Gencot automatically generates the required dummy declarations. The corresponding
macro calls always have the form of a single identifier occurring at positions where a C expression is expected. The type
of the identifier is irrelevant, hence Gencot always uses type \code{int} for the dummy declarations. For every preprocessor
constant definition of the form 
\begin{verbatim}
  #define NNN XXX
\end{verbatim}
a dummy declaration of the form
\begin{verbatim}
  int NNN;
\end{verbatim}
is generated. This is implemented by the additional filter \code{gencot-gendummydecls}. It is applied to the result of 
\code{gencot-selppconst}. The resulting dummy declarations are prepended to the input of the language-c preprocessor
since this prevents the lines from being counted for the \code{<stdin>} part.

Flag macro calls do not occur in C code, hence no dummy declarations are required for them.

For all other macros the required dummy declarations must be created manually and added to the Gencot macro call conversion.
Even if no macro call conversion is needed because the macro calls are already in C syntax, it may be necessary to
add dummy declarations to satisfy the requirements of the language-c analysis step.

\subsection{Generating Cogent Code}
\label{impl-ccode-gencog}

When Gencot generates its Cogent target code it uses the data structures defined by the Cogent compiler for representing
its AST after parsing Cogent code. The motivation to do so is twofold. First, the AST omits details such as using code layout
and parentheses for correct code structure and the Cogent compiler provides a prettyprint function for its AST which cares
about these details. Hence, it is much easier to generate the AST and use the prettyprinter for output, instead of generating
the final Cogent program text. Second, by using the Cogent AST the generated Cogent code is guaranteed to be syntactically correct and
current for the Cogent language version of the used compiler version. Whenever the Cogent language syntax is changed
in a newer version, this will be detected when Gencot is linked to the newer compiler version.

\subsubsection{Cogent Surface Syntax Tree}

The data structures for the Cogent surface syntax AST are defined in the module Cogent.Surface. It defines parameterized types
for the main Cogent syntax constructs (\code{TopLevel}, \code{Alt}, \code{Type}, \code{Polytype}, \code{Pattern}, 
\code{IrrefutablePattern}, \code{Expr}, and \code{Binding}), where the type parameters determine the types of the 
sub-structures. Hence the AST types
can easily be extended by wrapping the existing types in own extensions which are then also used as actual type parameters.

Cogent itself defines two such wrapper type families: The basic unextended types \code{RawXXX} and the types \code{LocXXX}
where every construct is extended by a representation of its source location. 

All parameterized types for syntax constructs and the \code{RawXXX} and \code{LocXXX} types are defined as instances of 
class \code{Pretty} from
module \code{Text.PrettyPrint.ANSI.Leijen}. This prettyprinter functionality is used by the Cogent compiler for outputting
the parsed Cogent source code after some processing steps, if requested by the user.

As source location representation in the \code{LocXXX} types Cogent uses the type \code{SourcePos} from Module 
\code{Text.Parsec.Pos} in package \code{parsec}.
It contains a file name and a row and column number. This information is ignored by the prettyprinter.

\subsubsection{Extending the Cogent Surface Syntax}

Gencot needs to extend the Cogent surface syntax for its generated code in two ways:
\begin{itemize}
\item origin markers must be supported, as described in Section~\ref{impl-origin},
\item C function bodies must be supported in Cogent function definitions, as described in Section~\ref{design-fundefs-body}.
\end{itemize}

\paragraph{Origin Markers}

The origin markers are used to optionally surround the generated target code parts, which may be arbitrary syntactic constructs
or groups of them. Hence it would be necessary to massively extend the Cogent surface syntax, if they are added as explicit 
syntactic constructs. Instead, Gencot optionally adds the information about the range of source lines to the syntactic
constructs in the AST and generates the actual origin markers when the AST is output. 

Although the \code{LocXXX} types already support a source position in every syntactic construct, it cannot be used by Gencot,
since it represents only a single position instead of a line range. Gencot uses the \code{NodeInfo} values, since they represent
a line range and they are already present in the C source code AST, as described in Section~\ref{impl-ccode-read}. Hence, they
can simply be transferred from the source code part to the corresponding target code part. For the case that there is no
source code part in the input file (such as for code generated for external name references), or there is no position 
information available for the source code part, the \code{NodeInfo} is optional.

It may be the case that a target AST node is generated from a source code part which is not a single source AST node. Then
there is no single \code{NodeInfo} to represent the origin markers for the target AST node. Instead, Gencot uses the 
\code{NodeInfo} values of the first and last AST nodes in the source code part.

It may also be the case that a structured source code part is translated to a sequence of sub-part translations without target
code for the main part. In this case the \code{\#ORIGIN} marker for the main part must be added before the \code{\#ORIGIN} 
marker of the first target code part and the \code{\#ENDORIG} marker for the main part must be added after the \code{\#ENDORIG} 
marker of the last target code part. 

To represent all these cases, the origin information for a construct in the target AST consists of two lists of \code{NodeInfo}
values. The first list represents the sequence of \code{\#ORIGIN} markers to be inserted before the construct, here only the
start line numbers in the \code{NodeInfo} values are used. The second list represents the sequence of \code{\#ENDORIG} markers 
to be inserted after the construct, here only the end line numbers in the \code{NodeInfo} values are used. If no marker of
one of the kinds shall be present, the corresponding list is empty.

Additional information must be added to represent the marker extensions for placing the comments (the trailing ``+'' signs).
Therefore, a boolean value is added to all list elements.

Together, Gencot defines the type \code{Origin} for representing the origin information, with the value \code{noOrigin}
for the case that no markers will be generated:
\begin{verbatim}
  data Origin = Origin { 
    sOfOrig :: [(NodeInfo,Bool)], 
    eOfOrig :: [(NodeInfo,Bool)] } 
  noOrigin = Origin [] []
\end{verbatim}
Gencot adds an \code{Origin} value to every Cogent AST element. The type \code{Origin} is defined in the module 
\code{Gencot.Origin}

\paragraph{Embedded C Code}

Cogent function definitions are represented by the \code{FunDef} alternative of the type for toplevel syntactic constructs:

\begin{verbatim}
  data TopLevel t p e = 
    ... | FunDef VarName (Polytype t) [Alt p e] | ...
\end{verbatim}
The type parameter \code{e} for representing syntactic expressions is only used in this alternative and in the alternative
for constant definitions. Cogent constant definitions are generated by Gencot only from C enum constants (preprocessor
constants are processed by \code{gencot-prcconst} which is not implemented in Haskell). The defined value for a C enum
constant is represented in the C AST by the type for expressions. Together, instead of Cogent expressions, Gencot always
uses either a C expression or a Cogent expression together with a C function body (which syntactically is a statement) 
in the Cogent AST. 

To modify the Cogent syntax in this way, Gencot defines an own expression type with two alternatives for a C expression 
and for a Cogent expression together with a C statement:
\begin{verbatim}
  data GenExpr = ConstExpr Exp
               | FunBody RawExpr Stm
\end{verbatim}
where \code{Exp} and \code{Stm} are the types for C expressions and statements as defined by the language-c-quote AST 
(see Section~\ref{impl-ccode-expr}). Note that no \code{Origin} components are added, since the types \code{Exp} and 
\code{Stm} already contain \code{Origin} information. The type \code{RawExpr} is used for the dummy result expression.
It has no origin in the C source, therefore the raw type without origin information is sufficient. 

Other than for the dummy result expression, the Cogent AST expression type is not used by Gencot. 
Since bindings only occur in expressions, the AST type for Cogent bindings is not used either.

For the type parameters \code{t} and \code{p} for representing types and patterns, respectively, the normal types for 
the Cogent constructs are used, since Gencot generates both in Cogent syntax. The pattern generated for a function
definition is always a tuple pattern, which is irrefutable. Gencot never generates other patterns, hence the AST
type for irrefutable patterns is sufficient. 

Together, Gencot uses the following types to represent its extended Cogent surface AST:
\begin{verbatim}
  data GenToplv =
    GenToplv Origin (TopLevel GenType GenIrrefPatn GenExpr)
  data GenAlt =
    GenAlt Origin (Alt GenIrrefPatn GenExpr)
  data GenIrrefPatn = 
    GenIrrefPatn Origin (IrrefutablePattern VarName GenIrrefPatn)
  data GenType = 
    GenType Origin (Type GenExpr GenType)
  data GenPolytype = 
    GenPolytype Origin (Polytype GenType)
\end{verbatim}
The first parameter of \code{Type} for expressions is only used for Cogent array types, which are currently 
not generated by Gencot.

All five wrapper types are defined as instances of class \code{Pretty}, basically by applying the Cogent prettyprint
functionality to the wrapped Cogent AST type.

\subsection{Mapping Names}
\label{impl-ccode-names}

Names used in the target code are either mapped from a C identifier or introduced, as described in 
Section~\ref{design-names}. Different schemas are used depending on the kind of name to be generated.
The schemas require different information as input.

\subsubsection{General Name Mapping}

The general mapping scheme is applied whenever a Cogent name is generated from an existing C identifier.
Its purpose is to adjust the case, if necessary and to avoid conflicts between the Cogent name and
the C identifier.

As input this scheme only needs the C identifier and the required case for the Cogent name.
It is implemented by the function
\begin{verbatim}
  mapName :: Bool -> Ident -> String
\end{verbatim}
where the first argument specifies whether the name must be uppercase.

\subsubsection{Cogent Type Names}

A Cogent type name (including the names of primitive types) may be generated as translation of a C 
primitive type, a C typedef name, a C struct/union/enum type reference, or a C derived type. 

A C primitive type is translated according to the description in Section~\ref{design-types}. Only the
type specifiers for the C type are required for that.

A C typedef name is translated by simply mapping it with the help of \code{mapName} to an uppercase name.
Only the C typedef name is required for that.

A C struct/union/enum type reference may be tagged or tagless. If it is tagged, the Cogent type name is
constructed from the tag as described in Section~\ref{design-names}: the tag is mapped with the help of
\code{mapName} to an uppercase name, then a prefix \code{Struct\_}, \code{Union\_} or \code{Enum\_} is 
prepended. For this mapping the tag and the kind (struct/union/enum) are required. Both are contained
in the language-c type \code{TypeName} which is used to represent a reference to a struct/union/enum.

If the reference is untagged, Gencot nevertheless generates a type name, as motivated and described 
in Section~\ref{design-names}. As input it needs the kind and the position of the struct/union/enum 
definition. The latter is not contained in the \code{TypeName}, it contains the position of the reference
itself. To access the position of the definition, the definition must be retrieved from the symbol table
in the monadic state. Hence, the mapping function is defined as a monadic action.

Together the function for translating struct/union/enum type references is
\begin{verbatim}
  transTagName :: TypeName -> FTrav String
\end{verbatim}

Since an untagged struct/union/enum can be contained in any type specification and type specifications
may occur in all other C constructs, the \code{GlobalDecls} map must be passed as argument to all translation
functions from C constructs to Cogent constructs.

If the definition itself is translated, it is already available and need not be retrieved from the map. 
However, as described in Section~\ref{impl-ccode-gencog}, the map may be needed to map the generic name
\code{<stdin>} to the true source file name. Therefore Gencot uses function \code{transTagName} also when
translating the definition.

A C derived type is translated to a Cogent type name by translating the name of the basic type as described
above, and then prepending the encoded sequence of derivation steps, as
described in Section~\ref{design-names}. The information about the derivation steps is contained in the 
type construct, no information in addition to that required for translating the basic type name is needed.

\subsubsection{Cogent Function Names}

Cogent function names are generated from C function names. A C function may have external or internal
linkage, according to the linkage the Cogent name is constructed either as a global name or as a name specific
to the file where the function is defined. For deciding which variant to use for a function name reference,
its linkage must be determined. It is available in the definition or in a declaration for the function name,
either of which must be present in the symbol table. The language-c analysis module replaces all 
declarations in the tyble by the
definition, if that is present in the parsed input, otherwise it retains a declaration. 

A global function name is generated by mapping the C function name with the help of \code{mapName} to
a lowercase Cogent name. No additional information is required for that.

For generating a file specific function name, the file name of the definition is required. Note that 
this is only done for a function with internal linkage, where the definitions must be present in
the input whenever the function is referenced. The definition contains the position information
which includes the file name. Hence, the symbol table is sufficient for translating the name,
to make it available the translation function is defined as a monadic action.

In C bodies function names cannot be syntactically distinguished from variable names. Therefore, Gencot
uses a common function for translating function and variable names. For a description how variable
names are translated see Section~\ref{impl-ccode-expr}.
\begin{verbatim}
  transObjName :: Ident -> FTrav String
\end{verbatim}

Similar as for tags, the function is also used when translating a function definition, although the 
definition is already available.

\subsubsection{Cogent Constant Names}

Cogent constant names are only generated from C enum constant names. They are simply translated
with the help of \code{mapName} to a lowercase Cogent name. No additional information is required.

\subsubsection{Cogent Field Names}

C member names and parameter names are translated to Cogent field names. Only if the C name is
uppercase, the name is mapped to a lowercase Cogent name with the help of \code{mapName}, 
otherwise it is used without change. Only the C name is required for that, in both cases it is
available as a value of type \code{Ident}. The translation is implemented by the function
\begin{verbatim}
  mapIfUpper :: Ident -> String
\end{verbatim}

\subsection{Generating Origin Markers}
\label{impl-ccode-origin}

For outputting origin markers in the target code, the AST prettyprint functionality must be extended.

The class \code{Pretty} used by the Cogent prettyprinter defines the methods
\begin{verbatim}
  pretty :: a -> Doc
  prettyList :: [a] -> Doc
\end{verbatim}
but the method \code{prettyList} is not used by Cogent. Hence, only the method \code{pretty} needs to be defined
for instances. The type \code{Doc} is that from module \code{Text.PrettyPrint.ANSI.Leijen}.

The basic approach is to wrap every syntactic construct in a sequence of \code{\#ORIGIN} markers and 
a sequence of \code{\#ENDORIG} markers according to the origin information for the construct in the extended AST. 
This is done by an instance definition of the form
\begin{verbatim}
  instance Pretty GenToplv where
    pretty (GenToplv org t) = addOrig org $ pretty t
\end{verbatim}
for \code{GenToplv} and analogous for the other types. The function \code{addOrig} has the type
\begin{verbatim}
  addOrig :: Origin -> Doc -> Doc
\end{verbatim}
and wraps its second argument in the origin markers according to its first argument.

The Cogent prettyprinter uses indentation for subexpressions. Indentation is implemented by the \code{Doc} type, 
where it is called ``nesting''. The prettyprinter maintains a
current nesting level and inserts that amount of spaces whenever a new line starts. 

The origin markers must be positioned in a separate line, hence \code{addOrig} outputs a newline before and after
each marker. This is done even at the beginning of a line, since due to indentation it cannot safely be determined
whether the current position is at the beginning of a line. Cogent may change the nesting of the next line after \code{addOrig}
has output a marker (typically after an \code{\#ENDORIG} marker). The newline at the end of the previous marker 
still inserts spaces according to the old nesting level, which determines the current position at the begin of
the following marker. This is not related to the new nesting level. 

This way many additional newlines are generated, in
particular an empty line is inserted between all consecutive origin markers. The additional newlines are later removed
together with the markers, when the markers are processed. Note that, if a syntactic construct
is nested, the indentation also applies to the origin markers and the line after it. To completely remove an
origin marker from the target code it must be removed together with the newline before it and with the newline 
after it and the following indentation. The following indentation can be determined since it is the same as that 
for the marker itself (a sequence of blanks of the same length). 

\subsubsection{Repeated Origin Markers}

Normally, target code is positioned in the same order as the corresponding source code. This implies, that
origin markers are monotonic. A repeated origin marker is a marker with the same line number as its previous marker.
Repeated origin markers of the same kind must be avoided, since they would result in duplicated comments or 
misplaced directives.
Repeated origin markers of the same kind occur, if a subpart of a structured source code part begins or ends 
in the same line as its main part. In this case only the outermost markers must be retained.

An \code{\#ENDORIG} marker repeating an \code{\#ORIGIN} marker means that the source code
part occupies only one single line (or a part of it), this is a valid case. 
An \code{\#ORIGIN} marker repeating an \code{\#ENDORIG} marker means that the previous source code
part ends in the same line where the following source code part begins. In this case the markers are
irrelevant, since no comments or directives can be associated with them. However, if they are
present they introduce unwanted line breaks, hence they also are avoided by removing both of them.

Together, the following rules result. In a sequence of repeated \code{\#ORIGIN} markers, only the first one 
is generated. In a sequence of repeated \code{\#ENDORIG} markers only the last one is generated.
If an \code{\#ORIGIN} marker repeats an \code{\#ENDORIG} marker, both are omitted.

There are several possible approaches for omitting repeated origin markers:
\begin{itemize}
\item omit repeated markers when building the Cogent AST
\item traverse the Cogent AST and remove markers to be omitted
\item output repeated markers and remove them in a postprocessing step
\end{itemize}
Note, that it is not possible to remove repeated markers already in the language-c AST, since there a \code{NodeInfo}
value always corresponds to two combined markers.

Handling repeated markers in the Cogent AST is difficult, because for an \code{\#ORIGIN} marker the context
before it is relevant whereas for an \code{\#ENDORIG} marker the context after it is relevant. An additional
AST traversal would be required to determine the context information. The first approach is even more complex
since the context information must be determined from the source code AST where the origin markers are not
yet present. 

For this reason Gencot uses the third approach and processes repeated markers in the generated target code text,
independent from the syntactical structure.

\subsubsection{Filter for Repeated Origin Marker Elimination}

The filter \code{gencot-reporigs} is used for removing repeated origin markers. It is implemented as an awk script.

It uses five string buffers: two for the previous two origin markers read, and three for the code before,
between, and after both markers. Whenever all buffers are filled (the buffer after both markers with a 
single text line; this line exists, since consecutive markers are always separated by an empty line),
the markers are processed as follows, if they have the same line number: in the case of two 
\code{\#ORIGIN} markers the second is deleted, in the
case of two \code{\#ENDORIG} markers the first is deleted, and in the case of an \code{\#ORIGIN}
marker after an \code{\#ENDORIG} marker both are deleted. In the latter case the line number of the
\code{\#ORIGIN} marker is remembered and subsequent \code{\#ORIGIN} markers with the same line number
are also deleted.

When both markers have different line numbers or if an \code{\#ENDORIG} marker follows an \code{\#ORIGIN}
marker the first marker and the code before it are output and the buffers are filled until the next marker
has been read.

\subsection{Generating Expressions}
\label{impl-ccode-expr}

For outputting the Cogent AST the prettyprint functionality must be extended to 
output C function bodies and the C expressions used for constant definitions. Additionally, at least in function bodies,
origin markers must be generated to be able to re-insert comments and preprocessor directives. Finally, all names
occurring free in a function body or a constant expression must be mapped to Cogent names.

The language-c prettyprinter is defined in module \code{Language.C.Pretty}. It defines an own class \code{Pretty} with 
method \code{pretty} to convert the AST types to a \code{Doc}. However, other than the Cogent prettyprinter, it uses 
the type \code{Doc} from module \code{Text.PrettyPrint.HughesPJ} instead of module \code{Text.PrettyPrint.ANSI.Leijen}.
This could be adapted by rendering the \code{Doc} as a string and then prettyprinting this string to a \code{Doc}
from the latter module. This way, a prettyprinted function body could be inserted in the document created by the
Cogent prettyprinter.

\subsubsection{Origin Markers}

For generating origin markers, a similar approach is not possible, since they must be inserted between single statements,
hence, the function \code{pretty} must be extended. Although it does not use the \code{NodeInfo}, it is only defined for
the AST type instances with a \code{NodeInfo} parameter and has no genericity which could be exploited for extending it.
Therefore, Gencot has to fully reimplement it. 

In the prettyprint reimplementation the target code parts must be wrapped by origin markers
in the same way as for the Cogent AST. However, for the type \code{Doc} from module \code{Text.PrettyPrint.HughesPJ} 
this is not possible, since newlines are only
available as separators between documents and cannot be inserted before or after a document. An alternative choice
would be to use the type \code{Doc} from \code{Text.PrettyPrint.ANSI.Leijen}, as the Cogent prettyprinter does.
However, the approach of both modules is quite different so that it would be necessary to write a new C 
prettyprint implementation nearly from scratch. 

It has been decided to use another approach which is expected to be simpler. The alternative C parser language-c-quote 
also has a prettyprinter. It generates a type \code{Doc} defined by a third module \code{Text.PrettyPrint.Mainland}.
It is similar to \code{Text.PrettyPrint.ANSI.Leijen} and also supports adding newlines before and after a document.
The language-c-quote prettyprinter is defined in the module \code{Language.C.Pretty} of language-c-quote and consists
of the method \code{ppr} of the class \code{Pretty} defined in module \code{Text.PrettyPrint.Mainland.Class.Pretty}.
This method is not generic at all, hence it must be completely reimplemented to extend it for generating origin 
markers. However, this reimplementation is straightforward and can be done by copying the original implementation
and only adding the origin marker wrappings. The resulting Gencot module is \code{Gencot.C.Output}.

Whereas the type \code{Doc} from \code{Text.PrettyPrint.ANSI.Leijen} provides a \code{hardline} document which always
causes a newline in the output, the type \code{Doc} from \code{Text.PrettyPrint.Mainland} does not. Normal line breaks
are ignored in certain contexts, if there is enough room. Using normal line breaks around origin markers could result
in origin markers with other code in the same line before or after the marker.

For the reimplemented language-c-quote prettyprinter Gencot defines its own \code{hardline} by using a newline 
which is hidden for type \code{Doc}. This could be implemented without nesting the marker and the subsequent line.
However, if at the marker position a comment is inserted, the subsequent line should be correctly indented.
To achieve this, the \code{hardline} implementation also adds the current nesting after the newline.

Hiding the newline from \code{Doc} implies that the ``current column'' maintained by \code{Doc} is not
correct anymore, since it is not reset by the \code{hardline}. Every \code{hardline} will instead advance the current
column by the width of the marker and twice the current nesting. This has two consequences.

First, in some places the language-c-quote prettyprinter uses ``alignment'' which means an indentation of subsequent lines
to the current column. This indentation will be too large after inserted markers. Gencot handles this by replacing 
alignment everywhere in the prettyprint implementation by a nesting of two additional columns. 

Second, the language-c-quote prettyprinter is parameterized by a ``document width''. It automatically breaks lines 
when the current column exceeds the document width. The incorrect column calculation causes many additional such
line breaks, since the current column increases much faster than normal. Gencot handles this by setting the document
width to a very large value (such as 2000 instead of 80) to compensate for the fast column increase.

\subsubsection{Using the language-c-quote AST}

Language-c-quote uses a different C AST implementation than language-c. To use its reimplemented prettyprinter, the 
language-c AST must be translated to a language-c-quote AST. This is not trivial, since the structures are somewhat
different, but it seems to be simpler than implementing a new C prettyprinter. The translation is implemented in
the module \code{Gencot.C.Translate}. 

Additionally the language-c-quote AST must be extended by \code{Origin} values. The language-c-quote AST already 
contains \code{SrcLoc} values which are similar to the \code{NodeInfo} values in language-c. Like these they cannot
be used as origin marker information since they cannot represent begin and end markers independently. Therefore
Gencot also reimplements the language-c-quote AST by copying its data types and replacing the \code{SrcLoc}
values by \code{Origin} values. This is implemented in module \code{Gencot.C.Ast}.

Together, this approach yields a similar structure as for the translation to Cogent: The Cogent AST is extended 
by the structures in \code{Gencot.C.Ast} to represent function bodies and constant expressions. The function for
translating from language-c AST to the Cogent AST is extended by the functions in \code{Gencot.C.Translate} to
translate function bodies and constant expressions from the language-c AST to the reimplemented language-c-quote 
AST, and the Cogent prettyprinter is extended by the prettyprinter
in \code{Gencot.C.Output} to print function bodies and constant expressions with origin markers.

In addition to translating the C AST structures from language-c to those of language-c-quote, the translation
function in \code{Gencot.C.Translate} implements the following functionality:
\begin{itemize}
\item generate \code{Origin} values from \code{NodeInfo} values,
\item map C names to Cogent names.
\end{itemize}

\subsubsection{Name Mapping}

Name mapping depends on the kind of name and may additionally depend on its type. Both information is
available in the symbol table (see Section~\ref{impl-ccode-trav}). However, the scope cannot be queried
from the symbol table. Hence it is not possible to map names depending on whether they are locally defined
or globally.

The following kinds of names may occur in a function body: primitive types, typedefs, tags, members, 
functions, global variables, enum constants, preprocessor constants, parameters and local variables.

Primitive type names and typedef names can only occur as name of a base type in a declaration. Primitive
type names are mapped to Cogent primitive type names as described in Section~\ref{design-types-prim}.

A typedef name may also occur in a declarator of a local typedef which defines the name. 
In both cases, as described in Section~\ref{design-fundefs-body}, Gencot
only maps the plain typedef names, not the derived types. The typedef names are mapped according to
Section~\ref{design-types-typedef}: If they ultimately resolve to a struct, union, or array type they
are mapped with an unbox operator applied, otherwise they are mapped without.

A tag name can only occur as base type in a declaration. It is always mapped to a name with a prefix 
of \code{Struct\_}, \code{Union\_}, or \code{Enum\_}. Tagless structs/unions/enums are not mapped at all.
Tag names are mapped according to Sections~\ref{design-types-enum} and~\ref{design-types-struct}: struct
and union tags are mapped with an unbox operator applied, enum tags are mapped without.

Gencot also maps defining tag occurrences. Thus an occurrence of the form 
\begin{verbatim}
  struct s { ... }
\end{verbatim}
is translated to
\begin{verbatim}
  struct #Struct_s { ... }
\end{verbatim}

Every occurrence of a field name can be syntactically distinguished. It is mapped according to 
Section~\ref{design-names} to a lowercase Cogent name if it is uppercase, otherwise it is unchanged.
Field names are also mapped in member declarations in locally defined structures and unions.

All other names syntactically occur as a primary expression. They are mapped depending on their semantic
information retrieved from the symbol table. In a first step it distinguishes objects, functions, 
and enumerators.

An object identifier may be a global variable, parameter, or local variable. It may also be a preprocessor 
constant since for them dummy declarations have been introduced which makes them appear as a global variable
for the C analysis. For the mapping the linkage is relevant, this is also available from the symbol table.

Identifiers for global variables may have external or internal linkage and are mapped depending on the
linkage. Identifiers for parameters always have no linkage and are always mapped like field names. Identifiers
for local variables either have no linkage or external linkage. In the first case they are mapped like
field names. In the second case they cannot be distinguished from global variables with external linkage,
and are mapped to lowercase. The dummy declarations introduced for preprocessor constants
always have external linkage, the identifiers are mapped to lowercase. Together, object identifiers with 
internal linkage are mapped as described in Section~\ref{design-names}, object identifiers with external
linkage are mapped to lowercase, and object identifiers with no linkage are mapped to lowercase if they are
uppercase and remain unchanged otherwise.

An identifier for a function has either internal or external linkage and is mapped depending on its linkage.
An identifier for an enumerator is always mapped to lowercase, like preprocessor constants.

Identifiers for local variables may also occur in a declarator of a local object definition which defines 
the name. They are also mapped depending on their linkage, as described above.

\subsection{Traversing the C AST}
\label{impl-ccode-trav}

The package language-c uses a monad for traversing and analysing the C AST. The monad is defined in module 
\code{Language.C.Analysis.TravMonad} and mainly provides the symbol table and user state during the traversal.
The traversal itself is implemented by a recursive descent according to the C AST using a separate function
for analysing every syntactic construct. 

When processing the semantic map resulting from the language-c analysis Gencot implements a similar recursive 
descent using a processing function for every syntactic construct. For this it uses the same monad for two reasons.
\begin{itemize}
\item the definitions and declarations for the global identifiers are needed for mapping the identifiers to
Cogent names,
\item additionally, the definitions and declarations for locally defined identifiers are needed in C function
bodies for mapping the identifiers.
\end{itemize}

The global definitions and declarations in the symbol table correspond to the semantics map which is the result 
of the language-c analysis step. It is created from the symbol table after traversing the C AST. Although Gencot 
processes the content of the semantics map, it is not available as a whole in the processing functions. Instead
of passing the semantics map as an explicit parameter to all processing functions, Gencot uses a monadic traversal
through the relevant parts of the semantics map, which implicitly makes the symbol table available to all 
processing functions. This is achieved by reusing the symbol table after the analysis phase for the traversal
of the semantics map.

Additionally, when processing the C function bodies, the symbol table is used for managing the local declarations. 
This is possible because although the analysis phase translates global declarations and definitions to a 
semantic representation, it does not modify function bodies and returns them as the original C AST. Since
the information about local declarations is discarded at the end of its scope, the information is not 
present anymore in the symbol table after the analysis phase. Gencot uses the symbol table functionality
to rebuild this information during its own traversal and uses it for name mapping.

The user state is used by Gencot to provide additional information, depending on the purpose of the traversal.
A common case is to make the actual name of the processed file available during processing. In the \code{NodeInfo}
values in the AST it is always specified as \code{<stdin>} since the input is read from a pipe. All C processing 
filters take the name of the original C source file as an additional argument. It is added to the user state 
of traversal monads so that it can be used during traversal.

This is supported by defining in module \code{Gencot.Traversal} the class \code{FileNameTrav} as
\begin{verbatim}
  class (Monad m) => FileNameTrav m where
    getFileName :: m String
\end{verbatim}
so that the method \code{getFileName} can be used to retrieve the source file name from all traversal monads of 
this class.

In module \code{Gencot.Names} the monadic action \code{srcFileName} is defined to returns the file name for a 
\code{NodeInfo} value and replace it by the original source file name if it is equal to \code{<stdin>}.

The utilities for the monadic traversal of the semantics map are defined in module \code{Gencot.Traversal}. 
The minimal monadic type is defined as
\begin{verbatim}
  type FTrav = Trav String
\end{verbatim}
where \code{String} is the type used for storing the original C source file name in the user state. It is an
instance of \code{FileNameTrav}. As execution 
function for the monadic actions the functions
\begin{verbatim}
  runFTrav :: FTrav a -> IO a
  runWithTable :: DefTable -> String -> FTrav a -> IO a
\end{verbatim}
are defined. The second one takes the symbol table and the original C source file name as arguments to initialize 
the state. The first one leaves both empty. The functions are themselves 
\code{IO} actions and print error messages generated during traversal to the standard output.

In the monadic actions the symbol table can be accessed by actions defined in the modules 
\code{Language.C.Analysis.TravMonad} and \code{Language.C.Analysis.DefTable}. An identifier can be
resolved using the actions
\begin{verbatim}
  lookupTypeDef :: Ident -> FTrav Type
  lookupObject :: Ident -> FTrav (Maybe IdentDecl)
\end{verbatim}
For resolving tag definitions the symbol table must be retrieved by
\begin{verbatim}
  getDefTable :: FTrav DefTable
\end{verbatim}
then the struct/union/enum reference can be resolved by
\begin{verbatim}
  lookupTag :: SUERef -> DefTable -> Maybe TagEntry
\end{verbatim}

Additionally, there are actions to enter and leave a scope and actions for inserting definitions.

An error can be recorded in the monad using the action
\begin{verbatim}
  recordError :: Language.C.Data.Error.Error e => e -> m () 
\end{verbatim}

\subsection{Creating the C Call Graph}
\label{impl-ccode-callgraph}

In some Gencot components we use the C call graph. This is the mapping from functions to the functions
invoked in their body. Here we describe the module \code{Gencot.Util.CallGraph} which provides
utility functions for creating the call graph.

The set of invoked functions is determined by traversing the bodies of all function definitions after the analysis
phase. The callback handler is not used since it is only invoked for declarations and definitions and does not help
for processing function invocations.

Invocations can be identified purely syntactically as C function call expressions. The invoked function is usually 
specified by an identifier, however, it can be specified as an arbitrary C expression. The expression can use 
identifiers which are locally declared, such as a parameter of the function where the invocation occurs. The 
declaration information of these identifiers would not be available after the traversal which builds the call graph.

To make the full information about the invoked functions available, Gencot inserts the declarations into the call graph 
instead of the identifiers. This cannot be done for arbitrary expressions, hence Gencot only handles two special cases 
of specifying the invoked function: as a single (globally or locally declared) identifier or as a member access from
a single identifier. All other invocation where the invoked function is specified in a different way are ignored and
not added to the call graph.

The call graph has the form of a mapping from function names (invoking function) to sets of declarations
(invoked functions):
\begin{verbatim}
  type CallGraph = Map Ident (Set CGDecl)
  data CGDecl =
    IdentDecl IdentDecl
    MemberDecl MmberDecl
\end{verbatim}
An invoked function can be specified by an \code{IdentDecl} (function or object of function pointer type),
or by a \code{MemberDecl} (struct or union member of function pointer type). Note that in a function definition
the parameters are represented in the symbol table by \code{IdentDecl}s, not by \code{ParamDecl}s.

To also access the declarations of locally declared identifiers, the symbol table with local declarations
must be available while building the call graph. Therefore we traverse the function bodies with the help of
the \code{FTrav} monad and \code{runWithTable} as described in Section~\ref{impl-ccode-trav}.

The call graph is constructed by the monadic action
\begin{verbatim}
  getCallGraph :: [DeclEvent] -> FTrav CallGraph
\end{verbatim}
It processes all function definitions in its argument list and ignores all other \code{DeclEvent}s.

As a utility the monadic action
\begin{verbatim}
  getInvocations :: [DeclEvent] -> FTrav (Set CGDecl)
\end{verbatim}
returns the union of all declarations of invoked functions in the call graph. It is implemented by an own
traversal of the function bodies, directly building the union of all invocations.

The declaration of an invoked function also tells 
whether the function or object is defined or only declared. Note that the traversal for collecting invocations is a ``second 
pass'' through the C source after the analysis phase of language-c. During analysis language-c replaces
declarations in the symbol table whenever it finds the corresponding definition.


\section{C Processing Components}
\label{impl-ccomps}
As described in Section~\ref{design-modular} there are several different Gencot components which process C code and generate 
target code.

\subsection{Filters for C Code Processing}
\label{impl-ccomps-filters}

All filters which parse and process C code are implemented in Haskell and read the
C code as described in Section~\ref{impl-ccode-read}.

The following filters always process the content of a single C source file and produce the content for a single 
target file.
\begin{description}
\item[\code{gencot-translate}] translates a single file \code{x.c} or \code{x.h} to the Cogent code to be put in file
\code{x.cogent} or \code{x-incl.cogent}. It processes typedefs, struct/union/enum definitions, and function
definitions. 
\item[\code{gencot-entries}] translates a single file \code{x.c} to the antiquoted C entry wrappers to be put in
file \code{x-entry.ac}. It processes all function definitions with external linkage.
\item[\code{gencot-remfundef}] processes a single file \code{x.c} by removing all function definitions. The output
is intended to be put in file \code{x-globals.c}
\item[\code{gencot-deccomments}] processes a single file \code{x.c} or \code{x.h} to generate the list of
all declaration positions.
\item[\code{parmod-gen}] processes a single file \code{x.c} and generates the 
function parameter modification descriptions (see Section~\ref{impl-parmod}).
\end{description}

All these filters take the name of the original source file as additional first
argument, since they need it to generate Cogent names for C names with internal linkage and for tagless C struct/union
types.

There are other target files which are generated for the whole Cogent compilation unit. The filters for generating these target files 
take as input the list of file names to be processed (see Section~\ref{impl-ccode-package}). Filters of this kind 
are called ``processors'' in the following.

Usually only \code{.c} files need to be specified as input to processors.
In contrast to the single-file filters, the original file name is not required for the files input to a 
Gencot processor. A processor only processes items which are external to all input files, whereas the original
file name is needed for items which are defined in the input files. Therefore the list of input file names is
sufficient and the input files may have arbitrary names which need not be related to the names of the original 
\code{.c} files.

Gencot uses the following processors of this kind:
\begin{description}
\item[\code{gencot-exttypes}] generates the content to be put in the file \code{<package>-exttypes.cogent}. It 
processes externally referenced typedefs, tag definitions and enum constant definitions.
\item[\code{gencot-dvdtypes}] generates the content to be put in the file \code{<package>-dvdtypes.cogent}. It 
processes derived types.
\item[\code{gencot-dvdtypesah}] generates the content to be put in the file \code{<package>-dvdtypes.ah}. It 
processes derived types.
\item[\code{gencot-externs}] generates the abstract function definitions of the exit wrappers to be put in the file 
\code{<package>-externs.cogent}. It processes the declarations of externally referenced functions and variables.
\item[\code{gencot-externsac}] generates the exit wrappers to be put in the file \code{<package>-externs.ac}. It processes
the declarations of externally referenced functions.
\item[\code{parmod-externs}] generates the list of function identifiers of all externally referenced functions and
function pointer (array)s.
\end{description}

\subsection{Safe Pointer List}
\label{impl-ccomps-pointer}

As default Gencot maps every C pointer type to a Cogent type of the form \code{MayNull P}, indicating that the value
\code{NULL} may occur for the typed entity (see Section~\ref{design-types-pointer}). The ``Gencot safe pointer list''
can be used to manually define exceptions from this behavior, when it is known that an entity is never \code{NULL}.

The Gencot safe pointer list is a list of textually specified C entities or types stored in a file. This file is read by 
all Gencot components which read C code for translation to Cogent.

The following kinds of entities of a non-function pointer type may be specified in the Gencot safe pointer list:
\begin{itemize}
\item global and local variables
\item function parameters
\item members of struct and union types
\end{itemize}
Additionally, functions returning a non-function pointer type may be specified. In this case the function result
value is specified to be not null. It is not possible to specify a function pointer, because that may have a type
name as its type, so that the result type cannot be modified individually.

Entities are specified in a similar way as the function identifiers used for parameter modification descriptions
(see Section~\ref{impl-parmod-ids}): Functions and global variables of external linkage are specified by their name.
For internal linkage \code{<filename>:} is prepended. Function parameters and local variables are specified by 
their name with \code{<funid>/} prepended where \code{<funid>} is a function identifier for the containing function 
as described in \ref{impl-parmod-ids}. Function parameters can also be specified by their position (starting with 1)
in the case that \code{<funid>} specifies a function type where only the parameter types are given.
Struct and union members are specified by their name with \code{<tag>.} prepended. Specifying members of untagged
structs and unions is not supported. 

It is also possible to specify non-function pointer types in the list. The meaning is that all entities having this
type are never \code{NULL}. Pointer types can be specified in the form 
\begin{verbatim}
  typedef | <typedef name>
\end{verbatim}
if they are named by \code{<typedef name>}. In this case only entities declared using this typedef name are specified 
to be not \code{NULL}, not entities declared to have the type to which the type name resolves. Pointer types can also be 
specified in the forms
\begin{verbatim}
  * typedef | <typedef name>
  * <tag>
\end{verbatim}
where \code{<typedef name>} is a name for a non-function type and \code{<tag>} is the tag name of a tagged struct or
union type. The first form applies only to entities declared to have type \code{*<typedef name>}, not entities declared
to be a pointer to the type to which the typedef name resolves. The second form applies to all entities declared to
be a pointer to the struct or union, possibly after fully resolving all typedef names used in the declaration. 

\subsection{Main Translation to Cogent}
\label{impl-ccomps-main}

The main translation from C to Cogent is implemented by the filter \code{gencot-translate}. It translates \code{DeclEvent}s
of the following kinds:
\begin{itemize}
\item struct/union definitions
\item enum definitions with a tag
\item enum constant definitions
\item function definitions
\item object definitions
\item type definitions
\end{itemize}
The remaining global items are removed by the predicate passed to \code{Gencot.Input.getDeclEvents}: all declarations, 
and all tagless enum definitions. No Cogent type name is generated for a tagless enum definition,
references to it are always directly replaced by type \code{U32}.

The translation does not use the callback handler to collect information during analysis. It only uses the semantics map
created by the analysis and processes the \code{DeclEvent} sequence created by preprocessing as described in 
Section~\ref{impl-ccode-read}.

The translation of the \code{DeclEvent} sequence is implemented by the function
\begin{verbatim}
  transGlobals :: [DeclEvent] -> FTrav [GenToplv]
\end{verbatim}
in module \code{Gencot.Cogent.Translate}.
It performs the monadic traversal as described in Section~\ref{impl-ccode-trav} and returns the list of toplevel
Cogent definitions of type \code{GenToplv}. It is implemented by mapping the function \code{transGlobal} to the 
\code{DeclEvent} sequence.

A struct/union/enum definition corresponds to a full specifier, as described in Section~\ref{design-decls-tags}.
The language-c analyser already implements moving all nested full specifiers to separate global definitions and the
sorting step done by \code{Gencot.Input.getDeclEvents} creates the desired ordering. Therefore, basically the 
\code{DeclEvent}s in the list could be processed by \code{transGlobal} independently from each other.

However, this approach would create inappropriate origin markers which could result in duplicated conditional preprocessor
directives and comments. If a struct definition in lines $b1 < e1$ contains a nested struct definition in lines $b2 < e2$
independent translation would wrap the translated nested struct always in origin markers for $b2$ and $e2$. If $e1 = e2$
the corresponding origin marker would occur twice. A better approach is to use origin markers for $b1, e1$ to wrap the 
whole sequence consisting of the translated struct together with the translations of all nested structs. Then, the translated
nested struct would be followed by the two origin markers for $e2$ and $e1$ which can be reduced to a single marker if
$e2 = e1$.

Therefore function \code{transGlobal} is defined as
\begin{verbatim}
  transGlobal :: DeclEvent -> FTrav [GenToplv]
\end{verbatim}
It translates every struct definition together with all nested struct/union/enum definitions and returns the sequence of 
the corresponding Cogent toplevel definitions, wrapped by the origin markers as described.

Since the nested definitions still occur seperately in the list of \code{DeclEvent}s processed by \code{transGlobals}, their processing
must be suppressed. This is implemented by sorting the list according to the position of the first character in the source file
(which will put all nested definitions after their surrounding definitions), marking processed nested definitions in the 
monadic user state, and testing every struct/union/enum definition found by \code{transGlobal} in the \code{DeclEvent} list 
whether it is marked as already processed.

A struct/union definition is translated to a Cogent type definition where the type name is constructed as described 
in Section~\ref{design-names}. A struct is translated to a corresponding record type, a union is translated to an 
abstract type, as described in Section~\ref{design-types}.
In both cases the type name names the boxed type, i.e., it corresponds to the C type of a pointer to the struct/union.

An enum definition with a tag is translated to a Cogent type definition where the type name is constructed as described 
in Section~\ref{design-names}. The name is always defined for type \code{U32}, as described in Section~\ref{design-types}.

An enum constant definition is translated to a Cogent constant definition where the name is constructed as described 
in Section~\ref{design-names} and the type is always \code{U32}, as described in Section~\ref{design-types}.

A function definition is translated to a Cogent function definition, as described in Section~\ref{design-fundefs}.

A type definition is translated to a Cogent type definition as described in Section~\ref{design-decls-typedefs}.

A type reference is translated by the function
\begin{verbatim}
  transType :: Bool -> String -> Type -> FTrav GenType
\end{verbatim}

The first parameter specifies whether typedef names should be resolved when translating the type. The second parameter 
specifies  the parameter modification function identifer (see Section~\ref{impl-parmod-ids}) for the type. It is used to
access the corresponding parameter modification description for translating derived function types. They are only available
for function types, function pointer types, and function pointer array types, for all other types the empty string is 
specified. For function pointer types and function pointer array types the identifier is passed without change to the 
base type translation, since in both cases the associated function type is uniquely determined. The identifier is 
actually processed only for the derived function types.

The reason why typedef names are resolved as part of the type translation function is that typedef names may have
an associated parameter modification function identifier which is based on the typedef name. If typedef names would
be resolved in a separate function the parameter modification function identifier could not be associated anymore
afterwards. This would cause the parameter modification specifications for typedef names to be ignored.

A type reference may be a direct type, a derived type, or a typedef name. For every typedef name a Cogent type
name is defined, as described in Sections~\ref{design-names} and~\ref{design-types-typedef}. A direct type is either
the type \code{void}, a primitive C type, which is mapped to the name of a primitive Cogent type, or it is a 
struct/union/enum type reference for which Gencot also introduces a Cogent type name or maps it to the 
primitive Cogent type \code{U32} (tagless enums). Hence, both direct types and typedef names can always be mapped
to Cogent type names, with the exception of type \code{void}, which is mapped to the Cogent unit type \code{()}.
For struct and union types the unbox operator must be applied to the Cogent type name.
If a typedef name references (directly or indirectly) a struct or union type, the corresponding Cogent
type name references the boxed type. Therefore, it must also be modified by applying the unbox operator. 

Primitive types or typedef names referencing a primitive type are semantically unboxed. However, if marked as unboxed, 
the Cogent prettyprint function will add an explicit unbox operator. For better readability Gencot specifies these types
as boxed, to suppress the redundant unbox operator.

A derived type is either a pointer type, an array type, or a function type. It is derived from a base type
which in case of a function type is the type of the function result. The base type may again be a derived
type, ultimately it is a direct type or a typedef name.

For a pointer type the translation depends on the base type. If it is a struct, union, or array type or a typedef
name referencing such a type, the pointer type is translated to the translation of the base type in its
boxed form. If it is a function type (possibly after resolving type names), the function type
is encoded as described in Section~\ref{design-types-pointer} using the function \code{encodeType}, then the function
pointer type is constructed from the encoding as described in Section~\ref{design-types-pointer}. The monadic operation
\begin{verbatim}
  encodeType :: Bool -> String -> Type -> FTrav String
\end{verbatim}
works similar to \code{transType} but returns the encoding as a string instead of the translated 
Cogent type. In particular, it optionally resolves typedef names and respects parameter modification specifications.
If the base type is \code{void} the special type name \code{CVoidPtr} is used.

In all other cases, as described in Section~\ref{design-types}, 
the pointer type is translated to the parametric type \code{CPtr} with the translated base type as type argument.

For an array type, the translation is the parameterized type for an array according to the number of elements. 
The translated base type is used as type argument. In case that the base type is again an array type (multidimensional
array), an explicit unbox operator is applied to the type argument to differentiate the multidimensional array 
from an array of array pointers, as described in Section~\ref{design-types-array}.

A function type is always translated to the corresponding Cogent function type, where a tuple type is
used as parameter type if there is more than one parameter, and the unit type is used if there is
no parameter. If a parameter modification specification is available, it is used to determine whether a parameter type
should be translated as readonly, or whether it should be returned as part of a result tuple. Otherwise
a default parameter modification specification is used which is constructed from the parameter type.
It assumes that a parameter is modified if its type is not qualified as \code{const} and all directly or indirectly 
contained pointers are also qualified as \code{const}.

For a qualified C type Gencot only respects the \code{const} qualifier. For a direct type the \code{const}
qualifier is ignored, since in Cogent values of unboxed and regular types are always immutable. For
a function type the qualifier is also ignored since function types are regular in Cogent. All other array types 
and all pointer types are translated to linear types which can be mutable in
Cogent. Whenever the C type is intrinsically immutable, it is translated to a readonly Cogent type by 
applying a bang operator to it. The immutability is determined by function\code{isReadOnlyType}. It is implemented
by a monadic action since it must access the symbol table to resolve references to tagged C structs.

\subsection{External Functions and Variables}
\label{impl-ccomps-externs}

For the exit wrapper functions Gencot generates the Cogent abstract function definitions and the implementations
in antiquoted C. Exit wrappers are generated for all functions which are invoked but not defined in the Cogent code. 
The information required for each of these functions is contained in the C function declaration.

Additionally, Gencot generates Cogent abstract function definitions for access functions to selected external
variables without providing an implementation in antiquoted C.

\subsubsection{Determining External Functions}

The easiest approach would be to generate an exit wrapper for every function which is declared but 
not defined in the C program. However, often an
included C source file contains many function declaration where only some of them are invoked. Therefore, Gencot
determines the declarations of all actually invoked external functions with the help of the call graph as described in 
Section~\ref{impl-ccode-callgraph}. Only for them declarations are translated.

The external invoked functions are further filtered as follows:
\begin{itemize}
\item If the type is a function pointer, invocations are translated to Cogent in a different way as 
described in Section~\ref{design-types-function}. Therefore we omit all function pointer invocations.
\item A function which is only declared may be incomplete, i.e., its parameters (number and types) are not known. 
We could translate such function declarations to Cogent abstract functions with Unit as argument type, however,
that would not be related to the invocations and thus be useless. Therefore we omit invocations
of incompletely declared functions.
\end{itemize}

An identifier seen by Gencot in a function invocation may also be a macro call which shall be translated by Gencot. 
In this case a dummy function declaration must be manually provided for it as part of the Gencot macro call conversion
(see Section~\ref{design-preprocessor-macros}), so that the language-c parser can parse the invocations. By 
using either an incomplete declaration or a complete declaration it can be controlled whether a Cogent abstract 
function definition is generated for it. If the macro is translated to a macro in Cogent, no abstract function
definition shall be generated and an incomplete dummy declaration must be used.

There are cases where an external function is used without being seen by the call graph. This is the case when
the function is assigned to a function pointer or when the function invocation is created by a macro call and
is thus not visible for the call graph. For these cases Gencot supports a list where additional functions
to be processed can be specified in an auxiliary input file by their names, each in a seperate line. Only
functions with external linkage can be specified, for them the name is unique in the <package>.

\subsubsection{Determining External Variables}

As for functions, Gencot also restricts the processed external variables to a subset of the declared variables.
It would be possible to determine the accessed variables in a similar way as the functions are determined with
the help of the call graph. Gencot does this for all variables of function pointer (array) type, using the 
call graph. Variables accessed in other ways or of other types may be specified explicitly as a list. This
provides more control about the processed external variables for the developer.

The external variables are specified in the same list as the additional external functions.

\subsubsection{Processor \code{gencot-externs}}

The processor \code{gencot-externs} generates the Cogent abstract function definitions. It is invoked as
\begin{verbatim}
  gencot-externs [<parmod file> [<varlist file>]]
\end{verbatim}
where the optional arguments are a file with parameter modification descriptions and with the names of external
variables to be processed.

The processor parses and analyses
all C source files specified in the input file name list, resulting in the list of corresponding symbol tables, as
described in Section~\ref{impl-ccode-package}. Then
it determines for every table the invoked functions and function pointer (array)s as described in 
Section~\ref{impl-ccode-callgraph}. This must
be done before combining the tables, since invocations may also occur in the bodies of functions with internal
linkage. These functions are removed during table combination.

After determining the invoked functions, the tables are combined as described in Section~\ref{impl-ccode-package}
and the invocations are reduced to those for
which a declarations is present in the combined table. These are the invocations of functions and function
pointer (array) variables external to all
read C sources. The processor translates their declarations using the function \code{transGlobals} of
module \code{Gencot.Cogent.Translate}. It translates C function and variable declarations to abstract function 
definitions
in Cogent. It reads and uses parameter modification descriptions to determine readonly parameter types, the
descriptions are generated by \code{parmod-externs} as described in Section~\ref{impl-ccomps-parmod}.
The result is output using \code{prettyTopLevels} from module \code{Gencot.Cogent.Output}.

The processor takes two files as optional arguments. The first file contains parameter modification descriptions
and is used for translating the types of functions and function pointers. If it is not specified all linear
parameters are assumed to be modifyable by the function. The second file contains a list of newline-separated 
names of external variables to be processed in addition to the invoked function pointer (array)s. This file can
only be specified if the parameter modification description file is also specified. If no parameter modifications
shall be used, a file containing the empty list \code{[]} can be used instead.

\subsubsection{Processor \code{gencot-externsac}}

The processor \code{gencot-externsac} generates the wrapper implementations in antiquoted C code.

\subsection{External Cogent Types}
\label{impl-ccomps-exttypes}

The processor \code{gencot-exttypes} generates all Cogent type definitions which origin from a C system include file.
It is invoked as
\begin{verbatim}
  gencot-exttypes [<parmod file> [<varlist file>]]
\end{verbatim}
where the optional arguments are a file with parameter modification descriptions and with the names of external
variables to be processed.

Since often only a small selection of the types defined by a system include file is used by a program, Gencot performs 
an analysis to reduce the type definitions to be translated. It determines all types actually \textit{used} by
referencing them somewhere in the program.

C Types are used in a C source in the following cases:
\begin{itemize}
\item As type of a global or local variable.
\item As result or parameter type of a derived function type.
\item As base type of a derived pointer or array type.
\item As type of a field in a composite type (struct or union).
\item As defining type for a typedef name.
\end{itemize}
There are other uses of types in C, such as in \code{sizeof} expressions, these cases are ignored by Gencot and 
must be handled manually.

We call the items which may have a separate syntactically
specified C type a ``type carrier''. These are variables, functions, fields, and typedef names. Since the language-c
parser constructs for every defined or declared function the derived function type, variables and functions
can be treated in the same way, they only differ in their type: functions have a derived function type, whereas
variables may have all other types, including types derived from function types such as function pointer types. Fields 
always belong to a composite type, therefore we use as type carrier the composite type instead of the single fields.
Together we have three kinds of type carriers: variable/functions, composite types, and typedef names.

Every type carrier has one or (in the case of a composite type) several associated ``syntactic types''
which are the type expressions as specified in the C source. If a syntactic type is a derived type it contains
``syntactic part types'' (base types, parameter and result types). 

We also include enum types as type carriers. They have no associated types, since they always correspond to a name 
for a primitve (numeric) type. However, \code{gencot-exttypes} must determine and translate enum types which 
are defined in system include files similar to a composite type.

Generally, Gencot translates type carriers to Cogent. It translates functions as described in Section~\ref{design-fundefs}
and it translates enum types, composite types and type definitions as described in Section~\ref{design-types}.
Gencot translates global variables to an abstract access function, using the translated type of the variable.
Gencot does not translate local variables, this must be done manually. To support the manual translation, 
however, Gencot treats local variables as if it would translate them to Cogent code referencing the translated 
declared type.

The task of \code{gencot-exttypes} is to translate all type carriers referenced in the translated code for which
no definition has been generated by translating the C sources of the Cogent translation unit, i.e., which have
been defined in a system include file.

When Gencot translates a type carrier, it references at most the associated syntactic types in the resulting Cogent code.
This is the case for struct types and for functions. Union types are translated without referencing the field types
(see Section~\ref{design-types-struct}). To support a manual translation \code{gencot-exttypes} treats them in 
the same way as struct types.

Gencot translates derived types to a parameterized Cogent type with the base type as type argument. Hence the syntactic 
part types are always required as type arguments. Therefore \code{gencot-exttypes} also handles all syntactic
part types associated with a type carrier.

For functions, the associated syntactic type is either a typedef name (resolving to a derived function type) 
or a syntactic derived function type. In the latter case no Cogent type definition is needed for it, since Gencot
directly translates it to a Cogent type expression (Section~\ref{design-fundefs}). 

\subsubsection{Collecting Used Types}

A type can be used by code in a translated C soure file or by a declaration of an external function or variable.
As described in Section~\ref{impl-ccomps-externs}, Gencot only translates external functions and variables if they are 
actually invoked or explicitly listed by the developer.
Therefore, \code{gencot-exttypes} determines the external invoked functions and variables in the same way. Note that
the external invoked functions include those defined in the <package> outside the Cogent compilation unit, as well
as those declared in a system include file.

Enum types, composite types and typedef names are reduced by an analysis of type usage by reference. Every type carrier can be
associated with the enum types, composite types and typedef names referenced in all associated syntactic part types. For 
composite types and typedef names this usage relation is transitive. To determine the relevant enum types, composite types 
and typedef names defined in system include files, \code{gencot-exttypes} starts with all initial type carriers belonging
to the Cogent compilation unit, together with all external invoked functions and the processed external variables. 
It then determines all transitively
used enum types, composite types and typedef names (which all must be defined in system include files).

To further reduce type carriers, the types associated with external type carriers are ``fully resolved'', i.e.~all
typedef names occurring in them are transitively resolved and replaced by their definition. The exception are
typedef names which are referenced by the initial type carriers, they are preserved since they are needed anyways.

For the resulting set of type carriers the required type definitions are generated.

The initial type carriers are collected with the help of the callback handler during analysis, as described in 
Section~\ref{impl-ccode-read}. All type carriers can be wrapped as a \code{DeclEvent}. The callback handler 
is automatically invoked for all type carriers in the parsed
C source files, including local variables (which are not present anymore in the symbol table after the analysis phase).

Gencot uses the callback handler
\begin{verbatim}
  collectTypeCarriers :: DeclEvent -> Trav [DeclEvent] ()
\end{verbatim}
in module \code{Gencot.Util.Types}. It adds all \code{DeclEvent}s with relevant type carriers to the list in the 
user state of the \code{Trav} monad. The handler is used for every \code{.c} file separately, the collected
type carriers must be combined afterwards to get all type carriers in the Cogent compilation unit. This can be
done by first combining the symbol tables as described in Section~\ref{impl-ccode-package}, and then using the union of all 
type carriers which are local, have internal linkage, or are still contained in the resulting combined table. 
This will avoid duplicate occurrences of type carriers.

All parameter declarations are ignored by the callback handler, since for every parameter declaration there must be 
a \code{DeclEvent} for the containing function, which is processed by the handler. Function and global variable declarations 
are also ignored, since they represent external functions and variables which are separately determined. Also,
a function or variable declared in one source file may be defined in another source file of the Cogent compilation unit and thus 
be not external. Type carriers where all associated types are primitive (including enum types) are also ignored, 
since they may neither contain derived types nor reference other type carriers.

Enum types, composite types and typedef names declared in system include files are omitted, since they will be determined 
using the type usage relation. To determine whether
a type carrier is defined in a system include file, the simple heuristics is used that the source file name is specified
as an absolute pathname in the \code{nodeinfo}. System includes are typically accessed using absolute pathnames of the 
form \code{/usr/include/...} (after being searched by cpp). If include paths are specified explicitly for Gencot, 
system include paths must be 
specified as absolute paths whereas include paths belonging to the <package> must be specified as relative paths
to make the heuristics work.

After the language-c analysis phase the declarations of all invoked external functions are determined 
as described in Section~\ref{impl-ccomps-externs}. If the associated syntactic type is a typedef name, it is resolved
to a derived function type. Afterwards, all invoked external functions where not all associated syntactic types
are primitive are added to the combined type carriers determined by the callback handler. No duplicate type carriers
will result here since the combined symbol table contains every external declaration only once.

The resulting set of type carriers is transitively completed by adding all type carriers used by referencing them,
as described above. This will add all required enum types, composite types and typedef names defined in system include files.
Here, type carriers where all associated types are primitive are not omitted, since 
\code{gencot-exttypes} must translate all added type carriers. The same holds for enum types. 
Duplicates could occur if a type carrier is referenced
in several different types, they must be detected and avoided. Since all type carriers correspond to 
syntactic entities in a source file, they can be identified and compared by their position and source file.

\subsubsection{Translating Type Definitions}

For all external enum types, composite types and typedef names in the resulting set \code{gencot-exttypes} translates the
definition to Cogent as described in Sections~\ref{design-types}.
All types used in these definitions are fully resolved, as described above.

To support the full resolving of types, the normal monadic action \code{transGlobals} used by the main translation 
to Cogent described in Section~\ref{impl-ccomps-main} cannot be used. The resolving needs access to all
type names directly referenced from the Cogent compilation unit to stop the resolving there.
The translation of external types is implemented by the monadic action 
\begin{verbatim}
  transExtGlobals :: [String] -> [DeclEvent] -> FTrav [GenToplv]
\end{verbatim}
in module \code{Gencot.Cogent.Translate} where the string list contains the referenced type names.

\subsection{Derived Types}
\label{impl-ccomps-dvdtypes}

The processor \code{gencot-dvdtypes} generates all Cogent type definitions required for the derived types
used in the C program. It is invoked as
\begin{verbatim}
  gencot-dvdtypes [<varlist file>]
\end{verbatim}
where the optional argument is a file with the names of external variables to be processed.

A C derived type, is a pointer type, array type or function type. As described in 
Section~\ref{design-types}, Gencot maps all these types to parameterized Gencot types. The generic types \code{CPtr},
\code{CFunPtr} and \code{CFunInc} used for pointer and function types are predefined by Gencot. However, function
pointer types use abstract types for encoding the base function type and the generic 
types of the form \code{CArr<size>} used for array types syntactically contain the array size, both
depend on the translated C program. For these types Cogent definitions must be provided.

Derived types are used in C in the form of type expressions, i.e., base types, result and parameter types are 
syntactically included in the type specification. With some exceptions, derived types may occur wherever a type
may be used. Therefore, to determine all used derived types, Gencot determines all used type carriers as
described for \code{gencot-exttypes} in Section~\ref{impl-ccomps-exttypes}. These include all type carriers used 
in translated C sources, in declarations of external functions and variables and those defined in system
include files.

For the same reason as for \code{gencot-exttypes} Gencot also processes the syntactic part types of derived types,
if they are again derived types.

Together, \code{gencot-dvdtypes} processes
\textit{all} type carriers for all associated syntactic part types which are a derived array or function pointer type.

\subsubsection{Generating Type Definitions}

For all derived array types associated as syntactic part type with a type carrier in the used set, two abstract 
generic type definitions must be generated
(see Section~\ref{design-types-array}). However, only one such pair may be generated for every array \code{<size>}, 
although array types with a specific size may be associated with several different type carriers.

For all derived function pointer types an abstract type definition must be generated for encoding the function
type used as its base type.

Gencot first generates the translated Cogent type names for all derived array and function pointer types associated 
as syntactic part type with a type carrier. Then it generates the corresponding definitions for every type name
used in all these derived types. If a translated array type uses the generic name \code{CArrXX} no definition
is generated since this type name is predefined by Gencot.

Generating the type definitions for derived array and function pointer types is implemented by the monadic action
\begin{verbatim}
  genTypeDefs :: [DeclEvent] -> FTrav [GenToplv]
\end{verbatim}
in module \code{Gencot.Cogent.Translate}, where the list of \code{DeclEvent}s corresponds to the type carriers
to be processed.

\subsubsection{Processor \code{gencot-dvdtypesah}}

\subsection{Declarations}
\label{impl-ccomps-decls}

The filter \code{gencot-deccomments} is used to provide the information about the positions of all declarations with
external linkage in
a translated file, as described in Section~\ref{impl-comments-decl}. This information is used to move declaration
comments to the corresponding definitions, as described in Section~\ref{design-comments-decl}.

Only global declarations are processed by this filter. Therefore the filter processes the list of \code{DeclEvents}
returned by \code{getDeclEvents} (see Section~\ref{impl-ccode-read}). The predicate for filtering the list
selects all declarations with external linkage (i.e. removes all object/function/enumeration/type definitions and 
all declarations with internal or no linkage). 
The main application case are functions declared as external, but moving the comments may also be useful for 
objects.

If a function or object with external linkage is declared and defined in the same file, the language-c analysis step
replaces the declaration by the corresponding definitions in the semantic map, hence it will not be found and processed
by \code{gencot-deccomments}. However, it is assumed that in this case the documentation is assocatied with the definition
and need not be moved from the declaration to the definition.

For processing the declarations we do not need a monad since every declaration can be processed on its own. This is implemented
by the function
\begin{verbatim}
  transDecl :: DeclEvent -> String
\end{verbatim}
in module \code{Gencot.Text.Decls}. It is mapped over the list of \code{DeclEvents} and the resulting string list
is output one string per line.

\subsection{Parameter Modification}
\label{impl-ccomps-parmod}

\subsubsection{\code{parmod-gen}}

The filter \code{parmod-gen} is used to generate the Json parameter modification description from a C source file.
It may be invoked in two forms:
\begin{verbatim}
  parmod-gen <source file name>
  parmod-gen <source file name> close
\end{verbatim}
The second form runs the filter in ``closing mode''.

In normal mode the filter processes all \code{DeclEvent}s in the source file which are function
definitions or definitions for objects with function pointer (array) type. Every such definition is translated to an
entry in the parameter modification description. A function definition is translated to the sequence of its
own description entry and the entries for all invoked local function pointer (array)s and all parameters for which
the type is a function or function pointer (array) directly including the derived function type.

The filter processes every definition of a composite type in the source file and translates all
its members with function pointer (array) type directly including the derived function type to the corresponding
description. The filter processes every type name definition in the source file where the definig type
is a function (pointer (array)) type directly including the derived function type, and translates it to
the corresponding description.

The descriptions of functions, their function parameters, composite type members, and typedef names are 
intended to be used by \code{gencot-translate}
when it translates the function and type definitions. The descriptions of invoked local function pointer (array)s
are intended for evaluating dependencies as described in Section~\ref{impl-parmod}.

In closing mode the filter processes all \code{DeclEvent}s in the source file and in all included files which
are declarations for functions or function pointer (array)s. Every declaration is translated to an entry in the parameter
modification description. These descriptions are only intended for ``closing'' the dependencies for the evaluation.

After analysing the C code as described in Section~\ref{impl-ccode} the call graph is generated as described
in Section~\ref{impl-ccode-callgraph}. Then another traversal is performed using the \code{CTrav} monad and
\code{runWithCallGraph} with action
\begin{verbatim}
  transGlobals :: [DeclEvent] -> CTrav Parmods
\end{verbatim}
defined in module \code{Gencot.Json.Translate}. The resulting list of JSON objects is output as described 
in Section~\ref{impl-parmod-json}.

\subsubsection{Processor \code{parmod-externs}}

The processor \code{parmod-externs} is used to generate the parameter modification descriptions intended
to be used by \code{gencot-externs} (see Section~\ref{impl-ccomps-externs}) and \code{gencot-exttypes} 
(see Section~\ref{impl-ccomps-exttypes}). It is invoked as
\begin{verbatim}
  parmod-externs [<varlist file>]
\end{verbatim}
where the optional argument is a list of names of external variables to be processed.

Since all these parameter modification descriptions are already
generated by \code{parmod-gen}, the processor only generates a list of function identifiers, as described in 
Section~\ref{impl-parmod-ids}. Every identifier is output in a separate line. The identifiers can be used
to specifically select and manually process only those descriptions which are actually required by
\code{gencot-externs} and \code{gencot-exttypes}. 

However, only identifiers for those descriptions are listed,
which are not yet needed by \code{gencot-translate} to translate the source files of the Cogent compilation unit.
The idea is to merge these with the descriptions selected by the listed identifiers to determine the 
descriptions needed by \code{gencot-externs} and \code{gencot-exttypes}.

The processor \code{gencot-externs} processes declarations of external invoked functions. It needs parameter
modification descriptions for these functions and for their parameters of function (pointer (array)) type.
The descriptions for the parameters are needed by \code{gencot-externs} to translate the parameter type as part of the 
generated abstract Cogent function definition. 

Although \code{gencot-externs} only processes complete function declarations of non-variadic external functions, 
\code{parmod-externs} also lists identifiers for incompletely declared functions and for variadic functions,
if they are invoked in the Cogent compilation unit. This is intended as an 
information for the developer who has to translate invocations of such functions manually. All these
external invoked functions are determined with the help of the call graph in the same way as described for 
\code{gencot-externs} in Section~\ref{impl-ccomps-externs}.

The processor \code{gencot-externs} also processes external variables, if they are invoked or listed explicitly
by the developer. Parameter modification descriptions are only relevant for variables of function pointer (array)
type. There may be such variables which are not invoked, but explicitly specified by the developer. For this
reason \code{parmod-externs} also reads the list of external variables to be processed in addition to the
invoked variables.

External invoked composite type members are not included, since they will be added through the used types below. To invoke 
a composite type member an object must exist which is a type carrier for the composite type. The parameter modification
description identifier for the member will be listed when this type carrier or a typedef referenced by it
is processed.

The processor \code{gencot-exttypes} processes derived types of all type carriers and translates enum types,
composite types and typedef names defined in system include files. It needs parameter modification descriptions
for both cases. 

Since parameter modification descriptions must be associated with a function identifier, they cannot be used
for arbitrary types. They are only used for composite type members with a function pointer (array) type and 
for typedef names where the defining type is a function (pointer (array)) type. In both cases the derived
function type must be syntactically included. If it is specified using a typedef name, the parameter modification
description is only used for translating the corresponding type definition.

Accordingly, \code{parmod-externs} lists for all processed type carriers the function ids of all composite type
members and of all typedef names which have a type of this kind. These descriptions cover all cases needed
by \code{gencot-exttypes}.

The type carriers to be processed are determined in the same way as by \code{gencot-exttypes}. Note that the 
external invoked functions and variables are already included there, however, those with only primitive types are omitted.
Here we also include them, since \code{gencot-externs} processes them. 
For external function pointer (array)s descriptions for parameters are not listed, since
for the parameter types of function pointers no parameter modification descriptions are used by Gencot, they
are always translated assuming as default that all nonlinear parameters are modified.

Descriptions for composite types and typedef names need only be processed if they are defined in a system include
file. In all other cases they are already needed by \code{gencot-translate} for translating the source files
of the Cogent compilation unit and are omitted here.

For the determined type carriers \code{parmod-externs} lists the identifiers of 
the parameter modification descriptions for the following entities:
\begin{itemize}
\item declarations of external invoked functions, together with all their parameters, if the parameter
type is a function (pointer (array)) type directly specifying the function type,
\item declarations of external invoked or listed function pointer (array)s, if the declared type directly specifies
the function type,
\item members of all composite types defined in system include files, if their type resolves to a function pointer (array) type.
\item typedef names defined in a system include file, if they resolve to a function (pointer (array)) type.
\end{itemize}

Note that typedef names and member types in system include files need not specify a function type directly for being listed, 
since types in external type carriers are fully resolved by \code{gencot-exttypes} and the resulting type is needed to translate 
the type definition.


\section{Other Components}
\label{impl-ocomps}
The following auxiliary Gencot components exist which do not process C source code:
\begin{description}
\item[\code{parmod-proc}] processes parameter modification descriptions in JSON format (see Section~\ref{impl-parmod}).
\item[\code{items-proc}] processes item property declarations in text format (see Section~\ref{impl-itemprops}).
\item[\code{auxcog-macros}] processes Cogent source code to select macro definitions.
\item[\code{auxcog-mapback}] processes Cogent source code to map Cogent constants back to preprocessor constants.
\item[\code{gencot-prclist}] processes list files to remove comments.
\end{description}

\subsection{Parameter Modification Descriptions}
\label{impl-ocomps-parmod}

Processing parameter modification descriptions is implemented by the filter \code{parmod-proc}. It reads a parameter
modification description from standard input as described in Section~\ref{impl-parmod-json}. The first command line 
argument acts as a command how to process the parameter modification description. The filter implements the following
commands with the help of the functions from module \code{Gencot.Json.Process}  (see Section~\ref{impl-parmod-modules}):
\begin{description}

\item[\code{check}]
Verify the structure of the parameter modification description 
according to Section~\ref{impl-parmod-struct} and lists all errors found (not yet implemented).

\item[\code{unconfirmed}] 
List all unconfirmed parameter descriptions using function \code{showRemainingPars}.

\item[\code{required}]
List all required invocations by their function identifiers, using function \code{showRequired}.

\item[\code{funids}]
List the function identifiers of all functions described in the parameter modification description.

\item[\code{sort}]
Takes an additional file name as command line argument. The file contains a list of function identifiers.
The input is sorted using function \code{sortParmods}. This command
is intended to be applied after the \code{merge} command to (re)establish a certain ordering.

\item[\code{filter}]
Takes an additional file name as command line argument. The file contains a list of function identifiers.
The input is filtered using function \code{filterParmods}.

\item[\code{merge}]
Takes an additional file name as command line argument. The file contains a parameter modification description 
in JSON format. Both descriptions are merged using function \code{mergeParmods}
where the first parameter is the description read from standard input.
After merging, function \code{addParsFromInvokes} is applied.
Thus, if the merged information contains
an invocation with more arguments than before, the function description is automatically extended.

\item[\code{eval}]
Using the functions \code{showRemainingPars} and \code{showRequired} it is verified that the parameter modification 
description contains no
unconfirmed parameter descriptions and no required dependencies. Then it is evaluated using function \code{evalParmods}.
The resulting parameter modification description contains no parameter dependencies and
no invocation descriptions. It is intended to be read by the filters which translate C function types and function 
definitions.

\item[\code{out}]
Using the function \code{convertParmods} the parameter modification description is converted to an item property 
map. The map is written to the standard output in the format described in Section~\ref{impl-itemprops-decl}.
\end{description}

All lists mentioned above are structured as a sequence of text lines.

If the result is a parameter modification description in JSON format it is written to the output as described in 
Section~\ref{impl-parmod-json}.

The first three commands are intended as a support for the developer when filling the description manually. The goal is that
for all three the output is empty. If there are unconfirmed parameters or heap use specifications they must be inspected and confirmed. This usually 
modifies the list of required invocations. They can be reduced by generating and merging corresponding descriptions
from other source files.

\subsection{Item Property Declarations}
\label{impl-ocomps-items}

Processing item property declarations is implemented by the filter \code{items-proc}. It reads item property declarations
from standard input as a sequence of text lines in the format described in Section~\ref{impl-itemprops-decl}. The first command line 
argument acts as a command how to process the item property declarations. The filter implements the following
commands:
\begin{description}

\item[\code{merge}]
Takes an additional file name as command line argument. The file contains additional item property declarations.
Both declarations are merged. If both contain properties for the same item the union of properties is declared for it.
If the union contains negative properties they are removed together with all positive occurrences of the same
property. 
\item[\code{idlist}]
Outputs the list of item identifiers for all toplevel items occuring in the declaration.
\item[\code{filter}]
Takes an additional file name as command line argument. The file contains a list of item identifiers.
The declarations are filtered, only declarations for those items remain where a prefix of the identifier is in the 
item identifier list. Usually, this command is applied to find the property declarations for all subitems of
a given list of toplevel item identifiers.

\end{description}

\subsection{Processing Cogent Code}
\label{impl-ocomps-cogent}

There are some Gencot components which process a Cogent source to generate auxiliary files in antiquoted C or normal C code
or other output.
All these components are invoked with the name of a single Cogent source file as argument. Additionally, directories for
searching files included by cpphs can be specified with the help of \code{-I} options. The result is always written to 
standard output.

\subsubsection{\code{auxcog-remcomments}}

This component is a filter which removes comments from Cogent sources. Note that the Cogent preprocessor \code{cpphs}
does not remove comments, so it cannot be used for this task.

The filter is implemented as an awk script.

\subsection{Using Macros in Generated C Code}
\label{impl-ocomps-macros}

The components described here support the transferral of some preprocessor macro definitions from the Cogent source
to the generated C source. This feature has been used when array types have been mapped to abstract Cogent types. 
Since array types are now mapped to Cogent builtin array types, it is no more required and has been deprecated. For
information reasons the concept is still documented here.

Note that in a definition 
\begin{verbatim}
  type CArrX<id>X el = {arrX<id>X: el#[<id>]}
\end{verbatim}
using a builtin array type the macro name \code{<id>} is still present in the type name on the left side, however, its
second occurrence on the right side will be expanded, so the numerical value of the array size is known and used by the Cogent compiler
when it generates the C source.

\subsubsection{\code{auxcog-macros}}

Gencot array types may have the form \code{CArrX<id>X} where \code{<id>} is a preprocessor constant (parameterless macro) 
identifier specifying the array size (see Section~\ref{design-types-array}). If these types are implemented using
abstract types in Cogent, they must be implemented in antiquoted C. The implementation has the form 
\begin{verbatim}
typedef struct $id:(UArrX<id>X el) {
  $ty:el arrX<id>X[<id>];
} $id:(UArrX<id>X el);
\end{verbatim}
where the identifier \code{<id>} occurs as size specification in the C array type. Cogent translates this antiquoted 
C definition to normal C definitions for all instances of the generic type used in the Cogent program. These C definitions
are included in the C program generated by Cogent.
Thus the definition of \code{<id>} must be available when the C program is processed by the C compiler or by Isabelle.

The definition of \code{<id>} may depend on other preprocessor macros, it may even depend on macros with parameters.
Therefore all macro definitions must be extracted from the Cogent source and made available as auxiliary C source.
This is implemented by \code{auxcog-macros} with the help of \code{cpp} and an \code{awk} script.

The macro definitions are extracted from the Cogent source \code{file.cogent} by executing the C preprocessor as
\begin{verbatim}
  cpp -P -dD -I<dir1> -I<dir2> ... -imacros file.cogent /dev/null 2> /dev/null
\end{verbatim}
The option \code{-imacros} has the effect that \code{cpp} only reads the macro definitions from \code{file.cogent}.
The actual input file is \code{/dev/null} so that no other input is read. The option \code{-dD} has the effect
to generate as output macro definitions for all known macros. This will also generate line directives, these are
suppressed by the option \code{-P}. The \code{-I} options must specify all directories where Cogent source files
are included by the preprocessor. All uses of the unbox operator \code{\#} in the Cogent code are signaled as an 
error by \code{cpp}, but the output is not affected by these errors. The error messages are suppressed by redirecting
the error output to \code{/dev/null}.

Note that \code{cpp} interpretes all include directives and conditional directives in the Cogent source. If for a macro
different definitions occur in different branches of a conditional directive, depending on a configuration flag,
\code{cpp} will select the definition according to the actual flag value.

The C preprocessor also adds definitions of internal macros. Some of them (those starting with \code{\_\_STDC}) are 
signaled as redefined when the output is processed again by \code{cpp}. This is handled
by postprocessing the output with an \code{awk} script. It removes all definitions of macros starting with \code{\_\_STDC}.

Since \code{cpp} does not recognize the comments in Cogent format, it will not eliminate them from the macro definitions 
in the output. Therefore it should be postprocessed by \code{auxcog-remcomments}. For a better result it should also 
be postprocessed by \code{auxcog-numexpr}.

\subsubsection{\code{auxcog-numexpr}}

The macro definitions selected by \code{auxcog-macros} are normally written as part of the Cogent program to expand 
to Cogent code. However, to be included in the C code they must expand to valid C code. This must basically be done
by the developer of the Cogent program. 

The filter \code{auxcog-numexpr} supports this by performing a simple conversion of numerical expressions from Cogent
to C. Numerical expressions are mostly already valid C code. Currently, the filter only removes all occurrences
of the \code{upcast} operator.

The filter is implemented as a simple \code{sed} script.

\subsubsection{\code{auxcog-mapback}}

The \code{auxcog-mapback} component supports this by converting specific kinds of Cogent code back to C. It handles 
the use of Cogent constants in integer value expressions. This occurs in code generated by Gencot: every reference 
to another preprocessor constant in the definition of a preprocessor constant is translated to a reference of the 
corresponding Cogent constant.

This is handled by including another C source file before the file generated by \code{auxcog-macros}. This
file is generated by \code{auxcog-mapback}. It contains a preprocessor macro definition for every Cogent constant
where the constant name is a mapped preprocessor constant name, which maps the Cogent constant name back to 
the preprocessor constant name. For example, if the preprocessor constant name \code{VAL1} has been mapped by
Gencot to the Cogent constant name \code{cogent\_VAL1}, the generated macro definition is
\begin{verbatim}
  #define cogent_VAL1 VAL1
\end{verbatim}

The component \code{auxcog-mapback} is implemented in Haskell and uses the Cogent parser to read the Cogent code.
It then processes every Cogent constant definition.

\subsection{Processing List Files}
\label{impl-ocomps-prclist}

Gencot uses some files containing simple lists of text strings, one in each line: the item property declaration list (see 
Section~\ref{impl-ccomps-itemprops}), and the list of used external items to be processed by \code{gencot-externs}, 
\code{gencot-exttypes}, and \code{gencot-dvdtypes}.

To be able to specify these lists in a more flexible way, Gencot allows comments of the form
\begin{verbatim}
  # ...
\end{verbatim}
where the hash sign must be at the beginning of the line.

The filter \code{gencot-prclist} is used to remove these comment lines (and also all empty lines). It is implemented
as an awk script.

\subsection{Processing Isabelle Code}
\label{impl-ocomps-isabelle}

Gencot processes Isabelle code as described in Section~\ref{impl-isabelle}. The extension of the shallow embedding
(see Section~\ref{impl-isabelle-shallow}) is implemented with the help of the filters \code{auxcog-embedding} and
\code{auxcog-shallow}. Both filters take as input an Isabelle theory such as that generated by Cogent in file
\code{X\_ShallowShared.thy} for the proof name \code{X}.

\subsubsection{auxcog-embedding}

This filter takes as additional argument the name of a file containing the Gencot standard components list (see 
Section~\ref{impl-ocomps-main}). It generates the Isabelle theory to be stored in file \code{X\_Shallow\_Embedding.thy}
and writes it to standard output.

The filter determines the theories to be imported from the entries \code{"gencot: <name>"} in the Gencot standard
components list. From its input it determines the array sizes for which locale interpretations must be created.

The filter is implemented as an awk script.

\subsubsection{auxcog-shallow}

This filter takes as additional argument the name of a file containing the textual struct type information which 
has been generated by filter \code{auxcog-ctypstruct} (see Section~\ref{impl-ccomps-ctypinfo}). It uses this 
information to determine the size of array types which have been defined using a constant name as size specification.

This filter replaces some type declarations and type synonyms in its input and writes the modified Isabelle theory
to standard output.

The filter is implemented as an awk script.

\subsection{Generating Main Files}
\label{impl-ocomps-main}

The main files as described in Section~\ref{design-files} are \code{<unit>.cogent} and \code{<unit>.c}. They
contain include directives for all parts of the Cogent unit sources, the former for the Cogent sources, the latter 
for the C sources resulting from translating the Cogent sources with the Cogent compiler.

The content of these main files depends on the set of \code{.c} files which comprise the Cogent compilation unit.
A list of these files, as described in Section~\ref{impl-ccode-package}, is used as input for the components generating
the main files. In contrast to the components reading the package C code, these components only process the file 
names and not their content. The file name \code{additional\_includes.c} is ignored in the list, it can be used
to specify additional \code{.h} files for Gencot components reading the C code, which are not used in the Cogent 
code or after compiling Cogent back to C. For an application see Section~\ref{app-transfunction-pointer}.

The content of the main files also depends on the additional standard library components used from the Gencot distribution
or the Cogent distribution. These components must be explicitly specified by the ``Gencot standard components list'' which
is a sequence of lines of the form
\begin{verbatim}
  <kind>: <name>
\end{verbatim}
where \code{<kind>} may be one of \code{gencot}, \code{common}, or \code{anti}. In the first case the file
\code{include/gencot/<name>.cogent} from the Gencot distribution is included in the Cogent program, in the 
second case the file \code{lib/gum/common/<name>.cogent} from the Cogent distribution is included in the 
Cogent program, and in the third case the file \code{lib/gum/anti/<name>.ac} from the Cogent distribution 
is processed by Cogent so that it becomes a part of the Cogent compilation unit.

\subsubsection{Generating the Cogent Main File}

The component \code{gencot-mainfile} generates the content of the Cogent main file \code{<unit>.cogent}. 
It is implemented as a shell script. It reads the list of
\code{.c} files comprising the Cogent compilation unit as its input and it takes the unit name \code{<unit>}
and the name of a file containing the Gencot standard components list as additional arguments. The generated content 
is written to the standard output. 

The component adds include directives for the following files:
\begin{itemize}
\item The files to be included according to the Gencot standard components list.
\item The file \code{x.cogent} for every source file \code{x.c} specified as input.
\item The files \code{<unit>-externs.cogent}, \code{<unit>-exttypes.cogent}, \code{<unit>-dvdtypes.cogent}.
\item The file \code{<unit>-manabstr.cogent}, if it exists in the current directory. This file can be used to
manually define additional abstract types and functions used in the translated program.
\end{itemize}

The files \code{x.cogent} must be included first, because they may contain definitions of preprocessor macros
which may also be used in the remaining files.

\subsubsection{Generating the C Main File}

The component \code{auxcog-mainfile} generates the content of the C main file \code{<unit>.c}. 
It is implemented as a shell script. It reads the list of
\code{.c} files comprising the Cogent compilation unit as its input and it takes the unit name \code{<unit>}
and the name of a file containing the Gencot standard components list as additional arguments. The generated content 
is written to the standard output. 

The component adds include directives for the following files:
\begin{itemize}
\item The file \code{<unit>-gencot.h} for non-generic abstract standard types used by Gencot.
\item The files \code{<unit>-dvdtypes.h} and \code{<unit>-exttypes.h}, if they exist in the current directory. 
They contain definitions for the non-generic abstract types defined in \code{<unit>-dvdtypes.cogent} and 
\code{<unit>-dvdtypes.cogent}, respectively.
\item The file \code{<unit>-manabstr.h}, if it exists in the current directory. This file can be used to
manually provide definitions for the non-generic abstract types defined in \code{<unit>-manabstr.cogent}.
\item The file \code{<unit>-gen.c} which must be the result of the translation of \code{<unit>.cogent}
with the Cogent compiler.
\item The file \code{<unit>-externs.c} with the implementations of all exit wrappers.
\item The file \code{<unit>-gencot.c}, if it exists in the current directory. It contains the 
implementations of all generic abstract functions the program uses from the Gencot standard library.
\item The file \code{<unit>-manabstr.c}, if it exists in the current directory. This file can be used to
manually provide implementations of the abstract functions defined in \code{<unit>-manabstr.cogent}.
\item The file \code{<unit>-cogent-common.c} with implementations of functions defined in \code{gum/common/common.cogent}
but not implemented by Cogent.
\item The file \code{x-entry.c} for every source file \code{x.c} specified as input. These files contain
the implementations of the entry wrappers.
\item The file \code{std-<name>.c} for every line \code{"anti: <name>"} specified in the Gencot standard
components list.
\end{itemize}

The files \code{<unit>-*.c} and \code{*-entry.c} result from postprocessing the corresponding 
\code{\_pp\_inferred.c} files by \code{auxcog}. The files \code{std-<name>.c} result from renaming the files
\code{<name>\_pp\_inferred.c} which have been generated by Cogent by processing \code{<name>.ac} as antiquoted C code.


\section{Putting the Parts Together}
\label{impl-all}
The single filters and processors of Gencot are combined for the typical use cases in the shell scripts
\code{gencot}, \code{items}, \code{parmod}, and \code{auxcog}. The script \code{items} is intended for handling all 
aspects of working with the item property declarations (see Section~\ref{impl-itemprops}), the script \code{parmod} 
is intended for handling all aspects
of working with the parameter modification descriptions (see Section~\ref{impl-parmod}), the script
\code{gencot} is intended for handling all other aspects of translating from C to Cogent. The script \code{auxcog}
is intended for handling all aspects of additional processing after or in addition to processing the Cogent program by the 
Cogent compiler.

\subsection{The \code{gencot} Script}
\label{impl-all-gencot}

\subsubsection{Usage}

The overall synopsis of the \code{gencot} command is
\begin{verbatim}
  gencot <options> <subcommand> [<file>]
\end{verbatim}

The \code{gencot} command supports the following subcommands:
\begin{description}
\item[\code{check}] Test the specified C source file or include file for parsability by Gencot (see Section~\ref{app-prep} for how to 
make a C source parsable, if necessary).

\item[\code{cfile}] Translate the specified C source file to Cogent. If the source file name is \code{x.c} the result is
written to file \code{x.cogent}

\item[\code{hfile}] Translate the specified C include file to Cogent. If the source file name is \code{x.h} the result is
written to file \code{x-incl.cogent}

\item[\code{config}] Translate the specified C include file to Cogent with specific support for configuration files as described
in Section~\ref{impl-preprocessor-config}. If the source file name is \code{x.h} the result is
written to file \code{x-incl.cogent}

\item[\code{unit}] Generate additional Cogent files for the Cogent compilation unit.

\item[\code{cgraph}] Print the call graph for the Cogent compilation unit.

\end{description}

The input file \code{<file>} must be specified for the first four subcommands and it must be omitted for the latter two.
The subcommands \code{unit} and \code{cgraph} use the ``unit name'' to determine the files to be processed.
If the unit name is \code{x} the subcommand \code{unit} generates the following additional files (see Section~\ref{design-files}):
\begin{itemize}
\item \code{x.cogent}
\item \code{x-externs.cogent}
\item \code{x-exttypes.cogent}
\item \code{x-dvdtypes.cogent}
\end{itemize}

The \code{gencot} command supports the following options:
\begin{description}
\item[\code{-I}] used to specify directories where included files are searched, like for \code{cpp}. The 
option can be repeated to specify several directories, they are searched in the order in which the options
are specified.

\item[\code{-G}] Directory for searching Gencot auxiliary files. Only one can be given, default is \code{"."}.

\item[\code{-C}] Directory for retrieving stored declaration comments. Default is \code{"./comments"}.

\item[\code{-u}] The unit name. Default is \code{"all"}.

\item[\code{-k}] Keep directory with intermediate files. This is only intended for debugging.

\end{description}

As described in Section~\ref{impl-comments-decl}, Gencot moves comments from C function declarations to translated function
definitions. Since a declaration may be specified in a different C source file (often an include file) than the definition, 
the subcommands \code{cfile}, \code{hfile}, and \code{config} write all declaration comments in the processed file to
the directory specified by the \code{-C} option and use all comments found there when translating function definitions.

\subsubsection{Auxiliary Files}

Gencot uses an approach, where all manual annotations added to the C program for configuring the translation to Cogent 
are stored in auxiliary files separate from the original C program sources. This way it is possible to update the C 
sources to a newer version without the need to re-insert the annotations.

There are different auxiliary files for different purposes. The \code{gencot} script determines the auxiliary files using
a predefined naming scheme. These files are searched in the directory specified by the \code{-G} option. All auxiliary files
are optional, if a file is not found it is assumed that no corresponding annotations are needed.

When processing a file \code{x.<ext>} where \code{<ext>} is \code{c} or \code{h}, the \code{gencot} command uses the 
following auxiliary files, if they exist:
\begin{description}
\item[\code{x.gencot-addincl}] The content of this file is prepended to the processed file before processing.
\item[\code{x.gencot-noincl}] List of ignored include files, as described in Section~\ref{impl-ccode-include}.
\item[\code{x.gencot-omitincl}] The Gencot include omission list (see Section~\ref{design-preprocessor-incl}).
\item[\code{x.gencot-ppretain}] The Gencot directive retainment list (see Section~\ref{design-preprocessor-config}).
\item[\code{x.gencot-chsystem}] The content of this file is inserted immediately after the last system include directive.
\item[\code{x.gencot-manmacros}] The Gencot manual macro list (see Section~\ref{design-preprocessor-macros}).
\item[\code{x.gencot-macroconv}] The Gencot macro call conversion (see Section~\ref{design-preprocessor-macros}).
\item[\code{x.gencot-macrodefs}] The Gencot macro translation for the processed file (see Section~\ref{design-preprocessor-macros}).
\end{description}

For all cases with the exception of the last two, additionally a file with name \code{common} instead of \code{x} ist used, when it
is present. If both files are present the concatenation of them is used. In this way it is possible to specify common annotations 
for all C sources and add specific annotations in the source file specific files.

The subcommands \code{unit} and \code{cgraph} use corresponding auxiliary files for every file \code{x.c} processed by them.

The subcommands \code{cfile}, \code{hfile}, \code{config}, and \code{unit} additionally use the auxiliary file named
\code{<file>-itemprops} which must contain all relevant item property declarations (see Section~\ref{impl-itemprops-decl}).

These subcommands also use the auxiliary file \code{common.gencot-namap} which must contain the name prefix map, 
as described in Section~\ref{impl-ccode-names}. No source file specific versions of the name prefix map are supported,
since the map is used globally for all names translated to Cogent.

Another kind of auxiliary files describe the Cogent compilation unit. These files are required to be present 
in the directory specified by the \code{-G} option. If they are not found, an error or a warning is signaled.

If the unit name is \code{<uname>} the auxiliary files for the Compilation unit are
\begin{description}
\item[\code{<uname>.unit}] The ``unit file''. It contains the list of C source file names
(one per line) which together constitute the Cogent compilation unit. Only the original source files \code{x.c} must 
be listed, include files \code{x.h} must not be listed.
\item[\code{<uname>-external.items}] The list of used external items, as generated by the processor \code{items-used}
(see Section~\ref{impl-ccomps-items}).
\end{description}

The first file is only used by the subcommands \code{unit} and \code{cgraph}. They process all files listed in the unit file
together with all included files. The second file is used by all \code{gencot} subcommands with the exception of \code{check}.

\subsubsection{Implementation}

**todo**

The intended use of filter \code{gencot-remcomments} is for removing all comments from input to the language-c parser.
This input always consists of the actual source code file and the content of all included files. The simplest approach
would be to use the language-c preprocessor for it, immediately before parsing. 

However, it is easier for the filter \code{gencot-rempp} to remove the preprocessor directives when the comments are 
not present anymore. Therefore, Gencot applies the filter \code{gencot-remcomments} in a separate step before applying
\code{gencot-rempp}, immediately after processing the quoted include directives by \code{gencot-include}.
 
The filters \code{gencot-selcomments} and \code{gencot-selpp} for selecting comments and preprocessor directives, however, are
still applied to the single original source files, since they do not require additional information from the included files.


\subsection{The \code{items} Script}
\label{impl-all-items}

\subsubsection{Usage}

The overall synopsis of the \code{items} command is
\begin{verbatim}
  items <options> <subcommand> [<file>] [<file2>]
\end{verbatim}

The \code{items} command supports the following subcommands:
\begin{description}
\item[\code{file}] Create default property declarations for all items defined in the specified C source file
or include file \code{<file>}. 

\item[\code{unit}] Create default property declarations for all external items used in the Cogent compilation 
unit. 

\item[\code{used}] List external toplevel items used in the Cogent compilation unit. 

\item[\code{merge}] Merge the declarations in \code{<file>} and \code{<file2>}. Two declarations for the 
same item are combined by uniting the properties.

\item[\code{mergeto}] Add properties from \code{<file2>} to items in \code{<file>}. Properties are only added
if their item is already present in \code{<file>}.

\item[\code{omitfrom}] Omit (remove) properties in \code{<file2>} from items in \code{<file>}. 

\end{description}

If the unit name is \code{<uname>}, subcommand \code{used} writes its output to the file \code{<uname>-external.items}
The presence of this file is expected by most other Gencot command scripts. All other subcommands write their result to 
standard output.

For the subcommands \code{unit} and \code{used} no \code{<file>} must be specified, instead, they use the unit name
in the same way as the subcommand \code{unit} of the \code{gencot} command.

The \code{items} command supports the options \code{-I}, \code{-G}, \code{-u}, \code{-k} with the same meaning as
for the \code{gencot} command.

\subsubsection{Auxiliary Files}

The subcommand \code{file} uses the same auxiliary files as the subcommands \code{cfile} and \code{hfile} of \code{gencot} with
the exception of \code{common.gencot-namap} and \code{<file>-itemprops}.
The subcommand \code{unit} uses the same auxiliary files as the subcommand \code{unit} of \code{gencot} with the exception
of \code{common.gencot-namap} and \code{<file>-itemprops}.
The subcommand \code{used} uses the unit file \code{<uname>.unit} like subcommand \code{unit} and additionally
the optional auxiliary file 
\begin{description}
\item[\code{<uname>.unit-manitems}] Manually specfied external items, as described in Section~\ref{impl-ccomps-items} as
input to the processor \code{items-used}.
\end{description}

\subsubsection{Implementation}

** todo **

The \code{unit} command is implemented by the processor \code{items-externs} (see Section~\ref{impl-ccomps-itemprops}). 
The postprocessing step for removing declarations of non-external subitems is implemented by the filter
\code{items-proc} with command \code{filter} (see Section~\ref{impl-ocomps-items}).

\subsection{The \code{parmod} Script}
\label{impl-all-parmod}

\subsubsection{Usage}

The overall synopsis of the \code{parmod} command is
\begin{verbatim}
  parmod <options> <subcommand> <file> [<file2>]
\end{verbatim}

The \code{parmod} command supports the following subcommands:
\begin{description}
\item[\code{file}] Create parameter modification descriptions for all functions defined in C source file 
or include file \code{<file>}.
The result is written to standard output.

\item[\code{close}] Create parameter modification descriptions for all functions declared for the C source file
\code{<file>} in the file itself or in a file included by it.
The result is written to standard output.

\item[\code{unit}] Select entries from \code{<file>} (a file in JSON format containing parameter modification
descriptions) for all external functions used in the Cogent compilation unit. 
The result is written to standard output.

\item[\code{show}] Display on standard output information about the parameter modification description in \code{<file>}.

\item[\code{idlist}] List on standard output the item identifiers of all functions described in the 
paramer modification description in \code{<file>}.

\item[\code{diff}] Compare the parameter modification descriptions in \code{<file>} and \code{<file2>}. The output
has the same form as the Unix \code{diff} command, however, entries of functions occurring in both files are 
directly compared.

\item[\code{iddiff}] Compare the item identifiers of all functions described in \code{<file>} and \code{<file2>}.
The output has the same form as the Unix \code{diff} command.

\item[\code{addto}] Add to \code{<file>} all entries for required dependencies found in \code{<file2>}. Both files must contain 
parameter modification descriptions in JSON format. The result is written to \code{<file>}.

\item[\code{mergin}] Merge the parameter modification description entries in \code{<file>} and \code{<file2>} by building 
the union of the described functions. If a function is described in both files the entry with more confirmed parameter 
descriptions is used. The result is written to \code{<file>}.

\item[\code{replin}] Replace in \code{<file>} all function entries by an entry for the same function in \code{<file2>}
if it is present and has not less confirmed parameters. Both files must contain 
parameter modification descriptions in JSON format. The result is written to \code{<file>}.

\item[\code{eval}] Evaluate the parameter modification description in \code{<file>} as described in 
Section~\ref{impl-parmod-eval}.  The resulting parameter modification description is written to standard output.

\item[\code{out}] Convert the evaluated parameter modification description in \code{<file>} to item property declarations.
The result is written to the two files \code{<file>-itemprops} (properties to be added to the default properties) 
and \code{<file>-omitprops} (properties to be removed from the default properties).

\end{description}

The subcommand \code{unit} expects no argument \code{<file>}, instead, it uses the unit name in the same way as the subcommand \code{unit} 
of the \code{gencot} command.

The \code{parmod} command supports the options \code{-I}, \code{-G}, \code{-u}, \code{-k} with the same meaning as
for the \code{gencot} command. They are only used for the subcommands \code{file}, \code{close}, and \code{unit}.

\subsubsection{Auxiliary Files}

The subcommands \code{file} and \code{close} use the same auxiliary files as the subcommands \code{cfile} and \code{hfile} of \code{gencot} with 
the exception of \code{common.gencot-namap} and \code{<file>-itemprops}.
The subcommand \code{unit} uses the same auxiliary files as the subcommand \code{unit} of \code{gencot} with the exception
of \code{common.gencot-namap} and \code{<file>-itemprops}.

\subsubsection{Implementation}

\paragraph{The subcommand \code{unit}} 
First, all C source files are prepared for parsing, as in command \code{gencot unit}. Then the list of used external toplevel 
items \code{<uname>-external.items} is passed to processor \code{items-extfuns} to determine the list of 
used external functions. This list is then used to filter the descriptions in \code{<file2>} with \code{parmod-proc filter}.

Additionally the result is tested, whether all external functions determined by \code{items-extfuns} have been found in 
\code{<file2>}. This is done by calculating the function ids of all descriptions in the result with \code{parmod-proc funids}, 
sorting both lists and comparing them with \code{diff}. If the result is not empty, a warning is displayed on standard error.

\subsection{The \code{auxcog} Script}
\label{impl-all-auxcog}

\subsubsection{Usage}

The overall synopsis of the \code{auxcog} command is
\begin{verbatim}
  auxcog <options> <subcommand> [<file>]
\end{verbatim}

The \code{auxcog} command supports the following subcommands:
\begin{description}
\item[\code{unit}] Generate additional C files for the Cogent compilation unit.

\item[\code{comments}] Remove comments from the Cogent source \code{<file>} and write the result to standard output.

\end{description}

The subcommand \code{unit} expect no argument \code{<file>}, instead, it uses the unit name in the same way as the subcommand \code{unit} 
of the \code{gencot} command. If the unit name is \code{x} the following additional files are generated 
(see Section~\ref{design-files}):
\begin{itemize}
\item \code{x.c}
\item \code{x-macros.h}
\item \code{x-gencot.c} (only if \code{x-gencot\_pp\_inferred.c} is present)
\item \code{x-externs.c}
\end{itemize}
and for all files \code{y.c} listed in the unit file \code{x.unit}:
\begin{itemize}
\item \code{y-entry.c}
\end{itemize}

The subcommand \code{comments} is mainly intended for testing the comment structure in a Cogent source generated by
\code{gencot}. In rare cases it may happen that a closing comment delimiter and a subsequent opening comment delimiter
are omitted, so that the Cogent compiler will not detect the wrong comment structure.

The \code{auxcog} command supports the options \code{-I}, \code{-u} with the same meaning as
the \code{gencot} command.

The \code{auxcog} command only uses the unit file \code{<uname>.unit} as auxiliary file for subcommand \code{unit}.

\subsubsection{Implementation}

**todo**



\end{document}
