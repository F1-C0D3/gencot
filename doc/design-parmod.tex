Gencot uses the information, whether a parameter value may be modified by a function, when it translates
the function, as described in Section~\ref{design-types-function}.

A parameter is modified, when a part of it is changed, which is referenced through a pointer. Only then the 
modification is shared with the caller and persists after the function call.

\subsection{Detecting Parameter Modifications}

A parameter may be modified by an assignment to a referenced part. Such an assignment
is easy to detect, if it directly involves the parameter name. However, the assignment can also be indirect, where
the parameter or part of it is first assigned to a local variable or another structure and then the assignment
is applied to this variable or structure. This implies that a full data flow analysis would be necessary
to detect all parameter modifications.

Gencot uses a simpler semiautomatic approach. It detects all direct parameter modifications which involve the 
parameter name automatically. It then generates a JSON description which lists all functions with their parameters
and the information about the detected modification cases. The developer has to check this description and
manually add additional cases of other parameter modifications. The resulting description is then read by
Gencot and used during translation.

Gencot also treats the special cases, where a parameter is discarded (by directly or indirectly invoking the C standard
function \code{free}) or is modified but already returned as the (single) result of the C function. These cases 
must also be detected by the developer.

Parameter modifications are only relevant if the parameter is a pointer or if a pointer can be reached by the parameter
and if at least one such pointer is not declared to have a \code{const} qualified referenced type. Gencot detects
this information from the C type and adds it to the JSON description, so that the developer can easily identify
the relevant cases where to look for modifications.

In the case of a variadic function, the number of its parameters cannot be determined from the function definition.
Here Gencot uses the invocation with the most arguments to determine the number of parameter descriptions added
to the JSON description.

A parameter may be modified by a C function locally, but it can 
also be modified by passing it or a part of it as parameter to an invoked function which modifies its corresponding
parameter. Gencot supports these dependent parameter modifications, by detecting dependencies on parameter
modifications by invoked functions and adding them to the JSON description. Again, only dependencies which
involve the parameter name are detected automatically, other dependencies must be added by the developer.

When Gencot reads the JSON template it calculates the transitive closure of all dependencies and uses it to
determine the parameter modification information. Thus the developer has only to look at every function locally,
it is not necessary to take into account the effects of invoked functions.

\subsection{Required Invocations}

To support the incremental translation of single C source files, Gencot determines the parameter modification
description for single C source files. It processes all function definitions in the file by analysing their 
function bodies. However, for an invoked function the definition may not be available, it may be defined externally
in another C source file. This situation must be handled manually: the developer has to process additional
C source files where the invoked functions are defined, to add their descriptions.

Gencot specifically selects only the relevant invocations, which are required because a parameter modification depends on it.
If a parameter is already modified locally, additional dependencies are ignored and the corresponding invocations
are not selected. In this way Gencot keeps the JSON descriptions minimal.

An invoked function may also be specified using a function pointer. In this case the possible function bodies cannot 
be determined from the program. Again, it is left to the developer to determine, whether such invocations may modify
their parameters. Gencot supports this by also adding parameter description templates for function pointers defined in the 
source file, to be filled by the developer. 

In contrast to functions, function pointers can also be defined locally in a function\footnote{The current version of
Gencot does not support C extensions for nested function definitions.} (as local variable or as parameter). Gencot
automatically adds description templates for all invoked local function pointers.

\subsection{External Invocations}

Finally, a required invocation may be external to the package which is processed by Gencot, such as an invocation of
a function in the C standard library. When all remaining required invocations are external in this sense, Gencot
can be told to ``close'' the JSON description. Then it uses the declarations of all required invocations to generate
a description template for each of them. Since no bodies are available, the developer must fill in the information
for all these external functions (and function pointers).

Invoked function pointers may also be members of a struct or union type. For them no definition exists, they are also 
handled by Gencot when closing a JSON description: for all required invocations of a struct or union member a 
description template is added which is identified by the member name and the tag of the struct or union type. Invocations
of function pointer members in anonymous struct or union types are ignored by the current Gencot version.
 
\subsection{Described Entities}

Gencot uses the information about parameter modifications when it maps a C function type or function pointer type
to Cogent, as described in Section~\ref{design-types-function}. 

C function types are only used when functions are declared or defined. Function definitions are translated to
Cogent for all functions defined in the Cogent compilation unit. Function declarations are translated to Cogent
for all functions invoked but not defined in the Cogent compilation unit. In both cases Gencot automatically 
determines the corresponding definitions or declarations and generates the JSON descriptions for them, to be
filled by the developer and read when translating the C code to Cogent.

C function pointer types can be used as type for several entities: for global and local variables, for members of struct
and union types, for parameters and the result of declared and defined functions, and as base type or parameter 
types in derived types.

As described in Section~\ref{design-fundefs-body}, local variable declarations are not translated by Gencot, hence
their types need not be mapped and no JSON description is provided if it is a function pointer type. In all other 
cases, however, the declaration is translated to Cogent and the type must be mapped. For every C source file (also 
\code{.h} files), the current version of Gencot provides JSON descriptions of all function pointer types which are 
used
\begin{itemize}
\item for a global variable defined in the file (including those with internal linkage, see Section~\ref{design-files}),
\item for a member of a tagged struct type defined in the file,
\item for a parameter of a function defined in the file,
\item as element type of an array type used for one of the kinds of entities specified above.
\end{itemize}
These are the cases where it is easy to create a unique identifier for the function pointer entities, which can be
understood by the developer and used to associate the description to the entity when translating it to Cogent. The
description is provided independent of whether the entity has parameters of linear type or not. If not, it is
not needed by Gencot for translating the entity definition, however it may be needed as required invocation for
calculating information for parameters depending on it.

In all other cases, such as a parameter type of another function pointer type, or a member of a tagless struct,
Gencot does not provide or use
JSON descriptions when mapping the type. For parameters of linear type it always assumes that they may be modified
and are not discarded and maps the type to a corresponding Cogent type name.

\subsection{Function Pointer Type Names}
\label{design-parmod-typedef}

In C, in the definition of an entity of function pointer type, the type may be specified directly as a derived
type, or it may be specified by a typedef name. In the latter case, several entities may share a common
function pointer type which is specified in the type name definition.

The parameter modification descriptions are needed for all single function pointer entities, since these are
used in invocations and may be required for calculating dependencies for parameters of other functions. If 
several entities use a common typedef name, they must agree on the parameter modification description and
the common description must be known to Gencot when translating the type name definition.

The situation can be even more complex if first a typedef name for a function type is defined and then several
different typedef names for function pointer types or function pointer array types are derived from it. All
these typedef names and the corresponding entities must agree on the parameter modification description.

However, entities using a common typedef name may be defined in different C source files and even in different
C compilation units, some of which belong to the Cogent compilation unit and others are external to it, but may
be added later. This makes it impossible to guarantee that the descriptions for all entities are available
and agree, if an incremental translation to Cogent shall be supported.

For this reason Gencot does not relate entities with a common typedef name and does not test whether their
descriptions agree, they are processed completely independently. For translating the type name definition,
Gencot uses another parameter modification description, which is also independent from those of the entities.
It is the task of the developer to take care that all these descriptions agree. If they do not, this will 
cause inconsistent uses of the Cogent type to which the common type name is mapped in the generated Cogent code.

Note that, in contrast to the entities, the defined type may directly be a function type. As described in 
Section~\ref{design-types-typedef} such a type definition is not translated to a definition for a Cogent function
type, instead it is translated in the same way as for a function pointer type, mapping to a generated Cogent
abstract type name. This makes it possible to translate derived pointer types and pointer array types to 
Cogent using the mapped type name only, the parameter modification description is only needed for translating
the original function type definition. For example the C type name definitions
\begin{verbatim}
  typedef int fun(int,char*);
  typedef fun *pfun;
\end{verbatim}
where the second parameter may be modified, are translated to the Cogent type definitions
\begin{verbatim}
  type Cogent_fun = #(CFun ((U32,CPtr U8) -> (U32,CPtr U8)))
  type Cogent_pfun = #Cogent_fun
\end{verbatim}
so that the parameter modification description is only needed for the first type definition translation.

The additional parameter modification descriptions are associated with the common type name of the definition
which directly contains the derived function type. For every C source file (also 
\code{.h} files) Gencot provides JSON descriptions for all such type name definition in the file, where the defined
type is a function type, a function pointer type, or an array type with a function pointer type
as element type. The
description is provided independent of whether the function type has parameters of linear type or not. If not,
the description is not used at all by Gencot, but it may be useful as additional information for the developer.

More complex types involving a derived function type are ignored. As for the corresponding
entities Gencot assumes for their translation that all parameters of linear type may be modified.

