Gencot supports basic operations either by providing a Cogent implementation, or by providing
a C implementation for functions defined as abstract in Cogent.

\subsection{Implementing Polymorphic Functions}
\label{impl-operations-poly}

If a basic function is implemented in Cogent, it is automatically available for all possible argument types, according
to the Cogent definition of the argument types. 

Basic functions which cannot be implemented in Cogent are defined as abstract in Cogent and are implemented in
antiquoted C. Antiquotation is used to specify Cogent types, in particular, the argument types of polymorphic
functions. Then the Cogent mechanism for creating all monomorphic instances used in a program.

However, for some of the basic functions defined by Gencot the argument types are restricted in ways which cannot
be expressed in Cogent. An example is the restriction to linear types. In Cogent the ``permissions'' \code{DS} express 
the restriction that an argument type must be sharable and discardable, i.e., \textit{not} linear. But the opposite
restriction cannot be expressed. 

In such cases Gencot only provides function implementation instances for the restricted argument types. If the function
is used with other argument types in a program this is not detected as error by the Cogent compiler. However, due to 
the missing C implementation the resulting C program will not compile.

\subsubsection{Genops}

Additionally, there are cases where the Cogent antiquotation mechanism is not sufficient for generating the 
monomorphic instances from a common polymorphic template. The antiquotation mechanism can replace Cogent type
and type argument specifications by their translation to a C type, but it cannot generate other C code parts 
depending on these types. This is required for several of the Gencot basic functions.

To provide instances also for these functions, Gencot extends the Cogent antiquotation mechanism by a postprocessing
mechanism called ``Genops''. It replaces specific code templates in the monomorphic function instances. The Genops
templates are specified in the antiquoted C definition of the polymorphic function and contain an (antiquoted) 
Cogent type. When Cogent generates the monomorphic instance it replaces the type in the template by the 
corresponding C type. Afterwards, Genops reads the template, and expands it depending on the contained C type. 
If the C type results from a Cogent type for which Gencot does not provide a function instance, Genops signals
an error. The required information about the generated C types is accessed by reading the struct and function type
information described in Section~\ref{impl-ctypinfo-repr}.

The Genops template syntax is defined in a way that its insertion results in syntactically valid C code. Thus, 
Cogent can read and process the antiquoted C code with embedded Genops templates as usual and the resulting 
monomorphic instances are syntactically valid (plain) C code. Genops is implemented by reading the monomorphic 
instances, detecting and expanding the templates, and outputting the result again as valid C code.

Genops templates have the form of a C function invocation
\begin{verbatim}
  genopsTemplate("XXX",(typ1) expr1, "genopsEnd")
\end{verbatim}
where \code{XXX} names the template and \code{typ1} is an arbitrary C type embedded in the template, and \code{expr1}
is an arbitrary C expression embedded in the template. Both \code{typ1} and \code{expr1} are read by Genops and are 
used to determine the template expansion. If more than one type and/or expression is required, additional function
arguments are used to specify them, each but the last followed by the string \code{"genopsNext"}. If only a type is 
required, \code{0} is used as expression. If only an expression is required, the cast with the type is omitted. The Genops 
templates always expand to a C expression, thus they are syntactically valid wherever the template may occur.

The syntax of the templates is selected for simple parsing, so that it is not required to use a parser for C types and 
expressions. It is assumed that the \code{typi} and \code{expri} in the template do not contain the strings \code{"genopsNext"}
and \code{"genopsEnd"}, so they can be separated by searching for these strings. All templates can be found by searching for 
\code{genopsTemplate}. The \code{typi} and \code{expri} can be separated because the \code{typi} generated from Cogent 
types never contain parantheses (see Section~\ref{impl-ctypinfo}).

Using a C parser is avoided because the result of processing an antiquoted C file by Cogent is usually only a fragment of
a C program which contains types but not their definitions. The C syntax is not context free and a C parser needs to 
know which identifiers name types and which name values, therefore the files could not be parsed without additional information
from the rest of the program.

\subsection{Implementing Default Values}
\label{impl-operations-default}

Default values are specified using the abstract polymorphic function \code{defaultVal} (see Section~\ref{design-operations-default}).
This function is implemented with the help of the Genops template
\begin{verbatim}
  genopsTemplate("DefaultVal",(typ) 0, "genopsEnd")
\end{verbatim}
where \code{typ} is the type for which to generate the default value. The Genops mechanism replaces this template by the value
as specified in Section~\ref{design-operations-default}.

The Cogent type is determined by its translation as C type \code{typ}. The primitive numerical types are translated to \code{u8},
\code{u16}, \code{u32}, \code{u64}, the unit type is translated to \code{unit\_t}, the type \code{String} is translated to \code{char*}.
Gencot function pointer types are represented by abstract types with names of the form \code{CFunPtr\_...} and have the same name when
translated to C.
Tuple types, unboxed record types, and variant types, are translated to types \code{tNN} which are described in the struct type 
information (see Section~\ref{impl-ctypinfo-repr}). Unboxed Gencot array types are implemented as specific unboxed record types and thus
have the same form. Tuple and record types are treated in the same way. Variant types are recognized by the first member having 
type \code{tag\_t}. Gencot array types are recognized as double wrapper records with a member of array type.

Default values for aggregate types (structs and arrays, including the unit type which is translated to a struct type \code{unit\_t})
are implemented with the help of C initializers. For these types the 
\code{"DefaultVal"} template expands to an initializer expression, it may thus only be used in an object definition. The initializer 
only specifies a value for the first member or element, according to the C standard this causes the remaining members or elements to
be initialized according to the C default initialization, which corresponds to the Gencot default values as specified in 
Section~\ref{design-operations-default}. This also applies to objects of automatic storage duration, which are not initialized when
no initializer is provided. So the template can be used to initialize a local stack-allocated variable in a function to the default 
value and use it to copy the value elsewhere.

\subsection{Implementing Create and Dispose Functions}
\label{impl-operations-create}

\subsection{Implementing Initialize and Clear Functions}
\label{impl-operations-init}

The functions \code{initFull} and \code{clearFull} are implemented in antiquoted C by moving the boxed value from stack
to heap and back in a single C assignment.

The functions \code{initHeap}, \code{clearHeap}, \code{initSimp}, and \code{clearSimp} must be implemented by treating
each component of a struct type and each element of an array type separately. This is done with the help of type specific
functions for initializing and clearing, which are generated automatically. They are invoked by two Genops templates:
\begin{verbatim}
  genopsTemplate("InitStruct",(typ) expr, "genopsEnd")
  genopsTemplate("ClearStruct",(typ) expr, "genopsEnd")
\end{verbatim}
In the first case \code{typ} must be the translation of a Gencot empty-value type, in the second case it must be the translation 
of a Gencot valid-value type. In both cases \code{expr} must be an expression for a value of type \code{typ}. The templates
expand to an invocation of a generated function which initializes or clears this value as described in 
Section~\ref{design-operations-init}.

For every type \code{t} occurring as \code{typ} in an \code{"InitStruct"} template, and also for every pointer type \code{t} which 
transitively occurs in a component or element of \code{typ}, Genops generates a C function \code{gencotInitStruct\_t} which implements
the initialization. Analogously Genops generates C functions \code{gencotClearStruct\_t} for the \code{"ClearStruct"} templates.

The generated function definitions must be placed at toplevel before the first occurrence of the corresponding template. Since
Genops does not parse the C file it uses a marker, which must be manually inserted, to determine that position. The marker has 
the form
\begin{verbatim}
  int genopsInitClearDefinitions;
\end{verbatim}
and is replaced by the sequence of all generated function definitions. The function definitions are ordered so that functions
for components or elements of a type precede the function for the type and can be invoked there.

\subsection{Implementing Pointer Types}
\label{impl-operations-pointer}

\subsection{Implementing Function Pointer Types}
\label{impl-operations-function}

\subsection{Implementing MayNull}
\label{impl-operations-null}

The functions of the abstract data type defined in Section~\ref{design-operations-null} are implemented in antiquoted
C, no Genops templates are required. 

The function \code{isNull} and the part access operation functions could be implemented in Cogent, but they are also 
defined as abstract and implemented in antiquoted C, for efficiency reasons. The implementation of \code{modifyNullDflt}
uses the Genops template \code{"DefaultVal"} (see Section~\ref{impl-operations-default}) to generate the default value
for type \code{out}.

\subsection{Implementing Record Types}
\label{impl-operations-record}

\subsection{Implementing Array Types}
\label{impl-operations-array}

\subsubsection{Creating and Disposing Arrays}

The \code{create} instance is implemented by simply allocating the
required space on the heap, internally using the translation of type \code{\#(CArr<size> El)} to specify
the amount of space. 

\subsubsection{Initializing and Accessing Arrays}

All abstract functions for initializing and accessing arrays defined in Section~\ref{design-operations-array} are implemented
in antiquoted C using the following two Genops templates:
\begin{verbatim}
  genopsTemplate("ArrayPointer",(typ) expr,"genopsEnd")
  genopsTemplate("ArraySize",(typ) expr,"genopsEnd")
\end{verbatim}
In both templates the \code{typ} must be the translation of a Gencot array type \code{CArr<size> El}. In the first template \code{expr}
must be an expression of type \code{typ}, which denotes a wrapper record according to Section~\ref{design-types-array}.
The template is expanded to an expression for the corresponding C array (pointer to the first element), which is constructed by 
accessing the array in the wrapper record. In the second template \code{expr} is arbitrary and is ignored by Genops, by convention it 
should also be an expression of type \code{typ} as in the first case. The template is expanded to the size of all arrays of type 
\code{typ}, specified as a numerical literal.

\subsection{Implementing Explicitly Sized Array Types}
\label{impl-operations-esarray}

