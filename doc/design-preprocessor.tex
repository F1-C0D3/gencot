
A preprocessor directive always occupies a single logical line, which may consist of several actual lines where 
intermediate line ends are backslash-escaped. No C code can be in a logical line of a preprocessor directive.
However, comments may occur before or after the directive in the same logical line. Therefore, every preprocessor 
directive may have an associated comment before-unit and after-unit, which are transferred as described in 
Section~\ref{design-comments}. Comments embedded in a preprocessor directive are discarded.

We differentiate the following preprocessor directive units:
\begin{itemize}
\item Constant definitions of the form \code{\#define XXX val}, where \code{val} is a C constant or the defined 
name of a constant,
\item all other macro definitions and \code{\#undef} directives,
\item conditional directives (\code{\#if, \#ifdef, \#ifndef, \#else, \#elif, \#endif}),
\item include directives (quoted or system)
\item all other directives, like \code{\#error} and \code{\#warning}
\end{itemize}

To identify constant definitions we assume that the names of constants are exactly the parameterless macro names 
starting with an uppercase letter. A C constant starts with a digit, a minus sign, or a single quote 
(string literals syntactically do not belong to the C constants). All constant definitions are processed
as described in Section~\ref{design-const}.

For comment processing every preprocessor directive is treated as an unstructured source code part.

\subsection{Configurations}

Conditional directives are often used in C code to support different configurations of the code. Every configuration
is defined by a combination of preprocessor definitions. Using conditional directives in the code, whenever the
code is processed only the code for one configuration is selected by the preprocessor.

In Gencot the idea is to process all configurations at the same time. This is done by removing the conditional 
directives from the code, process it, and re-insert the conditional directives into the generated Cogent code.

However, it may be the case that Gencot cannot process two configurations at the same time, because they contain
conflicting information needed by Gencot. An example would be different definitions for the same type which
shall be translated from C to a Cogent type by Gencot.

For this reason Gencot supports a list of conditions for which the corresponding conditional directives are not 
removed and thus only one configuration is processed at the same time. Then Gencot has to be run separately for every
such configuration and the results must be merged manually.

Conditional directives which are handled this way are still re-inserted in the generated target code. This
usually results in all branches being empty but the branches which correspond to the processed configuration.
Thus the branches in the results from separate processing of different configurations can easily be merged manually
or with the help of tools like diff and patch.

Keeping Conditional directives for certain configurations in the processed code makes only sense if the corresponding
macro definitions which are tested in the directives are kept as well. Therefore also define directives can be
kept. The approach in Gencot is to specify a list of regular expressions in the format used by awk. All directives
which mach one of these regular expressions are kept in the code to be interpreted before processing the code.

\subsection{Conditional Directives}

Conditional directives are used to suppress some source code according to specified conditions. Gencot aims to
carry over the same suppression to the generated code.

\subsubsection{Associating Conditional Directives to Target Code}

Conditional directives form a hierarchical block structure consisting of ``sections'' and ``groups''. A group
consists of a conditional directive followed by other code. Depending on the directive there are ``if-groups''
(directives \code{\#if, \#ifdef, \#ifndef}), ``elif-groups'' (directive \code{\#elif}), and ``else-groups''
(directive \code{\#else}). A section consists of an if-group, an optional sequence of elif-groups, an optional
else-group, and an \code{\#endif} directive. A group may contain one or more sections in the code after the
leading directive.

Basically, Gencot transfers the structure of conditional directives to the target code. Whenever a source code
part belongs to a group, the generated target code parts are put in the corresponding group. 

This only works if the source code part structure is compatible with the conditional directive structure.
In C code, theoretically, both structures need not be related. Gencot assumes the following restrictions:
Every source code part which overlaps with a section is either completely enclosed in a group or
contains the whole section. It may not span several groups or contain only a part of the section. If a
source code part is structured, contained sections may only overlap with subparts, not with code belonging
to the part itself. 

Based on this assumption, Gencot transfers conditional directives as follows. If a section is contained in an 
unstructured source code part, its directives are discarded. If a section is contained in a structured source
code part, its directives are transferred to the target code part. Toplevel sections which are not contained in
a source code part are transferred to toplevel. Generated target code parts are put in the same group which
contained the corresponding source code part.

It may be the case that for a structured source code part a subpart target must be placed separated from the
target of the structured part. An example is a struct specifier used in a member declaration. In Cogent, the 
type definition generated for the struct specifier must be on toplevel and thus separate from the generated member.
In these cases the condition directive structure must be partly duplicated at the position of the subpart target,
so that it can be placed in the corresponding group there.

Since the target code is generated without presence of the conditional directives structure, they must be 
transferred afterwards. This is done using the same markers \code{\#ORIGIN} and \code{\#ENDORIG} as for the
comments. Since every conditional directive occupies a whole line, the contents of every group consists of
a sequence of lines not overlapping with other groups on the same level. If every target code part is marked 
with the begin and end line of the corresponding source code part, the corresponding group can always be
determined from the markers.

The conditional directives are transferred literally without any changes, except discarding embedded comments. 
For every directive inserted in the target code origin markers are added, so that its associated comment before-
and after-unit will be transferred as well, if present.

\subsection{Macro Definitions}

Macro definitions are transferred literally, the intention is that they are used in a similar way in the Cogent code.
If the definition occurs in a file \code{x.h} it is transferred to file \code{x-types.cogent} to a corresponding position,
if it occurs in a file \code{x.c} it is transferred to file \code{x-impl.cogent} to a corresponding position.

This implies that the macro definitions are not available in the files \code{x-decls.cogent}, 
\code{x-globals.cogent} and in the files with
antiquoted Cogent code. If they are used there (which mainly is the case if they are used in a conditional
preprocessor directive which is transferred there), a manual solution is required. 

The replacement text for a defined macro usually is C code. Thus the definition can normally not be used directly in
the Cogent file, it must be adapted manually. 

If different C compilation units use the same name for different macros, conflicts are caused in the integrated Cogent
source. These conflicts are not detected by the Cogent compiler. A renaming scheme based on the name of the file 
containing the macro definition would not be safe either, since it breaks situations where a macro is deliberately
redefined in another file. Therefore, Gencot provides no support for macro name conflicts, they must be detected and
handled manually.

\subsection{Include Directives}

In C there are two forms of include directives: quoted includes of the form
\begin{verbatim}
  #include "x.h"
\end{verbatim}
and system includes of the form
\begin{verbatim}
  #include <x.h>
\end{verbatim}
Files included by system includes are assumed to be not a part of the translated <package>, therefore system
include directives are discarded.

\subsubsection{Translating Quoted Include Directives}

Quoted include directives are always translated to the corresponding Cogent language include statement
\begin{verbatim}
  include "x-types.h"
\end{verbatim}
If the original include directive occurs in file \code{y.c} the translated statement is put into the files 
\code{y-impl.c} and \code{y-decls.c}. In the latter it may be required to make the types for parameters and
function results available.

If the original include directive occurs in file \code{y.h} the translated statement is put into the file
\code{y-types.h}. 

It may be the case that a file \code{x.h} included by a quoted include is not a part of the translated <package>. 
In this case nevertheless the file \code{x-types.cogent} should be created and filled with abstract Cogent type
definitions for all types used from \code{x.h}. This can be done either manually or using Gencot.

\subsection{Other Directives}

All other preprocessor directives are discarded. Gencot displays a message for every discarded directive.

 
