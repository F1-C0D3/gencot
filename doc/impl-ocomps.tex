The following auxiliary Gencot components exist which do not process C source code:
\begin{description}
\item[\code{parmod-proc}] processes parameter modification descriptions in JSON format (see Section~\ref{impl-parmod}).
\end{description}

\subsection{Parameter Modification Descriptions}

Processing parameter modification descriptions is implemented by the filter \code{parmod-proc}. It reads a parameter
modification description from standard input as described in Section~\ref{impl-parmod-json}. The first command line 
argument acts as a command how to process the parameter modification description. The filter implements the following
commands with the help of the functions from module \code{Gencot.Json.Process}  (see Section~\ref{impl-parmod-modules}):
\begin{description}

\item[\code{check}]
Verify the structure of the parameter modification description 
according to Section~\ref{impl-parmod-struct} and lists all errors found (not yet implemented).

\item[\code{unconfirmed}] 
List all unconfirmed parameter descriptions using function \code{showRemainingPars}.

\item[\code{required}]
List all required invocations by their function identifiers, using function \code{showRequired}.

\item[\code{funids}]
List the function identifiers of all functions described in the parameter modification description.

\item[\code{sort}]
Takes an additional file name as command line argument. The file contains a list of function identifiers.
The input is sorted using function \code{sortParmods}. This command
is intended to be applied after the \code{merge} command to (re)establish a certain ordering.

\item[\code{filter}]
Takes an additional file name as command line argument. The file contains a list of function identifiers.
The input is filtered using function \code{filterParmods}.

\item[\code{merge}]
Takes an additional file name as command line argument. The file contains a parameter modification description 
in JSON format. Both descriptions are merged using function \code{mergeParmods}
where the first parameter is the description read from standard input.
After merging, function \code{addParsFromInvokes} is applied.
Thus, if the merged information contains
an invocation with more arguments than before, the function description is automatically extended.

\item[\code{eval}]
Using the functions \code{showRemainingPars} and \code{showRequired} it is verified that the parameter modification 
description contains no
unconfirmed parameter descriptions and no required dependencies. Then it is evaluated using function \code{evalParmods}.
The resulting parameter modification description contains no parameter dependencies and
no invocation descriptions. It is intended to be read by the filters which translate C function types and function 
definitions.
\end{description}

All lists mentioned above are structured as a sequence of text lines.

If the result is a parameter modification description in JSON format it is written to the output as described in 
Section~\ref{impl-parmod-json}.

The first three commands are intended as a support for the developer when filling the description manually. The goal is that
for all three the output is empty. If there are unconfirmed parameters they must be inspected and confirmed. This usually 
modifies the list of required invocations. They can be reduced by generating and merging corresponding descriptions
from other source files.

