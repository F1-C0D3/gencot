The following auxiliary Gencot components exist which do not process C source code:
\begin{description}
\item[\code{parmod-proc}] processes parameter modification descriptions in JSON format (see Section~\ref{impl-parmod}).
\end{description}

\subsection{Parameter Modification Descriptions}

Processing parameter modification descriptions is implemented by the filter \code{parmod-proc}. It reads a parameter
modification description from standard input as described in Section~\ref{impl-parmod-json}. The kind of processing
depends on whether an additional parameter is specified for the filter.

If no additional parameter is specified, the filter verifies the structure of the parameter modification description 
according to Section~\ref{impl-parmod-struct} and lists all errors found.
Additionally, it generates a textual summary of the description. It lists
all unconfirmed parameter descriptions in the form
\begin{verbatim}
  <function identifier> / <parameter identifier>
\end{verbatim}
and it lists all required invocations by their function identifiers.

This summary is intended as a support for the developer when filling the description manually. The goal is that
the summary is empty. If there are unconfirmed parameters they must be inspected and confirmed. This usually 
modifies the list of required invocations. They can be reduced by generating and adding corresponding descriptions
from other source files.

If an additional parameter is specified, it must be either the string \code{"eval"} or the file name of another 
parameter modification description.

In the latter case all required invocations which are found in this second description are added (appended) to the main 
description and the result is written to the output as described in Section~\ref{impl-parmod-json}.

If the string \code{"eval"} is specified, the parameter modification description must be structurally correct, contain no
unconfirmed parameter descriptions and contain no required dependencies (i.e. the filter invocation without arguments
must produce an empty output). Then all dependencies are eliminated as described in Section~\ref{impl-parmod-eval}.
The resulting parameter modification description is written to the output. It contains no parameter dependencies and
no invocation descriptions. It is intended to be read by the filters which translate C function types and function 
definitions.

