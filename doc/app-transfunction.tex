Manual actions in a function body translation are required if Gencot cannot translate a C code fragment.
In this case it inserts a dummy expression, as described in Section~\ref{design-cstats-dummy}.

In the following sections we provide some rules and patterns how to translate such cases manually.
These rules and patterns are not exhaustive but try to cover most of the common cases.

\subsection{Pointers}
\label{app-transfunction-pointers}

Cogent treats C pointers in a special way as values of ``linear type'' and guarantees that no memory is shared
among different values of these types. More general, all values which may contain pointers (such as a struct with
some pointer members) have this property. All other values are of ``nonlinear type'' and never have common parts 
in C.

If a C program uses sharing between values of linear type, it cannot be translated directly to Cogent. Gencot always 
assumes that named C objects never share parts with the help of pointers. If this is not the case in the original 
C program, it must be manually modified. The following approach can be used for a translation of such cases. 

If several parameters of a function may share common memory, they are grouped together to a Cogent record or
abstract data type and treated as a single parameter. All functions which operate on one of the values are
changed to operate on the group value. Then no sharing occurs between the remaining function parameters.

If several local variables share common memory they can be treated in the same way. 

If a variable shares memory with a parameter this solution is not applicable. In this case the variable must be
eliminated. This is easy if the variable is only used as a shortcut to a part of the parameter value, then it
can be replaced by explicit access to the part of the parameter. For example, in the C function
\begin{verbatim}
  void f (struct{int i; x *p;} p1) {
    x *v = p1.p; ...
  }
\end{verbatim}
the occurrences of variable \code{v} can be replaced by accesses to \code{p1.p}.

Alternatively, in Cogent a \code{take} operation can be used to bind the value of \code{p} to a variable \code{v}.
This prevents accessing the value through \code{p1}. Before the end of the scope of \code{v} the value must
be put back into \code{p1} using a \code{put} operation:
\begin{verbatim}
  let p1 { p = v }
  and ...
  in p1 { p = v }
\end{verbatim}

In other cases individual solutions must be found. Note that parameters and variables of nonlinear type never
cause such problems.

\subsection{Function Pointer Invocation}
\label{app-transfunction-pointer}

When a function pointer \code{fptr} shall be invoked in Cogent, it must be converted to the corresponding Cogent function
using \code{fromFunPtr[...](fptr)} or the macro \code{FROMFUNPTR(<enc>)(fptr)} where \code{<enc>} is the encoding of the 
corresponding function type (see Section~\ref{design-operations-function}).

In C a function pointer may be \code{NULL}, therefore it is typically tested for being valid before the
referenced function is invoked, such as in 
\begin{verbatim}
  if( fptr == NULL ) ...
  else fptr( params );
\end{verbatim}
When applied to \code{NULL} the conversion function \code{fromFunPtr} returns a result which can be invoked in Cogent.
To treat the \code{NULL} case separately it must be tested using the function \code{nullFunPtr}:
\begin{verbatim}
  if nullFunPtr(fptr) then ...
  else fromFunPtr(fptr)(params)
\end{verbatim}

In Cogent the \code{then} case only covers the case where the function pointer is \code{NULL}, it does not cover
the cases where the function referenced by the pointer is not known by Cogent. A function is only known by Cogent
if there is a Cogent definition for it, either as a Cogent function or as an abstract function.

Function pointers are not needed for Cogent functions which are only invoked from Cogent, then the Cogent function can 
be assigned and stored directly. Function pointers are only relevant if they are also used in C code external to
the Cogent compilation unit. So there are two application cases:
\begin{itemize}
\item A function which is defined in C and is passed as a function pointer to Cogent for invocation there. As
described in Section~\ref{design-modular}, this invocation requires an exit wrapper. The function \code{fromFunPtr}
converts the function pointer to the exit wrapper function. To be consistent, function \code{toFunPtr} converts
every exit wrapper to a pointer to the wrapped external function.
\item A function which is defined in Cogent and is passed as a function pointer to C for invocation there. As
described in Section~\ref{design-modular}, this invocation requires an entry wrapper. Entry wrappers are automatically
generated for all translated functions which had external linkage in C. For all these functions \code{toFunPtr}
converts them to a pointer to the entry wrapper. To be consistent, function \code{fromFunPtr} converts pointers
to entry wrappers to the corresponding Cogent functions. For translations of functions with internal linkage
\code{toFunPtr} converts to a pointer to the Cogent function. If such a pointer is invoked from C the invocation
will normally fail, because of different parameter and result types. In the current version of Gencot this must
be handled manually.
\end{itemize}

Exit wrappers are automatically generated by Gencot for functions which are invoked from the Cogent compilation
unit (see Section~\ref{impl-ccomps-externs}), determined with the help of the call graph (see 
Section~\ref{impl-ccode-callgraph}). However, functions which are \textit{only} invoked through function pointers
are not detected by the call graph (the invoked function pointer is detected but not which actual functions
may have been assigned to the function pointer), so no exit wrapper will be generated and \code{fromFunPtr}
will return an invalid result. Moreover, if there are no other Cogent functions
of the same function type, the Cogent program will be translated to inconsistent C code: the invocation of the 
converted function pointer will be translated using a dispatcher function which does not exist.

Gencot supports forcing exit wrapper generation for functions not detected by the call graph, by specifying them
explicitly in the file \code{<package>.gencot-externs} when generating the unit files by executing \code{gencot unit}
(see Section~\ref{impl-all-gencot}). The following additional steps are
required when forcing an exit wrapper for an external C function \code{f} which is not detected by the call graph:
\begin{itemize}
\item To generate the correct type for the exit wrapper a parameter modification description is required for 
\code{f}. Since the required descriptions are also determined using the call graph, \code{f} must additionally
be specified explicitly when generating descriptions for external functions using \code{parmod externs}. The
\code{parmod} script reads the same file \code{<package>.gencot-externs} (see Section~\ref{impl-all-parmod}),
so this happens automatically. However, \code{parmod externs} does not generate the descriptions, it only selects 
them from a file given as input. Therefore the file containing the definition of \code{f} must be processed 
by \code{parmod init} in addition to the files processed as described in Section~\ref{app-parmod-extern} and the 
resulting descriptions must be added to the file input to \code{parmod externs}.
\item Even if explicitly specified, \code{gencot unit} generates exit wrappers only for functions which are declared
in one of the source files belonging to the Cogent compilation unit (usually the declaration will be in a \code{.h}
file included by a source file). If \code{f} is only invoked through a function pointer it may be the case that
the declaration of \code{f} is not present in either Cogent compilation unit source file. In this case the ``pseudo
source'' \code{additional\_externs.c} must be added to the Cogent compilation unit which only includes the \code{.h} 
file where \code{f} is declared. More generally, it should include all \code{.h} files with declarations of functions 
for which an exit wrapper shall be forced and which are not included by regular sources in the Cogent compilation unit.
As described in Section~\ref{impl-ocomps-main}, the name \code{additional\_externs.c} is treated in a special
way by Gencot, it is ignored when the main Cogent and C source files are generated.
\end{itemize}

\subsection{The Null Pointer}
\label{app-transfunction-null}

Translating C code which uses the null pointer is supported by the abstract data type \code{MayNull} defined
in \code{include/gencot/MayNull.cogent} (see Section~\ref{design-operations-null}).

A typical pattern in C is a guarded access to a member of a struct referenced by a pointer:
\begin{verbatim}
  if (p != NULL) res = p->m;
\end{verbatim}
In Cogent the value \code{p} has type \code{MayNull R} where \code{R} is the record type with field \code{m}. 
Then a translation to Cogent is
\begin{verbatim}
  let res = roNotNull p 
            | None -> dflt
            | Some s -> s.m
      !p
\end{verbatim}
Note that a value \code{dflt} must be selected here to bind it to res if the pointer is \code{NULL}. Also note that the access is
done in a banged context for p. Therefore it is only possible if the type of \code{m} is not linear, since otherwise the result
cannot escape from the context.

Another typical pattern in C is a guarded modification of the referenced structure:
\begin{verbatim}
  if (p != NULL) p->m = v;
\end{verbatim}
A translation to Cogent is
\begin{verbatim}
  let p = notNull p 
          | None -> null ()
          | Some s -> mayNull s{m = v}
\end{verbatim}
where the reference to the modified structure is bound to a new variable with the same name \code{p}. 
Note that in the None-case the result cannot be specified as 
\code{p} since this would be a second use of the linear value \code{p} which is prevented by Cogent.

Alternatively this can be translated using the function \code{modifyNullDflt}:
\begin{verbatim}
  let p = fst (modifyNullDflt (p, (setFld_m, v)))
\end{verbatim}
using a function \code{setFld\_m} for modifying the structure.

\subsection{The Address Operator \code{\&}}
\label{app-transfunction-addrop}

The address operator \code{\&} is used in C to determine a pointer to data which is not yet accessed through a 
pointer. The main use cases are
\begin{itemize}
\item determine a pointer to a local or global variable as in the example
\begin{verbatim}
  int i = 5;
  int *ptr = &i;
\end{verbatim}

\item determine a pointer to a member in a struct as in the example
\begin{verbatim}
  struct ii { int i1; int i2; } s = {17,4};
  int *ptr = &(s.i2);
\end{verbatim}

\item determine a pointer to an array elemnt as in the example
\begin{verbatim}
  int arr[20];
  int *ptr = &(arr[5]);
\end{verbatim}

determine a pointer to a function as in the example
\begin{verbatim}
  int f(int p) { return p+1; }
  ...
  int *ptr = &f;
\end{verbatim}
\end{itemize}

In all these cases the pointer is typically used as reference to pass it to other functions or store it
in a data structure.

The binary compatible Cogent equivalent of the pointer is a value of a linear type. However, there is
no Cogent functionality to create such values. Hence it must be implemented by an abstract function.

\subsubsection{Address of Variable}

In the first use case there are several problems with this approach. First, there is no true equivalent
for C variables in Cogent. Second, it is not possible to pass the variable to the implementation of the
abstract function, without first determining its address using the address operator. Third, if the address
operator is applied to the variable in the implementation of the abstract function, the Isabelle c-parser
will not be able to process the abstract function if the variable is local, since it only supports the address
operator when the result is a heap address or a global address.

All these problems can be solved by allocating the variable on the heap instead. Then the variable definition
must be replaced by a call to the polymorphic function \code{create} (see Section~\ref{design-operations-pointer}) and 
at the end of its scope a call to the polymorphic function \code{dispose} must be added. Then the address
operator need not be translated, since \code{create} already returns the linear value which can be 
used for the same purposes. The resulting Cogent code for a variable of type \code{int} initialized to \code{5}
would be
\begin{verbatim}
  create[EVT(CPtr U32)] heap
  | Success (ptr,heap) ->
    fst INIT(Full,CPtr U32) (ptr,#{cont=5})
    | Success ptr ->
      let  ... use ptr ...
      in dispose (fst CLEAR(Simp,CPtr U32) (ptr,()),heap)
    | Error eptr -> dispose (eptr,heap)
  | Error heap -> heap
\end{verbatim}
where \code{INIT(Full,CPtr U32)} is used to initialize the referenced value. Here, the result of the expression is
only the heap, in more realistic cases it would be a tuple with additional result values.

The drawback of this solution is that it is less efficient to allocate the variable on the heap, than to use a 
stack allocated variable. If it is only used for a short time, a better solution should be created manually.

\subsubsection{Address of Struct Member}

The second use case cannot be translated in this way, since the referenced data is a part of a larger structure.
If it is separated from the structure and allocated on the heap, the structure is not binary compatible any more.

If the overall structure is allocated on the stack, the same three problems apply as in the first use case.
This can be solved in the same way, by moving the overall structure to the heap. Then it can be represented
by the linear Cogent type
\begin{verbatim}
  type Struct_Cogent_ii = { i1: U32, i2: U32 }
\end{verbatim}

If the overall structure in Cogent is readonly an abstract polymorphic function \code{getrefFld\_i2}
can be used to access the Cogent field through a pointer. This is an instance of the general \code{getref} operation
for records as described in Section~\ref{design-operations-record}.

If the overall structure is modifyable, determining a pointer to the field would introduce sharing for the
field, since it can be modified through the pointer or by modifying the structure. A safe solution is to use 
an abstract polymorphic function \code{modrefFld\_i2} which is an instance of the general \code{modref} operation
for records as described in Section~\ref{design-operations-record}.

\subsubsection{Address of Array Element}

The third use case is similar to the second. It often occurs if the array itself is represented by a pointer to its first
element (see Section~\ref{app-transtype-arrpoint}). This case could simply be replaced by using the element index instead of a 
pointer to the element. The array index is nonlinear and thus easier to work with. However, this solution is not binary
compatible, if the element pointer is also accessed outside the Cogent compilation unit.

A binary compatible solution can be achieved in a similar way as for struct members. The first prerequisite for it is
to allocate the array on the heap.

Then the polymorphic functions \code{getrefArr}, \code{getrefArrChk}, and \code{modrefArr}
(see Section~\ref{design-operations-array}) can be used for working with pointers to elements.

\subsubsection{Address of Function}

The last use case is translated in a specific way for function pointers, using the translation function
\code{toFunPtr} (see Section~\ref{design-operations-function}), which is generated by Gencot for all function
pointer types. The resulting Cogent code has the form
\begin{verbatim}
  Cogent_f : U32 -> U32
  Cogent_f p = p+1
  ...
  let ptr = TOFUNPTR(FXU32X_U32) Cogent_f
  in ...
\end{verbatim}
In this case \code{ptr} has the nonlinear type \code{\#CFunPtr\_FXU32X\_U32} in Cogent and may freely be copied and
discarded in its scope.

