
A C declaration consists of zero or more declarators, preceded by information applying to all declarators together.
Gencot translates every declarator to a separate Cogent definition, duplicating the common information as needed.
The Cogent definitions are generated in the same order as the declarators.

A C declaration may either be a \code{typedef} or an object declaration. A typedef can only occur on toplevel or 
in function bodies in C.
For every declarator in a toplevel typedef Gencot generates a Cogent type definition at the corresponding position. Hence
all these Cogent type definitions are on toplevel, as required in Cogent. Typedefs in function bodies are not
processed by Gencot, as described in Section~\ref{design-fundefs}.

A C object declaration may occur 
\begin{itemize}
\item on toplevel (called an ``external declaration'' in C),
\item in a struct or union specifier for declaring members,
\item in a parameter list of a function type for declaring a parameter,
\item in a compound statement for declaring local variables.
\end{itemize}

External declarations are simply discarded by Gencot. In Cogent there is no corresponding concept, it is not needed
since the scope of a toplevel Cogent definition is always the whole program. 

Compound statements in C only occur 
in the body of a function definition. Declarations embedded in a body are processed when the body is translated. They
are translated to a binding of a local variable to the value specified by the initializer or to a default value (see
Section~\ref{design-cstats}).

Union specifiers are always translated to abstract types by Gencot, hence declarations for union members are
never processed by Gencot.

The remaining cases are struct member declarations and function parameter declarations. 
For every declarator in an object declaration, Gencot generates a Cogent record field definition, if the C declaration
declares struct members, or it generates a tuple field definition, if the C declaration declares a function parameter.

\subsection{Target Code for struct/union/enum Specifiers}
\label{design-decls-tags}

Additionally, whenever a struct-or-union-specifier or enum-specifier occurring in the C declaration has a body and
a tag, a Cogent type definition is generated for the corresponding type, since it may be referred in C by its tag from
other places. A C declaration may contain atmost one 
struct-or-union-specifier or enum-specifier directly. Here we call such a specifier the ``full specifier'' of 
the declaration, if it has a body. 

Since Cogent type definitions must be on toplevel, Gencot defers it to the next possible toplevel position after the
target code generated from the context of the struct/union/enum declaration. If the context is a typedef, it is placed
immediately after the corresponding Cogent type definition. If the typedef contains several full specifiers (which
may be nested), all corresponding Cogent type definitions are positioned on toplevel in the order of the beginnings
of the full specifiers in C (which corresponds to a depth-first traversal of all full specifiers).

If the context is a member declaration in a struct-or-union-specifier, the Cogent type definition is placed after that
generated for its context. 

If the context is a parameter declaration it may either be embedded in a function definition or in a declarator of another
declaration. Function definitions in C always occur on toplevel, the Cogent type definitions for all struct/union/enum
declarations in the parameter list are placed after the target code for the function definition (which may be unusual for 
manually written Cogent code, but it is easier to generate for Gencot). In all other cases the Cogent type definitions
for struct/union/enum declarations in a parameter list are treated in the same way as if they directly occur in the surrounding
declaration.

Note, that a struct/union/enum tag declared in a parameter list has only ``prototype scope'' or ``block scope'' which
ends after the function type or definition. Gencot nevertheless generates a toplevel type definition for it, since the
tag may be used several times in the parameter list or in the corresponding body of a function definition. Note that 
this may introduce name conflicts, if the same tag is declared in different parameter lists. Since declaring tags in a 
parameter list is very unusual in C, Gencot does not try to solve these conflicts, they will be detected by the Cogent
compiler and must be handled manually.

A full specifier without a tag can only be used at the place where it statically occurs in the C code, however, it
may be used in several declarators. Therefore Gencot also generates a toplevel type definition for it, with an 
introduced type name as described in Section~\ref{design-names}.

\subsection{Relating Comments}

A declaration is treated as a structured source code part. The subparts are the full specifier, if present, and all
declarators. Every declarator includes the terminating comma or semicolon, thus there is no main part code between or after
the declarators. The specifiers may consist of a single full specifier, then there is no main part code at all.

The target code part generated for a declaration consists of the sequence of target code parts generated for the declarators,
and of the sequence of target code parts generated for the full specifier, if present. No target code is generated for the 
main part itself. In both sequences the subparts are 
positioned consecutively, but the two sequences may be separated by other code, since the second sequence consists of 
Cogent type definitions which must always be on toplevel. 

According to the rules defined in Sectiom~\ref{design-comments-relate}, the before-unit of the declaration is put before
the target of the first subpart, which is that for the full specifier, if present, otherwise it is the target for the
first declarator. In the first case the comments will be moved to the type definition for the full specifier. The rationale
is that often a comment describing the struct/union/enum declaration is put before the declaration which contains it.

The after-unit of the declaration is always put behind the target of the last declaration.

A declarator may derive a function type specifying a parameter-type-list. If that list is not \code{void}, the 
declarator is a structured source code part with the parameter-declarations as embedded subparts. Every
parameter-declaration includes the separating comma after it, if another parameter-declaration follows,
thus there is no main part code between the parameter-declarations. The parentheses around the parameter-type-list
belong to the main part, thus a comment is only associated with a parameter if it occurs inside the parentheses.

In all other cases a declarator is an unstructured
source code part.

\subsection{Typedef Declarations}
\label{design-decls-typedefs}

For a C typedef declaration Gencot generates a separate toplevel Cogent type definition for every declarator.

For every declarator a C type is determined from the declaration specifiers together with the derivation specified
in the declarator. As described in Section~\ref{design-types}, either a Cogent type expression is determined from this C type,
or the Cogent type is decided to be abstract. 

The defined type name is generated from the C type name according to the mapping described in Section~\ref{design-names}.
Type names used in the C type specification are mapped to Cogent type names in the Cogent type expression in the same way.

\subsection{Object Declarations}

C object declarations are processed if they declare struct members or function parameters.

For such a C object declaration Gencot generates a separate Cogent field 
definition for every declarator. This is a named record field definition if the declaration is embedded in the
body of a struct-or-union-specifier, it is an unnamed tuple field definition if the declaration is embedded in the
parameter-type-list of a function type. In the first case declarators with function type are not allowed, in the
second case they are adjusted to function pointer type. In both cases the Cogent field type is determined from the 
declarator's C type as described in Section~\ref{design-types}. 

In the case of a named record field the Cogent
field name is determined from the name in the C declarator as described in Section~\ref{design-names}. In the 
case of an unnamed tuple field a name specified in the C parameter declaration is always discarded.

\subsection{Struct or Union Specifiers}

For a full specifier with a tag Gencot generates a Cogent type definition. The name of the defined type is generated
from the tag as described in Section~\ref{design-names}. For a union specifier the type is abstract, no defining type
expression is generated. For a struct specifier a (boxed) Cogent record type expression is generated, which has a field
for every declared struct member which is not a bitfield. Bitfield members are aggregated as described in 
Section~\ref{design-types-struct}. 

A specifier without a body must always have a tag and is used in C to reference the full specifier with the same tag.
Gencot translates it to the Cogent type name defined in the type definition for the full specifier.

Note that the Cogent type defined for the full specifier corresponds to the C type of a pointer to the struct or union, 
whereas the unboxed Cogent type corresponds to the C struct or union itself. This is adapted by Gencot when translating
the C specifier embedded in a context to the corresponding Cogent type reference.

\subsection{Enum Specifiers}

For a full enum specifier with a tag Gencot generates a Cogent type definition immediately followed by Cogent object
definitions for all enum constants. The name of the defined type is generated
from the tag as described in Section~\ref{design-names}. The defining Cogent type is always \code{U32}, as described in
Section~\ref{design-types-enum}. 

A specifier without a body must always have a tag and is used in C to reference the full specifier with the same tag.
Gencot translates it to the Cogent type name defined in the type definition for the full specifier.

 
