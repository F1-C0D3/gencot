The single filters and processors of Gencot are combined for the typical use cases in the shell scripts
\code{gencot}, \code{items}, \code{parmod}, and \code{auxcog}. The script \code{items} is intended for handling all 
aspects of working with the item property declarations (see Section~\ref{impl-itemprops}), the script \code{parmod} 
is intended for handling all aspects
of working with the parameter modification descriptions (see Section~\ref{impl-parmod}), the script
\code{gencot} is intended for handling all other aspects of translating from C to Cogent. The script \code{auxcog}
is intended for handling all aspects of additional processing after or in addition to processing the Cogent program by the 
Cogent compiler.

\subsection{The \code{gencot} Script}
\label{impl-all-gencot}

\subsubsection{Usage}

The overall synopsis of the \code{gencot} command is
\begin{verbatim}
  gencot <options> <subcommand> [<file>]
\end{verbatim}

The \code{gencot} command supports the following subcommands:
\begin{description}
\item[\code{check}] Test the specified C source file or include file for parsability by Gencot (see Section~\ref{app-prep} for how to 
make a C source parsable, if necessary).

\item[\code{cfile}] Translate the specified C source file to Cogent and generate the entry wrappers. If the source file name 
is \code{x.c} the results are written to files \code{x.cogent} and \code{x-entry.ac}.

\item[\code{hfile}] Translate the specified C include file to Cogent. If the source file name is \code{x.h} the result is
written to file \code{x-incl.cogent}

\item[\code{config}] Translate the specified C include file to Cogent with specific support for configuration files as described
in Section~\ref{impl-preprocessor-config}. If the source file name is \code{x.h} the result is
written to file \code{x-incl.cogent}

\item[\code{unit}] Generate additional Cogent files for the Cogent compilation unit.

\item[\code{cgraph}] Print the call graph for the Cogent compilation unit.

\end{description}

The input file \code{<file>} must be specified for the first four subcommands and it must be omitted for the latter two.
The subcommands \code{unit} and \code{cgraph} use the ``unit name'' to determine the files to be processed.
If the unit name is \code{x} the subcommand \code{unit} generates the following additional files (see Section~\ref{design-files}):
\begin{itemize}
\item \code{x.cogent}
\item \code{x-externs.cogent}
\item \code{x-exttypes.cogent}
\item \code{x-dvdtypes.cogent}
\end{itemize}

The \code{gencot} command supports the following options:
\begin{description}
\item[\code{-I}] used to specify directories where included files are searched, like for \code{cpp}. The 
option can be repeated to specify several directories, they are searched in the order in which the options
are specified.

\item[\code{-G}] Directory for searching Gencot auxiliary files. Only one can be given, default is \code{"."}.

\item[\code{-C}] Directory for retrieving stored declaration comments. Default is \code{"./comments"}.

\item[\code{-u}] The unit name. Default is \code{"all"}.

\item[\code{-k}] Keep directory with intermediate files. This is only intended for debugging.

\end{description}

As described in Section~\ref{impl-comments-decl}, Gencot moves comments from C function declarations to translated function
definitions. Since a declaration may be specified in a different C source file (often an include file) than the definition, 
the subcommands \code{cfile}, \code{hfile}, and \code{config} write all declaration comments in the processed file to
the directory specified by the \code{-C} option and use all comments found there when translating function definitions.

\subsubsection{Auxiliary Files}

Gencot uses an approach, where all manual annotations added to the C program for configuring the translation to Cogent 
are stored in auxiliary files separate from the original C program sources. This way it is possible to update the C 
sources to a newer version without the need to re-insert the annotations.

There are different auxiliary files for different purposes. The \code{gencot} script determines the auxiliary files using
a predefined naming scheme. These files are searched in the directory specified by the \code{-G} option. All auxiliary files
are optional, if a file is not found it is assumed that no corresponding annotations are needed.

When processing a file \code{x.<ext>} where \code{<ext>} is \code{c} or \code{h}, the \code{gencot} command uses the 
following auxiliary files, if they exist:
\begin{description}
\item[\code{x.gencot-addincl}] The content of this file is prepended to the processed file before processing.
\item[\code{x.gencot-noincl}] List of ignored include files, as described in Section~\ref{impl-ccode-include}.
\item[\code{x.gencot-omitincl}] The Gencot include omission list (see Section~\ref{design-preprocessor-incl}).
\item[\code{x.gencot-ppretain}] The Gencot directive retainment list (see Section~\ref{design-preprocessor-config}).
\item[\code{x.gencot-chsystem}] The content of this file is inserted immediately after the last system include directive.
\item[\code{x.gencot-manmacros}] The Gencot manual macro list (see Section~\ref{design-preprocessor-macros}).
\item[\code{x.gencot-macroconv}] The Gencot macro call conversion (see Section~\ref{design-preprocessor-macros}).
\item[\code{x.gencot-macrodefs}] The Gencot macro translation for the processed file (see Section~\ref{design-preprocessor-macros}).
\end{description}

For all cases with the exception of the last two, additionally a file with name \code{common} instead of \code{x} ist used, when it
is present. If both files are present the concatenation of them is used. In this way it is possible to specify common annotations 
for all C sources and add specific annotations in the source file specific files.

The subcommands \code{unit} and \code{cgraph} use corresponding auxiliary files for every file \code{x.c} processed by them.

The subcommands \code{cfile}, \code{hfile}, \code{config}, and \code{unit} additionally use the auxiliary file named
\code{<file>-itemprops} which must contain all relevant item property declarations (see Section~\ref{impl-itemprops-decl}).

These subcommands also use the auxiliary file \code{common.gencot-namap} which must contain the name prefix map, 
as described in Section~\ref{impl-ccode-names}. No source file specific versions of the name prefix map are supported,
since the map is used globally for all names translated to Cogent.

Another kind of auxiliary files describe the Cogent compilation unit. These files are required to be present 
in the directory specified by the \code{-G} option. If they are not found, an error or a warning is signaled.

If the unit name is \code{<uname>} the auxiliary files for the Compilation unit are
\begin{description}
\item[\code{<uname>.unit}] The ``unit file''. It contains the list of C source file names
(one per line) which together constitute the Cogent compilation unit. Only the original source files \code{x.c} must 
be listed, include files \code{x.h} must not be listed.
\item[\code{<uname>-external.items}] The list of used external items, as generated by the processor \code{items-used}
(see Section~\ref{impl-ccomps-items}).
\end{description}

The first file is only used by the subcommands \code{unit} and \code{cgraph}. They process all files listed in the unit file
together with all included files. The second file is used by all \code{gencot} subcommands with the exception of \code{check}.

The subcommand \code{unit} also processes the following optional auxiliary files by generating include directives in the 
main Cogent source \code{<uname>.cogent}, if these files exist:
\begin{description}
\item[\code{gencot-incl.cogent}] This file is intended for specifying include directives for the files
where the Cogent types predefined by Gencot are defined (see Sections~\ref{design-types} and~\ref{design-operations}).
The file names should be specified in the form \code{gencot/CPointer.cogent}. 
\item[\code{<uname>-manabstr.cogent}] This file is intended for manually specifying additional abstract functions used 
in the Cogent program.
\end{description}

\subsubsection{Implementation}

**todo**

The intended use of filter \code{gencot-remcomments} is for removing all comments from input to the language-c parser.
This input always consists of the actual source code file and the content of all included files. The simplest approach
would be to use the language-c preprocessor for it, immediately before parsing. 

However, it is easier for the filter \code{gencot-rempp} to remove the preprocessor directives when the comments are 
not present anymore. Therefore, Gencot applies the filter \code{gencot-remcomments} in a separate step before applying
\code{gencot-rempp}, immediately after processing the quoted include directives by \code{gencot-include}.
 
The filters \code{gencot-selcomments} and \code{gencot-selpp} for selecting comments and preprocessor directives, however, are
still applied to the single original source files, since they do not require additional information from the included files.


\subsection{The \code{items} Script}
\label{impl-all-items}

\subsubsection{Usage}

The overall synopsis of the \code{items} command is
\begin{verbatim}
  items <options> <subcommand> [<file>] [<file2>]
\end{verbatim}

The \code{items} command supports the following subcommands:
\begin{description}
\item[\code{file}] Create default property declarations for all items defined in the specified C source file
or include file \code{<file>}. 

\item[\code{unit}] Create default property declarations for all external items used in the Cogent compilation 
unit. 

\item[\code{used}] List external toplevel items used in the Cogent compilation unit. 

\item[\code{merge}] Merge the declarations in \code{<file>} and \code{<file2>}. Two declarations for the 
same item are combined by uniting the properties.

\item[\code{mergeto}] Add properties from \code{<file2>} to items in \code{<file>}. Properties are only added
if their toplevel item is already present in \code{<file>}.

\item[\code{omitfrom}] Omit (remove) properties in \code{<file2>} from items in \code{<file>}. 

\end{description}

If the unit name is \code{<uname>}, subcommand \code{used} writes its output to the file \code{<uname>-external.items}
The presence of this file is expected by most other Gencot command scripts. All other subcommands write their result to 
standard output.

For the subcommands \code{unit} and \code{used} no \code{<file>} must be specified, instead, they use the unit name
in the same way as the subcommand \code{unit} of the \code{gencot} command.

The \code{items} command supports the options \code{-I}, \code{-G}, \code{-u}, \code{-k} with the same meaning as
for the \code{gencot} command.

\subsubsection{Auxiliary Files}

The subcommand \code{file} uses the same auxiliary files as the subcommands \code{cfile} and \code{hfile} of \code{gencot} with
the exception of \code{common.gencot-namap} and \code{<file>-itemprops}.
The subcommand \code{unit} uses the same auxiliary files as the subcommand \code{unit} of \code{gencot} with the exception
of \code{common.gencot-namap} and \code{<file>-itemprops}.
The subcommand \code{used} uses the unit file \code{<uname>.unit} like subcommand \code{unit} and additionally
the optional auxiliary file 
\begin{description}
\item[\code{<uname>.unit-manitems}] Manually specfied external items, as described in Section~\ref{impl-ccomps-items} as
input to the processor \code{items-used}.
\end{description}

\subsubsection{Implementation}

** todo **

The \code{unit} command is implemented by the processor \code{items-externs} (see Section~\ref{impl-ccomps-itemprops}). 
The postprocessing step for removing declarations of non-external subitems is implemented by the filter
\code{items-proc} with command \code{filter} (see Section~\ref{impl-ocomps-items}).

\subsection{The \code{parmod} Script}
\label{impl-all-parmod}

\subsubsection{Usage}

The overall synopsis of the \code{parmod} command is
\begin{verbatim}
  parmod <options> <subcommand> <file> [<file2>]
\end{verbatim}

The \code{parmod} command supports the following subcommands:
\begin{description}
\item[\code{file}] Create parameter modification descriptions for all functions defined in C source file 
or include file \code{<file>}.
The result is written to standard output.

\item[\code{close}] Create parameter modification descriptions for all functions declared for the C source file
\code{<file>} in the file itself or in a file included by it.
The result is written to standard output.

\item[\code{unit}] Select entries from \code{<file>} (a file in JSON format containing parameter modification
descriptions) for all external functions used in the Cogent compilation unit. 
The result is written to standard output.

\item[\code{show}] Display on standard output information about the parameter modification description in \code{<file>}.

\item[\code{idlist}] List on standard output the item identifiers of all functions described in the 
paramer modification description in \code{<file>}.

\item[\code{diff}] Compare the parameter modification descriptions in \code{<file>} and \code{<file2>}. The output
has the same form as the Unix \code{diff} command, however, entries of functions occurring in both files are 
directly compared.

\item[\code{iddiff}] Compare the item identifiers of all functions described in \code{<file>} and \code{<file2>}.
The output has the same form as the Unix \code{diff} command.

\item[\code{addto}] Add to \code{<file>} all entries for required dependencies found in \code{<file2>}. Both files must contain 
parameter modification descriptions in JSON format. The result is written to \code{<file>}.

\item[\code{mergin}] Merge the parameter modification description entries in \code{<file>} and \code{<file2>} by building 
the union of the described functions. If a function is described in both files the entry with more confirmed parameter 
descriptions is used. The result is written to \code{<file>}.

\item[\code{replin}] Replace in \code{<file>} all function entries by an entry for the same function in \code{<file2>}
if it is present and has not less confirmed parameters. Both files must contain 
parameter modification descriptions in JSON format. The result is written to \code{<file>}.

\item[\code{eval}] Evaluate the parameter modification description in \code{<file>} as described in 
Section~\ref{impl-parmod-eval}.  The resulting parameter modification description is written to standard output.

\item[\code{out}] Convert the evaluated parameter modification description in \code{<file>} to item property declarations.
The result is written to the two files \code{<file>-itemprops} (properties to be added to the default properties) 
and \code{<file>-omitprops} (properties to be removed from the default properties).

\end{description}

The subcommand \code{unit} expects no argument \code{<file>}, instead, it uses the unit name in the same way as the subcommand \code{unit} 
of the \code{gencot} command.

The \code{parmod} command supports the options \code{-I}, \code{-G}, \code{-u}, \code{-k} with the same meaning as
for the \code{gencot} command. They are only used for the subcommands \code{file}, \code{close}, and \code{unit}.

\subsubsection{Auxiliary Files}

The subcommands \code{file} and \code{close} use the same auxiliary files as the subcommands \code{cfile} and \code{hfile} of \code{gencot} with 
the exception of \code{common.gencot-namap} and \code{<file>-itemprops}.
The subcommand \code{unit} uses the same auxiliary files as the subcommand \code{unit} of \code{gencot} with the exception
of \code{common.gencot-namap} and \code{<file>-itemprops}.

\subsubsection{Implementation}

\paragraph{The subcommand \code{unit}} 
First, all C source files are prepared for parsing, as in command \code{gencot unit}. Then the list of used external toplevel 
items \code{<uname>-external.items} is passed to processor \code{items-extfuns} to determine the list of 
used external functions. This list is then used to filter the descriptions in \code{<file2>} with \code{parmod-proc filter}.

Additionally the result is tested, whether all external functions determined by \code{items-extfuns} have been found in 
\code{<file2>}. This is done by calculating the function ids of all descriptions in the result with \code{parmod-proc funids}, 
sorting both lists and comparing them with \code{diff}. If the result is not empty, a warning is displayed on standard error.

\subsection{The \code{auxcog} Script}
\label{impl-all-auxcog}

\subsubsection{Usage}

The overall synopsis of the \code{auxcog} command is
\begin{verbatim}
  auxcog <options> <subcommand> [<file>]
\end{verbatim}

The \code{auxcog} command supports the following subcommands:
\begin{description}
\item[\code{unit}] Generate additional C files for the Cogent compilation unit.

\item[\code{comments}] Remove comments from the Cogent source \code{<file>} and write the result to standard output.

\end{description}

The subcommand \code{unit} expect no argument \code{<file>}, instead, it uses the unit name in the same way as the subcommand \code{unit} 
of the \code{gencot} command. If the unit name is \code{x} the following additional files are generated 
(see Section~\ref{design-files}):
\begin{itemize}
\item \code{x.c}
\item \code{x-macros.h}
\item \code{x-gencot.c} (only if \code{x-gencot\_pp\_inferred.c} is present)
\item \code{x-externs.c}
\end{itemize}
and for all files \code{y.c} listed in the unit file \code{x.unit}:
\begin{itemize}
\item \code{y-entry.c}
\end{itemize}

The subcommand \code{comments} is mainly intended for testing the comment structure in a Cogent source generated by
\code{gencot}. In rare cases it may happen that a closing comment delimiter and a subsequent opening comment delimiter
are omitted, so that the Cogent compiler will not detect the wrong comment structure.

The \code{auxcog} command supports the options \code{-I}, \code{-u} with the same meaning as
the \code{gencot} command.

The \code{auxcog} command only uses the unit file \code{<uname>.unit} as auxiliary file for subcommand \code{unit}.

\subsubsection{Implementation}

**todo**

