The following auxiliary Gencot components exist which do not process C source code:
\begin{description}
\item[\code{parmod-proc}] processes parameter modification descriptions in JSON format (see Section~\ref{impl-parmod}).
\item[\code{auxcog-macros}] processes Cogent source code to select macro definitions.
\item[\code{auxcog-mapback}] processes Cogent source code to map Cogent constants back to preprocessor constants.
\item[\code{gencot-prclist}] processes list files to remove comments.
\end{description}

\subsection{Parameter Modification Descriptions}
\label{impl-ocomps-parmod}

Processing parameter modification descriptions is implemented by the filter \code{parmod-proc}. It reads a parameter
modification description from standard input as described in Section~\ref{impl-parmod-json}. The first command line 
argument acts as a command how to process the parameter modification description. The filter implements the following
commands with the help of the functions from module \code{Gencot.Json.Process}  (see Section~\ref{impl-parmod-modules}):
\begin{description}

\item[\code{check}]
Verify the structure of the parameter modification description 
according to Section~\ref{impl-parmod-struct} and lists all errors found (not yet implemented).

\item[\code{unconfirmed}] 
List all unconfirmed parameter descriptions using function \code{showRemainingPars}.

\item[\code{required}]
List all required invocations by their function identifiers, using function \code{showRequired}.

\item[\code{funids}]
List the function identifiers of all functions described in the parameter modification description.

\item[\code{sort}]
Takes an additional file name as command line argument. The file contains a list of function identifiers.
The input is sorted using function \code{sortParmods}. This command
is intended to be applied after the \code{merge} command to (re)establish a certain ordering.

\item[\code{filter}]
Takes an additional file name as command line argument. The file contains a list of function identifiers.
The input is filtered using function \code{filterParmods}.

\item[\code{merge}]
Takes an additional file name as command line argument. The file contains a parameter modification description 
in JSON format. Both descriptions are merged using function \code{mergeParmods}
where the first parameter is the description read from standard input.
After merging, function \code{addParsFromInvokes} is applied.
Thus, if the merged information contains
an invocation with more arguments than before, the function description is automatically extended.

\item[\code{eval}]
Using the functions \code{showRemainingPars} and \code{showRequired} it is verified that the parameter modification 
description contains no
unconfirmed parameter descriptions and no required dependencies. Then it is evaluated using function \code{evalParmods}.
The resulting parameter modification description contains no parameter dependencies and
no invocation descriptions. It is intended to be read by the filters which translate C function types and function 
definitions.
\end{description}

All lists mentioned above are structured as a sequence of text lines.

If the result is a parameter modification description in JSON format it is written to the output as described in 
Section~\ref{impl-parmod-json}.

The first three commands are intended as a support for the developer when filling the description manually. The goal is that
for all three the output is empty. If there are unconfirmed parameters they must be inspected and confirmed. This usually 
modifies the list of required invocations. They can be reduced by generating and merging corresponding descriptions
from other source files.

\subsection{Processing Cogent Code}
\label{impl-ocomps-cogent}

There are some Gencot components which process a Cogent source to generate auxiliary files in antiquoted C or normal C code
or other output.
All these components are invoked with the name of a single Cogent source file as argument. Additionally, directories for
searching files included by cpphs can be specified with the help of \code{-I} options. The result is always written to 
standard output.

\subsubsection{\code{auxcog-remcomments}}

This component is a filter which removes comments from Cogent sources. Note that the Cogent preprocessor \code{cpphs}
does not remove comments, so it cannot be used for this task.

The filter is implemented as an awk script.

\subsubsection{\code{auxcog-macros}}

Gencot array types may have the form \code{CArrX<id>X} where \code{<id>} is a preprocessor constant (parameterless macro) 
identifier specifying the array size (see Section~\ref{design-types-array}). These types are generic in the element type 
and are thus implemented in antiquoted C. The implementation has the form 
\begin{verbatim}
typedef struct $id:(UArrX<id>X el) {
  $ty:el arr[<id>];
} $id:(UArrX<id>X el);
\end{verbatim}
where the identifier \code{<id>} occurs as size specification in the C array type. Cogent translates this antiquoted 
C definition to normal C definitions for all instances of the generic type used in the Cogent program. These C definitions
are included in the C program generated by Cogent.
Thus the definition of \code{<id>} must be available when the C program is processed by the C compiler or by Isabelle.

The definition of \code{<id>} may depend on other preprocessor macros, it may even depend on macros with parameters.
Therefore all macro definitions must be extracted from the Cogent source and made available as auxiliary C source.
This is implemented by \code{auxcog-macros} with the help of \code{cpp} and an \code{awk} script.

The macro definitions are extracted from the Cogent source \code{file.cogent} by executing the C preprocessor as
\begin{verbatim}
  cpp -P -dD -I<dir1> -I<dir2> ... -imacros file.cogent /dev/null 2> /dev/null
\end{verbatim}
The option \code{-imacros} has the effect that \code{cpp} only reads the macro definitions from \code{file.cogent}.
The actual input file is \code{/dev/null} so that no other input is read. The option \code{-dD} has the effect
to generate as output macro definitions for all known macros. This will also generate line directives, these are
suppressed by the option \code{-P}. The \code{-I} options must specify all directories where Cogent source files
are included by the preprocessor. All uses of the unbox operator \code{\#} in the Cogent code are signaled as an 
error by \code{cpp}, but the output is not affected by these errors. The error messages are suppressed by redirecting
the error output to \code{/dev/null}.

Note that \code{cpp} interpretes all include directives and conditional directives in the Cogent source. If for a macro
different definitions occur in different branches of a conditional directive, depending on a configuration flag,
\code{cpp} will select the definition according to the actual flag value.

The C preprocessor also adds definitions of internal macros. Some of them (those starting with \code{\_\_STDC}) are 
signaled as redefined when the output is processed again by \code{cpp}. This is handled
by postprocessing the output with an \code{awk} script. It removes all definitions of macros starting with \code{\_\_STDC}.

Since \code{cpp} does not recognize the comments in Cogent format, it will not eliminate them from the macro definitions 
in the output. Therefore it should be postprocessed by \code{auxcog-remcomments}. For a better result it should also 
be postprocessed by \code{auxcog-numexpr}.

\subsubsection{\code{auxcog-numexpr}}

The macro definitions selected by \code{auxcog-macros} are normally written as part of the Cogent program to expand 
to Cogent code. However, to be included in the C code they must expand to valid C code. This must basically be done
by the developer of the Cogent program. 

The filter \code{auxcog-numexpr} supports this by performing a simple conversion of numerical expressions from Cogent
to C. Numerical expressions are mostly already valid C code. Currently, the filter only removes all occurrences
of the \code{upcast} operator.

The filter is implemented as a simple \code{sed} script.

\subsubsection{\code{auxcog-mapback}}

The \code{auxcog-mapback} component supports this by converting specific kinds of Cogent code back to C. It handles 
the use of Cogent constants in integer value expressions. This occurs in code generated by Gencot: every reference 
to another preprocessor constant in the definition of a preprocessor constant is translated to a reference of the 
corresponding Cogent constant.

This is handled by including another C source file before the file generated by \code{auxcog-macros}. This
file is generated by \code{auxcog-mapback}. It contains a preprocessor macro definition for every Cogent constant
where the constant name is a mapped preprocessor constant name, which maps the Cogent constant name back to 
the preprocessor constant name. For example, if the preprocessor constant name \code{VAL1} has been mapped by
Gencot to the Cogent constant name \code{cogent\_VAL1}, the generated macro definition is
\begin{verbatim}
  #define cogent_VAL1 VAL1
\end{verbatim}

The component \code{auxcog-mapback} is implemented in Haskell and uses the Cogent parser to read the Cogent code.
It then processes every Cogent constant definition.

\subsection{Processing List Files}
\label{impl-ocomps-prclist}

Gencot uses some files containing simple lists of text strings, one in each line: the safe pointer list (see 
Section~\ref{impl-ccomps-pointer}), and the list of external variables to be processed by \code{gencot-externs}, 
\code{gencot-exttypes}, and \code{gencot-dvdtypes}.

To be able to specify these lists in a more flexible way, Gencot allows comments of the form
\begin{verbatim}
  # ...
\end{verbatim}
where the hash sign must be at the beginning of the line.

The filter \code{gencot-prclist} is used to remove these comment lines (and also all empty lines). It is implemented
as an awk script.

\subsection{Generating Main Files}
\label{impl-ocomps-main}

The main files as described in Section~\ref{design-files} are \code{<package>.cogent} and \code{<package>.c}. They
contain include directives for all parts of the Cogent unit sources, the former for the Cogent sources, the latter 
for the C sources resulting from translating the Cogent sources with the Cogent compiler.

The content of these main files depends on the set of \code{.c} files which comprise the Cogent compilation unit.
A list of these files, as described in Section~\ref{impl-ccode-package}, is used as input for the components generating
the main files. In contrast to the components reading the package C code, these components only process the file 
names and not their content. The file name \code{additional\_includes.c} is ignored in the list, it can be used
to specify additional \code{.h} files for Gencot components reading the C code, which are not used in the Cogent 
code or after compiling Cogent back to C. For an application see Section~\ref{app-transfunction-pointer}.

\subsubsection{Generating the Cogent Main File}

The component \code{gencot-mainfile} generates the content of the Cogent main file \code{<package>.cogent}. 
It is implemented as a shell script. It reads the list of
\code{.c} files comprising the Cogent compilation unit as its input and it takes the package name \code{<package>}
as additional argument. The generated content is written to the standard output. 

The component adds include directives for the following files:
\begin{itemize}
\item The file \code{gencot.cogent}, if it exists in the current directory. This file can be used to include standard 
definitions provided by Gencot such as \code{gencot/Memory.cogent}.
\item The files \code{<package>-externs.cogent}, \code{<package>-exttypes.cogent}, \code{<package>-dvdtypes.cogent}.
\item The file \code{<package>-manabstr.cogent}, if it exists in the current directory. This file can be used to
manually define additional abstract types and functions used in the translated program.
\item The files \code{x.cogent} and \code{x-entry.cogent} for every source file \code{x.c} specified as input.
\end{itemize}

\subsubsection{Generating the C Main File}

The component \code{auxcog-mainfile} generates the content of the C main file \code{<package>.c}. 
It is implemented as a shell script. It reads the list of
\code{.c} files comprising the Cogent compilation unit as its input and it takes the package name \code{<package>}
as additional argument. The generated content is written to the standard output. 

The component adds include directives for the following files:
\begin{itemize}
\item The file \code{gencot.h} for non-generic abstract standard types used by Gencot.
\item The files \code{<package>-mapback.h} and \code{<package>-macros.h}, if they exist in the current directory.
They are used to make preprocessor constants from the Cogent program available in C.
\item The files \code{<package>-dvdtypes.h} and \code{<package>-exttypes.h}, if they exist in the current directory. 
They contain definitions for the non-generic abstract types defined in \code{<package>-dvdtypes.cogent} and 
\code{<package>-dvdtypes.cogent}, respectively.
\item The file \code{<package>-manabstr.h}, if it exists in the current directory. This file can be used to
manually provide definitions for the non-generic abstract types defined in \code{<package>-manabstr.cogent}.
\item The file \code{<package>-gen.c} which must be the result of the translation of \code{<package>.cogent}
with the Cogent compiler.
\item The file \code{<package>-externs.c} with the implementations of all exit wrappers.
\item The file \code{<package>-gencot.c}, if it exists in the current directory. It contains the 
implementations of all generic abstract functions the program uses from the Gencot standard library.
\item The file \code{<package>-manabstr.c}, if it exists in the current directory. This file can be used to
manually provide implementations of the abstract functions defined in \code{<package>-manabstr.cogent}.
\item The file \code{cogent-common.c} with implementations of functions defined in \code{gum/common/common.cogent}
but not implemented by Cogent.
\item The files \code{x-entry.c} for every source file \code{x.c} specified as input. These files contain
the implementations of the entry wrappers.
\end{itemize}

The files \code{<package>-*.c} and \code{*-entry.c} result from postprocessing the corresponding 
\code{\_pp\_inferred.c} files by \code{auxcog}.
