\documentclass[a4paper]{report}
%\usepackage[bookmarks]{hyperref}
\usepackage[utf8]{inputenc}
\usepackage[T1]{fontenc} % needed for italic curly braces

\usepackage{isabelle,isabellesym}
\usepackage{pdfsetup}

% urls in roman style, theory text in math-similar italics
\urlstyle{rm}
\isabellestyle{it}

\newcommand{\code}[1]{\textnormal{\texttt{#1}}}

\begin{document}

\title{Report on Gencot Development}
\author{Gunnar Teege}

\maketitle

\chapter{Introduction}

This report is a living document which is constantly modified to reflect the development of Gencot. 
Not all parts described in the design chapter need already be implemented.

Gencot (GENerating COgent Toolset) is a set of tools for generating Cogent code from C code. 

Gencot is used for parsing the C sources and generating templates for the required Cogent sources, 
antiquoted Cogent sources, and auxiliary C code. 

Gencot is not intended to perform an automatic translation, it prepares the manual translation by 
generating templates and perfoming some mechanic steps.

Roughly, Gencot supports the following tasks:
\begin{itemize}
\item translate C preprocessor constant definitions and enum constants to Cogent object definitions,
\item generate function invocation entry and exit wrappers,
\item generate Cogent abstract function definitions for invoked exit wrappers,
\item translate C type definitions to default Cogent type definitions,
\item generate C type mappings for abstract Cogent types referring to existing C types,
\item generate Cogent function definition skeletons for all C function definitions,
\item rename constants, functions, and types to satisfy Cogent syntax requirements and avoid collisions,
\item convert C comments to Cogent comments and insert them at useful places in the Cogent source files,
\item generate the main files \code{<package>.cogent} and \code{<package>.ac} for Cogent compilation.
\end{itemize}

To do this, Gencot processes in the C sources most comments and preprocessor directives, all declarations (whether
on toplevel or embedded in a context), and all function definitions. It does not process C statements other than
for processing embedded declarations.

\chapter{Design}

\section{General Context}

We assume that there is a C application <package> which consists of C source files \code{.c} and \code{.h}. The 
\code{.h} files are included by \code{.c} files and other \code{.h} files. There may be included \code{.h} which
are not part of <package>, such as standard C includes, all of them must be accessible. Every \code{.c} file
is a separate compilation unit. There may be a \code{main} function but Gencot provides no specific support for it.

From the C sources of <package> Gencot generates Cogent source files \code{.cogent} and antiquoted Cogent source
files \code{.ac} as a basis for a manual translation from C to Cogent. All function definition bodies have to be
translated manually, for the rest a default translation is provided by Gencot.

Gencot supports an incremental translation, where some parts of <package> are already translated to Cogent and
some parts consist of the original C implementation, together resulting in a runnable system.

\subsection{Mapping Names from C to Cogent}
\label{design-names}

Names used in the C code shall be translated to similar names in the Cogent code, since they usually are descriptive for the
programmer. Ideally, the same names would be used. However, this is not possible, since Cogent differentiates between 
uppercase and lowercase names and uses them for different purposes. Therefore, atleast the names in the ``wrong'' case
need to be mapped.

Additionally, when the Cogent compiler translates a Cogent program to C code, it transfers the names without changes to
the names for the corresponding C items. To distinguish these names from the names in the original C code Gencot uses 
name mapping schemas which support mapping all kinds of names to a different name in Cogent. Generally, this is done by 
substituting a prefix of the name.

Often, a <package> uses one or more specific prefixes for its names, at least for names with external linkage. In this case
Gencot should be able to substitute these prefixes by other prefixes specific for the Cogent translation of the <package>.
Therefore, the Gencot name mapping is configurable. For every <package> a set of prefix mappings can be provided which is
used by Gencot. Two separate mappings are provided depending on whether the Cogent name must be uppercase or lowecase, so 
that the target prefixes can be specified in the correct case.

If a name must be mapped by Gencot which has neither of the prefixes in the provided mapping, it is mapped 
by prepending the prefix \code{cogent\_} or \code{Cogent\_}, depending on the target case.

\subsubsection{Name Kinds in C}

In C code the tags used for struct, union and enum declarations constitute an own namespace separate from the ``regular''
identifiers. These tags are mapped to Cogent type names by Gencot and could cause name conflicts with regular identifiers
mapped to Cogent type names. To avoid these conflicts Gencot maps tags by prepending the prefixes \code{Struct\_}, 
\code{Union\_}, or \code{Enum\_}, respectively, after the mapping described above. Since tags are always translated to Cogent 
type names, which must be uppercase, only one case variant is required.

Member names of C structs or unions are translated to Cogent record field names. Both in C and Cogent the scope of these
names is restricted to the surrounding structure. Therefore, Gencot normally does not map these names and uses them unmodified
in Cogent. However, since Cogent field names must be lowercase, Gencot applies the normal mapping for lowercase target 
names to all uppercase member names (which in practice are unusual in C). Moreover, Cogent field names must not begin
with an underscore, so these names are mapped as well, by prepending \code{cogent\_} (which results in two consecutive 
underscores).

C function parameter names are translated to Cogent variable names bound in the Cogent function body expression. Hence, both
in C and Cogent the scope of these names is restricted to the function body. They are treated by Gencot in the same way as 
member names and are only mapped if they are uppercase in C, which is very unusual in practice.

The remaining names in C are type names, function names, enum constant names, and names for global and local variables.
Additionally, there may be C constant names defined by preprocessor macro directives.
Local variables only occur in C function bodies which are not translated by Gencot. The other names are always mapped by
Gencot, irrespective whether they have the correct case or not.

\subsubsection{Names with internal linkage}

In C a name may have external or internal linkage. A name with internal linkage is local to the compilation unit in which it
is defined, a name with external linkage denotes the same item in all compilation units. Since the result of Gencot's 
translation is always a Cogent program which is translated to a single compilation unit by the Cogent compiler, names 
with internal linkage could cause conflicts when they origin in different C compilation units.

To avoid these conflicts, Gencot uses a name mapping scheme for names with internal linkage which is based on the 
compilation unit's file name. Names with internal linkage are mapped by substituting a prefix by the prefix \code{local\_x\_}
where \code{x} is the basename of the file which contains the definition, which is usually a file \code{x.c}. The default
is to substitute the empty prefix, i.e., prepend the target prefix. The mapping can be configured by specifying prefixes
to be substituted. This is motivated by the C programming practice to sometimes also use a common prefix for names 
with internal linkage which can be removed in this way.

Name conflicts could also occur for type names and tags defined in a \code{.h} file. This would be the case if different
C compilation units include individual \code{.h} files which use the same identifier for different purposes. However, most
C packages avoid this to make include files more robust. Gencot assumes that all identifiers defined in a \code{.h} file
are unique in the <package> and does not apply a file-specific renaming scheme. If a <package> does not satisfy this assumption
Gencot will generate several Cogent type definitions with the same name, which will be detected and signaled by the Cogent 
compiler and must be handled manually.

\subsubsection{Introducing Type Names}

There are cases where in Cogent a type name must be introduced for an unnamed C type (directly specified by a C type 
expression). Then the Cogent type name cannot be generated by mapping the C type name.

Unnamed C types are tagless struct/union/enum types and all derived types, i.e., array types, pointer types and 
function types. Basically, an unnamed C type could be mapped to a corresponding Cogent type expression. However,
this is not always possible or feasible.

Tagless enum types are always mapped to a primitive type in Cogent.

A tagless C struct could be mapped to a corresponding Cogent record type expression. However, the tagless struct
can be used in several declarators and several different types can be derived from it. In this case the Cogent record
expression would occur syntactically in several places, which is semantically correct, but may not be feasible for
large C structs. Therefore, Gencot introduces a Cogent type name for every tagless C struct and union.

Tagless structs and unions syntactically occur at only a single place in the source. The unique name is derived from 
that place, using the name of the corresponding source file and the line number where the struct/union begins
in that file (this is the line where the struct or enum keyword occurs).
The generated names have the forms
\begin{verbatim}
  <kind><lnr>_x_h
  <kind><lnr>_x_c
\end{verbatim}
where the suffix is constructed from the name \code{x.h} or \code{x.c} of the source file. \code{<kind>} is one of
\code{Struct} or \code{Union}, and \code{<lnr>} is the line number in the source file.

Derived C types are pointer types, function types, and array types. They always depend on a base type which in the C program 
must be defined before the base type is used.
In Cogent a similar dependency can only be expressed ba a generic type: the Cogent compiler takes care that in the generated C code
each instance
of a generic type is introduced after all its type arguments have been defined. Therefore Gencot translates every C derived
types to a generic Cogent type with its base type as its (single) type argument. 

For C pointer types a fixed set of generic types is used by Gencot (see Section~\ref{design-types-pointer}). Function types 
are translated to Cogent function type expressions. Only for array types generic type names are introduced which are specific
for the translated C program.

To be able to process every source file independently from all other source files, Gencot uses a schema which generates
a unique generic type name for every C array type. Derived types may syntactically occur at many places in a C program, so
it is not feasible to generate the name from a position in the source file.

For C array types its size (the number of elements) may be part of the derived type specification. Gencot uses two seperate generic
type names for every size for which an array type is used in the C program. If the size is specified by a literal the names
have the form
\begin{verbatim}
  CArr<size>
  UArr<size>
\end{verbatim}
where \code{<size>} is the literal size specification. If the size is specified by a single identifier the names have the form
\begin{verbatim}
  CArrX<size>X
  UArrX<size>X
\end{verbatim}
where \code{X} is a letter not occurring in the identifier.
In all other cases (also if no size is specified) the predefined names
\begin{verbatim}
  CArrXX
  UArrXX
\end{verbatim}
are used which may lead to name conflicts in Cogent and must be handled manually. 

Note that the generated Cogent type names could still cause conflicts with mapped type names. These conflicts can be
avoided if no configured mapping prefix starts with one of the \code{<kind>} strings
or the strings \code{"CArr"}, \code{"UArr"} used for mapping C array types, or any other predefined type used by Gencot.

\subsubsection{Encoding C Function Types}

Gencot represents C function pointer types with the help of abstract Cogent types which encode C function types. These
function type names are constructed from encoding of all parameter types and the result type. Hence Gencot implements
a schema for encoding all C types as Cogent abstract type names.

A C type is either a primitive type, a derived type, a typedef name, or a struct/union/enum type. Gencot maps all
struct/union/enum types to a Cogent type name. Primitive types and typedef names are specified by a single identifier
in C and are also mapped to a Cogent type name by Gencot. These names are directly used as encoding. Thus only for the 
derived types (pointer, array, and function types) a nontrivial encoding must be defined.

Every derived type in C has a single base type from which it is derived. The base type of a derived
type is always either another derived type, or it is mapped by Gencot to a Cogent type name. Hence every derived 
type can be uniquely characterized by a sequence of derivation steps starting with a type name. The sequence of 
derivation steps is syntactically encoded in the generated name as follows.

A pointer derivation step is encoded by a single letter \code{"P"}. 

An array derivation step without size
specification is encoded by a single letter \code{"A"}. An array derivation step with a literal
as size specification is encoded in the form
\begin{verbatim}
  A<size>
\end{verbatim}
where \code{<size>} is the size specification. If the size is specified by a single identifier the 
step is encoded in the form
\begin{verbatim}
  AX<size>X
\end{verbatim}
where \code{X} is a letter not occurring in the identifier.
In all other cases an array derivation step is encoded by
\begin{verbatim}
  AXX
\end{verbatim}
which may lead to name conflicts in Cogent and must be handled manually. As described in 
Section~\ref{design-types-array}, a C array type may be mapped to an unboxed or boxed form, depending
on its usage context. The unboxed form is encoded by prepending the letter \code{"U"} to the step
encoding.

A function derivation step is encoded in the form
\begin{verbatim}
  FX<P1>X<P2>X...X<Pn>X
\end{verbatim}
where the \code{<Pi>} are the encodings of the parameter types and \code{X} is a letter not occurring in 
any of the parameter type encodings. A parameterless function type is encoded as \code{FXX}, whereas an
incomplete function type where no parameter types are specified is encoded as \code{F}.
If the function type is variadic, an additional pseudo parameter type \code{VariadicCogentParameters}
is added as last parameter. 

In some cases Gencot maps a C pointer type to a boxed Cogent type and its base type to the corresponding
unboxed type by applying the unbox operator \code{\#} to the Cogent type. In these cases the pointer derivation
step is omitted in the encoding, and the base type is encoded by applying a pseudo derivation step 
of the form \code{"U"}.

Function parameters of linear type may be mapped by Gencot as readonly or not (see Section~\ref{design-types-readonly}). 
This is encoded by pseudo derivation steps of the form \code{"R"} for readonly and \code{"M"} (``modifyable'')
otherwise. Parameters of nonlinear type are not marked by either of these steps.

For all derivation steps which are applied to a base type, their encodings are concatenated, beginning with the 
last derivation step, with an underscore \code{\_} as separator. For a derived pointer or 
function type the base type can be the pseudo type \code{void}. In these cases the identifier \code{Void} is
used as base type encoding.

Hence, for example for the C type
\begin{verbatim}
  int (* [5])(int, const short*)
\end{verbatim}
the encoding is
\begin{verbatim}
  A5_P_FXU32XR_P_U16X_U32
\end{verbatim}


\subsection{Modularization and Interfacing to C Parts}
\label{design-mdular}

Every C compilation unit produces
a set of global variables and a set of defined functions. Data of the same type may be used in
different compilation units, e.g. by passing it as parameter to an invoked function. In this case type compatibility in C is
only guaranteed by including the \code{.h} file with the type definition in both compilation units. In the compiled
units no type information is present any more. 

This organisation makes it possible to use different \code{.h} files in different compilation units. Even the type definitions
in the included files may be different, as long as they are binary compatible, i.e., have the same memory layout.

We exploit this organisation for an incremental translation from C to Cogent as follows. At every stage we replace some 
C compilation units by Cogent sources. All C data types used both in C units and in Cogent units are mapped to binary compatible
Cogent types. Compiling the Cogent sources again produces C code which together with the remaining C units are linked to
the target program. The C code resulting from Cogent compilation is completely separated from the code of the remaining C units,
common include files are only used for types which are abstract in Cogent, i.e., have no Cogent definition.

All interfacing between C compilation units is done by name. All names of C objects with external linkage can be referred
from other compilation units. This is possible for functions and for global variables. Interfacing from and to Cogent works
in the same way. 

\subsubsection{Interfacing to Functions}

Cogent functions always take a single parameter, the same is true for the C functions generated by the Cogent compiler. Hence
for interfacing from or to an arbitrary C function, wrapper functions are needed which convert between arbitrary many parameters
and a single structured parameter. These wrapper functions are implemented in C. 

The ``entry wrapper'' for invoking a Cogent function 
from C has the same name as the original C function, so it can be invoked transparently. Thus the Cogent implementation of
the function must have a different name so that it does not conflict with the name of the wrapper. This is guaranteed by the 
Gencot renaming scheme as described in Section~\ref{design-names}.

The Cogent implementation of a C function generated by Gencot is never polymorphic. This implies that the Cogent compiler
will always translate it to a single C function of the same name.

The ``exit wrapper'' for invoking a C function from Cogent invokes the C function by its original name, hence the wrapper
must have a different name. We use the same renaming scheme for these wrappers as for the defined Cogent functions.
This implies that every exit wrapper
can be replaced by a Cogent implementation without modifying the invocations in existing Cogent code. Note that for every
function either the exit wrapper or the Cogent implementation must be present, but not both, since they have the same name.

To use the exit wrapper from Cogent, a corresponding abstract function definition must be present in Cogent.

Note that if the C function has only one parameter, a wrapper is not required. For consistency reasons we generate and use
the wrappers also for these functions.

Cogent translates all function definitions to C definitions with internal linkage. To make them accessible the entry wrappers
must have external linkage. They are defined in an antiquoted Cogent (.ac) file which includes the complete code generated from 
Cogent, there all functions translated from Cogent are accessible from the entry wrappers. The exit wrappers are only invoked 
from code generated from Cogent. They are defined with internal linkage in an included antiquoted Cogent file.

\subsubsection{Interfacing to Global Variables}

Accessing an existing global C variable from Cogent is not possible in a direct way, since there are no ``abstract constants'' in 
Cogent. Access may either be implemented with the help of abstract functions which are implemented externally in additional C code and
access the global variable from there. Or it may be implemented by passing a pointer as (part of) a ``system state'' to the
Cogent function which performs the access.

Accessing a Cogent object definition from C is not possible, since the Cogent compiler does not generate a definition for them, it
simply substitutes all uses of the object name by the corresponding value. Hence, all global variable definitions need to remain 
in C code to be accessible there.

Since the way how global state is treated in a Cogent program is crucial for proving program properties, Gencot does not 
provide any automatic support for accessing global C variables from Cogent, this must always be implemented manually.

\subsubsection{Cogent Compilation Unit}

As of December 2018, Cogent does not support modularization by using separate compilation units. A Cogent program may be distributed
across several source files, however, these must be integrated on the source level by including them in a single compilation unit.
It would be possible to interface between several Cogent compilation units in the same way as we interface from C units to Cogent
units, however this will probably result in problems when generating proofs. 

Therefore Gencot always generates a single Cogent compilation unit for the <package>. 
At every intermediate stage of the incremental translation the package consists of one Cogent compilation unit 
together with all remaining original C compilation units and optionally additional C compilation units (e.g., for implementing 
Cogent abstract data types).

Conflicts for names with internal linkage originating in different C compilation units are avoided by Gencot's name mapping scheme
as described in Section~\ref{design-names}.

\subsubsection{External Name References}
\label{design-modular-extref}

To successfully compile the Cogent compilation unit all referenced identifiers must be declared in the C code. Those references which are
used in the generated Cogent code must additionally be defined in Cogent. A non-local reference in a C source file is every
identifier which is used in the file but not defined or declared in the same file.

In the original C source for every non-local reference there must be a declaration or definition present in one of the included
files (its ``origin file''). If the origin file of a non-local reference is a file which has already been translated by 
Gencot, the required information about the identifier is already present in the Cogent compilation unit. If the origin file
of a non-local reference has not yet been translated, or is not a part of the <package> (which normally is the case for all
system includes), we call it an ``external reference''. For external references additional information must be created and 
made available in the Cogent compilation unit.

On the C level the information is provided by simply including the origin files of all external references. On the Cogent
level the information is provided as follows.
\begin{itemize}
\item If the external reference is a type name or a struct/union/enum tag, a Cogent type definition is generated for the mapped name.
The defined Cogent type is determined
from the C type referenced by the type name as described in Section~\ref{design-types}. The only difference is that all C type
names used directly or indirectly by the C type are resolved, if they are not already external references. This is done to avoid 
introducing type names which are never referenced from any other place in the generated Cogent program. 
\item If the external reference is a function name, an exit wrapper and the corresponding Cogent abstract function 
definition is generated.
\item If the external reference is the name of a global variable, no information is generated for Cogent, since Gencot does not
support accessing global C variables from Cogent.
\item If the external reference is the name of an enum constant or a preprocessor defined constant, a Cogent constant definition 
is generated.
\item An external reference may be the name of a member in a struct or union. In this case also the struct or union tag must
be externally referenced and the corresponding Cogent type definition is generated, as described above. Note that for a union
member this will always be an abstract type which does not provide access to the member in Cogent.
\end{itemize}

\subsection{Cogent Source File Structure}
\label{design-files}

Although the Cogent source is not structured on the level of compilation units, Gencot intends to reflect the structure of 
the C program at the level of Cogent source files. 

Note, that there are four kinds of include statements available in Cogent source files. One is the \code{include} statement which
is part of the Cogent language. When it is used to include the same file several times in the same Cogent compilation unit,
the file content is automatically inserted only once. However, the Cogent preprocessor is executed separately for every file included 
with this \code{include} statement, thus preprocessor macros defined in an included file are not available in all other files. For 
this reason it cannot be used to reflect the file structure of a C program.

The second kind is the Cogent preprocessor \code{\#include} directive, it works like the C preprocessor \code{\#include} directive
and is used by Gencot to integrate the separate Cogent source files. 
The third kind is the preprocessor \code{\#include} directive 
which can be used in antiquoted C files where the Cogent \code{include} statement is not available. This is only possible 
if the included content is also an antiquoted C file. The fourth kind
is the \code{\#include} directive of the C preprocessor which can be used in antiquoted C files in the form 
\code{\$esc:(\#include ...)}. It is only executed when the C code generated by the Cogent compiler is processed by the C compiler.
Hence it can be used to include normal C code.

Gencot assumes the usual C source structure: Every \code{.c} file contains definitions with internal or external linkage.
Every \code{.h}
file contains preprocessor constant definitions, type definitions and function declarations. The constants and type definitions 
are usually mainly those which are needed for the function declarations. Every \code{.c} file includes the \code{.h} file which
declares the functions which are defined by the \code{.c} file to access the constants and type definitions. Additionally it may
include other \code{.h} files to be able to invoke the functions declared there. A \code{.h} file may include other \code{.h} files
to reuse their constants and type definitions in its own definitions and declarations.

\subsubsection{Cogent Source Files}

In Cogent a function which is defined may not be declared as an abstract function elsewhere in the program. If the types and constants,
needed for defining a set of functions, should be moved to a separate file, like in C, this file must not contain the 
function declarations for the defined functions. Declarations for functions defined in Cogent are not needed at all, since the Cogent 
source is a single compilation unit and functions can be invoked at any place in a Cogent program, independently whether their definition 
is statically before or after this place.

Therefore we map every C source file \code{x.c} to a Cogent source file \code{x.cogent} containing definitions of the same 
functions. We map every C include file \code{x.h} to a Cogent source file \code{x-incl.cogent} 
containing the corresponding constant and type definitions, but omitting any function declarations. The include relations among 
\code{.c} and \code{.h} files are directly transferred to \code{.cogent} and \code{-incl.cogent} using the Cogent preprocessor 
\code{\#include} directive. 

The file \code{x-incl.cogent} also contains Cogent value definitions generated from C preprocessor
constant definitions and from enumeration constants (see below). It would be possible to put the value definitions in a 
separate file. However, then for other preprocessor macro definitions it would not be clear where to put them, since they could
be used both in constant and type definitions. They cannot be moved to a common file included by both at the beginning,
since their position relative to the places where the macros are used is relevant.

In some cases an \code{x.h} file contains function definitions, typically for inlined functions. They are translated to Cogent
function defintions in the \code{x-incl.cogent} file in the usual way.

This file mapping implies that for every translated \code{.c} file all directly or indirectly included \code{.h} files must be 
translated as well.

\subsubsection{Wrapper Definition Files}

The entry wrappers for the functions defined with external linkage in \code{x.c} are implemented in antiquoted C code and
put in the file \code{x-entry.ac}. 

The exit wrappers for invoking C functions from Cogent are only created for the actual
external references in a processing step for the whole <package>. They are implemented in antiquoted C
and put in the file \code{<package>-externs.ac}.

\subsubsection{External Name References}

For external name references Gencot generates the information required for Cogent. 
All generated type and constant definitions are put in the file \code{<package>-exttypes.cogent}.

If a Cogent function in \code{x.cogent} invokes a function which is externally referenced and not defined in another
file \code{y.cogent}, this function must be declared as an abstract function in Cogent. These abstract function declarations
are only created for the actual
external references in a processing step for the whole <package>. They are put in the file \code{<package>-externs.cogent}.
The corresponding exit wrappers are put in file \code{<package>-externs.ac} as described above.

For external variables Gencot creates declarations in antiquoted C. Since they are only accessed in 
the entry wrapper functions they are put into the files \code{x-entry.ac} at the beginning before the entry wrapper 
definitions. For all
external variables which are used by a function defined in \code{x.c} and must be passed as argument to the 
translated function a declaration is generated in \code{x-entry.ac}.

\subsubsection{Derived Types}

Gencot generates definitions of unary generic type definitions for derived array types 
used in the C source. These definitions are put in the file \code {<package>-dvdtypes.cogent}.

Corresponding type implementations in C are generated by Gencot in the file \code{<package>-dvdtypes.h}.

\subsubsection{Global Variables}

In C a compilation unit may define global variables. Gencot does not generate a direct access interface to these variables
from Cogent code. However, the variables must still be present in a compilation unit, since they may be accessed
from other C compilation units (if they have external linkage). 

Gencot assumes that global variables are only defined in \code{.c} files. For every file \code{x.c} Gencot generates
antiquoted C definitions for all global variables (toplevel object) defined in \code{x.c}. If the variable has 
external linkage the definition uses the original name, thus it can be accessed from outside the Cogent compilation unit. 
If the variable has internal linkage the definition uses the mapped name so that it is unique in the Cogent compilation
unit. 

In the Cogent compilation unit the defined global variables are only accessed in 
the entry wrapper functions. Therefore Gencot puts them into the files \code{x-entry.ac} at the places where 
their definition occurred in the original source. Additionally they are declared at the beginning of the file, 
so that they can be accessed in entry wrappers defined before that position.

\subsubsection{Predefined Gencot Types and Functions}

Gencot provides several Cogent types and functions which are used to translate C types and operators for them
(see Sections~\ref{design-types} and~\ref{design-operations}). Cogent definitions for all these types and functions
are provided in files which are part of the Gencot distribution and are contained in a directory \code{include/gencot}.
When Cogent is used to process a source generated by Gencot the directory \code{include} must be made known to the 
Cogent preprocessor using the compiler option \code{--cogent-pp-args}. Then the files can be included in the Cogent
source in the form
\begin{verbatim}
  #include "gencot/xxx.cogent"
\end{verbatim}

Some of the types and functions are abstract, the corresponding definitions in antiquoted C are provided in single
files \code{gencot.ah} and \code{gencot.ac} in the Gencot distribution. These files must be processed by the Cogent 
compiler using the options \code{--infer-c-types} and \code{--infer-c-funcs}, respectively, to generate the C source 
as part of the Cogent compilation unit. For generic abstract types the Cogent compiler generates a C source file
for every instance used in the Cogent program, these files are generated in the subdirectory \code{abstract} of the
directory where the Cogent compilation unit is compiled. The file \code{gencot.ac} is compiled to a single file 
\code{gencot\_pp\_inferred.c}. Since nearly all predefined Gencot functions are polymorphic the file \code{gencot.ac}
is used as a template and only the C source code for those function instances are generated in \code{gencot\_pp\_inferred.c}
which are actually used in the Cogent program.

\subsubsection{Abstract Data Types}

There may also be cases of C types where no corresponding Cogent type can be defined or is predefined be Gencot.
In this case it must be manually mapped to an 
abstract data type T in Cogent, consisting of an abstract type together with abstract functions. Both are put in 
a file \code{T.cogent} which must be included manually by all \code{x-incl.cogent} where it is used. The types and 
functions of T must be implemented in additional C code. In contrast to the abstract functions defined in 
\code{<package>-externs.cogent},
there are no existing C files where these functions are implemented. The implementations are provided as antiquoted C
code in the file \code{T.ac}. If T is generic, the additional file \code{T.ah} is required for 
implementing the types, otherwise they are implemented in \code{T.h}. In this case \code{T.h} must be \code{\$esc}-included
in \code{T.ac} so that it is included in the final C source of the Cogent compilation unit.

Gencot does not provide any support for using abstract data types, they must be managed manually according to the following
proposed schema. An abstract data type T is implemented in the following files:
\begin{description}
\item[\code{T.ac}] Antiquoted Cogent definitions of all functions of T. 
\item[\code{T.ah}] Antiquoted Cogent definition for T if T is generic.
\item[\code{T.h}] Antiquoted Cogent definitions of all non-generic types of T.
\end{description}
Using the flag \code{--infer-c-types} the Cogent compiler generates from \code{T.ah} files \code{T\_t1...tn.h} for all 
instantiations of T with type arguments t1...tn used in the Cogent code.

\subsubsection{File Summary}

Summarizing, Gencot uses the following kinds of Cogent source files for existing C source files \code{x.c} and \code{x.h}:
\begin{description}
\item[\code{x.cogent}] Implementation of all functions defined in \code{x.c}. For each file \code{y.h} included by
  \code{x.c} the file \code{y-incl.cogent} is included.
\item[\code{x-incl.cogent}] Constant and type definitions for all constants and types defined in \code{x.h}. 
  If possible, for every C type definition a binary compatible Cogent type 
  definition is generated by Gencot. Otherwise an abstract type definition is used. Includes
  all \code{y-incl.cogent} for which \code{x.h} includes \code{y.h}.
\item[\code{x-entry.ac}] Antiquoted Cogent definitions of entry wrapper functions for all function definitions with external linkage
  defined in \code{x.c}. Declarations of all external global variables used in the entry wrappers, and definitions
  of all global variables defined in \code{x.c}
\end{description}

For the Cogent compilation unit the following common files are used:
\begin{description}
\item[\code{<package>-exttypes.cogent}] Type and constant definitions for all external type and constant references.
\item[\code{<package>-externs.cogent}] Abstract function definitions for all external function and variable references.
\item[\code{<package>-dvdtypes.cogent}] Generic type definitions for all used derived array types.
\item[\code{<package>-externs.ac}] Exit wrapper definitions for all external function references.
\item[\code{<package>-dvdtypes.h}] Implementations of abstract types defined in \code{<package>-dvdtypes.cogent}.
\end{description}

\subsubsection{Main Files}

To put everything together we use the files \code{<package>.cogent} and \code{<package>.c}. The former includes all 
existing \code{x.cogent} files and the files \code{<package>-exttypes.cogent}, \code{<package>-externs.cogent}, 
\code{<package>-dvdtypes.cogent}, and all required files \code{gencot/Xxx.cogent} for predefined types and functions.
It is the file processed by the Cogent compiler which translates it to files \code{<package>-gen.c} 
and \code{<package>-gen.h} where \code{<package>-gen.c} includes \code{<package>-gen.h} and \code{<package>-gen.h} includes all files
in subdirectory \code{abstract} which have been generated from \code{gencot.ah} and any
other file \code{T.ah} for manually implemented abstract data types. 

Cogent also compiles all files \code{x-entry.ac}, \code{T.ac}, \code{<package>-externs.ac}, and \code{gencot.ac} to corresponding 
files \code{x-entry\_pp\_inferred.c}, \code{T\_pp\_inferred.c}, \code{<package>-externs\_pp\_inferred.c}, and \code{<package>-gencot\_pp\_inferred.c}, respectively. Gencot postprocesses these files to yield corresponding files 
\code{x-entry.c}, \code{T.c}, \code{<package>-externs.c}, and \code{<package>-gencot.c}, respectively.

The file \code{<package>.c} includes all existing files 
\code{x-entry.c}, \code{T.c}, \code{<package>-externs.c}, \code{<package>-gencot.c}, 
together with \code{<package>-dvdtypes.h} and \code{<package>-gen.c} and constitutes the C code of the Cogent compilation unit. 
It is the file to be compiled 
by the C compiler to produce the executable program and to be read by the Isabelle C parser when checking the refinement proof.



\section{Processing Comments}
\label{design-comments}

The Cogent source generated by Gencot is intended for further manual modification. Finally, it should be used as a 
replacement for the original C source. Hence, also the documentation should be transferred from the C source to
the Cogent source.

Gencot uses the following heuristics for selecting comments to be transferred: All comments at the beginning or end 
of a line and all comments on one or more full lines are transferred. Comments embedded in C code in a single line
are assumed to document issues specific to the C code and are discarded.

\subsection{Identifying and Translating Comments}

Gencot processes C block comments of the form \code{/* ... */} possibly spanning several lines, and C line comments
of the form \code{// ...} ending at the end of the same line.

Identifying C comments is rather complex, since the comment start sequences \code{/*} and \code{//} may also occur
in C code in string literals and character constants and in other comments. 

Comments are translated to Cogent comments. Every C block comment is translated to a Cogent block comment of the form
\code{\{- ... -\}}, every C line comment is translated to a Cogent line comment of the form \code{-- ...}. Only the 
start and end sequences of identified comments are translated, all other occurrences of comment start and end sequences
are left unchanged.

If a Cogent block comment end sequence \code{-\}} occurs in a C block comment, the translated Cogent block comment
will end prematurely. This will normally cause syntax errors in Cogent and must be handled manually. It is not
detected by Gencot.

\subsection{Comment Units}

Gencot assembles sequences of transferrable comments which are only separated by whitespace together to comment units
as follows. All comments starting in the same line after the last existing source code are concatenated to become 
one unit. Such units are called ``after-units''. All comments starting in a separate line with no existing source code 
or before all existing source code in that line are concatenated to become one unit. Such units are called ``before-units''. 

Additionally, all remaining comments at the end of a file after the last after-unit are concatenated to become the 
``end-unit''. At the beginning of a file there is often a schematic copyright comment. To allow for a specific treatment
a configurable number of comments at the beginning of a file are concatenated to become the ``begin-unit''. The default
number of comments in the begin-unit is 1.

As a result, every transferrable comment is either part of a comment unit and every comment unit
can be uniquely identified by its kind and by the source file line numbers where it starts and where it ends.

Heuristically, a before-unit is assumed to document the code after it, whereas an after-unit is assumed to document
the code before it. Based on this heuristics, comment units are associated to code parts. A begin-unit and an end-unit
is assumed to document the whole file and is not associated with a code part.

\subsection{Relating Comment Units to Documented Code}
\label{design-comments-relate}

Basically, Gencot translates source code parts to target code parts. Source code parts may consist of several lines,
so there may be several before- and after-units associated with them: The before-unit of the first line, the after-unit
of the last line and possibly inner units. Target code parts may also consist of several lines. The before-unit of
the first line is put before the target code part, the after-unit of the last line is put after the target code part.

If there is no inner structure in the source code part which can be mapped to an inner structure of the target code
part, there are no straightforward ways where to put the inner comment units. They could be discarded or they could be
collected and inserted at the beginning or end of the target code part. If they are collected no information is lost 
and irrelevant comments can be removed manually. However, in well structured C code inner comment units are rare,
hence Gencot discards them for simplicity and assumes, that this way no relevant information will be lost.

If the source code part has an inner structure units can be associated with subparts and transferred to subparts of the
target code part. Gencot uses the following general model for a structured source code part: It may have one or more
embedded subparts, which may be structured in a similar way. Every subpart has a first line where it begins and a last 
line where it ends. Before and after a subpart
there may be lines which contain code belonging to the surrounding part. Subparts may overlap, then the last line of 
the previous subpart is also the first line of the next subpart. Subparts may overlap with the surrounding part, then 
the first or last line of the subpart contains also code from the surrounding part.

For a structured source code part Gencot generates a target code part for the main part and a target code part for every 
subpart. The subpart targets may be embedded in the main part target or not. If they are embedded they may be reordered.

The inner comment units of a structured source code part can now be classified and associated. Every such unit is either
an inner unit of the main part, a before-unit of the first line of a subpart if that does not overlap, an inner 
unit of a subpart, or an after-unit of the last line of a subpart, if that does not overlap. The units associated with a subpart
are transferred to the generated target according to the same rules as for the main part. 

If there is no main source code before the first subpart (e.g., a declaration starting with a struct definition), the
before-group of the first line is nevertheless associated with the main part and not with the first subpart. The after-group
at the end of a part is treated in the analogous way.

Inner units of the main part may be before the first subpart, between two subparts, or after the last subpart. Following 
the same argument as for inner units of unstructured source code parts, Gencot simply discards all these inner units.

As a result, for every source code part atmost the before-unit of the first line and the after-unit of the last line 
is transferred to the target part. If the source code part is structured the same property holds for every embedded 
subpart. If no target code is generated for the main part but for subparts, the before-unit of the main part immediately
precedes the before-unit of the first subpart, if both exist, and analogously for the after-units.

Target code for a part may be generated in several separated places. If no code
is generated for the main part, it must be defined to which group of subpart targets the comments associated with the
main part is associated.

\subsection{Declaration Comments}

Since toplevel declarations are not translated to a target code part in Cogent, all comments associated with them would
be lost. However, often the API documentation of a function or global variable is associated with its declaration instead of the
definition. 

Therefore Gencot treats before- and after-units associated with a toplevel declaration in a specific way and 
moves them to the target code part generated for the corresponding definition. There they are placed around 
the comments associated with the definition itself. 

Gencot assumes, that only one declaration exists for each definition. If there are more than one declarations 
in the C code the comments associated with one of them are moved to the definition, the comments associated with
the other declarations are lost. 

 


\section{Processing Constants Defined as Preprocessor Macros}
\label{design-const}
Often a C source file contains constant definitions of the form
\begin{verbatim}
  #define CONST1 123
\end{verbatim}
The C preprocessor substitutes every occurrence of the identifier \code{CONST1} in every C code after the definition 
by the value 123. This is a special application of the C preprocessor macro feature.

Constant names defined in this way may have arbitrary C constants as their value. Gencot only handles integer,
character, and string constants, floating constant are not supported since they are not supported by Cogent.

\subsection{Processing Direct Integer Constant Definitions}

Constant definitions of this form could be used directly in Cogent, since they are also supported by the Cogent preprocessor.
By transferring the constant definitions to the corresponding file \code{x-incl.cogent} the identifiers are available
in every Cogent file including \code{x-incl.cogent}. 

However, for generating proofs it should be better to use Cogent value definitions instead of having unrelated 
literals spread across the code. The Cogent value definition corresponding to the constant definition above can either 
be written in the form
\begin{verbatim}
  #define CONST1 123
  const1: U8
  const1 = CONST1
\end{verbatim}
preserving the original constant definition or directly in the shorter form
\begin{verbatim}
  const1: U8
  const1 = 123
\end{verbatim}
Since the preprocessor name \code{CONST1} may also be used in \code{\#if} directives, we use the first form. A typical pattern 
for defining a default value is
\begin{verbatim}
  #if !defined(CONST1)
  #define CONST1 123
  #endif
\end{verbatim}
This will only work if the preprocessor name is retained in the Cogent preprocessor code.

If different C compilation units use the same preprocessor name for different constants, the generated Cogent value definitions
will conflict. This will be detected and signaled by the Cogent compiler. Gencot does not apply any renaming to prevent these
conflicts.

For the Cogent value definition the type must be determined. It may either be the smallest primitive type covering the value 
or it may always be U32 and, if needed, U64. The former requires to insert upcasts whenever the value is used for a different 
type. The latter avoids the upcast in most cases, however, if the value should be used for a U8 or U16 that is not possible 
since there is no downcast in Cogent. Therefore the first approach is used.

Constant definitions are also used to define negative constants sometimes used for error codes. Typically they are used for 
type \code{int}, for example in function results. Here, the type cannot be determined in the way as for positive values, since the 
upcast does not preserve negative values. Therefore we always use type U32 for negative values, which corresponds to type 
\code{int}. This may be wrong, then a better choice must be used manually for the specific case.

Negative values are specified as negative integer literals such as -42. To be used in 
Cogent as a value of type U32 the literal must be converted to an unsigned literal using 2-complement by: 
\code{complement(42 - 1)}.
Since Cogent value definitions are translated to C by substituting the \textit{expression} for every use, it should be as 
simple as possible, such as \code{complement 41} or even \code{0xFFFFFFD6} which is \code{4294967254} in decimal notation.

As described in Section~\ref{design-names}, names for preprocessor defined constants are always mapped to a different
name for the use in Cogent. This is not strictly necessary, if a preprocessor name is lowercase. By convention, C preprocessor 
constant definitions use uppercase identifiers, thus they normally must be mapped anyways.

For comment processing, every preprocessor constant definition is treated as an unstructured source code part.

\subsection{Processing Direct Character and String Constant Definitions}

A character constant definition has the form
\begin{verbatim}
  #define CONST1 'x'
\end{verbatim}
It is translated to a Cogent value definition similar as for integer constants. As type always \code{U8}
is used, the constant is transferred literally.

A string constant definition has the form
\begin{verbatim}
  #define CONST1 "abc"
\end{verbatim}
It is translated to a Cogent value definition similar as for integer constants. As type always \code{String}
is used.

In C it is also possible to specify a string constant by a sequence of string literals, which will be concatenated.
A corresponding string constant definition has the form
\begin{verbatim}
  #define CONST1 "abc" "def"
\end{verbatim}
Since there is no string concatenation operator in Cogent, the concatenation is performed by Gencot and
a single string literal is used in the Cogent value definition.

\subsection{Processing Indirect Constant Definitions}

A constant definition may also reference a previously defined constant in the form 
\begin{verbatim}
  #define CONST2 CONST1
\end{verbatim}

In this case the Cogent constant definition uses the same type as that for \code{CONST1} and
also references the defined Cogent constant and has the form
\begin{verbatim}
  #define CONST2 CONST1
  const2: U8
  const2 = const1
\end{verbatim}

\subsection{Processing Expression Constant Definitions}

A constant definition in C may also specify its value by an expression. In this case the C preprocessor will replace
the constant upon every occurrence by the expression, every expression according to the C syntax is admissible. 

In this case Gencot also generates a Cogent value definition and transfers the expression. Gencot does not evaluate
or translate the expression, however, it maps all contained names of other preprocessor defined constants to their
Cogent form, so that they refer the corresponding Cogent value name. As type for an expression Gencot always assumes
\code{int}, i.e. \code{U32} in Cogent.

If the expression is of type \code{int} and only uses operators which also exist in Cogent, positive integer constants 
and preprocessor defined constant names, the resulting expression will be a valid Cogent expression. In all other cases
the Cogent compiler will probably detect a syntax error, these cases must be handled manually.

\subsection{External Constant References}

If the constant \code{CONST1} is an external reference in the sense of Section~\ref{design-modular-extref}, a corresponding
Cogent constant definition is generated in the file \code{<package>-exttypes.cogent}. It has the same form
\begin{verbatim}
  #define CONST1 123
  const1: U8
  const1 = CONST1
\end{verbatim}
as for a non-external reference. Thus we define the original preprocessor constant name \code{CONST1} here, although
it is already defined in the external origin file. The reason for this approach is that the define directive here 
is intended to be processed by the Cogent preprocessor. Therefore we cannot include the origin file to make the name
available, since that would also include the C code in the origin file.

If the external definition is indirect, the value used in the define directive is always resolved to the final
literal or to an existing external reference. This is done for determining the Cogent type for the constant 
and avoids introducing unnecessary intermediate constant names.



\section{Processing Other Preprocessor Directives}
\label{design-preprocessor}

A preprocessor directive always occupies a single logical line, which may consist of several actual lines where 
intermediate line ends are backslash-escaped. No C code can be in a logical line of a preprocessor directive.
However, comments may occur before or after the directive in the same logical line. Therefore, every preprocessor 
directive may have an associated comment before-unit and after-unit, which are transferred as described in 
Section~\ref{design-comments}. Comments embedded in a preprocessor directive are discarded.

We differentiate the following preprocessor directive units:
\begin{itemize}
\item Preprocessor constant definitions
\item all other macro definitions and \code{\#undef} directives,
\item conditional directives (\code{\#if, \#ifdef, \#ifndef, \#else, \#elif, \#endif}),
\item include directives (quoted or system)
\item all other directives, like \code{\#error} and \code{\#warning}
\end{itemize}

To identify constant definitions we resolve all macro definitions as long as they are defined by another single
macro name. If the result is a C integer constant (possibly negative) or a C character constant the macro is assumed
to be a constant definition. All constant definitions are processed
as described in Section~\ref{design-const}.

For comment processing every preprocessor directive is treated as an unstructured source code part.

\subsection{Configurations}
\label{design-preprocessor-config}

Conditional directives are often used in C code to support different configurations of the code. Every configuration
is defined by a combination of preprocessor macro definitions. Using conditional directives in the code, whenever the
code is processed, only the code for one configuration is selected by the preprocessor.

In Gencot the idea is to process all configurations at the same time. This is done by removing the conditional 
directives from the code, process the code, and re-insert the conditional directives into the generated Cogent code.

Only directives which belong to the <package> are handled this way, i.e., only directives which occur in source
files belonging to the <package>. For directives in other included files, in particular in the system include files,
this would not be adequate. First, normally there is no generated target code where they could be re-inserted.
Second, configurations normally do not apply to the system include files.

However, it may be the case that Gencot cannot process two configurations at the same time, because they contain
conflicting information needed by Gencot. An example would be different definitions for the same type which
shall be translated from C to a Cogent type by Gencot.

For this reason Gencot supports a list of conditions for which the corresponding conditional directives are not 
removed and thus only one configuration is processed at the same time. Then Gencot has to be run separately for every
such configuration and the results must be merged manually.

Conditional directives which are handled this way are still re-inserted in the generated target code. This
usually results in all branches being empty but the branches which correspond to the processed configuration.
Thus the branches in the results from separate processing of different configurations can easily be merged manually
or with the help of tools like diff and patch.

Retaining Conditional directives for certain configurations in the processed code makes only sense if the corresponding
macro definitions which are tested in the directives are retained as well. Therefore also define directives can be
retained. The approach in Gencot is to specify a list of regular expressions in the format used by awk. All directives
which match one of these regular expressions are retained in the code to be interpreted before processing the code.
The list is called the ``Gencot directive retainment list'' and may be specified for every invocation of Gencot.

The retained directives affect only the C code, the preprocessor directives are already selected for separate processing.
To suppress directives which belong to a configuration not to be translated, macro definitions must be explicitly
suppressed with the help of the Gencot manual macro list (see below) and include directives must be suppressed
with the help of the Gencot include omission list (see below). Conditional directives are automatically omitted
if their content is omitted.

\subsection{Conditional Directives}

Conditional directives are used to suppress some source code according to specified conditions. Gencot aims to
carry over the same suppression to the generated code.

\subsubsection{Associating Conditional Directives to Target Code}

Conditional directives form a hierarchical block structure consisting of ``sections'' and ``groups''. A group
consists of a conditional directive followed by other code. Depending on the directive there are ``if-groups''
(directives \code{\#if, \#ifdef, \#ifndef}), ``elif-groups'' (directive \code{\#elif}), and ``else-groups''
(directive \code{\#else}). A section consists of an if-group, an optional sequence of elif-groups, an optional
else-group, and an \code{\#endif} directive. A group may contain one or more sections in the code after the
leading directive.

Basically, Gencot transfers the structure of conditional directives to the target code. Whenever a source code
part belongs to a group, the generated target code parts are put in the corresponding group. 

This only works if the source code part structure is compatible with the conditional directive structure.
In C code, theoretically, both structures need not be related. Gencot assumes the following restrictions:
Every source code part which overlaps with a section is either completely enclosed in a group or
contains the whole section. It may not span several groups or contain only a part of the section. If a
source code part is structured, contained sections may only overlap with subparts, not with code belonging
to the part itself. 

Based on this assumption, Gencot transfers conditional directives as follows. If a section is contained in an 
unstructured source code part, its directives are discarded. If a section is contained in a structured source
code part, its directives are transferred to the target code part. Toplevel sections which are not contained in
a source code part are transferred to toplevel. Generated target code parts are put in the same group which
contained the corresponding source code part.

It may be the case that for a structured source code part a subpart target must be placed separated from the
target of the structured part. An example is a struct specifier used in a member declaration. In Cogent, the 
type definition generated for the struct specifier must be on toplevel and thus separate from the generated member.
In these cases the condition directive structure must be partly duplicated at the position of the subpart target,
so that it can be placed in the corresponding group there.

Since the target code is generated without presence of the conditional directives structure, they must be 
transferred afterwards. This is done using the same markers \code{\#ORIGIN} and \code{\#ENDORIG} as for the
comments. Since every conditional directive occupies a whole line, the contents of every group consists of
a sequence of lines not overlapping with other groups on the same level. If every target code part is marked 
with the begin and end line of the corresponding source code part, the corresponding group can always be
determined from the markers.

The conditional directives are transferred literally without any changes, except discarding embedded comments. 
For every directive inserted in the target code origin markers are added, so that its associated comment before-
and after-unit will be transferred as well, if present.

\subsection{Macro Definitions}
\label{design-preprocessor-macros}

Preprocessor macros are defined in a definition, which specifies the macro name and the replacement text. Optionally,
a macro may have parameters. After the definition a macro can be used any number of times in ``macro calls''.
A macro call for a parameterless macro has the form of a single identifier (the macro name). A macro call for
a macro with parameters has the form of a C function call: the macro name is followed by actual parameter values
in parenthesis separated by commas. However, the actual parameter values need not be C expressions, they can be 
arbitrary text, thus the macro call need not be a syntactically correct C function call.

Macro calls can occur in C code or in other preprocessor directives (macro definitions, conditional directives, 
and include directives). All macro calls occuring in C code must result in valid C code after full expansion 
by the preprocessor.

\subsubsection{General Approach}

Gencot tries to preserve macros in the translated target code instead of expanding them. In general this requires
to implement two processing aspects: translating the macro definitions and translating the macro calls. Since Gencot
processes the C code separately from the preprocessor directives, macro call processing can be further distinguished
according to macro calls in C code and in preprocessor directives.

Gencot processes the C code by parsing it with a C parser. This implies that macro calls in C code must correspond
to valid C syntax, or they must be preprocessed to convert them to valid C syntax. Note that it is always possible
to do so by fully expanding the macro definition, however, then the macro calls cannot be preserved.

There are several special cases for the general approach of macro processing:
\begin{itemize}
\item if calls for a macro never occur in C code they need not be converted to valid C syntax and they need
only be processed in preprocessor directives. This is typically the case for ``flags'', i.e., macros with
an empty replacement text which are used as boolean flags in conditional directives.
\item if for a parameterless macro all calls in C code occur at positions where an identifier is expected,
the calls need not be converted to valid C syntax and can be processed in the C code. This approach applies
to the ``preprocessor defined constants'' as described in Section~\ref{design-const}. 
\item if for a macro with parameters all calls in C code are syntactically valid C function calls and
always occur at positions where a C function call is expected, the calls need not be converted to valid C 
syntax and can be processed in the C code.
\item if a macro need not be preserved by Gencot, its calls can be converted to valid C code by fully expanding
them. Then the calls are not present anymore and the calls and the definition need not be processed at all.
This approach is used for conflicting configurations as described in Section~\ref{design-preprocessor-config}.
\end{itemize}

When Gencot preserves a macro, there are several ways how to translate the macro definition and the macro calls.
An apparent way is to use again a macro in the target code. Then the macro definition is translated by translating
the replacement text and optionally also the macro name. If the macro name is translated then also all macro calls
must be translated, otherwise macro calls need only be translated if the macro has parameters and the actual 
parameter values must be translated.

How the macro replacement text is translated depends on the places where the macro is used. If it is only used 
in preprocessor directives, usually no translation is required. If it is used in C code parts which are translated to
Cogent code, the replacement text must also be translated to Cogent. If it is used in C function bodies it must
be translated in the same way as C function bodies, i.e., only the identifiers must be mapped and calls of other
macros must be processed.

A macro may also be translated to a target code construct. Then the macro definition is typically translated to 
a target code definition (such as a type definition or a function definition) and the macro calls are translated
to usages of that definition. This approach is used for the ``preprocessor defined constants'' as described in 
Section~\ref{design-const}: the macro definitions are translated to Cogent constant definitions and the macro 
calls occurring in C code are translated to the corresponding Cogent constant names. Additionally, the original
macro definitions are retained and used for all macro calls occurring in conditional directives, which are not
translated. Macro calls in the replacement text of other preprocessor constant definitions are translated to 
the corresponding Cogent constant names.

\subsubsection{Flag Translation}

A flag is a parameterless macro with an empty replacement text. Its only use is in the conditions of 
conditional preprocessor directives, hence macro calls for flags only occur in preprocessor directives.

Gencot translates flags by directly transferring them to the target code. Neither their macro definitions 
nor their macro calls are further processed by Gencot.

The translation of a flag can be suppressed with the help of the Gencot manual macro list (see below).

\subsubsection{Manual Macro Translation}

Most of the aspects of macro processing cannot be determined and handled automatically by Gencot. Therefore a
general approach is supported by Gencot where macro processing is specified manually for specific macros used in the
translated C program package.

The manual specification consists of the following parts:
\begin{itemize}
\item A specification of all parameterless macros which shall not be processed as preprocessor defined
constants or flags. This specification consists of a list of macro names, it is called the ``Gencot manual macro list''.
For all macros in this list a manual translation must be specified. Macros with parameters are never 
processed automatically, for them a manual translation must always be specified if they shall be preserved.
\item For all manually processed macros for which macro calls may occur in C code a conversion to valid C 
code may be specified.
This specification is itself a macro definition for the same macro, where the replacement text must be valid
C code for all positions where macro calls occur. A set of such macro definitions is called a ``Gencot macro 
call conversion''. It is applied to all macro calls and the result is fed to 
the Gencot C code translation and is processed in the usual way, no further manual specification for the macro call
translation can be provided. Since the conversion is applied after all preprocessor directives have been
removed, it has no effect on macro calls in preprocessor directives. 
\item For all manually translated macros a translation of the macro definition may be specified. It has the form 
of arbitrary text marked with a specification of the position of the original macro definition in its source file.
According to this position it is inserted in the corresponding target code file. A collection of such macro 
definition translations is always specifi for a single source file and is called the ``Gencot macro translation''
for the source file.
\end{itemize}
All four parts may be specified as additional input upon every invocation of Gencot.

The usual application for suppressing a flag definition with the Gencot manual macro list ist a parameterless macro
which is conditionally defined by a replacement text or as empty. The second definition then looks like a flag but
other than for flags the macro calls typically occur in C code. Usually in this case also the macro calls should
be suppressed, this can be done by adding an empty definition for the macro to the Gencot macro call conversion.

Macro definitions are always translated at the position where they occurred in the source file.
If the definition occurs in a file \code{x.h} it is transferred to file \code{x-incl.cogent} to a corresponding position,
if it occurs in a file \code{x.c} it is transferred to file \code{x.cogent} to a corresponding position.

This implies that translated macro definitions are not available in the file \code{x-globals.cogent} and in the files with
antiquoted Cogent code. If they are used there (which mainly is the case if macro calls occur in a conditional
preprocessor directive which is transferred there), a manual solution is required.

If different C compilation units use the same name for different macros, conflicts are caused in the integrated Cogent
source. These conflicts are not detected by the Cogent compiler. A renaming scheme based on the name of the file 
containing the macro definition would not be safe either, since it breaks situations where a macro is deliberately
redefined in another file. Therefore, Gencot provides no support for macro name conflicts, they must be detected and
handled manually.

\subsubsection{External Macro References}

Whenever a macro call occurs in a source file, it may reference a macro definition which is external in 
the sense of Section~\ref{design-modular-extref}. For such external references the (translated) definition 
must be made available in the Cogent compilation unit.

For all preprocessor defined constants (i.e. parameterless nonempty macros not listed in the Gencot manual 
macro list) Gencot adds the translated macro definition to the file \code{<package>-exttypes.cogent}. For
manually translated macros a separate Gencot macro translation must be specified for external macro definitions.
For them the position specification is omitted, they are simply appended to the file 
\code{<package>-exttypes.cogent}. If this is not sufficient, because macro calls already occur in 
\code{<package>-exttypes.cogent}, they must be inserted manually at the required position.

To avoid introducing additional external references, in the macro replacement text for preprocessor defined 
constants all macro calls are resolved to existing external reference names or until they are fully resolved.
Manually translated macro definitions should handle external macro calls in a similar way.

\subsection{Include Directives}
\label{design-preprocessor-incl}

In C there are two forms of include directives: quoted includes of the form
\begin{verbatim}
  #include "x.h"
\end{verbatim}
and system includes of the form
\begin{verbatim}
  #include <x.h>
\end{verbatim}
Additionally, there can be include directives where the included file is specified by a preprocessor macro call,
they have the form
\begin{verbatim}
  #include MACROCALL
\end{verbatim}
for them it cannot be determined whether they are quoted or system includes.

Files included by system includes are assumed to be always external to the translated <package>, therefore system
include directives are discarded in the Cogent code. The information required by external references from system 
includes is always fully contained in the file \code{<package>-exttypes.cogent}.

Macro call includes are normally assumed to be quoted includes and are treated similar.

Quoted includes and macro call includes can be omitted from the translation by adding the file specification
to the ``Gencot include omission list''. In every line it must contain the exact file specification, as it
appears in the include directive, for example
\begin{verbatim}
  "x.h"
  MACROCALL
\end{verbatim}

\subsubsection{Translating Quoted Include Directives}

Quoted include directives for a file \code{x.h} which belongs to the Cogent compilation unit are always translated 
to the corresponding Cogent preprocessor include directive
\begin{verbatim}
  #include "x-incl.h"
\end{verbatim}
If the original include directive occurs in file \code{y.c} the translated directive is put into the file 
\code{y.cogent}. If the original include directive occurs in file \code{y.h} the translated directive is put into the file
\code{y-incl.h}. 

Other quoted include directives and macro call includes are transferred to the Cogent source file without modifications, if
necessary they must be processed manually.

\subsection{Other Directives}

All other preprocessor directives are discarded. Gencot displays a message for every discarded directive.

 


\section{Including Files for C Code Processing}
\label{design-ccode}

After comments and preprocessor directives have been removed from a C source file, it is parsed and
the C language constructs are processed to yield Cogent language constructs. 

\subsection{Including Files for C Code Processing}

When Gencot processes the C code in a source file, it may need access to information in files included by
the source file. An example is a type definition for a type name used in the source file. Hence for
C code processing Gencot always reads the source file together with all included files. Since in the source
file all comments and preprocessor directives have been  removed, they must also be removed in the included files
which belong to the same <package>. Gencot assumes these are the files included by a quoted include directive.

There are two possible approaches how this can be done.

The first approach is to use the C preprocessor which is invoked by language-c before parsing. It processes
include directives as usual, hence it would be sufficient to leave the include directives in the source code
when removing the other preprocessor directives. However, this would include the original \code{.h} files and
\textit{process} all preprocessor directives there, instead of removing them. The directives have to be 
removed from all included files in <package> in a separate step, then the include directives have to be modified to include
the results of that step instead of the original files.

The second approach is to process all quoted include directives before the other directives are removed, 
resulting for every source
file in a single file containing all included information and all other preprocessor directives. Then the directives
are removed from this file with the exception of the system include directives. When the result is fed to the 
language-c parser its preprocessor will expand the system includes as usual, thus providing the complete 
information needed for processing the C code.

Gencot uses the second approach, since this way it can process every source file independently from previous steps for 
other source files and it needs no intermediate files which must be added to the include file path of the language-c
preprocessor.

In the included original files the comments are still present and must be removed as well. This could be done 
by the language-c preprocessor immediately before parsing. However, it is easier to remove the preprocessor
directives when the comments are not present anymore. Therefore, Gencot removes the comments immediately after
processing the quoted include directives.

\subsection{Processing the C Code}



 


\section{Mapping C Datatypes to Cogent Types}
\label{design-types}

Here we define rules how to map common C types to binary compatible Cogent types. Since the usefulness of a mapping
also depends on the way how values of the type are processed in the C program, the resulting types may require manual 
modification.

\subsection{Item Properties}
\label{design-types-itemprops}

The Cogent type system reflects more and other data properties than the C type system, e.g., linear types
and readonly types. Instead of simply determining these properties automatically, Gencot supports declaring 
them by the developer. Before translating a C program to Cogent, the developer may specify properties
for several ``items'' in the C program. Gencot reads these declarations in addition to the C source code
and uses them when determining the Cogent type for the translated item.

\subsubsection{Items}

An item for which properties can be declared for Gencot may be every entity in the C program which has a declared type.

Items may be global, such as global variables and functions, or they may be sub-items of other items, such as a
struct member or a function parameter. Sub-items are determined by the type of the main item. If that type is
specified by a common identifier (typedef name or tag name) for several main items together, properties for
a sub-item may not be declared individually anymore, instead, they must be declared for the corresponding sub-item
of all main items together. Such sub-items are called ``collective items'' here.

Items may also be local variables defined in a function body. Gencot identifies local items by their C identifier.
The C identifier need not be unique, not even in the same function, due to the block structure of C function bodies
with the possibility to define variables in every block. Therefore the identifiers of local items are not 
unique in the current Gencot version and thus it does not make sense to declare properties for local items. 

There are two kinds of global items: C variables (also called ``objects'' in the C standard) and C functions, which
are defined on toplevel (called an ``external definition'' in the C standard), i.e., not locally in a function. Note that
these two kinds only differ in their types: functions have a derived function type, objects may have all other types.

Global items are always named by a C identifier. Depending on the linkage, the identifier may be unique (external linkage)
or it may only be unique for the compilation unit (internal linkage). Gencot uses the C identifier to identify global 
items. In the case of internal linkage the file name of the compilation unit is used for disambiguation.

Gencot supports property declarations for the following kinds of sub-items:
\begin{itemize}
\item the members of a struct or union
\item the elements of an array
\item the data referenced by a pointer
\item the parameters and the result of a function
\end{itemize}
Members are identified by their name. Parameters are identified by their name or by their position. The other
sub-items are identified using specific mechanism.

Depending on their type, sub-items may again have sub-items. Gencot supports property declarations for arbitrary
deep sub-item nesting.

Global items are always individual. A sub-item is individual only if
\begin{itemize}
\item the containing item is individual
\item the containing item's type is not specified by a defined type identifier.
\end{itemize}
This means that the containing item's type must be an expression for a derived type or for a tagless struct or union.
If a struct or union declares a tag, all members can only be used as collective items, since the tag makes it possible
to assign the type to different items so that a member declaration may be shared by several sub-items. Gencot does
not check whether a tagged struct or union is only used for a single item, it always treats its members as collective.

For example, if the item \code{s} is declared by
\begin{verbatim}
  struct { int m1,m2; } s;
\end{verbatim}
then \code{m1} is an individual sub-item of \code{s}. If \code{s} is declared by
\begin{verbatim}
  struct str { int m1,m2; } s;
\end{verbatim}
then \code{m1} is not an individual sub-item of \code{s} anymore, it is a collective sub-item of all items with
type \code{struct str}.

Basically, collective items are identified by their type. If a property is declared for a type, it applies to 
all items which are declared to have this type. Additionally, for a collective item its sub-items can be determined
in the same way as for an individual item. 

Gencot does not support all kinds of C types to identify collective items. Only the following kinds are supported:
\begin{itemize}
\item typedef names,
\item tag names,
\item derived pointer types with a supported base type.
\end{itemize}
Note that collective items have the form of a generalization hierarchy, where a collective item may include one or 
more other collective items which are more specific. This hierarchy is built by using typedef names.

For example, if the C program contains the type name definition 
\begin{verbatim}
  typedef struct str { int m1,m2; } str_t;
\end{verbatim}
the collective item identified by \code{struct str} corresponds to all items declared to have the type \code{struct str},
whereas the collective subitem \code{m1} of \code{str} corresponds to all members \code{m1} of such items. The collective
item identified by the type \code{struct str*} corresponds to all items declared as pointers to \code{str} structs.
The collective item identified by the type \code{str\_t} identifies all items declared to have this type. In general 
this is semantically a subset of all items declared to have the type \code{struct str}, therefore \code{str\_t} corresponds
to a more specific collective item than \code{struct str}. Properties declared for \code{struct str} also apply to items
with declared type \code{str\_t} but not vice versa.

Gencot does not support primitive types for item identification, because (currently) there are no properties supported
for them and they have no sub-items. Moreover, it seems not useful to declare a property for \textit{all} items of
a primitive type. Gencot does not support derived array types for item identification because (currently) the only 
property supported for them is Read-only and it does not seem useful to declare this for all items of a certain array 
type. It could be useful to declare properties for the elements of all such items, but Gencot does not realize that.
For derived function types or their parameters and results it could be useful to declare properties, but this is not
implemented by Gencot either. The main reason for not supporting derived array and function types is the difficulty
for defining a unique syntactical representation which is needed for the current implementation of the item property
mechanism as described in Section~\ref{impl-itemprops}.

\subsubsection{Properties}

The properties which can be declared for an item are usually specific for the item's type. If a property is declared
for an item for which its type does not support the property, the property is silently ignored by Gencot.

Gencot supports declaring the following item properties.

\begin{description}
\item[Read-Only]

The Read-Only property is supported for items of arbitrary type. If declared, the type of the translated item is
made read-only by applying the Cogent bang operator \code{!} to it.

If the item's type is already translated to a read-only type (see Section~\ref{design-types-readonly}), the property 
is ignored. Note, that for several types the Cogent bang operator has no effect, these cases are equivalent
to ignoring the property.

Note that semantically the Read-Only property does not apply to the item iteself, but to the item's possible values.
The item's value can still be replaced although the item has been declared as Read-Only.

\item[Not-Null]

The Not-Null property is only supported for items of pointer type. If declared, the type of the translated item
is not wrapped by the generic \code{MayNull} type (see Section~\ref{design-types-pointer}). 

Although Gencot translates array types in the same way as pointer types, it never wraps them by \code{MayNull}.
This has the same effect as specifying the Not-Null property for all items of array type, therefore it is ignored
for items of array type.

\item[No-String]

The No-String property is only supported for items of type \code{const char *}. If declared, Gencot uses the  
type \code{(CPtr U8)!} instead of \code{String} for the translated item (see Section~\ref{design-types-pointer}).

\item[Heap-Use]

The Heap-Use property is only supported for items of function type. If declared, the parameters and
result of the translated function are extended by a component of type \code{Heap} (see Section~\ref{design-types-function}).

\item[Add-Result]

The Add-Result property is only supported for parameter items. If declared, a component of the same type
is added to the result, intended for returning a modified value of the parameter (see Section~\ref{design-types-function}). 

If the function has a non-void result and at least one parameter has the Add-Result property the result type is changed
to a tuple type with the original result as first component followed by components for all parameters with the Add-Result
property in the parameter order. If the function result is void the result tuple contains only the parameters with 
Add-Result property. If there is only one such parameter the result consists only of the corresponding component.

\item[Modification-Function]

The Modification-Function property is only supported for parameter items of pointer type. If declared, the parameters and
result of the translated function are rearranged to the form of a modification function (see 
Section~\ref{design-types-function}) where the modified parameter is the first one for which the Modification-Function 
property has been declared. Note that, although the property is declared for a parameter, the function is affected
by it as a whole.

The parameter type of the function is constructed as a pair consisting of the modified parameter and a tuple of all 
other parameters in the same order as in C. If the modified parameter also has the Add-Result property the result type 
of the function is constructed as a pair consisting of the modified parameter and a tuple of the original function result
and all other parameters with the Add-Result property. If the modified parameter does not have the Add-Result property
it is assumed that the original result returns the modified parameter value and the function result type is constructed 
as the pair consisting of the original result and the tuple of all parameters with the Add-Result property (which may 
be the unit type if there are no such parameters).

The Read-Only property is ignored, if it is combined with the Modification-Function property.

\end{description}

\subsection{Numerical Types}
\label{design-types-prim}

The Cogent primitive types are mapped to C types in \code{cogent/lib/cogent-defns.h} which is included by the Cogent compiler
in every generated C file with \code{\#include <cogent-defns.h>}. The mappings are: 
\begin{verbatim}
  U8 -> unsigned char
  U16 -> unsigned short 
  U32 -> unsigned int
  U64 -> unsigned long long
  Bool -> struct bool_t { unsigned char boolean }
  String -> char*
\end{verbatim}
The inverse mapping can directly be used for the unsigned C types. For the corresponding signed C types to be binary
compatible, the same mapping is used. Differences only occur when negative values are actually used, this must be handled by using specific functions for numerical operations in Cogent.

In C all primitive types are numeric and are mapped by Gencot to a primitive type in Cogent. Note that in C the representation 
of numeric types may depend on the C version and target system architecture. However, the main goal of Gencot is only to generate
Cogent types which are, after translation to C, binary compatible with the original C types. Hence it is sufficient for the numerical 
types to simply invert the Mapping used by the Cogent compiler.

Together we have the following mappings:
\begin{verbatim}
char, unsigned/signed char -> U8
short, unsigned/signed short -> U16
int, unsigned/signed int -> U32
long int, unsigned/signed long int -> U64
long long int, unsigned/signed long long int -> U64
\end{verbatim}

The only mapping not determined by the Cogent compiler mapping is that for \code{long int}. For the gcc C version 
it depends on the architecture and is either the same as \code{int} (on 32 bit systems) or \code{long long int}
(on 64 bit systems). Gencot assumes a 64 bit system and maps it like \code{long long int}.

\subsection{Enumeration Types}
\label{design-types-enum}

A C enumeration type of the form \code{enum e} is a subset of type \code{int} and declares enumeration 
constants which have type \code{int}. According to the C99 standard, an enumeration type may be implemented
by type \code{char} or any integer type large enough to hold all its enumeration constants.

A natural mapping for C enumeration types would be Cogent variant types. However, the C implementation
of a Cogent variant type is never binary compatible with an integer type (see below). 

Therefore C enumeration types must be mapped to a primitive integer type in Cogent. Depending on the C
implementation, this may always be type \code{U32} or it may depend on the value of the last enumeration
constant and be either \code{U8}, \code{U16}, \code{U32}, or maybe even \code{U64}. Under Linux, both cc
and gcc always use type \code{int}, independent of the value of the last enumeration constant. 
Therefore Gencot always maps enumeration types to Cogent type \code{U32}.

If an enumeration type has a tag, Gencot preserves the tag information for the programmer and uses
a type name of the form \code{Enum\_tag}, as described in Section~\ref{design-names}. For tagless enums
no type names are introduced, they are directly mapped to type \code{U32}.

The rules for mapping enumeration types are
\begin{verbatim}
  enum { ... } -> U32
  enum e { ... } -> Enum_e
  enum e -> Enum_e
\end{verbatim}

The enumeration constants must be mapped to Cogent constant definitions of the corresponding type. In 
C the value for an enumeration constant may be explicitly specified, this can easily be mapped to
the Cogent constant definitions.

An enumeration declaration of the form \code{enum e \{C1, C2, C3=5, C4\}} is translated as
\begin{verbatim}
  cogent_C1: U32
  cogent_C1 = 0
  cogent_C2: U32
  cogent_C2 = 1
  cogent_C3: U32
  cogent_C3 = 5
  cogent_C4: U32
  cogent_C4 = 6
\end{verbatim}
Note that the C constant names are mapped to Cogent names as described in Section~\ref{design-names}.

\subsection{Structure and Union Types}
\label{design-types-struct}

A C structure type of the form \code{struct \{ ... \}} is equivalent to a Cogent unboxed record type \code{\#\{ ... \}}.
The Cogent compiler translates the unboxed record type to the C struct and maps all fields in the same order.
If every C field type is mapped to a binary compatible Cogent field type both types are binary compatible as a whole.

*** This is not true! Record fields are reordered by the Cogent compiler. Dargent must be used to get binary compatible types.

\subsubsection{Mapping Struct and Union Types}
A C structure may contain bit-fields where the number of bits used for storing the field is explicitly specified.
Gencot maps every consecutive sequence of bit-fields to a single Cogent field with a primitive Cogent type.
The Cogent type is determined by the sum of the bits of the bit-fields in the sequence. It is the smallest 
type chosen from \code{U8, U16, U32, U64} which is large enough to hold this number of bits. 
***--> test whether this is correct.
The name of the
Cogent field is \code{cogent\_bitfield}<n> where <n> is the number of the bit-field sequence in the C structure.
Gencot does not generate Cogent code for accessing the single bit-fields. If needed this must be done manually in Cogent.
However, Gencot adds comments after the Cogent bitfield showing the original C bit-field declarations.

A C union type of the form \code{union \{ ... \}} is not binary compatible to any type generated by the Cogent compiler.
The semantic equivalent would be a Cogent variant type. However, the Cogent compiler translates every variant type
to a \code{struct} with a field for an \code{enum} covering the variants, and one field for every variant. Even if a variant
is empty (has no additional fields), in the C \code{struct} it is present with type \code{unit\_t} which
has the size of an \code{int}. Therefore Gencot maps every union type to an unboxed abstract Cogent type.

Another semantic equivalent would be a Cogent record type, where always all fields but one are taken. However,
this type is not binary compatible either, it is translated to a normal struct where every member has another
offset. Even if the translated type in C is manually changed to a union, the Cogent take and put operations cannot
be used since they respect the field offsets. 

Together we have the mapping rules:
\begin{verbatim}
  struct s -> #Struct_s
  union s -> #Union_s
\end{verbatim}
where \code{Struct\_s} is the Cogent name of a record type corresponding to \code{struct s} and \code{Union\_s} is the
name of the abstract type introduced for \code{union s}.

As explained in Section~\ref{design-names}, Gencot always introduces a Cogent type name for each struct and union,
even if no tag is present in C. Since the tag name is either defined to name a Cogent record or an abstract type,
it is always linear and names a boxed type which corresponds to a pointer. Hence, the type name generated for a struct
is always used to refer to the type ``pointer to struct'', the struct type itself is translated to the type name 
with the unbox operator applied. The same holds for union types.

\subsection{Array Types}
\label{design-types-array}

A C array type \code{t[n]} has the semantics of a consecutive sequence of n instances of type \code{t}. A value
of type \code{t[n]} is a pointer to the first element and therefore compatible to type \code{t*}.

Basically, Cogent does not support accessing elements by an index value in an array. 
This is an important security feature since the index value is computed at runtime and cannot be statically 
compared to the array length by the compiler. Therefore, a C array type can only be mapped to an abstract type 
in Cogent, which prevents accessing its elements in Cogent code. Element access must be implemented externally 
with the help of abstract functions.

The Cogent standard library includes three abstract data types for arrays (\code{Wordarray, Array, UArray}). 
However, they cannot be used as a binary compatible replacement for C arrays, because they are implemented by 
pointers to a \code{struct} containing the array length together with the pointer to the array elements. 
Only if the C array pointer is contained in such a struct, it is possible to use the abstract data types. 
In existing C code the array length is often present somewhere at runtime, but not in a single \code{struct}
directly before the array pointer.

As of December 2018 there is an experimental Cogent array type written \code{T[n]}. It is binary compatible 
with the C array type \code{t[n]}. It is not linear, however it only supports read access to the array elements, 
the element values cannot be replaced. Thus it can be used as replacement for a pure abstract type, if the array 
is never modified and if it does not contain any pointers (directly or indirectly). If it is modified, replacing
elements can be implemented externally with the help of abstract functions.

In C the incomplete type \code{t[]} can be used in certain places. It may be completed statically, e.g. 
when initialized. Then the number of elements is statically known and the type can be mapped like \code{t[n]}.
If the number of elements is not statically known the type cannot be mapped to a Cogent array, it must be mapped 
to an abstract type.

Since the Cogent array type is still under development, the current version of Gencot does not use it for
mapping C array types. Instead, all C array types are mapped to Cogent with the help of generated names for 
abstract types.

As a C derived type, every array type depends on its base type, which is the element type. It must be defined in
the C source before the array type can be used. The Cogent compiler can only respect this ordering requirement if
it knows that the abstract type to which the array type is mapped depends on the type to which the element type
is mapped. The only way to make this known to the Cogent compiler is to use a generic type for the array with
a single type parameter for the element type. 

When a C array type is used for a field in a record, after translation from Cogent to C a type
must be used which includes the array size. To be able to specify a corresponding C typedef for the 
Cogent abstract type name, array types with different size specifications must be mapped to different 
Cogent abstract type names. This is the reason why the size specification is encoded into the type name,
as described in Section~\ref{design-names}. So a different type name is required for every array size 
occurring in the C program.

\subsubsection{Mapping Array Types}

A C array type \code{t[n]} has two slightly different meanings. When it is used for allocating space in memory,
it is used to determine the required space as \code{n * sizeof(t)}. When it is used as declared type for an 
identifier, it means that the identifier names a (constant) value of type \code{t*}, since C arrays are 
always represented by a pointer to the first element. In Cogent both cases must be supported. However, since 
the first case corresponds to a nonlinear (unboxed) type and the second case corresponds to a linear type, 
different Cogent types must be used. If the types are directly implemented by \code{t*} and \code{t[n]}, the 
linear type is not the derived pointer type of the nonlinear type, then the difference cannot be implemented 
by using a single type name with the unbox operator applied or not.

If the nonlinear case is implemented as a struct with the array as its only member, the
linear case can be implemented by a pointer to this struct which corresponds in Cogent to the same type
with the unbox operator omitted. The pointer to the struct is binary compatible with the pointer to its
only member which is binary compatible with the pointer to the first array element. 
This solution depends on the property, that a C struct with a single array as member has the same memory
layout as the array itself. If the C implementation adds padding after the array (it
cannot add it before the array according to the C specification), another solution must be used.

Together, this way the C array type is mapped for literal size specifications using a single generic abstract Cogent type
\begin{verbatim}
  type UArr<size> el
\end{verbatim}
whith an antiquoted C type definition 
\begin{verbatim}
typedef struct $id:(UArr<size> el) {
  $ty:el arr[<size>];
} $id:(UArr<size> el);
\end{verbatim}

As described in Section~\ref{design-operations-create} Gencot provides for every mapped linear type a separate type
for ``empty'' values. For a record type \code{R} the type \code{R take (..)} is used. However, the 
\code{take} operator can normally not be applied to an abstract type. There are three possible approaches for mapped 
array types.

The first approach uses the Cogent compiler flag \code{--flax-take-put} which allows the \code{take} operator to
be applied to abstract types (including abstract generic types). However, as of October 2019 the Cogent type checker 
is instable for such types and cannot process all applications. Therefore this approach is not used by Gencot.

The second approach uses a wrapper struct also in Cogent:
\begin{verbatim}
  type CArr<size> el = {arr: #(UArr<size> el)}
  type UArr<size> el
\end{verbatim}
This makes it possible to use the Cogent take type operator for modelling empty-value array types.
The drawback is that two different Cogent type names are required and that arrays are wrapped twice as struct.

The double wrapping could be avoided if type \code{UArr<size>} is directly implemented by the array type
in C as in
\begin{verbatim}
  typedef $ty:el $id:(UArr<size> el)[<size>];
\end{verbatim}
However, there are two problems with this approach. First, the Cogent compiler does not support antiquoted
type definitions for generating this form (the type parameter \code{\$ty:el} is only known in a struct type
with the generic type as tag name). Second, Cogent translates take and put operations for a record field 
to an assignment in C. Therefore, if field \code{arr} in a value of type \code{CArr<size> El} is put or taken, 
the C code generated by Cogent will be wrong, since arrays cannot be assigned in C. 

The third approach uses a second generic abstract type 
\begin{verbatim}
  type UEArr<size> el
\end{verbatim}
for every \code{<size>} to represent the empty-value array type. It is defined in C in the same way as type \code{UArr<size>}.
This also avoids the double wrapping. It does not use the \code{take} operator to construct the empty-value type
for array types, so there is no relation between both which is known to Cogent. However, the only situation where this
relation would be useful is for type variables, but the \code{take} operator cannot be applied to type variables since
it is not possible in Cogent to restrict a type variable to record types. 

The drawback of the third approach is that the empty-value 
type is constructed in different ways for mapped arrays and for mapped structs. This makes it impossible to define 
a single common macro \code{EVT} for this type construction in Section~\ref{design-operations-create}. This further 
implies that no single macro is possible if it uses \code{EVT}, and, in particular, since the type for pointers to
an unboxed array is the boxed array type (see Section~\ref{design-types-pointer}) this case must be distinguished 
in Macros which apply \code{EVT} to a constructed pointer type.

Therefore the decision in Gencot is to use the second approach. For every array \code{<size>} used in the C 
program the two generic Cogent types \code{CArr<size>} and \code{UArr<size>} are used where the latter is
abstract and is defined in antiquoted C as described above.

\subsubsection{Specifying Array Types}

For constructing an array type from a given size and element type Gencot provides the preprocessor macro
\begin{verbatim}
  CARR(<size>,<ek>,el)
\end{verbatim}
defined in \code{include/gencot/CArray.cogent}. The parameter \code{<ek>} can be used to specify an unbox
operator for the element type, it may be \code{U} or empty. The form \code{CARR(<size>,U,el)} is equivalent
to \code{CARR(<size>,,\#el)}. The main reason for this parameter is compatibility to the macro \code{CPTR}
defined in Section~\ref{design-types-pointer}, so that it is possible to construct an array type and the 
pointer-to-element type from the same input. Both macros are mainly intended to be used for defining other
macros.

The unboxed type \code{\#(Carr<size> El)} is binary compatible with the C array type and can be used for
the nonlinear case. It is a usual Cogent unboxed record and can be used freely. 
If another Cogent record has a field of an unboxed array type, the take and put operations can be applied
in the usual way, assigning the array content as a whole.

If the element type \code{El} is again an array type, the meaning in C is a multidimensional array.
This case is mapped to Cogent in the straightforward way, using an unboxed inner array type as element type
for the outer array type. For example, the C type \code{int[2][7]} is mapped to the Cogent type
\begin{verbatim}
  CArr2 #(CArr7 U32)
\end{verbatim}
which in C is a doubly wrapped array of doubly wrapped elements which should nevertheless be binary compatible 
to \code{int[2][7]}. Using the macro \code{CARR} it can be specified as \code{CARR(2,U,CARR(7,,U32))}.

As described in Section~\ref{design-names} the form \code{CArrX<size>X}
is used if the size is specified by an identifier and the single predefined generic type \code{CArrXX} is used for all other
size specifications and for array types without size specification. In the latter case no antiquoted C type definition
for the corresponding type \code{UArrXX} is provided by Gencot, C type definitions for instances must be provided 
manually. The types \code{CArrXX} and \code{UArrXX} are defined in \code{include/gencot/CArray.cogent}.

To specify the \code{<size>} of an array type by an identifier, the identifier must be a preprocessor constant
defined in the Cogent source code. The expansion of the identifier must be a valid C expression. In simple cases,
where the expansion is a single integer literal or an expression built from literals and the operators \code{+ - * /}
the expansion is both valid C and Cogent code, then the identifier can also be used at other places in the Cogent
source code. An example usage is
\begin{verbatim}
  #ifdef LARGE
  #define SECT 20
  #else 
  #define SECT 5
  #endif
  #define SIZE 3*SECT
  type UArrXSIZEX el
\end{verbatim}
It defines the generic type \code{UArrXSIZEX} to have either 60 or 15 elements, depending on the preprocessor flag \code{LARGE}.
For additional information on possibilities how to specify the identifier see Section~\ref{impl-ocomps-cogent}.

Whether a C array type is mapped to the boxed or unboxed form depends on its usage context.
A C array type cannot be used as result type of a function. Thus, the remaining possible uses of a C array type are
\begin{itemize}
\item as type of a global variable. This is translated using an access function which returns the variable
value as a pointer of the boxed type \code{CArr<size> El}. Since in Cogent boxed types
can only be allocated on the heap, this means that the global array variable must be implemented by a global pointer
for which the array is allocated on the heap.
\item as type of a function parameter. In C this is ``adjusted'' to the pointer-to-element type. Since this
is binary compatible with type \code{CArr<size> El}, this type is used as parameter type in Cogent. Alternatively 
the adjusted type could be mapped, resulting in type \code{CPtr El} or \code{El} (in case of elements of unboxed 
record or array type). The type \code{CArr<size> El} is preferred here by Gencot since it preserves more information.
Although technically \code{NULL} may be passed as parameter value, the type is not wrapped by \code{MayNull} 
(see~\ref{design-types-pointer}), thus always requiring a non-null pointer as argument.
\item as type of a member in a struct type. Here the size of the array is relevant, therefore the unboxed type
\code{\#(CArr<size> El)} is used as type for the corresponding Cogent record field.
\item as element type of another array, resulting in a multidimensional array. Again the size of the array 
is relevant here, the unboxed type \code{\#(CArr<size> El)} is used as element type.
\item as base type of a derived pointer type. Here the binary compatible type is the boxed type \code{CArr<size> El}
since it corresponds to a pointer to the array. Since the derived pointer type includes the \code{NULL} value
it is wrapped by \code{MayNull} (see~\ref{design-types-pointer}).
\item as defining type for a typedef name. Here the type to be used depends on the context where the typedef
name is used. As described in Section~\ref{design-types-typedef}, Gencot defines the mapped 
typedef name always as an alias for the boxed type \code{CArr<size> El}. Then applying the unbox operator to the typedef name 
is equivalent to applying it to \code{CArr<size> El}.
\end{itemize}

Together we have the following mapping rules for C arrays with element type \code{el} and size 
specification \code{<size>}. Depending on the context of the C array type, additionally the unbox operator must be applied.
\begin{verbatim}
  el[<literal size>] -> CArr<literal size> El
  el[<size id>] -> CArrX<size id>X El
  el[<complex size>] -> CArrXX El
  el[] -> CArrXX El
  el[*] -> CArrXX El
\end{verbatim}
where El is the result of mapping the element type \code{el} to a Cogent type. If the size specification
is too complex (not a literal or single identifier) it is omitted and the type must be handled manually.

In Cogent linear types (pointers) are only supported if they point to data on the heap. Otherwise the Isabelle 
C parser will not accept the C code generated by Cogent. Therfore, whenever the boxed form is used for an array
(i.e., whenever it is passed as parameter to a function or defined as a global variable) it must be allocated
on the heap. If it is stack allocated in the C program this must be adapted manually by allocating and deallocating
it on the heap in the Cogent program.

\subsection{Function Types}
\label{design-types-function}

C function types of the form \code{t (...)} are used in C only for declaring or defining functions or
when a typedef name is defined for a function type. In all other
places they are either not allowed or automatically adjusted to the corresponding function pointer type
of the form \code{t (*)(...)}. 

\subsubsection{Mapping Function Parameters}

In Cogent every function
has only one parameter. To be mapped to Cogent, the parameters of a C function with more than one parameter must
be aggregated in a tuple or in a record. A C function type \code{t (void)} which has no parameters is mapped
to the Cogent function type \code{() -> T} with a parameter of unit type.

The difference between using a tuple or record for the function parameters is that the fields in a 
record are named, in a tuple they are not. In 
a C function definition the parameters may be omitted, otherwise they are specified with names in a prototype.
In C function types the names of some or all parameters may be omitted, specifying only the parameter type.

It would be tempting to map C function types to Cogent functions with a record as parameter, whenever parameter 
names are available in C, and use a tuple as parameter otherwise. However, in C it is possible to assign a 
pointer to a function which has been defined 
with parameter names to a variable where the type does not provide parameter names such as in 
\begin{verbatim}
  int add (int x, int y) {...}
  int (*fun)(int,int);
  fun = &add;
\end{verbatim}
This case would result in Cogent code with incompatible function types.

For this reason we always use a tuple as parameter type in Cogent. Cogent tuple types are equivalent, if they
have the same number of fields in the same order and the fields have equivalent types. To preserve the C parameter names in 
a function definition, the parameter is matched with a tuple pattern containing variables of these
names as fields.

C function types where a variable number of
parameters is specified such as in \code{t (...)} (``variadic function type'') must 
be treated manually in specific ways. Gencot maps variadic function types
with an additional last parameter type \code{VariadicCogentParameters}. This pseudo type is intended 
to inform the developer that manual action is required. The bang operator is applied as a hint
that no modifications are returned. For a function pointer, the corresponding Cogent type has
the form \code{\#(CFun ((...,VariadicCogentParameters!) -> <result type>))}.

C function types where the parameters are omitted, such as in \code{t ()} (``incomplete function type'') 
cannot be mapped to a Cogent function type in this way. 
They can only be mapped using an abstract type as parameter type. This can again lead to incompatible 
Cogent types if a function pointer is assigned where parameters have been specified, these cases must 
be treated manually in specific ways. Note that incomplete function types cannot be used for function
definitions, only for function pointers and for declarations of external functions. 
Gencot does not translate declarations of functions with incomplete types, these must be added manually.
This behavior of Gencot is also exploited to handle manually translated macros, as described in 
Section~\ref{impl-ccomps-externs}.

Together the rules for mapping function types are
\begin{verbatim}
  t(t1, ..., tn) -> (T1, ..., Tn) -> T
  t(void) -> () -> T
  t(t1,...,tn,...) -> (T1, ..., Tn, Variadic_Cogent_Parameters!) -> T
\end{verbatim}

\subsubsection{Linear and Readonly Parameter Types}

Like every other type, the type of a function parameter may be readonly because this property can be derived
from the C type information, as described in Section~\ref{design-types-readonly}, or because the Read-Only property
has been declared for the parameter, as described in Section~\ref{design-types-itemprops}.

For a parameter with linear type, the function can only be defined in Cogent if the parameter is not discarded,
i.e. it must be part of the result. Gencot assumes the most simple handling of this case, where the result
is a tuple consisting of the original result of the C function together with a component for every parameter
of linear type. 

A parameter of linear type may not be discarded in Cogent, but it may be passed to an abstract function which discards
it. In this case the parameter must not be returned, although it has linear type. This can be specified by the 
developer by declaring the Discard-Parameter property for the parameter (see Section~\ref{design-types-itemprops}).

Gencot uses the parameter types and declared properties to make parameter types readonly or
add parameters to the function result, according to the following rules:
\begin{itemize}
\item If the C type of a parameter is not linear according to Section~\ref{design-types-readonly}, or if it
has the Discard-Parameter property or the Result-Parameter property, it is translated as described before.
\item Otherwise, if the parameter is readonly according to its type or has the Read-Only property
its translated type is made readonly by applying the
bang operator \code{!}. For a single such parameter of type \code{r} the translation rule becomes
\begin{verbatim}
  t(t1, ... r, ... tn) -> (T1, ... R!, ... Tn) -> T
\end{verbatim}
\item Otherwise, the function result is changed to a tuple and the parameter is added as component to that tuple.
For a single such parameter of type \code{l} the translation rule becomes
\begin{verbatim}
  t(t1, ... l, ... tn) -> (T1, ... L, ... Tn) -> (T,L)
\end{verbatim}
\end{itemize}

If the result is modified to a tuple, the first component is the original function result and the remaining components
are the parameters of linear type in their order as they occur in the parameter tuple.

If the Modification-Function property has been declared for a parameter (see Section~\ref{design-types-itemprops}), 
the function type is translated according to the rule
\begin{verbatim}
  t(t1, ... l1, ... tn) -> (L1,(T1, ... ... Tn)) -> (L1,(T,L2,...,Lm))
\end{verbatim}
where \code{l1} is the type of the first parameter for which the Modification-Function parameter has been declared, \code{L1}
is its translation, and \code{L2,...,Lm} are the mapped types of all other parameter of linear type for which none of the 
properties Read-Only, Discard-Parameter, or Result-Parameter has been declared. 

The parameter type of
the translated function is a pair where the second component is the tuple of all other parameter types with \code{L1}
omitted and the result type is a pair where the second component is the tuple of the original result and all non-discarded
parameters of linear type. If there are no parameters \code{L2,...,Lm} the result type has the form \code{(L1,T)}. If there
are only two parameters the parameter tuple has the form \code{(L1,T1)}. If \code{l1} is the only parameter the parameter tuple
has the form \code{(L1,())}. If the parameter with the Modification-Function
property also has the Result-Parameter property, the result type has the form \code{(L1,(L2,...,Lm))}.

If the Heap-Use property has been declared for a function, its type is translated according to the rule
\begin{verbatim}
  t(t1, ..., tn) -> (T1, ..., Tn,Heap) -> (T,Heap)
\end{verbatim}
i.e., the heap is appended to the parameter and result tuple. If the C result type is void the result type is \code{Heap}.
If \code{n = 0} the parameter type is \code{Heap}.

If additionally the Modification-Function property has been declared for a parameter the heap is added as last components \code{Tn} and
and \code{Lm} in the additional tuple of the parameter and result pairs.

\subsection{Pointer Types}
\label{design-types-pointer}

In general, a C pointer type \code{t*} is the kind of types targeted by Cogent linear types. The linear type 
allows the Cogent compiler to statically guarantee that pointer values will neither be duplicated nor 
discarded by Cogent code, it will always be passed through. 

If a pointer points to a C \code{struct} there is additional support for field access available in Cogent by 
mapping the pointer to a Cogent boxed record type. For all other pointer types this support can be employed by
mapping the type to a Cogent record with a single field of the type referenced by the pointer.

In C, a pointer-to-array type is not the type of pointers to the array address, instead its values are array addresses.
The difference from the array type is only that when applying the index operator \code{[]}, the whole array is
selected instead of only the first element. In Cogent the corresponding type is a Cogent record with a single
field of the unboxed mapped array type. Note, that this is equivalent to the mapping of the array type itself, as defined
in Section~\ref{design-types-array}, so it is used by Gencot.

As described for array types in Section~\ref{design-types-array}, Cogent must know that a pointer type depends on 
its base type, which is the type of the referenced value. Therefore pointer types are mapped to generic types which
have the base type as single type parameter. Since there is no additional information to be encoded for a pointer
type other than its base type, a single generic type is sufficient for all pointer types. As described in 
Section~\ref{design-names} the type name \code{CPtr} is used for this purpose.

\subsubsection{Mapping Pointer Types}

A pointer type \code{t*} to a struct is mapped to the corresponding boxed type, 
that means, it is mapped like the struct type \code{t}, but the unbox operator is omitted.

A pointer type \code{t*} where \code{t} is a union type is mapped in the same way to the corresponding
boxed type, omitting the unbox operator from the mapped type \code{t}. Thus no support for accessing
the referenced union is provided. The reason is, that access to the union as a whole is mostly useless
and further access to the union members cannot be provided. For consistency, Gencot treats union types
in the same way as struct types.

A pointer type \code{t*} where \code{t} is a primitive type, an enum type, 
or again a pointer type is mapped to a boxed record with a single field \code{cont} of the type \code{T} to which \code{t} is 
mapped. For every such type the generic type \code{CPtr T} name is used. This makes the 
Cogent program slightly more readable. 

The resulting generic type can be completely defined in Cogent as
\begin{verbatim}
  type CPtr ref = { cont: ref }
\end{verbatim}
it is predefined in file \code{include/gencot/CPointer.cogent}.

Values of such types are binary compatible to the C pointer type and they can be dereferenced with the help of
the Cogent take and put operations, thus supporting the full functionality of the C pointer.

A pointer type \code{t*} where \code{t} is an array type is mapped to the mapped (boxed) array type 
\begin{verbatim}
  CArr<size> El
\end{verbatim}
This approach is similar as for struct and union types: the pointer type is mapped to the plain boxed type,
the base type is mapped to the corresponding unboxed type. The difference in case of array pointers is, that also the base
type is often ``adjusted'' to the boxed type, because in C a pointer is used instead. Other than in C, however, Gencot
represents adjusted arrays by a pointer to the array as a whole, instead of as a pointer to the first element, because
this preserves more information about the array.

Finally, a pointer to \code{void} is mapped to the abstract type
\begin{verbatim}
  CVoidPtr
\end{verbatim}
defined in \code{include/gencot/CPointer.cogent}.
It is intended as a placeholder. Here the type of the referenced data is unknown. In C it is typically used 
as a generic pointer type which is cast to specific types of referenced values. This cannot be transferred to
Cogent, to be used, the type must be replaced manually. No C type definition is provided for it by Gencot.

To make the construction of a pointer type for a given Cogent type more generic (in particular for its use 
in other preprocessor macros), Gencot provides in \code{include/gencot/CPointer.cogent} the preprocessor macro
\begin{verbatim}
  CPTR(<knd>,<type>)
\end{verbatim}
where \code{<type>} is a Cogent type and \code{<knd>} is \code{U} or empty. \code{CPTR(U,T)} expands to \code{T} and
must be used if the type to be pointed to is an unboxed type of the form \code{\#Struct\_s}, \code{\#Union\_s}, or
\code{\#(Carr<size> El)}. \code{CPTR(,T)} expands to \code{(CPtr T)} and must be used in all other cases.

For example, the C type \code{struct s*} is mapped to \code{CPTR(U,Struct\_s)} and the C type \code{int*} is mapped
to \code{CPTR(,U32)}.

\subsubsection{\code{NULL} Pointers}

A Cogent linear type has the additional property that its values are never \code{NULL}. For a boxed record type this
guarantees that the record fields can always be accessed without caring whether the pointer may be \code{NULL}.
A C pointer type, instead, may include the \code{NULL} value, so it cannot be mapped directly to the Cogent linear type.

This could be handled by replacing the \code{NULL} pointer by a valid pointer to dummy data which is never used.
The main problem here is that it must be possible to determine at runtime whether a value is null or not. So simply
allocating a ``dummy'' to get a valid non-null pointer is not sufficient, it must also be possible to recognize the 
dummy pointer.

One possibility for this is if there is a value referenced by the pointer which never occurs in normal execution,
it can be used to mark the pointer as dummy. 

Another possibility is to use a single dummy pointer for all values of a specific linear type which can be null and store it 
in a seperate place for comparing it. However, this cannot be done in Cogent since the dummy pointer would be a shared linear 
value. Even if implemened in C through abstract functions, to prove memory safety every access to such a value must 
be guarded with a test for the dummy pointer. Thus it is easier to actually use the null pointer instead of the dummy pointer,
in combination with a guard testing for the null pointer

Moreover using a dummy pointer is not binary compatible if the pointer is also accessed in external existing C code where 
it is set to \code{NULL} or tested for being \code{NULL}.

Since in Cogent the values of boxed record types must not be \code{NULL}, C pointer types must be mapped to other types, if 
they are used in a context where \code{NULL} values are allowed. Gencot uses the generic type \code{MayNull a} for this purpose
where \code{a} is the Cogent linear type for the same values without \code{NULL}.

Semantically, \code{MayNull} is a marker type, its values are the same as for type \code{a} with the exception that 
\code{NULL} is also allowed as value. The C implementation of \code{MayNull a} must be equivalent to that for \code{a}.

The simplest way would be to define \code{MayNull} as abstract, then its values can never be used to access the referenced
data, they can only be processed in specific abstract functions. However, this does not fit with the special Cogent type system.
If \code{T} is a linear type which is not escapeable (because it contains components of readonly type), the type \code{MayNull T}
is also linear, but escapeable, if \code{MayNull} is an abstract generic type. This could be used to store a value of readonly
type in a value of type \code{T} in a banged context, coerce the value of type \code{T} to type \code{MayNull T}, return it from
the banged context and then access the readonly value in it, thus bypassing the Cogent type checker guarantees. If \code{MayNull T}
instead is always defined as readonly it is not escapeable, but no more linear and could be shared and discarded, which also
violates Cogent type checker guarantees. 

Instead, Gencot uses the definition (in \code{include/gencot/MayNull.cogent})
\begin{verbatim}
  type MayNull a = { no_access: #a }
\end{verbatim}
Every instance \code{MayNull T} is linear and it is escapeable iff \code{a} is escapeable. Moreover, values of type \code{MayNull T}
can be consistently casted to type \code{T} in C implementations of abstract functions for processing these values. This is 
important because otherwise the C code generated by Cogent would not be accepted by the Isabelle C parser. The only
drawback of this approach is that the field \code{no\_access} could be accessed in Cogent which could cause a null pointer
derefence. However, this can be syntactically checked for a Cogent program so that it is easy to prove that no such null
pointer dereferences can occur for it.

In Cogent for a linear type \code{T} the type \code{\#(T!)} is equivalent to \code{\#T}. Therefore, \code{MayNull (T!)} is 
equivalent to \code{\{ no\_access: \#T \}} which is equivalent to \code{MayNull T}. Therefore, if a type \code{T} is readonly,
type \code{(MayNull T)!} must be used instead of \code{MayNull T}.

Whether a value may be \code{NULL} in C is not determined by its type, it depends on the way the value is used. It may be known
to the C developer that for a parameter of pointer type a function is never (intended to be) invoked with \code{NULL} as actual
parameter, but Gencot cannot determine that from the C program. Therefore Gencot maps pointer types to Cogent heuristically 
applying \code{MayNull} or not. Basically, pointer types are always mapped with \code{MayNull} applied, if they are not 
function pointers (for function pointers see the next section). Additionally, the item property Not-Null (see 
Section~\ref{design-types-itemprops}) is inspected, if it is declared for an item \code{MayNull} is omitted in the item's type. 

A C pointer type may also result from ``adjusting'' an array type used as type for a function parameter. In this case Gencot
assumes that the actual parameter is never \code{NULL} and omits \code{MayNull} in the mapped type of the parameter.
As described in Section~\ref{design-types-array} Gencot also translates an array type for a global variable to a linear
type of the form \code{CArr<size> El} and assumes that the array is allocated on the heap so that it is valid during all 
the global lifetime. Therefore Gencot also omits \code{MayNull} in this case. Together, whenever Gencot maps an array type
to the linear form \code{CArr<size> El} it omits \code{MayNull}. If, however, Gencot maps a pointer to an array to the 
same form, it applies \code{MayNull} if not prevented by the Not-Null item property.

So together we have the basic mapping rules for pointer types:
\begin{verbatim}
  void * -> MayNull CVoidPtr
  struct s * -> MayNull Struct_s = MayNull { ... }
  union s * -> MayNull Union_s
  (*el)[...] -> MayNull (CArr... El)
  otherwise: t * -> MayNull (CPtr T)
\end{verbatim}
where \code{Struct\_s} and \code{Union\_s} are the names introduced for the struct or union types and \code{El} 
and \code{T} are the Cogent types to which \code{el} and \code{t} are mapped, respectively. When property
Not-Null has been declared for the type's item the generic type \code{MayNull} is omitted.

\subsubsection{The \code{String} type}

Cogent supports the specific primitive type \code{String} which is translated by the Cogent compiler to \code{char*}.
It is the type of the Cogent string literals. In contrast to all other Cogent types which correspond to a C pointer
type it is neither linear nor readonly, it is a regular type for which no restrictions apply. This is motivated
by the property of C string literals that they cannot be modified and are neither allocated nor disposed.

String literals are not supported by the Isabelle C parser, therefore a Cogent program using string literals cannot
be processed and verified by Isabelle. However, string literals are useful during development time for debug 
output. Their use should be surrounded by conditional preprocessor directives so that they can be easily removed
from the program for verification and for the production version. 

Gencot supports using type \code{String} in Cogent by mapping string literals to values of this type.
Since it is not possible to distinguish C string literals by their C type, Gencot uses a heuristics and the item
properties No-String and Read-Only for this mapping.

Gencot basically maps the type \code{const char*} to type \code{String}. This type is often used in C as type
for values which may be string literals, since it reflects the property of string literals to be immutable.
The \code{String} type in Cogent has the additional property, that it is not possible to access the characters
which are elements of the string. Therefore the item property No-String (see Section~\ref{design-types-itemprops})
is used to modify the mapping. If it is declared for a C item of type \code{const char*} the item's type is
mapped as a normal pointer type, i.e. to the Cogent type \code{(CPtr U8)!}. The bang operator is applied because
the C type is intrinsically read-only according to Section~\ref{design-types-readonly}.

Gencot also maps the type \code{char*} to type \code{String} for all items for which the property Read-Only has
been declared. To prevent this, the property No-String can be added to the item.

\subsubsection{Mapping Function Pointers}

In Cogent the distinction between function types and function pointer types does not exist. 
A Cogent function type of the form \code{T1 -> T2} is used both when
defining functions and when binding functions to variables. If used in a function definition, it is mapped by
the Cogent compiler to the corresponding C function type, as described in Section~\ref{design-types-function}.

In other places, however, Cogent does not translate its function types to C function pointers. Instead, it uses 
a C enumeration type where every known function has an associated enumeration constant. Whenever a 
function is bound to a variable, passed as a parameter or is invoked through a function pointer, it is 
represented by this enumeration constant in C, i.e., by an integer value.
For function invocation Cogent generates dispatcher functions which receive the integer value as an argument
and invoke the corresponding C function. 

Binary compatibility is only relevant when a function is stored, then it is always represented by the enumeration
constant in C generated from Cogent. Thus, a C function pointer type cannot be mapped by Gencot to a Cogent function type,
since this will not be binary compatible. Instead, it must be mapped to a Cogent abstract type together with 
abstract functions which translate between the abstract type and the Cogent function type (needed when invoking 
the function in Cogent).

Together, Gencot treats C function types and C function pointer types in completely different ways. It maps
C function types to Cogent function types and it maps C function pointer types to Cogent abstract types.

These abstract types treat C function pointers in Cogent as fully opaque values. The only operation which can 
be applied to them is to translate them to the enumeration value used in Cogent. This is done by comparing the
actual pointer values for equality. Since this also works for the \code{NULL} pointer, no specific measures
are required for supporting \code{NULL} pointers for functions. In particular, the type \code{MayNull} is never
used for function pointers.

Since comparing pointer values can be done in C independent from the pointer's type, a single 
common C pointer type such as \code{void *} is sufficient. Accordingly, a single abstract type in Cogent is
sufficient to represent all C function pointers. In particular, no dependencies on the parameter or result types 
need to be known by Cogent to position their definitions before the first use of the type in C. 

However, for the developer the information about the parameter and result types are useful. Moreover, if different
Cogent types are used for different C function pointer types, the Cogent type checker can be used to find
mistakes. Since the full information about the parameter and result types is contained in the Cogent function types
the easiest way is to use a generic abstract type
\begin{verbatim}
  CFunPtr (P -> R)
\end{verbatim}
with a function type \code{P->R} as single type argument. It is used for complete function pointer types where the 
parameter types are specified or it is specified that the function has no parameters using the keyword \code{void}. 
This can be thought of as an ``annotated'' or ``wrapped'' function type, although the constraint that the argument
must be a function type cannot be specified or checked in Cogent.

For a complete C function pointer type its base type is mapped to a Cogent function type as described in
Section~\ref{design-types-function}. This includes the cases for variadic function types and the cases where
parameter types are converted to readonly types or cause additional components in the result type for returning 
modified parameters.

For incomplete C function pointer types, where only the result type
is specified but no parameter types, a second generic abstract type is used:
\begin{verbatim}
  CFunInc R
\end{verbatim}
where \code{R} is the result type. It may be an arbitrary 
mapped C type, only restricted by the C rules that it may not be a function or array type. Since however C array types
are always mapped by wrapping them in a record, the corresponding unboxed types of the form \code{\#(CArr<size> El)}
\textit{can} be used as function result types, although Gencot will never do so when translating a valid C program.

Both generic abstract types are defined in \code{include/gencot/CPointer.cogent}.

Although the C function pointer is a pointer, the pointer target value (the machine code implementing the function) 
normally cannot be modified through the pointer. Hence, semantically a function pointer type does not correspond to a linear type
in Cogent, it could be represented by a readonly type or by an unboxed type. Gencot uses an unboxed type
since values of readonly types cannot escape from banged contexts. Gencot automatically applies the 
unbox operator in the form \code{\#(CFunPtr (P->R))} whenever it uses one of the two generic function pointer types.

However, if a function pointer type of these forms occurs as type of a struct member, any direct or indirect reference of a
parameter or result type to the struct type will be detected by Cogent as a cyclic type dependency. Since 
these dependencies seem to be no problem for the validation, Gencot hides the parameter and result types from
Cogent by encoding the function type as an abstract type name (see next Section below). As a result, for 
every function type which is used as base type for a function pointer type there are now two forms: the Cogent
type expression of the form \code{P -> R} and the abstract type name. Semantically, both are equivalent, however,
this equivalence is deliberately hidden from the Cogent type checker. For incomplete function pointer types similar
abstract type names are required for arbitrary non-function types.

When generating the abstract type name the goal is to use the same name 
for every occurrence of a C function pointer type. Thus it is infeasible to generate the name from 
the occurrence position (file name and line number). If an arbitrarily generated name would be used, 
Gencot could not separately compile different C files. Therefore the name is generated by encoding the structure of the 
C function or result type. Gencot implements the encoding for all mapped C types. As described
below, the encoding also supports the type operators for unboxed and readonly types and for \code{MayNull} types.
Modified function parameters are not added to the result type, they are marked as parameter type instead.

Now both kinds of function pointer types are mapped to the forms
\begin{verbatim}
  CFunPtr Encfuntype
  CFunInc Encrestype
\end{verbatim}
with the abstract encodings as type arguments. Again, this constraint cannot be specified or checked in Cogent.

If a mapped function type and the corresponding mapped function pointer type are used in the same context it is useful to 
be able to syntactically generate both from a common specification with the help of a preprocessor macro. This can be
supported by using the abstract type name also to define a synonym for the function type using another parameterized
type of the form
\begin{verbatim}
  type CFun Encfuntype = P->R
\end{verbatim}
Note that this cannot be expressed in Cogent syntax or with the help of antiquoted C, since the type \code{P->R} must
correspond to the encoding used as type argument. The Gencot mechanism for defining instances of polymorphic types and
functions must be used here, as described in Section~\ref{impl-operations}.

Another possibility is to use monomorphic abstract types instead. They have the form
\begin{verbatim}
  CFunPtr_Encfuntype
  CFunInc_Encrestype
  type CFun_Encfuntype = P->R
\end{verbatim}
It has the advantage that the function type synonyms can be defined using the normal Cogent syntax.

Due to an error in the Cogent compiler which does not treat types of the form \code{\#(A B)} as unboxed, if \code{A}
and \code{B} are abstract, the current Gencot version uses this approach.

Together the rules for mapping a function pointer type are
\begin{verbatim}
  t (*)( ... ) -> #CFunPtr_encfuntype
  t (*)() -> #CFunInc_encrestype
\end{verbatim}
where \code{encfuntype} is the encoding of the function type \code{t( ... )} 
and \code{encrestype} is the encoding of the result type \code{t}.

As additional convenience, for every complete function pointer type a Cogent type name of the form \code{CFun\_encfuntype}
is introduced for the base function type. This makes it easy to generate both types from \code{encfuntype} in a preprocessor
macro.

For example, the C function pointer type \code{int (*)(int, int[10])} is mapped to the abstract type
\begin{verbatim}
  #CFunPtr_FXU32XA10_U32X_U32
\end{verbatim}
with generated type definitions
\begin{verbatim}
  type CFunPtr_FXU32XA10_U32X_U32
  type CFun_FXU32XA10_U32X_U32 =
    (U32,CArr10 U32) -> U32
\end{verbatim}

Note that the macro call \code{CPTR(,T)} expands to \code{(CPtr T)} also if \code{T} is a Cogent function type.
This corresponds to a pointer to the function enumeration value used by Cogent and is not related to the C
function pointer type. Therefore \code{CPTR} should not be used for function pointer types.

\subsubsection{Encoding C Function Types}

Gencot represents C function pointer types with the help of abstract Cogent types which encode C function types. These
function type names are constructed from encodings of all parameter types and the result type. Hence Gencot implements
a schema for encoding all C types as Cogent abstract type names.

A C type is either a primitive type, a derived type, a typedef name, or a struct/union/enum type. Gencot maps all
struct/union/enum types to a Cogent type name. Primitive types and typedef names are specified by a single identifier
in C and are also mapped to a Cogent type name by Gencot. These names are directly used as encoding. Thus only for the 
derived types (pointer, array, and function types) a nontrivial encoding must be defined.

Every derived type in C has a single base type from which it is derived. The base type of a derived
type is always either another derived type, or it is mapped by Gencot to a Cogent type name. Hence every derived 
type can be uniquely characterized by a sequence of derivation steps starting with a type name. The sequence of 
derivation steps is syntactically encoded in the generated name as follows.

A pointer derivation step is encoded by a single letter \code{"P"}. 

An array derivation step without size
specification is encoded by a single letter \code{"A"}. An array derivation step with a literal
as size specification is encoded in the form
\begin{verbatim}
  A<size>
\end{verbatim}
where \code{<size>} is the size specification. If the size is specified by a single identifier the 
step is encoded in the form
\begin{verbatim}
  AX<size>X
\end{verbatim}
where \code{X} is a letter not occurring in the identifier.
In all other cases an array derivation step is encoded by
\begin{verbatim}
  AXX
\end{verbatim}
which may lead to name conflicts in Cogent and must be handled manually. As described in 
Section~\ref{design-types-array}, a C array type may be mapped to an unboxed or boxed form, depending
on its usage context. The unboxed form is only relevant here, if the array type is the element type 
of another array. Then it is encoded by an additional pseudo derivation step encoded as \code{"U"}.

A function derivation step is encoded in the form
\begin{verbatim}
  FX<P1>X<P2>X...X<Pn>X
\end{verbatim}
where the \code{<Pi>} are the encodings of the parameter types and \code{X} is a letter not occurring in 
any of the parameter type encodings. A parameterless function type is encoded as \code{FXX}, whereas an
incomplete function type where no parameter types are specified is encoded as \code{F}.
If the function type is variadic, an additional pseudo parameter type \code{VariadicCogentParameters}
is added as last parameter type. 

In some cases Gencot maps a C pointer type to a boxed Cogent type and its base type to the corresponding
unboxed type by applying the unbox operator \code{\#} to the Cogent type. In these cases the pointer derivation
step is omitted in the encoding, and the base type is encoded by applying a pseudo derivation step 
of the form \code{"U"}.

C pointer types can be mapped as a linear type \code{P}, if the values are never \code{NULL}, otherwise they are mapped
as the specific type \code{MayNull P}. In a type encoding the application of \code{MayNull} is represented by
a pseudo derivation step of the form \code{"N"}.

Function parameters of linear type may be mapped by Gencot as readonly or not (see Section~\ref{design-types-function}). 
This is encoded by pseudo derivation steps of the form \code{"R"} for readonly and \code{"M"} (``modifyable'')
otherwise. Parameters of nonlinear type are not marked by either of these steps. Parameters of linear typ for 
which the item property Discard-Parameter or Result-Parameter has been declared are also not
marked by either of these steps.

For all derivation steps which are applied to a base type, their encodings are concatenated, beginning with the 
last derivation step, with an underscore \code{\_} as separator. For a derived pointer or 
function type the base type can be the pseudo type \code{void}. In these cases the identifier \code{Void} is
used as base type encoding.

Hence, for example for the C type
\begin{verbatim}
  int (* [5])(int, const short*)
\end{verbatim}
the encoding is
\begin{verbatim}
  A5_P_FXU32XR_N_P_U16X_U32
\end{verbatim}

These encodings are used for representing C function pointer types by prepending \code{CFunPtr\_} or \code{CFunInc\_}
to construct abstract Cogent type names. For example, the C function pointer type
\begin{verbatim}
  int (*)(int[16], struct str)
\end{verbatim}
is mapped to the abstract Cogent type defined as
\begin{verbatim}
  type CFunPtr_FXA16_U32XU_Struct_Cogent_strX_U32
\end{verbatim}

\subsection{Defined Type Names}
\label{design-types-typedef}

In C a typedef can be used to define a name for every possible type. In principle, it would be possible to
map a typedef name by resolving it to its type and then mapping this type as described above. However, the
typedef name often bears information for the programmer, hence the goal for Gencot is to preserve this information
and map the typedef name to the corresponding Cogent type name which is defined by translating the typedef
to a Cogent type definition.

Moreover, when a C type is derived from a typedef name, Gencot also maps the derived type using the mapped typedef name
as type argument in the corresponding parameterized type or type encoding. Since the mapped typedef name is defined to be a synonym 
for the mapped type definition, the resulting parameterized Cogent type is equivalent to the mapping of the derived 
type with the resolved typedef name as base type.

An exception from this rule are typedef names for struct, union, and array types. Mapping pointer types derived 
from such typedef names with the help of \code{CPtr} would result in the following situation:
\begin{verbatim}
  typedef struct s snam
  mapping: struct s -> #Struct_s
  mapping: struct s * -> Struct_s
  mapping: snam -> Cogent_snam
  mapping: snam * -> CPtr Cogent_snam
\end{verbatim}
where \code{Struct\_s} is the name of the Cogent record type corresponding to \code{struct s}. The problem here
is that for the mapped typedef name the pointer corresponds to a pointer to pointer to struct which is not binary 
compatible with the mapped struct pointer.

Therefore Gencot treats every typedef name resolving to a struct, union, or array type as if 
it would resolve to the corresponding pointer type. The plain name is mapped with the unbox operator 
applied (for arrays depending on the usage context), the pointer type derived from it is mapped without 
unbox operator applied. 

A similar approach is used for typedef names resolving to a C pointer type which is mapped to a Cogent type of the 
form \code{MayNull P} (see Section~\ref{design-types-pointer}). Since it may depend on the Not-Null property of the 
typed item whether its values may be \code{NULL} or not, it may be necessary to use the type \code{P} for some items.
To be able to do this also when a typedef name is defined for it, Gencot defines the typedef name as a synonym 
for the type \code{P} and translates its use by applying \code{MayNull} to the type synonym, if necessary.

Special treatment is also required for the abstract type names used for representing function pointer types, as
described in Section~\ref{design-types-pointer}. For them substitutability of contained type names (i.e. substituting
a Cogent type name by its definition results in an equivalent type) is no more automatic, if mapped typedef names 
are used in names constructed by the type encoding rules. To provide substitutability
for such types Gencot explicitly defines every occurring type encoding which contains mapped type names as a synonym for the
encoding of the fully resolved type. Through transitivity of the type equivalence relation this results in full
substitutability for all type encodings and makes it feasible to use mapped typedef names also in type encodings.

The form of the actual type synonym definitions depends on the approach used for constructing function pointer type 
mappings using type encodings (see Section~\ref{design-types-pointer}). If generic types are used which take the encoding
as type argument, only one synonym definition of the form 
\begin{verbatim}
  type Enctype = Encresolvtype
\end{verbatim}
is required for every occuring type encoding \code{Enctype} with embedded mapped typedef names. 

In particular, if the base type of a function pointer type is a typedef name for a function type, this makes the mapped
typedef name \code{Enctype} a synonym for the fully resolved function type \textit{encoding} instead of the function type itself.
However, this is not a problem: in C a typedef name for a function type cannot be used for defining a function of that type, 
since a definition must always include the parameter names. A C typedef name for a function type can only be used
for declaring functions and for constructing the corresponding C function pointer type, either explicitly or implicitly 
by adjustment. So for every use of the second kind it must be mapped to a (synonym for an) encoding anyways, and a synonym 
for the Cogent function type is never required. Function declarations are only translated to Cogent for external functions,
resulting in abstract function definitions for exit wrappers. Here the function type name is resolved to the function type.

If monomorphic types are
used, a synonym definition is required for every type constructed from such a type encoding:
\begin{verbatim}
  type CFunPtr_Encfuntype = CFunPtr_Encresolvfuntype
  type CFunInc_Encrestype = CFunInc_Encresolvrestype
\end{verbatim}
Here, if \code{Encfuntype} is a mapped function type name, no synonym is defined for it in isolation. Therefore it can 
be used as synonym for the corresponding Cogent function type. 

Names for function types are translated without unbox operator applied. It could be added to make apparent that it is
a nonlinear type, however, the current Cogent version detects a function type with unbox operator as not equivalent to
the function type without unbox operator.

The resulting mapping rules are for function type names:
\begin{verbatim}
  tn -> TN
  tn* -> #CFunPtr_TN
\end{verbatim}
for names of function pointer types:
\begin{verbatim}
  tn -> #TN
\end{verbatim}
for names of a struct or union type:
\begin{verbatim}
  tn -> #TN
  tn* -> MayNull TN
\end{verbatim}
for names of an array type:
\begin{verbatim}
  tn -> TN, #TN
  tn* -> Maynull TN
\end{verbatim}
depending on its usage context for \code{tn}, 
for names of a pointer type:
\begin{verbatim}
  tn -> MayNull TN
\end{verbatim}
and for all other type names:
\begin{verbatim}
  tn -> TN
\end{verbatim}
where \code{TN} is the name mapping of \code{tn}.

This implies, that also the Cogent type definitions generated from a C typedef have to be modified, if
the target type is a struct, union, or array type. In this case Gencot translates the typedef 
to a Cogent type definition which defines the mapped typedef name as a synonym 
for the corresponding boxed type in Cogent. If the target type is a pointer type Gencot omits the 
application of \code{MayNull} to the defining type.

\subsection{Linear and Readonly Types}
\label{design-types-readonly}

C types can be qualified as \code{const}. This means, the values of the type are immutable and could be stored in 
a readonly memory. A variable declared with a const qualified type is initialized with a value and cannot be modified 
afterwards. The immutability of an aggregate type also implies that values cannot be modified by modifying parts: 
for a struct the fields cannot be modified and for an array the elements cannot be modified. This behavior corresponds 
to the behavior of all primitive and unboxed types in Cogent. 

If a C type is not qualified as \code{const}, stored values of the type may be modified. This may have non-local
effects if the stored value is shared (part of several other values). In Cogent, values of primitive and unboxed types
cannot be shared (only copies can be part of other values). Therefore a modification of the C value always corresponds to
replacing the value bound to a variable in Cogent. This can be represented by binding the new value to a variable
of the same name which will shadow the previous binding. Together, this means that a \code{const} qualifier is
irrelevant whenever a C type is translated to a primitive or unboxed type in Cogent.

The situation is different for C pointer types which are translated to linear types in Cogent. Values of linear types
may be modified using put and take operations in Cogent, but they are restricted in their use. Put and take operations
correspond to modifications of the value referenced by the pointer. Thus, a \code{const} qualification of the pointer
type is still irrelevant for them, however, a \code{const} qualification of the pointer's \textit{base type} means
that put and take operations are not possible. This case is supported by Cogent as readonly types, which are not
restricted in their use in the same way as linear types.

Note, however, that Cogent does not separate between pointers and their referenced values: the referenced value
is treated as part of the linear value. If the referenced value itself contains references, the values referenced
by them are also treated as part of the overall value. This implies, that a readonly type in Cogent corresponds to
a C pointer type with \code{const} qualified base type where all components with a pointer type transitively have
the same property.

It further implies, that a C type also corresponds to a linear type in Cogent, if it directly or indirectly 
\textit{contains} pointers where the base type is not \code{const} qualified. This may be the case for struct or union
types (members may have such pointer types) or for array types (the elements may have such a pointer type).

An exception are C pointers to functions. It is assumed that the function code cannot be modified, hence a C pointer 
to function is treated like a primitive type in Cogent.

Gencot tests every C type for being a pointer or containing a pointer. If this is the case, the translated Cogent 
type is known to be linear. The C type is then further tested whether all pointers have a \code{const} qualified
base type. If this is the case, the type is translated to a Cogent readonly type,
by applying the bang operator \code{!} to the type after translating it as described in Section~\ref{design-types-pointer}.

In particular, the readonly property is valid for all pointer types where the base type is \code{const} qualified 
and contains no pointers, such as \code{const char*}.

Values of a C type where not all pointer base types are const qualified may still be used in an immutable way in the 
C program, if the C programmer forgot to specify the const qualifier or did not care about it. In such cases the item
property Read-Only (see Section~\ref{design-types-itemprops}) can be used to tell Gencot to apply the bang operator
when it maps the type to Cogent.


\section{Basic Operations for Datatypes}
\label{design-operations}
For working with the mapped C datatypes, in particular for those mapped using abstract Cogent types, Gencot provides support by
defining and implementing polymorphic Cogent functions, some of which are abstract and some of which are implemented in Cogent.

Although the operations provided for different kinds of datatypes have different semantics, they have common properties
and most are represented by common polymorphic functions. Here the operations are first introduced conceptually, together with
the polymorphic functions, then they are presented for specific kinds of data types.
Gencot provides the polymorphic function definitions in separate Cogent files which can be
included in a Cogent source.

\subsection{Dummy Expressions}
\label{design-operations-dummy}

For every mapped type Gencot defines a dummy expression of that type. It is used as a replacement for the actual 
body when translating C functions (see Section~\ref{design-fundefs}), so that the resulting Cogent code is 
syntactically valid and can be processed and checked by the Cogent compiler.

The dummy expression for all mapped numerical and enumeration types is the literal \code{0}.
The dummy expression for the unit type (used for functions which have \code{void} as result type) is the 
unit value \code{()}.

For all other types Gencot provides in \code{include/gencot/DummyExpr.cogent} the polymorphic abstract function
\begin{verbatim}
  gencotDummy: all(a). () -> a
\end{verbatim}
from the unit type to every possible Cogent type. Since the dummy expressions are intended to be eliminated before compiling the 
C code generated by the Cogent compiler, Gencot does not provide C definitions for this abstract function. 

If a function modifies parameters of linear type, it is translated by Gencot to return a tuple consisting of the 
original result and all such parameters (see Section~\ref{design-parmod}).
The dummy result expression is then built as a tuple with a dummy expression for the original result as the first component 
and the unmodified parameters as the remaining components.

\subsection{Default Values}
\label{design-operations-default}

Conceptually, Gencot provides a default value for every regular non-function Cogent type. For the primitive numeric types it is 
\code{0}, for type \code{String} it is \code{""}, and for type \code{()} it is the unit value \code{()}.

For all Gencot function pointer types (see Section~\ref{design-operations-function}) the default value is the \code{NULL}
pointer (which cannot be denoted directly in Cogent).

For types of the form \code{MayNull P} (see Section~\ref{design-types-pointer} a useful default value would be the 
\code{NULL} pointer. However, since these types are the only linear types with a default value, and the \code{NULL} value
can be denoted by the function \code{null} defined in Section~\ref{design-operations-null}, Gencot does not support
a default value for such types.

For a regular tuple (which has no component of linear or readonly type) the default value is the tuple of default values. For a
regular unboxed record it is the record where all fields have their default value. For a regular variant type it is
the default value of the first (in the C implementation) variant. If Dargent is not used, the first variant in C is
that with the lexicographically smallest constructor name.

Gencot defines the polymorphic abstract function
\begin{verbatim}
  defaultVal: all(out:<DSE). () -> out
\end{verbatim}
in \code{include/gencot/Default.cogent}. 

Gencot provides instances for all regular types which contain no Cogent functions.

\subsection{Creating and Disposing Boxed Values}
\label{design-operations-create}

Since all pointer types are mapped to Cogent linear types, Cogent does not provide support for creating values
of these types (``boxed values''). In C a pointer can be created using the address operator \code{\&} or by allocating data on
the heap using a C standard function such as \code{malloc}. The address operator is supported by Gencot
only for data on the heap, as explained in Section~\ref{app-transfunction-addrop}. Therefore, the basic functionality
for pointer creation is allocation on the heap. This must be provided as an abstract Cogent function.

Since values of a linear type cannot be discarded in Cogent, another abstract Cogent function is required for
disposing such values, implemented by using the C standard function \code{free}.

Gencot provides the polymorphic abstract functions defined in \code{include/gencot/Memory.cogent}:
\begin{verbatim}
  create : all(evt). Heap -> Result (evt,Heap) Heap
  dispose : all(evt). (evt,Heap) -> Heap
\end{verbatim}
for creating and disposing values of linear types. Gencot provides instance implementations for all linear
types, i.e. all (boxed) record types and abstract types. \code{Heap} is an abstract data type for modelling
the C heap where the data structures are allocated, it is defined in \code{include/gencot/Memory.cogent}.
\code{Result} is a variant type defined in the Cogent standard library (usually abbreviated as \code{R})
with variants for the success and error cases. In the success case \code{create} returns the newly allocated 
boxed value and the modified heap, in the error case it only returns the heap.

The \code{create} instances only allocate space on the heap but do not initialize it. To model this property in 
Cogent, Gencot uses two different Cogent types for every linear type to represent uninitialized (``empty'') and 
initialized (``valid'') values.
Conceptually, the instances of \code{create} and \code{dispose} are only defined for the empty-value types.
The function \code{create} returns empty values, the function \code{dispose} expects empty values. In particular,
empty values cannot contain linear parts (pointers to other memory regions) and can thus be safely discarded by 
deallocating their memory space. 

Note that according to their definition, the functions \code{create} and \code{dispose} are defined for arbitrary
types \code{evt}. However, Gencot only provides instances for (empty-value) linear types. Since it is not possible
to express this type property in Cogent, the use of unsupported instances of both functions are not detected by the 
Cogent compiler.

When Gencot maps a C type to a linear Cogent type, it always uses a Cogent record type (see Section~\ref{design-types}).
For these types the empty-value type is implemented by the record with all fields taken. Gencot defines in 
\code{include/gencot/Memory.cogent} the preprocessor macro
\begin{verbatim}
  EVT(vvt)
\end{verbatim}
which expands to the empty-value type \code{vvt take (..)} corresponding to the valid-value type \code{vvt}. 

It is not possible to define a generic type for this purpose, since in Cogent it is not possible to apply the 
\code{take} operator to a type variable.

The macro can be used to specify the correct
instances of \code{create} and \code{dispose} for all types implemented as Cogent record. For example, the 
\code{create} instance for a mapped pointer type \code{CPtr U32} can be specified as
\begin{verbatim}
  create[EVT(CPtr U32)]
\end{verbatim}

Note that although it is syntactically possible to use a type \code{MayNull P} as type argument for \code{create}
and \code{dispose}, it does not make sense, because a successful result of \code{create} is never \code{NULL} and
\code{dispose} need not be applied to \code{NULL}.

\subsection{Modifying Boxed Values}
\label{design-operations-modify}

Boxed records can be modified in Cogent by the put and take operations which in the simplest case have the form
\begin{verbatim}
  let r' = r { f = v } in ...
  let r' { f = p } = r in ...
\end{verbatim}
In the put operation \code{r} is the old record value, \code{f} is the field to be written, \code{r'} is the 
new (``modified'') record value and \code{v} is the new field value. The type of \code{r'} is either the same as 
for \code{r} or it differs, if field \code{f} was taken in \code{r}. In the take operation \code{r} is the old record value, \code{f} 
is the field to be taken, \code{r'} is the record without \code{f} and \code{p} is a pattern for binding the old 
field value. The type of \code{r'} is always different from that of \code{r}.

Note that in the C translation of this code the boxed records correspond to pointers to structs and for both operations
the pointers \code{r} and \code{r'} are the same.

We generalize this idea by defining operations for modifying boxed values in the form of Cogent functions.
To provide a common framework for modification operations, Gencot defines function types for such modification functions 
in \code{include/gencot/ModFun.cogent}. The basic function types are
\begin{verbatim}
  type ModTypeFun obj res arg out = (obj,arg) -> (res,out)
  type ModFun obj arg out = ModTypeFun obj obj arg out
\end{verbatim}
Here \code{obj} is the type of the value to be modified and \code{res} is its type after the modification.
Type \code{ModTypeFun} covers the general case where the type of the new value is different from that of the old value (although the pointer
values in C are the same!), \code{ModFun} covers the more special case where
the type is the same before and after the modification. In both cases a value of type \code{arg} is passed to the modification
function, it may provide information about how to modify the value, and the modification function additionally returns a value of
type \code{out}, e.g., an error code. The functions \code{fst} and \code{snd} defined in the Cogent standard library can be used
to retrieve the modified value and the additional output.

Note, that the additional property of a ``modification function'', that the result pointer must be the same as the argument pointer,
cannot be expressed in Cogent (or any other functional language). The type \code{ModTypeFun} is used by Gencot as an informal marker
for such functions. A function implemented in Cogent is a modification function, if it only applies take and put operations 
or other modification functions to its
first argument. An abstract Cogent function is a modification function
if the C implementation always returns the same pointer, without deallocating it in between.

For generality we extend the types \code{ModTypeFun} and \code{ModFun} to non-pointer types \code{obj} and \code{res}. There
they have the normal Cogent semantics and behave like the type \code{ChgFun} defined below. Note that the types \code{obj}
and \code{res} may still be linear if they correspond to structures containing pointers.

When the item property Modification-Function (see Section~\ref{design-types-itemprops}) has been declared for a C function, Gencot
translates it to the form of a modification function, i.e. it arranges the parameter and result types to be pairs of the type
of the first parameter of linear type and a tuple of all other relevant types. Gencot then uses a type of the form 
\code{ModFun Obj (...) (...)} as type of the translated function.

The put operation can now be represented as a function of type \code{ModTypeFun R (R put f) V ()} where \code{R} is the type of
\code{r} and \code{V} that of field \code{f} and value \code{v}. If the field is not taken in \code{R} the type is \code{ModFun R V ()}.
The take operation can be represented as a function of type \code{ModTypeFun R (R take f) () V}, it returns 
a pair of the remaining record and the taken field value.

In case of an error a \code{ModTypeFun} may not be able to produce a result value of type \code{res}. For this case Gencot defines 
a function type for ``transactional'' modification functions which either succeed or ``roll back'' the modification and return 
the original input value:
\begin{verbatim}
  type TModTypeFun obj res arg out = 
     (obj,arg) -> Result (res,out) (obj,out)
\end{verbatim}
using the variant type \code{Result} from the Cogent standard library. The result type is not defined in the form 
\code{(Result res obj,out)} because then it must always be bound explicitly before it can be used in a match expression.
With the definition used here the modification function call can be directly used in a match expression where the 
additional output is bound by the patterns in the alternatives.

Note that a \code{ModFun} can be transactional, if it
returns the information that an error occurred as part of the additional result of type \code{out}.

A typical pattern for modifying a record field \code{f} is a combination of a take and a put operation of the form
\begin{verbatim}
  let r' { f = h } = r 
  and r'' = r' { f = chg(h) }
\end{verbatim}
where a function \code{chg} is used to determine the new field value from the old field value. We can generalize function
\code{chg} to a function of type \code{(fld,arg)->(fld,out)} where \code{fld} is the type of field \code{f}. It takes 
additional input of type \code{arg} and returns additional output of type \code{out}. However, it need not be a modification
function, since type \code{fld} need not be linear and \code{chg} may map its first argument to an arbitrary other value
of the same type. Gencot defines the function type
\begin{verbatim}
  type ChgFun obj arg out = (obj,arg) -> (obj,out)
\end{verbatim}
for this kind of ``change functions''. Its meaning is the plain Cogent semantics of the type definition, no additional property is 
included.

Now we can define a higher order function for the combined field modification as
\begin{verbatim}
  modify (r,(chg,x)) = 
    let r' { f = h } = r
    and (v,y) = chg(h,x)
    in (r' { f = v }, y)
\end{verbatim}
It has the function type \code{ModFun R (ChgFun V arg out, arg) out}, accepts as additional input a pair of a change
function for the field value and its additional input, and returns the modified record and the additional result of
the field change function. Note that \code{modify} also respects the type constraints if the field type \code{V}
is linear. The field value is neither duplicated nor discarded.

For this kind of modifying a part Gencot defines the corresponding generalized function type
\begin{verbatim}
  type ChgPartFun obj prt arg out = 
    ModFun obj (ChgFun prt arg out, arg) out
\end{verbatim}
where \code{obj} is the type of the modified object, \code{prt} is the type of the part to be changed, \code{arg} is the type
of the information passed to the part change function, and \code{out} is the type of additional output of both functions.

Since every \code{ModFun} is also a \code{ChgFun}, modification functions of this type can be chained to modify parts arbitrarily 
deep embedded in other parts. 

As an example, to change a part \code{p} of type \code{P} in a part \code{q} of type \code{Q} in a value \code{r} of type \code{R}
an expression of the form
\begin{verbatim}
  modifyQInR (r, (modifyPInQ, (chg, arg)))
\end{verbatim}
modifies \code{r} by replacing \code{p} by \code{chg (p,arg)}. To make the modification functions generic for the type of the additional input
to the part modification function and for the type of its additional output they can be defined as polymorphic:
\begin{verbatim}
  modifyQInR: all(arg,out). ChgPartFun R Q arg out
  modifyPInQ: all(arg,out). ChgPartFun Q P arg out
  chg: ChgFun P A O
\end{verbatim}

If the change function \code{chg} needs as input information of nonlinear type from other linear parts of \code{q} or \code{r} it is not
possible to pass these parts to \code{chg}. For a record type \code{R} either they must be taken from \code{r} and put back in after the modification, then
the type of \code{r} is \code{R take (...)} and \code{modifyQInR} cannot be applied because of type incompatibility. Or the parts are
accessed as readonly in a banged context for \code{r}, then the readonly parts cannot escape from the banged context to be passed to the
modification operation (which must be outside of the banged context since it modifies \code{r}). Instead, the required nonlinear information must 
be retrieved from the linear parts in a banged context for \code{r}. Since it is nonlinear it may escape from the context and can be passed 
to the modification operation.

In C it is a common pattern to pass pointers to other parts of a data structure around for efficiency and access values through these pointers 
only when needed to modify parts of the structure. In the Cogent translation the values must be accessed separately in a banged context and then
passed to the modification operation as copies.

If type \code{prt} of the part to be modified is not linear, function \code{modify} removes the part's value from the structure,
passes it to the part change function and puts the result back into the structure. This is
inefficient for large parts when the part is only \textit{modified} by changing a small subpart.
In C the typical way of dealing with this situation is to pass a pointer to the part change function 
instead. 

The same approach can be used in Cogent defining a function \code{modref} which works like \code{modify}, but
uses a part modification function of type \code{ModFun} and passes a pointer to it instead of a copy of the part. 
Since it returns the same pointer, the part value can be modified ``in-place''.
In Cogent the pointer corresponds to a value of linear type, so the part modification function can usually be 
implemented in Cogent. Function \code{modref}, instead, must be abstract and implemented in C with the help of 
the address operator \code{\&}. 

Gencot defines the function type
\begin{verbatim}
  type ModPartFun obj pprt arg out = 
    ModFun obj (ModFun pprt arg out, arg) out
\end{verbatim}
which can be used to specify the type of \code{modref}:
\begin{verbatim}
  modref: ModPartFun obj pprt arg out
\end{verbatim}
Here \code{pprt} is the pointer type corresponding to the part's type \code{prt}. Note that Gencot's type mapping 
scheme supports for every mapped C type a corresponding pointer type.

Function \code{modref} can be chained with other \code{modref} functions and with \code{modify} functions in the
same way as described for \code{modify} functions.

Of course, using a pointer to the in-place part of the boxed value introduces sharing between both. However, the 
part modification function has no access to the boxed value other than by the pointer to the part. It could only get
access if the boxed value or a part of it would be passed to it using the additional input of type \code{arg}. But since
that is passed to \code{modref} together with the boxed value itself, the Cogent type checking rules prevent this
as a double use of the boxed value. Thus the approach is safe and the part modification function can work with the
pointer according to the usual Cogent rules, as long as it always returns the same pointer as result value. This
means it cannot dispose it and it cannot store it in another structured value and return a different pointer of
the same type. Function \code{modref} then simply discards the returned pointer so that the sharing ends when it
completes.

\subsection{Initializing and Clearing Boxed Values}
\label{design-operations-init}

Gencot uses the terms ``initialization'' and ``clearing'' for the conversions between empty and valid boxed values.
After a boxed value has been created it must be initialized to be used, before it is disposed it must be cleared.

Initialization must set every part of a structured value to a valid value. For parts of nonlinear type this is
straightforward, since values of nonlinear types can be directly denoted in Cogent programs in most cases. Parts
of unboxed record and abstract types are made valid by passing a pointer to the part to an initialization
function for the corresponding boxed part value (as described for function \code{modref} in 
Section~\ref{design-operations-modify}).

For parts of linear type there are two possible approaches: they can be created (allocated) during initialization 
or they can be passed as arguments to the initialization function. If they are created, the initialization function needs the heap as additional 
in- and output. Otherwise it takes all parts of linear type as additional input. Of course, if several parts
of linear type exist, some of them can be created and some passed as arguments.

Parts of readonly type \code{S!} cannot be initialized by creating a value for them. If a value is created in the 
initialization function it must be banged there but then it may not leave the banged context. The readonly value 
must either be created by a function which returns a value of type \code{S!}, or it must be passed as argument
to the initialization function. 

Clearing must convert every part of a structured value to an empty value. For parts of nonlinear type nothing
needs to be done, or the value can be explicitly set to a default value to overwrite the stored information for
security reason. Parts of unboxed record and abstract types are made empty by passing a pointer to a clearing 
function for the boxed part value.

Parts of linear type, dually to initializing them, can be disposed during clearing, or they can be returned
as result, so that they are not discarded. If they are disposed, the clearing function takes the heap as 
additional in- and output. Otherwise it returns all parts of linear type as output. If several parts
of linear type exist, some of them can be disposed and some returned as results.

Parts of readonly type need no specific treatment during clearing, since they can be discarded in the same way as
parts of nonlinear type.

Initialization and clearing functions are modification functions in the sense of Section~\ref{design-operations-modify}.
Translated to C they always return the same pointer they received as input.
Gencot defines the following function types in \code{Memory.cogent}:
\begin{verbatim}
  type IniFun evt vvt arg out = ModTypeFun evt vvt arg out
  type ClrFun vvt evt arg out = ModTypeFun vvt evt arg out
  type TIniFun evt vvt arg out = TModTypeFun evt vvt arg out
\end{verbatim}
They can be used by the developer to mark functions as initialization or clearing functions. This is purely 
informal since no constraint between the type parameters can be enforced by Cogent, so a correct pair of 
empty-value type and the corresponding valid-value type must be specified by the developer.

The third type supports ``transactional'' initialization functions which may fail and are rolled back, they return
a result of type \code{Result vvt evt}, as for type \code{TModTypeFun} (see Section~\ref{design-operations-modify}.
This is typically the case when initializing a part of the value includes allocating space on the heap.
If this is not done, initialization only consists of storing values, which cannot fail. For clearing we always 
assume that no error can occur, then the rollback for transactional initialization is always possible without 
causing another error.

Initialization and clearing functions are defined by Gencot to always expect additional input and output values of 
arbitrary types \code{arg} and \code{out}. The input value can be used to specify default values to be used,
the heap for creating and disposing parts of linear type, and initialization and clearing 
functions for parts of the value. If a function only uses fixed default values and functions for parts 
the unit type \code{()} is used as type \code{arg}. The output value can be used to return an error code,
the modified heap, or parts of linear type which have not been disposed.

An initialization or clearing function for a type with several parts must handle all parts together because it must
transform from the empty-value type, where all parts are taken, to the valid-value type, where all parts are present.
If an initialization or clearing function handles only one part, its type must respect which other parts are 
taken and which are not. This is not feasible for types with many parts. However, if all parts must be handled together,
there are many ways how to do so, especially if there are parts of linear and/or readonly type. 

Gencot provides the following polymorphic abstract initialization and clearing functions which are defined for only 
some of these cases:
\begin{verbatim}
  initFull : all(evt,vvt). IniFun evt vvt #vvt ()
  clearFull : all(vvt,evt). ClrFun vvt evt () #vvt
  initHeap : all(evt,vvt:<E). TIniFun evt vvt Heap Heap
  clearHeap : all(vvt:<E,evt). ClrFun vvt evt Heap Heap
  initSimp : all(evt,vvt:<E). IniFun evt vvt () ()
  clearSimp : all(vvt:<E,evt). ClrFun vvt evt () ()
\end{verbatim}
Generally, these functions are only defined if \code{evt} is the empty-value type corresponding to the valid-value
type \code{vvt}. Since this cannot be expressed by constraints in Cogent, Gencot defines the preprocessor macros:
\begin{verbatim}
  INIT(<k>,vvt) -> init<k>[EVT(vvt),vvt]
  CLEAR(<k>,vvt) -> clear<k>[vvt,EVT(vvt)]
\end{verbatim}
They can be used to specify valid instances of the functions with the correct types so that the constraints
between them are fulfilled and the Cogent typechecker can be used to test for correct application.
For example, \code{INIT(Full,vvt)} expands to \code{initFull[EVT(vvt),vvt]}. The macros can be used for 
all types \code{vvt} which are implemented as a Cogent record type.

Gencot also defines the macros
\begin{verbatim}
  INITTYPE(<k>,vvt)
  CLEARTYPE(<k>,vvt)
\end{verbatim}
which expand to the corresponding function types.

The first two functions pass the full content as argument: the additional input type for initialization and the additional
result type for clearing is the
unboxed type \code{\#vvt}. It is used to pass all content for the referenced memory region to \code{initFull} which
copies it there. For \code{clearFull} it is used to return all content, in particular, all content of linear type, so that 
it is not discarded.

The second pair passes the heap as additional input and result. Additionally, \code{vvt}
may not contain any readonly parts. One reason is that readonly values are a copy of a pointer, so they 
cannot be initialized internally. Another reason is that the current Gencot implementation cannot distinguish which 
parts are readonly and which parts are not, therefore it cannot treat them differently, which is necessary for 
clearing them: copies of a pointer must not be disposed. For this reason even parts of type \code{MayNull P} are
not allowed, although they could be initialized by setting them to \code{NULL}. 
The restriction not to contain readonly parts can be expressed
in Cogent by the type constraint \code{E}, this is specified for type \code{vvt} for both functions.

Function \code{initHeap} allocates all parts of linear type on the heap 
and \code{clearHeap} disposes them. Parts of type \code{MayNull a} are initialzed to \code{null} and are disposed only 
if they are not \code{null}. 
All parts of nonprimitive type are initialized or cleared using function \code{initHeap} or \code{clearHeap}, respectively.
All parts of primitive or Gencot function pointer type are initialized to their default value \code{defaultVal ()} and are cleared by
doing nothing. Whenever an allocation fails for 
\code{initHeap}, all other allocations are rolled back using \code{clearHeap}. The initialization function returns a variant
value which signals success or error.

The third pair passes no additional information. Moreover, \code{vvt} must have no readonly or linear parts other than 
parts of a type \code{MayNull P}.
Since \code{vvt} itself must be linear, the additional constraint cannot be expressed in Cogent, 
only readonly parts are excluded by specifying type constraint \code{E} for it. 
Functions \code{initSimp} and \code{clearSimp} work as \code{initHeap} or \code{clearHeap},
but the heap is not required since there are no linear parts which must be allocated or disposed.

The first pair is the most general, it is applicable to all kinds of valid-value types \code{vvt}. However, it bears the
most overhead, since all content must be passed as argument and copied to the memory to be initialized. The other
two pairs support ``in-place'' operation, however, they are more restrictive. 

Since the type of \code{initHeap} is different from that of the other initialization functions they cannot 
be passed as argument to a common function parameter. An alternative would have been to define all initialization
functions using the common type \code{TIniFun}, but then their use would always require unnecessary checks and
implementing an error case which can never occur.

Functions \code{clearFull}, \code{clearHeap}, and \code{clearSimp} do not overwrite the referenced memory with a 
``clearing value''. If this is required, a manually defined clearing function must be used instead.

For specific types \code{vvt} custom initialization and clearing functions can be defined manually. Typically, they 
pass values for some parts as parameters (in particular those of readonly types), and use default values or heap
allocation for the others.

\subsection{Accessing Parts of Structured Values}
\label{design-operations-parts}

For working with a structured value it is often necessary to access its parts for reading or modifying them.
Gencot supports the following conceptual operations on structured values:
\begin{description}
  \item[\code{get}] access the value of a part for reading,
  \item[\code{set}] replace the value of a part by a given value, discarding the old value,
  \item[\code{exchng}] replace the value of a part by a given value, returning the old value,
  \item[\code{modify}] apply a change function to a part.
\end{description}

Every function accesses only a single part of the structured value, if several parts must be accessed at the same
time custom operations must be defined manually.

Depending on the type of structured value and the kind of specifying the part, the part may safely exist or not.
For example, for a record field specified by its name it can be statically determined whether it exists,
for an array element specified by a calculated index value this is not the case. 

The actual function types of the operations differ depending on the way how the part is specified for them and
whether it safely exists. However, if the structured value has type \code{T} and the part has type \code{S} 
the conceptual functionalities are as follows:

The function \code{get} has functionality \code{T! -> S!}, if the part safely exists. It expects a readonly value as input and returns
a readonly copy of the part's value. This operation can be defined in Cogent for arbitrary types \code{S}
because the returned value can be shared with the value remaining in the structure since both are readonly.

If the part does not safely exists there are two possible functionalities: either the function returns a variant
value, or it returns a default value if the part does not exists. The variant value is the simpler and ``cleaner''
form, possible variant types are \code{Option} and \code{Result} from the Cogent standard library (the former treating
the part semantically as ``optional'', the latter treating its nonexistence as an ``error''). However, the use
of the variant value introduces an overhead, since Cogent implements it as a record. It must be constructed, passed
on the stack as function result, and then tested and deconstructed; for very frequent accesses to a part this may
cause a relevant performance reduction. Therefore an alternative form should be provided which always passes the
part's value directly, using a default if the part does not exist. It should only be used if it is clear from the
context that the part exists. A possible choice for the default value is \code{defaultVal} (defined in 
Section~\ref{design-operations-default}), however that is restricted to parts of regular type, therefore other 
solutions are required for other types.

An alternative to the function \code{get} would be a function which takes a modifyable structure and returns the structure
together with the element. However, this would be cumbersome in many applications and misleading for proofs, since 
the function never modifies the structure. 

The function \code{set} has functionality \code{(T,S) -> (T,())} and is a modification function in the sense of 
Section~\ref{design-operations-modify}, hence its type can be denoted as \code{ModFun T S ()}. 
It expects a structure and the new value as input
and returns the structure where only the value of the part has been replaced by the new value. The result value is 
structured as a pair, so that the function has the form of a modification function. Since the old
value of the part is discarded, function \code{set} is only defined if type \code{S} is discardable.
If the part does not exist, the function returns the unmodified structure.

The function \code{exchng} has functionality \code{(T,S) -> (T,S)} and is also a modification function, so its type
can be denoted as \code{ModFun T S S}. It works like \code{set}, but instead of
discarding the old value of the part it returns it in the result. This can be done for arbitrary types \code{S}
since values of this type are neither shared nor discarded. If the part does not exist, the function returns
the unmodified structure together with the input value.

The function \code{modify} is a modification function and has functionality \code{ChgPartFun T S A O} where 
\code{A} and \code{O} are arbitrary types.
According to the definition of \code{ChgPartFun} \code{modify} applies a part change function of
type \code{ChgFun S A O} to change the part. All types may be linear, since the corresponding values are
only passed through to the part change function and back.
The part change function determines the new part value from the old part value. If the part's type \code{S}
is linear it can be implemented by actually modifying the part's value. In particular, this can be done by
again using \code{modify} as part change function changing a part's part, as it has been described
for the \code{modify} function in Section~\ref{design-operations-modify}.

If the part does not exist, function \code{modify} returns the unmodified structure. However, it must also
return a value of type \code{O}, although the part change function is never executed. Similar as for 
function \code{get} there are two possibilities: returning a variant value or a default value. A third
possibility here is to use the same type \code{A} as additional input and result to the part change
function. Then, if the part change function is not executed, its additional input can be returned
as result. This works even for linear types \code{A} since this way the value is not discarded.

If the part's type \code{S} is not linear, it can be efficiently accessed using a pointer to it.
Gencot supports this with the following two operations:
\begin{description}
  \item[\code{getref}] return a pointer to a part,
  \item[\code{modref}] apply a modification function in-place to a part.
\end{description}

The function \code{getref} has functionality \code{T! -> PS!} where \code{PS} is the mapped type used by Gencot
for pointer to \code{S}. If \code{S} is an unboxed record or abstract type \code{\#B} then \code{PS} is the
corresponding boxed type \code{B}. Otherwise \code{PS} is the type \code{P\_S} generated by Gencot for pointers
to values of type \code{S}. The function cannot be implemented in Cogent and is usually implemented in C using the address 
operator \code{\&}. The operation is safe since both the structure and the result type are readonly. The shared
memory used by both can neither be modified through the structure nor through the resulting pointer. 

If the part does not safely exist, the situation is similat to function \code{get}. However, now the result 
type \code{PS} is always linear, so \code{defaultVal} cannot be used to provide a default value.

The function \code{modref} has functionality \code{ModPartFun T PS A O} where \code{A} and \code{O} are
arbitrary types. It behaves as described in Section~\ref{design-operations-modify}. If the part does not safely exist, 
the possible solutions are the same as for function \code{modify}.

Note that all part access functions are defined in a way that they never allocate or deallocate memory on the heap.
therefore they never need the heap as additional in- and output.

An alternative to the function \code{modref} would be a pair of functions \code{ref} and \code{deref} where \code{ref}
returns the pointer together with the structure converted to a type which marks the part as removed (for a record this
corresponds to the type with a field taken), and \code{deref} converts the type back to normal. 
Since the structure has the converted type the part cannot be accessed through it
as long as the type has not been changed back which is done by \code{deref}, consuming the pointer.
However, it is possible to apply \code{deref} to another pointer of the same type, causing sharing between the 
structure and the original pointer. To prevent this it must be proven for the Cogent program that the \code{deref}
operation is always applied to a pointer retrieved by a \code{ref} operation before any other \code{deref}
is applied to the structure. This implies that
for proving the type safety properties an arbitrary complex part of the Cogent program must be taken into
account. Therefore Gencot does not support such functions.

To make the other parts of the structure available in the modification operation the \code{modref} operation could
pass the structure to it together with the additional input, with the type converted to a readonly type where the modified
part is marked as removed. This is safe because in the modification function the part cannot be accessed
through the structure and the structure cannot be modified by inserting another value for the part since it is
readonly in the modification operation. Note however, that instead of passing the structure to the modification operation,
all (nonlinear) values required from it can also be retrieved outside of \code{modref} and passed to it as part of the 
additional information of type \code{A}. Therefore Gencot does not support this approach.

\subsection{Primitive Types}
\label{design-operations-prim}

The primitive Cogent types are the numerical types, \code{Bool} and \code{String}. 

\subsubsection{Creating and Disposing Values}

Values of primitive types cannot be created and disposed. If instances such as \code{create[U32]} are used in a program 
this is not detected as error by the Cogent compiler. However, since Gencot does not provide implementations for such
instances the resulting C program will not compile.

\subsubsection{Modifying Values}

Values of primitive types cannot be modified, they can only be replaced. The type constructor \code{ChgFun} 
can be used to define such replacement functions for primitive types. The type constructors for modification functions
defined in Section~\code{design-operations-modify} can also
be applied to primitive types, however the resulting function type does not have the
intended semantics of modification functions. Therefore such types should not be used.

\subsubsection{Initializing and Clearing Values}

Values of primitive types cannot be initialized or cleared. For primitive types there is no corresponding empty-value type.
Like for the modification function types the type constructors
\code{IniFun} and \code{ClrFun} can be applied to primitive types with useful results, but not with the intended semantics.
Therefore they should not be used.

\subsubsection{Accessing Parts of Values}

Primitive values have no parts, therefore the part access operations are not provided for them.

Although the type \code{String} has a structure consisting of a sequence of characters, access to characters is not supported
by Gencot because there is no known size for \code{String} values.

\subsection{Pointer Types}
\label{design-operations-pointer}

Here we denote as ``Gencot pointer types'' all Cogent types of the form \code{CPtr Ref} where \code{Ref} is an arbitrary non-function
Cogent type (except the unit type). These types are generated by Gencot for several C pointer types (see Section~\ref{design-types-pointer}).

Gencot pointer types always point to a value of primitive type, of an abstract type representing an unboxed array, or again a pointer 
(which may also be a function pointer, or a pointer used to represent an array or a boxed record).

Gencot pointer types do not include the type \code{CVoidPtr}. Since this is the mapping of the C type \code{void*} no information 
about the referenced data structure is available. Therefore Gencot cannot support any operations 
for it. Values of this type are fully opaque, they can be passed around but neither created, nor manipulated or disposed. 

It is possible for the developer to manually use additional Gencot pointer types by applying the generic type \code{CPtr} to other
Cogent types. Gencot provides the operation support described in the following sections also for such types.

\subsubsection{Creating and Disposing Pointers}

Since every Gencot pointer type is implemented by a Cogent record type, the corresponding empty-value type can be constructed by 
taking the single field: \code{(CPtr Ref) take (..)} which is equivalent to the expansion of the macro call \code{EVT(CPtr Ref)}
(see Section~\ref{design-operations-create}).

For every Gencot pointer type \code{CPtr Ref} used in the Cogent program Gencot automatically
generates instances of the functions \code{create} and \code{dispose} of the form:
\begin{verbatim}
  create[EVT(CPtr Ref)]
  dispose[EVT(CPtr Ref)]
\end{verbatim}

\subsubsection{Modifying Pointers}

Since Gencot pointer types are specific Cogent record types, modifications of pointer values can be done with the Cogent take and 
put functions.

The only modification function applicable to a pointer consists of replacing the referenced value. Such modifications correspond
either to initialization and clearing operations or to dereferencing operations, described in the next two sections.

\subsubsection{Initializing and Clearing Pointers}

Before a pointer returned by \code{create} can be used it must be initialized by storing a value into the referenced memory region.
Dually, before disposing a pointer the memory region may be cleared.

All instances of the functions \code{initFull/Heap/Simp} and \code{clearFull/Heap/Simp} described in 
Section~\ref{design-operations-init} are available
for Gencot pointer types. Functions \code{initFull} and \code{clearFull} additionally pass the value referenced
by the pointer (wrapped in an unboxed record). Functions \code{initSimp} and \code{clearSimp} set the referenced
value to its default value, discarding it upon clearing. Functions \code{initHeap} and \code{clearHeap} 
are required, if the referenced value is linear, for allocating or deallocating it on the heap. 

Functions
\code{initHeap} and \code{clearHeap} must also be used if the referenced value is of type \code{MayNull a} 
(see Section~\ref{design-operations-null}) since it has linear type. Note that function \code{initHeap} here
actually does not use the heap since it sets the referenced value to null. However, we assume that this is
feasible and do not provide alternative support for it, since it can also be implemented in Cogent using
the put operation.

For example a pointer \code{p} of type \code{EVT(CPtr U32)} can be initialized to the value \code{5} with typechecks by the 
Cogent expression
\begin{verbatim}
  INIT(Full,CPtr U32) (p,#{cont=5})
\end{verbatim}
and a pointer \code{p} of type \code{CPtr (CPtr U32)} can be cleared with typechecks by the Cogent expression
\begin{verbatim}
  CLEAR(Heap,CPtr (CPtr U32)) (p,heap)
\end{verbatim}
which clears and disposes the referenced pointer.

Alternatively values of Gencot pointer type can be initialized and cleared using the Cogent put and take
operations. Then the code for the initialization example above is
\begin{verbatim}
  p{cont=5}
\end{verbatim}
and for the clearing example is
\begin{verbatim}
  let p{cont=h}
  and heap = dispose(h,heap)
  in (p,heap)
\end{verbatim}
where clearing the value referenced by \code{h} is omitted since it is of primitive type.

\subsubsection{Dereferencing Pointers}

If a pointer is seen as a structured value, it has the referenced value as a single part.
Then the operations for accessing parts of a structured value can be defined for a Gencot pointer type as follows
The operation \code{getref} corresponds to the identity, the operation \code{modref} is equivalent to applying the
part modification function directly to the pointer. Both are not provided separately for pointers.
The other operations all dereference the pointer in some way. Gencot provides in \code{inlude/gencot/CPointer.cogent} 
the polymorphic functions
\begin{verbatim}
  getPtr: all(ptr,ref). ptr! -> ref! 
  setPtr: all(ptr,ref:<D). ModFun ptr ref ()
  exchngPtr: all(ptr,ref). ModFun ptr ref ref
  modifyPtr: all(ptr,ref,arg,out). ChgPartFun ptr ref arg out
\end{verbatim}
Instances of these functions are only defined if \code{ptr} is a Gencot pointer type and \code{ref} is the corresponding
type of the referenced values. 

The function \code{getPtr} dereferences a readonly pointer and returns the result as readonly. The function \code{setPtr}
replaces the referenced value by its second argument, discarding the old value. The function \code{exchngPtr} works like 
\code{setPtr} but returns the old referenced value as additional result. The function \code{modifyPtr} applies a change
function to the referenced value, replacing or modifying it.

Since a Gencot pointer type corresponds to a pointer which is guaranteed to be not null, the value referenced by the 
pointer safely exists and the functions need not handle the case where it does not exist.

\subsection{Function Pointer Types}
\label{design-operations-function}

As described in Section~\ref{design-types-function}, function pointer types are mapped by Gencot to abstract types of the form 
\code{CFunPtr\_EncFuntyp} where \code{EncFuntyp} encodes a C function type, or \code{CFunInc\_EncRestyp} where \code{EncRestyp}
encodes the C result type. They are not covered by Gencot pointer types since they behave differently.

\subsubsection{Creating and Disposing Function Pointers}

Function pointers cannot be created and disposed. Gencot does not provide instances of \code{create} and \code{dispose}
for function pointer types.

\subsubsection{Modifying Values}

Function pointers cannot be modified. As for primitive types the type constructors for modification functions should not be used for them.

\subsubsection{Initializing and Clearing Values}

Function pointers cannot be initialized or cleared, there is no corresponding empty-value type.
Like for the modification function types the type constructors
\code{IniFun} and \code{ClrFun} can be applied to function pointer types with useful results, but not with the intended semantics.
Therefore they should not be used.

\subsubsection{Accessing Parts of Values}

Function pointers have no parts, therefore the part access operations are not provided for them.

\subsubsection{Converting between Functions and Function Pointers}

For the values of the abstract type for function pointers in Cogent there are two relevant operations: invoking
it as a function and converting a Cogent function to a value of that type. Both are supported by Gencot
by providing polymorphic abstract functions in \code{inlude/gencot/CPointer.cogent} for the task.

The latter operation is supported by the polymorphic abstract function
\begin{verbatim}
  toFunPtr: all(fun,funptr). fun -> #funptr
\end{verbatim}
where \code{fun} is a Cogent function type and \code{funptr} is the Cogent abstract type representing the function pointer.

Invoking a function is supported by translating the function pointer to the Cogent function (equivalent to the enumeration value) which
then can be invoked in the usual way in Cogent. The translation is done by the polymorphic abstract function
\begin{verbatim}
  fromFunPtr: all(fun,funptr). #funptr -> Result fun ()
\end{verbatim}

All functions for which a function pointer is accepted or returned by these functions must be known to
Cogent so that there is an enumeration constant for it. It is not possible to pass a pointer to an arbitrary 
C function to Cogent as a parameter or as a field in a record. The function must either be defined in Cogent 
or it must be defined as an abstract function in Cogent. In both cases, the function may also be an instance of
a polymorphic Cogent function. 

Note, that if no function of type \code{fun} is 
defined in the Cogent program, the generated C code from an invocation of the result of \code{fromFunPtr} 
in Cogent is incomplete, since it invokes the dispatcher function which does not exist.

Illegal function pointers will be detected in the function \code{fromFunPtr}. The function must be implemented 
by selecting the result from a fixed list of known functions. Its result type uses the generic type \code{Result}
from the Cogent standard library. If the input pointer does not point to a 
function in the list (because it is NULL or points to an unknown function), an error is returned. Therefore,
whenever a function pointer shall be invoked, after translating it it must be checked whether the translation
was successful.

Note that \code{toFunPtr} is always successful, since it gets a known Cogent function as input and returns 
the corresponding function pointer.

Gencot defines in \code{inlude/gencot/CPointer.cogent} the preprocessor macros 
\begin{verbatim}
  TOFUNPTR(ENCFT)
  FROMFUNPTR(ENCFT)
\end{verbatim}
which expand to typed instances of both functions, where \code{ENCFT} is the encoding of the C function type.

Hence for example for the C type
\begin{verbatim}
  int *()(int, short)
\end{verbatim}
the following function instances are provided
\begin{verbatim}
  toFunPtr[(U32,U16) -> (CPtr U32),CFunPtr_FXU32XU16X_P_U32]
  fromFunPtr[(U32,U16) -> (CPtr U32),CFunPtr_FXU32XU16X_P_U32]
\end{verbatim}
which can be specified using the macros as
\begin{verbatim}
  TOFUNPTR(FXU32XU16X_P_U32)
  FROMFUNPTR(FXU32XU16X_P_U32)
\end{verbatim}

For incomplete function pointer types of the form \code{\#CFunInc\_ENCT} Gencot cannot provide instances of
\code{fromFunPtr} and \code{toFunPtr}, since it cannot determine the corresponding Cogent function type. Therefore
such function pointers cannot be used in any way in Cogent, they can only be passed through.

An alternative approach for invoking a function pointer would be a polymorphic abstract function 
\begin{verbatim}
  invkFunPtr: all(args,res). (#(CFunPtr (args->res)), args) -> res
\end{verbatim}
where \code{args} is the type of the single argument of the Cogent function (possibly a tuple) and \code{res}
is the result type of the Cogent function. In its C implementation \code{invkFunPtr} applies the function pointer
to the argument and returns its result. 
This approach always causes correct C code to be generated by the Cogent compiler. However, since the Isabelle
C parser does not support function pointer invocations, no refinement proof can be processed for the resulting
C program.

Therefore Gencot does not support this alternative approach.

\subsection{\code{MayNull} Pointer Types}
\label{design-operations-null}

The type safety of Cogent relies on the fact that the pointers representing values of linear types are never \code{NULL}.
If null pointers are used in the C source, Gencot reflects this in Cogent by marking the type with \code{MayNull} (see 
Section~\ref{design-types-pointer}. The way how to work with
null pointers in a binary compatible way depends on the way how the null pointers are used.

A null pointer can be used as struct member \code{f} to mark the corresponding part as ``uninitialized''. 
It is set when the struct is created and later
replaced by a valid pointer. It remains valid until the struct is disposed. If additionally the struct is used only in places
during the ``uninitialized state'' which are different from those afterwards, the ``uninitialized state'' can be represented
by marking the part \code{f} as not present in the type used for the struct. Setting the pointer to a non-null value 
changes the struct type to the normal type used for it in Cogent. In a similar way NULL pointers can be used in array
elements and in referenced values while the array or pointer has its empty-value type. In all these cases the pointer needs
no \code{MayNull} wrapper type, since the surrounding type already distinguishes between \code{NULL} values and valid values.

If the field \code{f} is initialized ``on demand'', i.e., not at a statical point in the program, or if not all elements
of an array are initialized together, this solution is not possible.
A null pointer can also be used as an ``error'' or ``escape'' value for function parameters or results. In all these cases
the \code{MayNull} marker type is required.

Since \code{MayNull} is an abstract type no predefined operations can be applied to its values. Gencot only defines abstract 
functions for
generating and testing the null pointer. Gencot provides the following generic abstract data type in \code{include/gencot/MayNull.cogent}.
It is available for all linear Cogent types, not only for types generated by Gencot by mapping a C pointer type.
\begin{verbatim}
  type MayNull a 
  null:      all(a:<E). () -> MayNull a
  roNull:    all(a). () -> (MayNull a)!
  mayNull:   all(a:<E). a -> MayNull a
  roMayNull: all(a). a! -> (MayNull a)!
  notNull:   all(a). MayNull a -> Option a
  roNotNull: all(a). (MayNull a)! -> (Option a)!
\end{verbatim}
The function \code{null} returns the null pointer, the function \code{mayNull} casts a non-null pointer of type \code{a}
to type \code{MayNull a}. For both the type \code{a} must be escapeable, otherwise it would be possible to wrap a value
of type \code{P!} which cannot escape from a banged context as a value of type \code{MayNull (P!)} which can escape.
The operations \code{roNull} and \code{roMayNull} must be used for readonly pointers, they return a result which is again 
readonly. This cannot be done 
by applying the bang operator to the result of \code{null} or \code{mayNull} since then the readonly result value
cannot escape the banged context. 

The type \code{Option} is used from the Cogent standard library. It is preferred over type \code{Result} because being null
is not interpreted as an error here.
The function \code{notNull} returns \code{None} if the argument is null and \code{Some x} if the argument \code{x} is not null.
The function \code{roNotNull} does the same for a readonly argument. 
Since \code{notNull} and \code{roNotNull} are the only functions which make the value available as a value of type \code{a} 
it is guaranteed by the Cogent type constraints that all accesses to the value are guarded by one of these two functions
(if no other abstract functions are introduced which convert from \code{Maynull a} to \code{a}).

As usual,
the type parameter \code{a} cannot be restricted by Cogent to linear types. However, Gencot provides instances of the functions only for
linear types \code{a}.

Based on the abstract functions the function
\begin{verbatim}
  isNull: all(a). (MayNull a)! -> Bool
\end{verbatim}
is defined for an explicit test for the null pointer. It could be implemented in Cogent based on function \code{roNotNull} but
is implemented in C for efficiency reasons, avoiding the intermediate use of a value of type \code{Option a}.

\subsubsection{General Operations}

Values of type \code{MayNull a} cannot be created, disposed, initialized, or cleared. Either they are null, or they are pointers of
a valid-value pointer type, for which the type-specific functions for creating, initializing, cearing, and disposing are available.

The type constructors for modification function (see Section~\ref{design-operations-modify}) can be applied to \code{MayNull a} 
with the usual intended semantics.

\subsubsection{Part Access Operations}

Conceptually, the type \code{MayNull a} can be seen as a structured value with the non-null pointer of type \code{a} being
an optional ``part''. Then the operations for accessing parts of a structured value can be defined for \code{MayNull a} as follows
The operation \code{get} corresponds to \code{roNotNull}. The operation \code{set} cannot be defined, since the type \code{a} of the 
``part'' is always linear and thus not discardable. The operations \code{getref} and \code{modref} cannot be defined, since
the ``part'' is identical to the container and need not be stored somewhere on the heap. 
All four operations are not provided by Gencot. The other two operations are provided
by the polymorphic functions
\begin{verbatim}
  exchngNull: all(a:<E).
     ModFun (MayNull a) a a
  modifyNull: all(a:<E,arg).
     ModPartFun (MayNull a) a arg arg
  modifyNullDflt: all(a:<E,arg:<D,out:<DSE).
     ModPartFun (MayNull a) a arg out
\end{verbatim}
Again, instances of the functions are only defined if \code{a} is a linear type. Additionally it must be escapeable, since
otherwise no values of type \code{Maynull a} can be constructed. The operations are not supported if \code{a} is not 
escapeable, since then the wrapped type would be \code{(MayNull a)!} and modifying its values would contradict the 
meaning of the bang operator. Like \code{isNull} the functions could be 
implemented in Cogent but are implemented in C for efficiency.

The function \code{exchngNull (mn, p)} takes as input two pointers, the first of which may be null. If \code{mn} is not null 
the result is the pair of \code{mayNull p} and the non-null pointer corresponding to \code{mn}.

The functions \code{modifyNull (mn, (modfun,addinput))} and \code{modifyNullDflt (mn, (modfun,addinput))} are
alternatives for operation \code{modify} as described in Section~\ref{design-operations-parts}.
If \code{mn} is not null it modifies the object referenced by \code{mn} by applying the part modification function \code{modfun} of type 
\code{ModFun a arg out} to \code{mn} and returns \code{mn} and the additional result of type \code{out}. Note that since here
the ``part'' is identical with the ``whole'', to modify the whole a modification function must be applied to the part instead of
a change function. Therefore the functions have type \code{ModPartFun} instead of \code{ChgPartFun}.

The case where the \code{MayNull a} value is null corresponds to the case where the part does not exist. According to the description in
Section~\ref{design-operations-parts} the function \code{roNotNull} corresponds to function \code{get} with a variant type as result.
Here a more efficient alternative passing the result value immediately is not possible, since in the null case a valid pointer must
be passed, which is not available. Therefore the typical C pattern of dereferencing a pointer without testing for null, because
we know from the context that it is not null, cannot be transferred directly to Cogent. However, a similar effect can be achieved
using type \code{a} for which it is guaranteed by the Cogent type system that it is a non-null pointer and which is always dereferenced
without testing it for null. 

Function \code{exchngNull} behaves in the null case as described in Section~\ref{design-operations-parts}, it returns its input \code{(mn,p)}.
Function \code{modifyNull} corresponds to the third case described in Section~\ref{design-operations-parts}. It uses the same type
for the additional input and result and returns the input to the part modification function as output when the part modification 
function is not used. This is the most general solution, therefore it is preferred over the other two. Since it restricts the function
to be used as part modification functions, the alternative \code{modifyNullDflt} is provided which discards the additional argument
and returns the default value \code{defaultVal[out]()} as additional result, however it is restricted in its types \code{arg} and
\code{out} as usual.

\subsection{Record Types}
\label{design-operations-record}

As described in Section~\ref{design-types-struct} C struct types are always mapped to Cogent record types.
Additional record types may be introduced manually in the translated Cogent program.

\subsubsection{Creating and Disposing Records}

The empty-value type corresponding to a record type \code{R} is the type \code{R take (..)} where all fields are taken,
which may be denoted by \code{EVT(R)} (see Section~\ref{design-operations-create}).

For every record type \code{R} used in the Cogent program (not only those generated by mapping a C type) Gencot automatically
generates instances of the functions \code{create} and \code{dispose} of the form:
\begin{verbatim}
  create[EVT(R)]
  dispose[EVT(R)]
\end{verbatim}

\subsubsection{Modifying Records}

The type constructors \code{ModFun}, \code{ModTypeFun}, \code{ChgPartFun}, and \code{ModPartFun} can be applied to record types with the 
usual intended semantics. 

A specific kind of modification functions for records are functions which put or take a record field. 
Gencot defines macros in \code{include/gencot/CStruct.cogent} to generate the type of simple put and take operations for
record types.

A macro call of the form 
\begin{verbatim}
  PUTFUN(R,f,A,O)
\end{verbatim}
expands to the type
\begin{verbatim}
  ModTypeFun (R take f) R A O
\end{verbatim}
for a function which puts field \code{f} in a record of type \code{R}. A macro call of the form
\begin{verbatim}
  TAKEFUN(R,f,A,O)
\end{verbatim}
expands to the type
\begin{verbatim}
  ModTypeFun R (R take f) A O
\end{verbatim}
for a function which takes field \code{f} in a record of type \code{R} (without returning the taken value).

A macro call of the form 
\begin{verbatim}
  PUTFUN<n>(R,(f1,...,fn),f,A,O)
\end{verbatim}
expands to the type
\begin{verbatim}
  ModTypeFun (R take (f,f1,...,fn)) (R take (f1,...,fn)) A O
\end{verbatim}
for a function which puts field \code{f} while the fields \code{fi} are already taken. A macro call of the form
\begin{verbatim}
  TAKEFUN<n>(R,(f1,...,fn),f,A,O)
\end{verbatim}
expands to the type
\begin{verbatim}
  ModTypeFun (R take (f1,...,fn)) (R take (f,f1,...,fn)) A O
\end{verbatim}
for a function which takes field \code{f} while the fields \code{fi} are already taken.

Gencot does not automatically define modification functions for record types, since a record can be modified using the
Cogent get and put operations for its fields. Only if a field of unboxed record type is modified in-place through a 
modification function for its context, the record fields
must be manipulated using modification functions for the record. Then such operations must be defined manually.

\subsubsection{Initializing and Clearing Records}

Before a record returned by \code{create} can be used it must be initialized by putting values for all fields. 
Dually, before disposing a record it must be cleared by taking all fields.

All instances of the functions \code{initFull/Heap/Simp} and \code{clearFull/Heap/Simp} described in 
Section~\ref{design-operations-init} are available
for record types. Functions \code{initFull} and \code{clearFull} pass a value of the corresponding
unboxed record type. Functions \code{initSimp} and \code{clearSimp} set all fields to
their default value, ignoring them upon clearing. Functions \code{initHeap} and \code{clearHeap} 
are required if the record contains fields of linear type, for allocating or deallocating values for them on the heap.

Functions
for initializing and clearing a record \code{r} of type \code{R} by treating the fields differently can be manually 
defined in Cogent by putting or taking values into/from all fields.

If fields of linear type or of unboxed record or abstract type should be initialized or cleared using specific initialization
or clearing functions, these functions can either be specified explicitly in the code or they can be passed as additional
parameter to the initialization or clearing function. 

A field \code{f} with an unboxed record type \code{\#S} corresponds to an embedded struct in C. The space for this struct is allocated together
with the space for \code{r} by function \code{create}, so it needs only be initialized. The \code{initFull} instance initializes 
it by writing the value in one assignment together with alll other fields. The \code{initHeap/Simp} instances instead pass a pointer 
to the embedded struct to corresponding \code{initHeap/Simp} instances for the part. 

When an initialization function for \code{R} is implemented
manually, it should be possible to do the same with an arbitrary initialization function of type \code{IniFun EVT(S) S arg out}
for field \code{f}. This is the same approach as for the operation \code{modref} described in Section~\ref{design-operations-modify}.
However, \code{modref} takes as argument a value of type \code{R}
where no field is taken, so it cannot be used to put the taken field \code{f} by initializing it. Separate modification functions
are required for the record with taken fields. If the record contains several embedded structs every modification function
initializes one field and the result type has one field less taken. Thus, defining such modification functions imposes an order
in which the embedded structs must be initialized and whether they are initialized before or after the other fields. A corresponding
modification function for field \code{f} which must be initialized before field \code{g} but after all other fields would be
\begin{verbatim}
  putFInR : all(arg,out). 
    PUTFUN1(R,(g),f,(IniFun EVT(S) S arg out,arg),out)
\end{verbatim}
It is invoked with the initialization function for the embedded record and its argument 
as additional information. In a similar way a field of unboxed record type \code{\#S} can be cleared in-place using an abstract
function of the form
\begin{verbatim}
  takeFInR : all(arg,out).
    TAKEFUN1(R,(g),f,(ClrFun S EVT(S) arg out,arg),out)
\end{verbatim}
which applies a clearing function of type \code{ClrFun S EVT(S) arg out} and takes the field \code{f} while field \code{g} 
is already taken. Of course, instead of polymorphic functions for all types \code{arg} and \code{out} the specific forms
corresponding to \code{initFull}, \code{initHeap}, or \code{initSimp} can be defined.

Another approach to record initialization could have been that the function \code{create[EVT(R)]} returns a value where 
only the fields of primitive and linear type
are taken and the embedded structs are present, but again with all primitive and linear fields taken. Then the initialization function for
field \code{f} could be directly applied to a pointer to field \code{f}. However, it is still necessary to retrieve the pointer to \code{f}
with the help of an abstract function which now has to respect the fact that other embedded record fields have types with
some fields taken. So the same number of additional abstract functions is needed as in the approach above and their argument
types are of similar complexity.

Initializing a field \code{f} with a function pointer type \code{\#C(FunPtr Funtyp)} can always be done using the Cogent put operation.
The value to be put must be constructed using the function \code{toFunPtr} (see Section~\ref{design-operations-function}). To
avoid null pointers to functions, as described in Section~\ref{app-transfunction-pointer}, a dummy function should be defined in
Cogent and passed to the \code{toFunPtr} function.

\subsubsection{Accessing Record Fields}

The parts of a record are its fields. The conceptual operations for accessing parts of a structured value can be defined for a record type 
\code{R} as follows.

The field to be accessed by the operation must be specified by its name. This can only be done as part of the function name.
Therefore Gencot does not define polymorphic functions for accessing arbitrary fields of arbitrary record types. For every record type
\code{R} a set of differently named functions must be defined for every field \code{f}. Since only some of the functions are needed, Gencot
does not automatically generate such functions, they must be defined manually, if required.

The access functions can be defined as polymorphic functions in respect to the record type. Then, for every field name
\code{f} there is a polymorphic function which can be used for all records with a field named \code{f}. Additionally, the
function must be polymorphic in the field type, so that it supports fields of different type. The resulting 
functions have the following forms
\begin{verbatim}
  getFld_f : all(rec,fld). rec! -> fld!
  setFld_f : all(rec,fld:<D). ModFun rec fld ()
  exchngFld_f : all(rec,fld). ModFun rec fld fld
  modifyFld_f : all(rec,fld,arg,out). ChgPartFun rec fld arg out
\end{verbatim}
where \code{rec} is the type of the record and \code{fld} is the type of the field.

The operation \code{getFld\_f} for a field \code{f} corresponds to the Cogent member access operation \code{r.f}.
Since the \code{get} operation is not passed as argument to other functions, it need not be defined for record fields, it is always possible
to use the Cogent member access operation instead.

The operation \code{setFld\_f} for a field \code{f} and a value \code{v} corresponds to the Cogent put operation r{f = v}, if
the field has a discardable type. Otherwise it is not supported.

The operation \code{exchngFld\_f} for a field \code{f} and a value \code{v} can be implemented in Cogent by first taking the 
old value of \code{f}, then putting \code{v} and finally returning the record together with the old value.

The operation \code{modifyFld\_f} for a field \code{f} and a field change function can be implemented in Cogent 
by first taking the value of \code{f}, then applying the change function to it, and finally putting the result back into
the field. 

All three operations may be passed as argument to other modification functions, then they must be defined and cannot be replaced
by a direct inline implementation in Cogent.

Defining the access functions as polymorphic in the record type has the drawback that they cannot be implemented in Cogent, 
although every single instance for a specific record type can be implemented in Cogent. So, an alternative approach is to
define the access functions for every record type \code{Rec} and every field \code{f} of \code{Rec} with type \code{Ftp} as
\begin{verbatim}
  getFld_F_InRec : Rec! -> Ftp!
  setFld_f_InRec : ModFun Rec Ftp ()
  exchngFld_f_InRec : ModFun Rec Ftp Ftp
  modifyFld_f_InRec : all(arg,out). ChgPartFun Rec Ftp arg out
\end{verbatim}
where \code{setFld\_f\_InRec} is only defined if \code{Ftp} is discardable.

The operations \code{getref} and \code{modref} cannot be implemented in Cogent, they must be defined as abstract function which are implemented in 
C with the help of the address operator \code{\&}. For a field \code{f} the corresponding function definitions have the form
\begin{verbatim}
  getrefFld_f : all(rec,pfld). rec! -> pfld!
  modrefFld_f : all(rec,pfld,arg,out). ModPartFun rec pfld arg out
\end{verbatim}
where \code{pfld} is the Gencot mapping of the type of pointer to the field type. 

Since whenever an instance of these functions is defined, the field is safely existing, the case of the nonexisting
part needs not be respected.

Remember that according to the definition of 
\code{ModPartFun} the function \code{modrefFld\_f} is invoked in the form
\begin{verbatim}
  (r',o) = modrefFld_f(r, (m,a))
\end{verbatim}
where \code{r} is a boxed record value which is not readonly, \code{m} is a modification function of type \code{ModFun pfld arg out} which
has the form \code{(pfld,arg) -> (pfld,out)}, \code{a} is the argument of type \code{arg} passed to \code{m}, \code{r'} is
the record value \code{r} after the modification, and \code{o} is the additional result of type \code{out} returned by \code{m}.

\subsection{Array Types}
\label{design-operations-array}

Gencot represents arrays of known size in Cogent always by a (boxed) type \code{CArr<size> El} which is a record 
type wrapping the array (see Section~\ref{design-types-array}).
We call these types ``Gencot array types'' here.

Gencot array types are complemented by abstract polymorphic functions for working with
arrays. For these functions Gencot automatically generates the instance 
implementations in C. All these implementations involve the array size in some way. The developer may manually introduce
additional Gencot array types by adhering to the Gencot naming schema for array types (see Section~\ref{design-types-array}). 
Then Gencot also supports these types and generates instances of all polymorphic functions for them.

If the array size is unknown (types of the form \code{CArrXX El})
no instances of the polymorphic functions are provided by Gencot. However, implementations may be provided manually. Often
this is possible by the developer determining the array size from the context. If several array types with 
different unknown sizes have been mapped to the same abstract type, they must be disambiguished manually.

Gencot only supports arrays which are allocated globally or on the heap. In C, arrays can also be introduced by defining
them as local variable, but Cogent has no language constructs which support this.

\subsubsection{Creating and Disposing Arrays}

Cogent does not provide a language construct to create or dispose values for Gencot array types.

Since Gencot array types are implemented as Cogent records, the empty-value type corresponding to type \code{CArr<size> El} 
is generated by the macro call \code{EVT(CArr<size> El)} as 
usual.

For every Gencot array type \code{CArr<size> El} used in the Cogent program Gencot automatically
generates instances of the functions \code{create} and \code{dispose} of the form:
\begin{verbatim}
  create[EVT(CArr<size> El)]
  dispose[EVT(CArr<size> El)]
\end{verbatim}

Both functions also support multidimensional array types of the form \code{CArr<size1> \#(CArr<size2> El)}.

\subsubsection{Modifying Arrays}

If an array is seen as a structured value, its elements are its parts.

If Cogent builtin array types are used for mapping C array types, the general case for accessing array elements must be 
performed using abstract functions. Only if the index value is statically known the builtin access operations can be used.
If Cogent builtin array types are not used the actual array type is always abstract, then all accesses to array elements 
must be performed using abstract functions. Gencot uses only abstract functions in both cases, in the current version
it does not try to optimize accesses using statically known indexes for builtin array types.

The type constructors for modification functions (see Section~\ref{design-operations-modify}) can be applied to 
Gencot array types with the usual intended semantics. 

Gencot provides no equivalent for single Cogent take and put operations for array elements. It would be
necessary to statically encode the set of array indices for which the elements have been taken in the type expression,
this is not feasible (although Cogent builtin array types support it).

\subsubsection{Initializing and Clearing Arrays}

An array initialization function sets a valid value for every element. An array clearing function clears every element, 
clearing and disposing all linear values contained in elements. Since all elements are of the same type, they can all be
treated in the same way by an initialization or clearing function for the element type. Array elements are similar to
an embedded struct: the element value is stored in-place in the memory region used for the array. Hence, initializing and 
clearing an element can also be done in-place by passing a pointer to the element to the element initialization or 
clearing function. As usual, an additional input given to the array function is passed through to every invocation 
of the element function. 

All instances of the functions \code{initFull/Heap/Simp} and \code{clearFull/Heap/Simp} described in 
Section~\ref{design-operations-init} are available
for Gencot array types. Functions \code{initFull} and \code{clearFull} take or return the complete array content
as an unboxed value of type \code{\#(CArr<size> El)}. Functions \code{initSimp} and \code{clearSimp} set all elements
to their default value, ignoring them upon clearing. Functions \code{initHeap} and \code{clearHeap} 
are required, if the element type is linear, for allocating or deallocating it on the heap.

For example an array \code{a} of type \code{EVT(CArr16 U32)} can be initialized by setting all elements to \code{0} 
(the \code{defaultVal[U32]()}) with typechecks by the Cogent expression
\begin{verbatim}
  INIT(Simp,CArr16 U32) (a,())
\end{verbatim}
and an array \code{a} of pointers to integers with Cogent type \code{CArr16 (CPtr U32)} can be cleared with typechecks 
by the Cogent expression
\begin{verbatim}
  CLEAR(Heap,CArr16 (CPtr U32)) (a,heap)
\end{verbatim}
which will clear and dispose all elements.

Additionally, Gencot provides operations where the user can specify how to initialize or clear a single element, which is then
applied to all elements. The specification for a single element is done by passing an initialization or clearing function which
can be applied to a pointer to the element. Gencot defines the corresponding polymorphic abstract functions
\begin{verbatim}
  initEltsParCmb: all(evt,vvt,epe,vpe,arg:<S,out). 
     IniFun evt vvt (IniFun epe vpe arg out, arg, (out,out)->out) out
  clearEltsParCmb: all(vvt,evt,vpe,epe,arg:<S,out). 
     ClrFun vvt evt (ClrFun vpe epe arg out, arg, (out,out)->out) out
  initEltsPar: all(evt,vvt,epe,vpe,arg:<S). 
     IniFun evt vvt (IniFun epe vpe arg (), arg) ()
  clearEltsPar: all(vvt,evt,vpe,epe,arg:<S). 
     ClrFun vvt evt (ClrFun vpe epe arg (), arg) ()
  initEltsSeq: all(evt,vvt,epe,vpe,arg). 
     IniFun evt vvt (IniFun epe vpe arg arg, arg) arg
  clearEltsSeq: all(vvt,evt,vpe,epe,arg). 
     ClrFun vvt evt (ClrFun vpe epe arg arg, arg) arg
\end{verbatim}
where \code{epe} is the empty-value type for a pointer to an element and \code{vpe} is the corresponding valid-value type.
The first pair of functions passes the additional input of type \code{arg} in parallel to every invocation of the element function, 
therefore it must be sharable. The outputs of type \code{out} are combined using the function passed as third part of
the additional input to \code{initEltsParCmb} or \code{clearEltsParCmb}. It is applied by ``folding'', starting at the first element.
The second pair uses an element function which does not return a result and needs no combination function.
The third pair of functions passes the additional result of every invocation of the element function as additional input 
to the invocation for the next element. Thus it must have a common type, which may be linear, since it is neither shared nor
discarded. 

All three initialization functions are only defined for the case that no error can occur. In particular, the function
\code{initHeap} cannot be used as element initialization function. If this is required, a corresponding array initialization
function must be implemented manually.

For every array type \code{CArr<size> El} Gencot provides corresponding instances of all six functions.
In these instances all types besides \code{A} and \code{O} are uniquely determined by the array size and the element type, although
this cannot be expressed by Cogent type constraints. Therefore Gencot defines in \code{include/gencot/CArray.cogent}
the preprocessor macros
\begin{verbatim}
  INITelts(<k>,<ek>,<size>,El,A,O)
  CLEARelts(<k>,<ek>,<size>,El,A,O)
\end{verbatim}
which expand to the corresponding instance specifications shown above and
\begin{verbatim}
  INITTYPEelts(<k>,<ek>,<size>,El,A,O)
  CLEARTYPEelts(<k>,<ek>,<size>,El,A,O)
\end{verbatim}
which expand to their function types. Parameter \code{<k>} is either \code{ParCmb}, \code{Par}, 
or \code{Seq}. The additional parameter \code{<ek>} is required for technical reasons and specifies the
kind of the element type, as for macro \code{CPTR} (see Section~\ref{design-types-pointer}. 
It must be \code{U} if the element is an unboxed record type or an unboxed Gencot array type, and it
must be empty otherwise. If \code{<k>} is not \code{ParCmb} the last parameter \code{O} is ignored and may be empty.

For example an array \code{a} of type \code{EVT(CArr16 U32)} can be initialized by setting all elements to \code{5} 
with typechecks by the Cogent expression
\begin{verbatim}
  INITelts(Par,,16,U32,U32,) (a,(INIT(Full,CPtr U32),#{cont=5}))
\end{verbatim}

It could be useful to also know the element index in the element function. This could be supported by passing the index
as additional input to the element functions. However, this implies that the normal initialization and clearing functions
for the element type cannot be used as element functions here, since they do not expect the index as additional input.
Therefore the functions automatically supported by Gencot for initializing and clearing arrays do not pass the index 
to the element functions. However, using \code{initEltsSeq} or \code{clearEltsSeq}, the index can be calculated by
passing it from one invocation of the element function to the next, counting it up in the element function.

\subsubsection{Accessing Array Elements}

All operations for accessing parts of a structured value are nontrivial for the elements of an array. Since elements are 
specified by an index value a specified element need not exist, this case must be handled by the part access operations.

Gencot provides polymorphic functions for all six operations with alternatives for the case where the element does not 
exist:
\begin{verbatim}
  getArr : all(arr,idx,el). (arr!,idx) -> el!
  getArrChk : all(arr,idx,el). (arr!,idx) -> Result el! ()
  setArr : all(arr,idx,el:<D). ModFun arr (idx,el) ()
  exchngArr : all(arr,idx,el). ModFun arr (idx,el) (idx,el)
  modifyArr : all(arr,idx,el,arg). 
    ModFun arr (idx, ChgFun el arg arg, arg) arg
  modifyArrDflt : all(arr,idx,el,arg:<D,out:<DSE). 
    ModFun arr (idx, ChgFun el arg out, arg) out
  getrefArr : all(arr,idx,pel). (arr!,idx) -> pel!
  getrefArrChk : all(arr,idx,pel). (arr!,idx) -> Result pel! ()
  modrefArr : all(arr,idx,pel,arg). 
    ModFun arr (idx, ModFun pel arg arg, arg) arg
  modrefArrDflt : all(arr,idx,pel,arg:<D,out:<DSE). 
    ModFun arr (idx, ModFun pel arg out, arg) out
\end{verbatim}
The type variable \code{arr} denotes the array type, \code{idx} denotes the index type,
\code{el} denotes the array element type, and \code{pel} denotes the type of pointers to elements. 

Instances of these functions are only defined if \code{arr} is a Gencot array type. Type \code{idx}
must be one of \code{U8, U16, U32, U64} according to the \code{<size>} of the array. Types \code{el} and
\code{pel} must be as determined by \code{arr}. \code{arg} and \code{out} may be arbitrary types
only constrained as specified above.

Function \code{getArr} retrieves the indexed element as readonly. If the element does not exist because
the specified index is not in the range \code{0..<size>-1} the function returns the element with 
index \code{0} which always exists since Gencot array types must have atleast one element. Note that
this solution is more general than returning \code{defaultVal} since it works for arbitrary element types.
As alternative, as described in Section~\ref{design-operations-parts}, the function
\code{getArrChk} returns a variant value using the value \code{Error()} if the element does not exist.
This function should be used whenever it cannot be proven that the index is valid.

Function \code{setArr}, as usual, is only defined for a discardable element type. 
It simply discards the old value of the indexed element and sets it to the specified value.

Function \code{exchngArr}
replaces the element at the specified position by the element passed as parameter and returns the old element in the result.
The index is always part of the result so that the additional output has the same type as the additional input,
so that \code{exchngArr} can be used as element function also for \code{modifyArr} and \code{modrefArr}.

If the specified
index is not in the range \code{0..<size>-1} both functions work as described in Section~\ref{design-operations-parts}:
\code{setArr} returns the unmodified array and \code{exchngArr} returns its unmodified input.

Function \code{modifyArr} changes the element at the specified position by applying the element change function
to it. If the specified
index is not in the range \code{0..<size>-1} the function returns the unmodified array together with the additional input
value for the element change function. This corresponds to the third solution described in Section~\ref{design-operations-parts}
and requires that the types of additional input and result are the same. Since this may prevent the use of some element
change functions an alternative is provided by the function \code{modifyArrDflt} which discards the additional
input and returns the default value \code{defaultVal[out] ()}. This allows different types for input and result, however,
the required constraints apply to these types.

Function \code{getrefArr} returns a pointer to the element in the array without copying the element. This is safe since 
both the array and the result type are readonly. If the element does not exist it behaves like \code{getArr}, returning 
a pointer to the element with index \code{0}. The alternative function \code{getrefArrChk} is provided and returns the 
corresponding variant value.

Functions \code{modrefArr} and \code{modrefArrDflt} work like \code{modifyArr} and \code{modifyArrDflt} but use an
element modification function and pass a pointer to the element to it, so that the element is modified in-place.

The types of \code{modifyArr}, \code{modifyArrDflt}, \code{modrefArr}, and \code{modrefArrDflt} are similar to a 
\code{ChgPartFun} or \code{ModPartFun}, but differ because in addition to the 
element function and its argument the element index must be passed as argument.

Note that it is not possible
to pass the array or parts of it as additional information in the parameter of type \code{arg} of the element
functions, since that would always be
a second use of the array of linear type. If several elements must be modified together, a specific modification function
must be defined and used instead of \code{modifyArr} or \code{modrefArr}.

For multidimensional array types of the form \code{CArr<n1> \#(CArr<n2> ...)} the element type is again an (unboxed) array type.
When \code{getrefArr} and \code{modrefArr} are used for the access in the outer array all access functions described here
can be used for the access in the inner array.

\subsubsection{Working with Unboxed Arrays}

For record types Cogent provides all operations required to work with unboxed values. For Gencot array types this is not the case. 
If Cogent builtin array types are used for mapping C array types, element access is restricted to statically known index values,
otherwise the array type is abstract and no operations are provided by Cogent. Therefore Gencot also defines and implements abstract
functions for working with unboxed Gencot array types.

The functions
\begin{verbatim}
  toArr : all(tup,arr). tup -> arr
  fromArr : all(tup,arr). arr -> tup
\end{verbatim}
convert between tuples and unboxed arrays. The functions are only defined if type \code{arr} is an unboxed Gencot
array type of size \code{n} and element type \code{el}, and \code{tup} is a Cogent tuple type with \code{n} components 
of type \code{el}. 

Function \code{toArr} constructs an unboxed array value from a tuple value, function \code{fromArr} returns the 
tuple of all element values. Both functions can also be used for a linear element
type \code{el}, since no elements are shared or discarded.

For example, an unboxed array of 3 elements of type \code{U16} can be constructed by
\begin{verbatim}
  toArr[(U16,U16,U16),#(CArr3 U16)] (1,2,3)
\end{verbatim}

Gencot does not support Gencot array types with taken elements other than empty-value types for boxed arrays. Therefore no
initialization or clearing operations are applicable to unboxed Gencot array types.

As for boxed Gencot array types Gencot only supports functions for accessing single elements of unboxed arrays. It supports the 
functions \code{getArr}, \code{getArrChk}, \code{setArr}, \code{exchngArr}, \code{modifyArr}, and \code{modifyArrDflt}.
The latter four functions return a modified copy of the array. As described in Section~\ref{design-operations-modify} 
the \code{ModFun} type is interpreted accordingly for unboxed values. Note that
an unboxed array may still be of linear type, if the elements are of linear type. Then all the type restrictions
defined for the access functions are also required for the unboxed arrays. Together with these restrictions the semantics
of these functions guarantees that elements of linear type are neither shared nor discarded.

The \code{getref} and \code{modref} functions are not supported. They would return or pass a pointer to an element.
Since unboxed Gencot arrays are stack allocated these would be pointers into the stack which are not supported by the 
Isabelle C parser.

\subsection{Explicitly Sized Arrays}
\label{design-operations-eearray}

A common pattern in C programs is to explicitly use a pointer type instead of an array type for referencing an array,
in particular if the array is allocated on the heap. This is typically done if the number of elements in the
array is not statically known at compile time. The C concept of variable length array types can sometimes be
used for a similar purpose, but is restricted to function parameters and local variables and cannot be used for
structure members.

In C the array subscription operator can be applied to terms of pointer type \code{p} in the form \code{p[i]}. 
The semantics is to access an element in memory at the specified offset after the element referenced by the pointer. 
To use this in a safe way, the number of array elements must be known. 

Gencot provides support for this kind of working with arrays, if the number of elements is specified explicitly as
an integer value. The element pointer and the element size are combined to a pair. The corresponding type is called
an ``explicitly sized array'' type. Gencot uses the generic type \code{CArrES}
to construct types for explicitly sized arrays, such as \code{CArrES PEl} for an explicitly sized array with elements
of type \code{El}, where \code{PEl} is the type of pointers to the elements. The corresponding type definition is
\begin{verbatim}
  type CArrES pel = (pel, U64)
\end{verbatim}

Since the type argument must be the pointer-to-element type Gencot provides in \code{include/gencot/ESArray.cogent}
the preprocessor macro
\begin{verbatim}
  CAES(<ek>,el)
\end{verbatim}
where \code{<ek>} describes the kind of element type as for the macro \code{CPTR} (see Section~\ref{design-types-pointer}).

For every instance of type \code{CArrES} Gencot provides operations as described in the following sections.

\subsubsection{Converting Arrays}

Gencot provides support for converting array values between explicitly sized arrays and the fixed size arrays of types
\code{CArr<size> El}. It supports the polymorphic abstract functions
\begin{verbatim}
  toCAES: all(arr,aes). arr -> aes
  fromCAES: all(aes,arr). aes -> Result arr aes
  rotoCAES: all(arr,aes). arr! -> aes!
  rofromCAES: all(aes,arr). aes! -> Result arr! aes!
\end{verbatim}
where \code{arr} is the fixed size array type and \code{aes} is the explicitly sized array type. Gencot provides instances
for all cases where \code{arr} and \code{aes} have the same element type. Since this contraint cannot be represented in
Cogent, Gencot provides the following preprocessor macros in \code{include/gencot/ESArray.cogent}:
\begin{verbatim}
  TOCAES(<size>,<ek>,El)
  FROMCAES(<size>,<ek>,El)
  ROTOCAES(<size>,<ek>,El)
  ROFROMCAES(<size>,<ek>,El)
\end{verbatim}
where \code{<size>} is the number of elements in the fixed size array type, \code{<ek>} is the kind of element type as above, 
and \code{El} is the element type. As example, the macro call \code{TOCAES(8,,U32)} expands to
\begin{verbatim}
  toCAES[CArr8 U32,CArrES (CPtr U32)]
\end{verbatim}
and the macro call \code{ROFROMCAES(5,,CPtr U16)} expands to
\begin{verbatim}
  rofromCAES[CArrES (CPtr (CPtr U16)),CArr5 (CPtr U16)]
\end{verbatim}

Function \code{toCAES} returns the pair of the pointer to the first array element and the constant known size, this
always succeeds. Function \code{fromCAES} compares the explicit size in the argument to the known fixed size of the result.
If they are equal it returns the element pointer, otherwise it returns an error value with the original input.

Whenever a pointer \code{p} to an element is given in a program and an expression \code{s} is known which specifies the corresponding
element size, the corresponding explicitly sized array value can be constructed manually as \code{(p,s)} in Cogent.

\subsubsection{Creating and Disposing Explicitly Sized Arrays}

The empty-value type corresponding to an explicitly sized array type \code{(PEl,U64)} is the type \code{(EVT(PEl),U64)}.
For specifying this type Gencot provides the preprocessor macro
\begin{verbatim}
  EVT_CAES(<ek>,El)
\end{verbatim}
where parameter \code{<ek>} is as described for \code{CAES} above.

For allocating an explicitly sized array the size must be specified, since it is not known as part of the type.
Gencot provides the polymorphic abstract functions
\begin{verbatim}
  createCAES: all(evp). (U64,Heap) -> Result ((evp,U64),Heap) Heap
  disposeCAES: all(evp). ((evp,U64),Heap) -> Heap
\end{verbatim}
for all cases where \code{evp} is a type of the form \code{EVT(PEl)}, i.e., an empty-value type for pointers to the 
array elements.

\subsubsection{Modifying Explicitly Sized Arrays}

Explicitly sized arrays can be modified in the same way as fixed size arrays. In particular, 
all type constructors for modification functions (see Section~\ref{design-operations-modify}) can be applied to 
explicitly sized array types with the usual intended semantics. 

\subsubsection{Initializing and Clearing Explicitly Sized Arrays}

Since the array size is known at runtime, all conceptual initialization and clearing operations defined for fixed
size arrays can also be provided for explicitly sized arrays. These are the general operations defined in 
Section~\ref{design-operations-init} and the array-specific operations defined in Section~\ref{design-operations-array}.

The functions \code{initFull} and \code{clearFull} are not supported, since they are defined using the unboxed type \code{\#vvt}
to represent the content. For an explicitly sized array type the unboxed type is the same tuple type. Thus \code{initFull} has not the
intended semantics of moving data from the stack to the heap. The other functions work as described.

Since the way how the empty-value type is constructed is different for explicitly sized arrays, the macros defined in
Section~\ref{design-operations-init} cannot be used. Gencot instead provides the macros
\begin{verbatim}
  INIT_CAES(<k>,<ek>,El)
  CLEAR_CAES(<k>,<ek>,El)
  INITTYPE_CAES(<k>,<ek>,El)
  CLEARTYPE_CAES(<k>,<ek>,El)
\end{verbatim}
where \code{<k>} is one of \code{Heap} or \code{Simp}, and \code{<ek>} and \code{El} are as for \code{CAES} above.

The functions \code{initEltsParCmb}, \code{clearEltsParCmb}, \code{initEltsPar}, \code{clearEltsPar}, \code{initEltsSeq}, 
\code{clearEltsSeq} are all supported for explicitly sized array types \code{vvt}. Gencot provides the macros
\begin{verbatim}
  INITelts_CAES(<k>,<ek>,El,A,O)
  CLEARelts_CAES(<k>,<ek>,El,A,O)
  INITTYPEelts_CAES(<k>,<ek>,El,A,O)
  CLEARTYPEelts_CAES(<k>,<ek>,El,A,O)
\end{verbatim}
corresponding to \code{INITelts}, \code{CLEARelts}, \code{INITTYPEelts}, and \code{CLEARTYPEelts}. As for them
\code{<k>} is one of \code{ParCmb}, \code{Par}, or \code{Seq} and \code{A} and \code{O} are arbitrary types for the 
additional input and output of the element function. \code{O} is ignored if \code{<k>} is not \code{ParCmb}.

\subsubsection{Accessing Array Elements}

All element access functions defined for fixed size arrays in Section~\ref{design-operations-array} are also supported for 
explicitly sized arrays, i.e., their type variable \code{arr} can be an explicitly sized array type. The index type \code{idx}
can be as described (one of U8, U16, U32, U64).


\section{Processing C Declarations}
\label{design-decls}

A C declaration consists of zero or more declarators, preceded by information applying to all declarators together.
Gencot translates every declarator to a separate Cogent definition, duplicating the common information as needed.
The Cogent definitions are generated in the same order as the declarators.

A C declaration may either be a \code{typedef} or an object declaration. A typedef can only occur on toplevel or 
in function bodies in C.
For every declarator in a toplevel typedef Gencot generates a Cogent type definition at the corresponding position. Hence
all these Cogent type definitions are on toplevel, as required in Cogent. Typedefs in function bodies are not
processed by Gencot, as described in Section~\ref{design-fundefs}.

A C object declaration may occur 
\begin{itemize}
\item on toplevel (called an ``external declaration'' in C),
\item in a struct or union specifier for declaring members,
\item in a parameter list of a function type for declaring a parameter,
\item in a compound statement for declaring local variables.
\end{itemize}

External declarations are simply discarded by Gencot. In Cogent there is no corresponding concept, it is not needed
since the scope of a toplevel Cogent definition is always the whole program. 

Compound statements in C only occur 
in the body of a function definition. Declarations embedded in a body are processed when the body is translated. They
are translated to a binding of a local variable to the value specified by the initializer or to a default value (see
Section~\ref{design-cstats}).

Union specifiers are always translated to abstract types by Gencot, hence declarations for union members are
never processed by Gencot.

The remaining cases are struct member declarations and function parameter declarations. 
For every declarator in an object declaration, Gencot generates a Cogent record field definition, if the C declaration
declares struct members, or it generates a tuple field definition, if the C declaration declares a function parameter.

\subsection{Target Code for struct/union/enum Specifiers}
\label{design-decls-tags}

Additionally, whenever a struct-or-union-specifier or enum-specifier occurring in the C declaration has a body and
a tag, a Cogent type definition is generated for the corresponding type, since it may be referred in C by its tag from
other places. A C declaration may contain atmost one 
struct-or-union-specifier or enum-specifier directly. Here we call such a specifier the ``full specifier'' of 
the declaration, if it has a body. 

Since Cogent type definitions must be on toplevel, Gencot defers it to the next possible toplevel position after the
target code generated from the context of the struct/union/enum declaration. If the context is a typedef, it is placed
immediately after the corresponding Cogent type definition. If the typedef contains several full specifiers (which
may be nested), all corresponding Cogent type definitions are positioned on toplevel in the order of the beginnings
of the full specifiers in C (which corresponds to a depth-first traversal of all full specifiers).

If the context is a member declaration in a struct-or-union-specifier, the Cogent type definition is placed after that
generated for its context. 

If the context is a parameter declaration it may either be embedded in a function definition or in a declarator of another
declaration. Function definitions in C always occur on toplevel, the Cogent type definitions for all struct/union/enum
declarations in the parameter list are placed after the target code for the function definition (which may be unusual for 
manually written Cogent code, but it is easier to generate for Gencot). In all other cases the Cogent type definitions
for struct/union/enum declarations in a parameter list are treated in the same way as if they directly occur in the surrounding
declaration.

Note, that a struct/union/enum tag declared in a parameter list has only ``prototype scope'' or ``block scope'' which
ends after the function type or definition. Gencot nevertheless generates a toplevel type definition for it, since the
tag may be used several times in the parameter list or in the corresponding body of a function definition. Note that 
this may introduce name conflicts, if the same tag is declared in different parameter lists. Since declaring tags in a 
parameter list is very unusual in C, Gencot does not try to solve these conflicts, they will be detected by the Cogent
compiler and must be handled manually.

A full specifier without a tag can only be used at the place where it statically occurs in the C code, however, it
may be used in several declarators. Therefore Gencot also generates a toplevel type definition for it, with an 
introduced type name as described in Section~\ref{design-names}.

\subsection{Relating Comments}

A declaration is treated as a structured source code part. The subparts are the full specifier, if present, and all
declarators. Every declarator includes the terminating comma or semicolon, thus there is no main part code between or after
the declarators. The specifiers may consist of a single full specifier, then there is no main part code at all.

The target code part generated for a declaration consists of the sequence of target code parts generated for the declarators,
and of the sequence of target code parts generated for the full specifier, if present. No target code is generated for the 
main part itself. In both sequences the subparts are 
positioned consecutively, but the two sequences may be separated by other code, since the second sequence consists of 
Cogent type definitions which must always be on toplevel. 

According to the rules defined in Sectiom~\ref{design-comments-relate}, the before-unit of the declaration is put before
the target of the first subpart, which is that for the full specifier, if present, otherwise it is the target for the
first declarator. In the first case the comments will be moved to the type definition for the full specifier. The rationale
is that often a comment describing the struct/union/enum declaration is put before the declaration which contains it.

The after-unit of the declaration is always put behind the target of the last declaration.

A declarator may derive a function type specifying a parameter-type-list. If that list is not \code{void}, the 
declarator is a structured source code part with the parameter-declarations as embedded subparts. Every
parameter-declaration includes the separating comma after it, if another parameter-declaration follows,
thus there is no main part code between the parameter-declarations. The parentheses around the parameter-type-list
belong to the main part, thus a comment is only associated with a parameter if it occurs inside the parentheses.

In all other cases a declarator is an unstructured
source code part.

\subsection{Typedef Declarations}
\label{design-decls-typedefs}

For a C typedef declaration Gencot generates a separate toplevel Cogent type definition for every declarator.

For every declarator a C type is determined from the declaration specifiers together with the derivation specified
in the declarator. As described in Section~\ref{design-types}, either a Cogent type expression is determined from this C type,
or the Cogent type is decided to be abstract. 

The defined type name is generated from the C type name according to the mapping described in Section~\ref{design-names}.
Type names used in the C type specification are mapped to Cogent type names in the Cogent type expression in the same way.

\subsection{Object Declarations}

C object declarations are processed if they declare struct members or function parameters.

For such a C object declaration Gencot generates a separate Cogent field 
definition for every declarator. This is a named record field definition if the declaration is embedded in the
body of a struct-or-union-specifier, it is an unnamed tuple field definition if the declaration is embedded in the
parameter-type-list of a function type. In the first case declarators with function type are not allowed, in the
second case they are adjusted to function pointer type. In both cases the Cogent field type is determined from the 
declarator's C type as described in Section~\ref{design-types}. 

In the case of a named record field the Cogent
field name is determined from the name in the C declarator as described in Section~\ref{design-names}. In the 
case of an unnamed tuple field a name specified in the C parameter declaration is always discarded.

\subsection{Struct or Union Specifiers}

For a full specifier with a tag Gencot generates a Cogent type definition. The name of the defined type is generated
from the tag as described in Section~\ref{design-names}. For a union specifier the type is abstract, no defining type
expression is generated. For a struct specifier a (boxed) Cogent record type expression is generated, which has a field
for every declared struct member which is not a bitfield. Bitfield members are aggregated as described in 
Section~\ref{design-types-struct}. 

A specifier without a body must always have a tag and is used in C to reference the full specifier with the same tag.
Gencot translates it to the Cogent type name defined in the type definition for the full specifier.

Note that the Cogent type defined for the full specifier corresponds to the C type of a pointer to the struct or union, 
whereas the unboxed Cogent type corresponds to the C struct or union itself. This is adapted by Gencot when translating
the C specifier embedded in a context to the corresponding Cogent type reference.

\subsection{Enum Specifiers}

For a full enum specifier with a tag Gencot generates a Cogent type definition immediately followed by Cogent object
definitions for all enum constants. The name of the defined type is generated
from the tag as described in Section~\ref{design-names}. The defining Cogent type is always \code{U32}, as described in
Section~\ref{design-types-enum}. 

A specifier without a body must always have a tag and is used in C to reference the full specifier with the same tag.
Gencot translates it to the Cogent type name defined in the type definition for the full specifier.

 


\section{Processing C Function and Object Definitions}
\label{design-fundefs}

A C function definition is translated by Gencot to a Cogent function definition. Old-style C function definitions
where the parameter types are specified by separate declarations between the parameter list and the function body
are not supported by Gencot because of the additional complexity of comment association.

The Cogent function name is generated from the C function name as described in Section~\ref{design-names}.

The Cogent function type is generated from the C function result type and from all C parameter types as described
in Section~\ref{design-types-function}. In a C
function definition the types for all parameters must be specified in the parameter list, if old-style function
definitions are ignored.

\subsection{Function Bodies}
\label{design-fundefs-body}

In C the function body consists of a compound statement. In Cogent
the function body consists of an expression. Gencot partially translates the C code to Cogent code as described 
in Section~\ref{design-cstats}.

As described in Section~\ref{design-types-function} Gencot translates all C functions to Cogent functions with
a single parameter. If the C function has more than one parameter it is a tuple of the C parameters. To make it
possible to use the original C parameter names in the translated function body
Gencot generates a Cogent pattern for the parameter of the Cogent function which 
consists of a tuple of variables with the names generated from the C parameter names. As described in 
Section~\ref{design-names} the C parameter names are only mapped if they are uppercase, otherwise they are
translated to Cogent unmodified. Since it is very unusual
to use uppercase parameter names in C, the Cogent function will normally use the original C parameter names.

If the item property Global-State (see Section~\ref{design-types-itemprops}) has been used to introduce additional 
function parameters, the corresponding parameter names are added to the tuple after the last C parameter name.
If the item property Heap-Use has been set for the function, an additional parameter named \code{heap} or 
\code{globheap<n>} (see~\ref{design-types-itemprops}) is added to the tuple after possible Global-State parameters.
If the Modification-Function property has been declared for a parameter the tuple is rearranged into a pair 
where the second part is a tuple as described in Section~\ref{design-types-function}.

If the function is variadic an additional last tuple component is added with a variable named
\code{variadicCogentParameters}, mainly to inform the developer that manual action is required.

The generated Cogent function definition has the form
\begin{verbatim}
  <name> :: (<ptype1>, ..., <ptypen>) -> <restype>
  <name> (<pname1>, ..., <pnamen>) = 
    <translated C function body>
\end{verbatim}

\subsection{Comments in Function Definitions}
\label{design-fundefs-comments}

A C function definition which is not old-style syntactically consists of a declaration with a single declarator
and the compound statement for the body.
It is treated by Gencot as a structured source code part with the declaration and the body as subparts
without any main part code. According to the structures of declarations the declaration has the single declarator as subpart
and optionally a full specifier, if present. The declarator has the parameter declarations as subparts.

\subsubsection{Function Header}

The target code part for the declaration and for its single declarator is the header of the Cogent function definition
(first two lines in the schema in the previous section). The target code part for the full specifiers with tags in
the declaration (which may be present for the result type and for each parameter) is a sequence of corresponding 
type definitions, as described for declarations in Section~\ref{design-decls-tags}, which is placed 
after the Cogent function definition. The target code part for full specifiers without tags is the generated type
expression embedded in the Cogent type for the corresponding parameter or the result.

All parameter declarations consist of a single declarator and the optional full specifier. The target code part for
a parameter declaration and its declarator is the corresponding parameter type in the Cogent function type expression.
Hence, comments associated with parameter declarations in C are moved to the parameter type expression in Cogent.

\subsubsection{Function Body}

Gencot inserts origin markers in the Cogent code generated for the function body. It also inserts them in embedded 
untranslated C code. Thus, comments and preprocessor directives will be reinserted into the code. However, since the
structure of the Cogent code may substantially differ from that of the C code, comments and directives may be misplaced 
or omitted, this must be checked and handled manually.

\subsection{Entry Wrappers}
\label{design-fundefs-wrapper}

The C function definitions are also used for generating the entry wrappers as described in Section~\ref{design-modular}.
For every definition of a C function with external linkage an entry wrapper is generated in antiquoted C.

The main task of the entry wrapper is to convert the seperate function arguments to an argument tuple which is passed
to the corresponding Cogent function. If the C function has only one argument it is passed directly. If the C 
function has no argument the unit value must be passed to the Cogent function.

The entry wrapper has the same interface (name, arguments, argument types, result type) as the original C function.
However, since the wrapper is a part of the Cogent compilation unit, the argument and result types used in the 
original function definition are not available. They must be replaced by the binary compatible types generated 
from their translation to Cogent types (see Section~\ref{design-types}). This is accomplished by using the Cogent
types in antiquoted form.

If the original C function definition has the signature
\begin{verbatim}
  <type0> fnam(<type1> arg1, ..., <typen> argn)
\end{verbatim}
the entry wrapper has the signature
\begin{verbatim}
  $ty:(<cogt0>) fnam($ty:(<cogt1>) arg1, ..., $ty:(<cogtn>) argn)
\end{verbatim}
where \code{<cogti>} is the translation of \code{<typei>} to Cogent. It would be possible to generate synthetic 
parameter names for the wrapper, however, since the original parameter names must always be specified in the 
original function definition, Gencot uses them in the wrapper signature for simplicity and better readability.

If there is more than one function argument (\code{n > 1}), the wrapper must convert them to the C implementation
of the Cogent tuple value used as argument for the Cogent function (see Section~\ref{design-types-function}).
Cogent tuples are implemented as C struct types with members named \code{"p1"}, \code{"p2"}, \ldots in the order
of the tuple components. Since tuple types are always unboxed, the C structs are allocated on the stack. Thus
the body of an entry wrapper with \code{n > 1} has the form
\begin{verbatim}
  $ty:((<cogt1>,...,<cogtn>)) arg = 
    {.p1=arg1,..., .pn=argn};
  return cogent_fnam(arg);
\end{verbatim}

If \code{n = 1} no tuple needs to be constructed and the wrapper body has the form
\begin{verbatim}
  return cogent_fnam(arg1);
\end{verbatim}

If \code{n = 0} the C implementation of the Cogent unit value must be constructed. It is a C struct with 
a single member named \code{"dummy"} of type \code{int}. The corresponding wrapper body has the form
\begin{verbatim}
  $ty:(()) arg = {.dummy=0};
  return cogent_fnam(arg);
\end{verbatim}

If the C function has no result, i.e., \code{<type0>} is \code{void}, no result is returned by the entry
wrapper. Then the last statement in its body has the form
\begin{verbatim}
  cogent_fnam(arg);
\end{verbatim}

If the C function has the Heap-Use property (see Section~\ref{design-types-itemprops}), a component of type
\code{Heap} is added to the Cogent function's argument tuple and the result is converted to a tuple where the
original result is the first component. In C the heap is accessed globally, therefore these components can 
be ignored. However, the wrapper implementation has to take care of them. Type \code{Heap} is implemented by
Gencot simply by the C type \code{int} and a \code{0} is passed as its value. The entry wrapper body for a
function with Heap-Use property has the form
\begin{verbatim}
  $ty:((<cogt1>,...,<cogtn>,Heap)) arg = 
    {.p1=arg1,..., .pn=argn, .pn+1=0};
  return cogent_fnam(arg).p1;
\end{verbatim}
if \code{n > 0} and
\begin{verbatim}
  return cogent_fnam(0).p1;
\end{verbatim}
if \code{n = 0}.

If atleast one of the C function arguments has the Add-Result property, the result is also converted to a tuple 
whith the original result as its first component. In this case the last statement in the wrapper's body also 
has the form
\begin{verbatim}
  return cogent_fnam(arg).p1;
\end{verbatim}
if the C function's result is not \code{void}.

If the C function has virtual parameters with Global-State properties declared in the item properties, references
to the corresponding global variables are passed to these parameters by the entry wrapper. For a parameter with 
Global-State property the corresponding variable is determined by searching all known toplevel items for an
item with a Global-State property with the same numerical argument. Since every global variable must use a 
different numerical argument, this search will result in a single toplevel item identifier. 

If the function does not access the global variable and uses the virtual parameter only to pass the variable 
reference to invoked function, it may be the case that the variable is not visible in the C source file, because
it is neither defined nor declared there. However, the name used in (antiquoted) C for the variable can always 
be constructed from the item identifier alone. No other information about the variable is needed to generate the 
entry wrapper.

Assume that a C function has two virtual parameters with Global-State properties with numerical arguments 
\code{0} and \code{3}. The global variable associated with argument \code{0} has name \code{gvar} and external linkage,
the global variable associated with argument \code{3} has name \code{table} and is defined in source file \code{scan.c}
with internal linkage. Then the entry wrapper body has the form
\begin{verbatim}
  $ty:((<cogt1>,...,<cogtn>,GlobState,GlobState3)) arg = 
    {.p1=arg1,..., .pn=argn, .pn+1=&gvar, .pn+2=&local_scan_table};
  return cogent_fnam(arg).p1;
\end{verbatim}


If a parameter has the Modification-Function property the argument and result of the Cogent functions are
represented as pairs where the second component may be a tuple (see Section~\ref{design-types-function}). 
The wrapper must rearrange the parameters and retrieve the original result. For example, if the C 
function has 3 parameters, the second parameter has the Modification-Function property and the Add-Result 
property, and the third parameter has the Add-Result property the Cogent function has type
\begin{verbatim}
  (<cogt2>, (<cogt1>, <cogt3>)) -> (<cogt2>, (<cogt0>, <cogt3>))
\end{verbatim}
and the entry wrapper body has the form
\begin{verbatim}
  $ty:((<cogt2>,(<cogt1>,<cogt3>))) arg = 
    {.p1=arg2,.p2={.p1=arg1,.p2=arg3}};
  return cogent_fnam(arg).p2.p1;
\end{verbatim}

Together, the information required to generate the entry wrapper is the same as that for translating the 
function's type to Cogent: the argument types and the result type must be translated to Cogent and the
function name must be mapped to determine the name of the Cogent function.

\subsection{Object Definitions}
\label{design-fundefs-object}

Gencot processes object declarations with block-scope (``local variables'') as described in Section~\ref{design-cstats-general}. 

A toplevel (``external'') object definition (with ``file-scope'') either has an initializer or is a ``tentative definition''
according to the C standard. Such an object definition defines a global variable. Gencot treats global variables
as described in Section~\ref{design-modular}.

If the global variable has a Global-State property (see Section~\ref{design-types-itemprops}) Gencot introduces a type 
synonym of the form \code{GlobState<i>}. The type synonym definition is generated in the translated Cogent source 
at the place where the object definition was in the C source and is marked with the corresponding origin. The 
type synonym name is determined form the numerical argument of the Global-State property, the right-hand side of
the type definition is the translation of the derived pointer type of the variable's type, translated according to
Section~\ref{design-types}.

If the global variable has the Const-Val property (see Section~\ref{design-types-itemprops}) Gencot introduces an abstract 
parameterless access function in Cogent in the form
\begin{verbatim}
  <name> : () -> <type>
\end{verbatim}
where \code{<name>} is the mapped variable name and \code{<type>} is the translation of the variable's defined type.
If the defined type is a struct, union, or array type, the translation of the derived pointer type is used instead,
with a bang operator applied to make it readonly. The access function definition is generated in the translated Cogent source 
at the place where the object definition was in the C source and is marked with the corresponding origin.

Independent from its item properties, the object definition is translated to antiquoted C and added to the file 
containing the entry wrappers. In this way all global variables are still present in the Cogent compilation unit.
The translated definition uses as type the antiquoted Cogent type which results from translating the type specified 
in the object definition. For an object with external linkage the name is the original name, so that the object can still
be referred from outside the Cogent compilation unit, and the linkage is external. For an object with internal linkage
the name is mapped as described in Section~\ref{design-names} so that it is unique in the Cogent compilation unit, and
the linkage is internal. If an initializer is present in the original definition it is transferred to the antiquoted
definition, mapping all referenced names of global variables with internal linkage.

The entry wrappers and object definitions are placed in the antiquoted C file in the same order as the original function
and object definitions. Additionally, all global variables are declared together with all external functions in a 
file included before all entry wrapper files, so that entry
wrappers defined before the variable can still pass a pointer to it to the wrapped Cogent function.

If the global variable has the Const-Val property Gencot additionally creates an implementation of the access function
in antiquoted C and puts it immediately after the variable definition. Depending on the variable's type it returns its
value or a pointer to the variable. 


\section{Processing C Statements and Expressions}
\label{design-cstats}
In C statements and expressions occur in the bodies of function definitions and in initializers of object definitions. Gencot translates
both to corresponding Cogent constructs, as described in Section~\ref{design-fundefs}. In both cases the C statements and expressions
must be translated to Cogent expressions. This means a transformation from imperative constructs with possible side effects in a global
state to purely functional constructs where all effects are explicitly represented in the construct itself. It also means a transformation
from arbitrary treatment of pointers to the Cogent uniqueness type system where pointers are never shared or discarded.

The first issue is handled by Gencot. In some cases a translation is not (yet) implemented. In such cases the generated
Cogent code contains a dummy expression which documents the reason for not translating it automatically, together with
the corresponding C code as a comment. So the generated Cogent code is always syntactically correct and can be parsed
by the Cogent compiler, although it may not reflect the complete original C code.

The second issue is not handled by Gencot. If the C code involves sharing or discarding of pointers the translation will result in Cogent
code which violates the uniqueness type assumptions. This will be detected by the Cogent compiler in its type checking phase and must be
handled manually, either by modifying and retranslating the C code, or by modifying the Cogent code.

\subsection{General}
\label{design-cstats-general}

The basic building blocks of C function bodies are declarations and statements. Here we only cover declarations of 
local variables. Such a declaration specifies a name for the local variable and optionally an initial value.

A C statement causes modifications in its context. Cogent, as a functional language, does not support this concept,
in particular, it does not support modifying the value of a variable. Cogent only supports expressions which functionally
depend on their input, and it supports introducing immutable variables by binding them to a value. 
The main task of translating function bodies is to translate C statements to Cogent expressions.

The main idea of the translation is to translate a C variable which is modified to a sequence of bindings of Cogent
variables to the values stored in the C variable over time. Everytime when a C statement modifies a variable, a new
variable is introduced in Cogent and bound to the new value. A major difference of both approaches is that in Cogent 
the old variable and its value are still available after the modification. To prevent this, we use the same name for
both variables. Then the new variable ``shadows'' the old one in its scope, making the old value unaccessible there.
For example the C code
\begin{verbatim}
  int i = 0; i = i+1; ...
\end{verbatim}
is translated to
\begin{verbatim}
  let i = 0 in let i = i+1 in ...
\end{verbatim}
where the single C variable \code{i} is translated to two Cogent variables which are both named \code{i}.

\subsubsection{Statements}

A C statement in a function can modify several of the following:
\begin{itemize}
\item function parameters
\item local variables
\item global variables
\end{itemize}
All of them are specified by an identifier which is unique at the position of the statement. The modification
may replace the value as a whole or only modify one or more parts in the case of a structured value.

As a first step modifications of global variables must be eliminated by passing all such variables as additional 
parameters to the function and returning them as additional component in the result. Then a modification of a
global variable becomes a modification of a function parameter and only the first two cases remain.

Global variables can be turned to parameters or access functions using the Global-State and Const-Val properties
(see Section~\ref{design-types-itemprops}). If neither property has been specified for a global variable, Gencot
does not translate accesses to it and inserts a dummy expression instead.

After this step the effect of the modification caused by a C statement can be described by a set of identifiers 
of all modified parameters and local variables together with the new values for them. Syntactically this corresponds
to a Cogent \textit{binding} of the form
\begin{verbatim}
  (id1,...,idn) = expr
\end{verbatim}
where \code{expr} is a Cogent expression for the tuple of new values for the identifiers.

\subsubsection{Pointers}

Cogent treats C pointers in a special way as values of ``linear type'' and guarantees that no memory is shared
among different values of these types. More general, all values which may contain pointers (such as a struct with
some pointer members) have this property. All other values are of ``nonlinear type'' and never have common parts 
in C.

Gencot assumes that no sharing occurs among the parameters and local variables in the function. This implies that 
a C statement can only modify parameters and variables for which the identifiers occur literally in the statement
source code text. Thus it is possible to determine the effect of the modification caused by a C statement
syntactically from the statement.

If the assumption is not true the Cogent Compiler will detect this when it typechecks the translated code. See
Section~\ref{app-transfunction-pointers} for proposals how to handle such cases manually.

\subsubsection{Variable Declarations}

An initializer \code{init} in a variable declaration \code{t v = init;} is either an expression or an initializer for
a struct or an array. The C declaration can be rewritten as
\begin{verbatim}
  t v;
  v = init;
\end{verbatim}
with a separate statement for initializing the variable. If \code{init} is an expression this is valid C code. Otherwise,
if \code{init} is an initializer for a struct or array, the statement is not valid in C, but Gencot will translate it 
according to the intended semantics. For a struct type Cogent provides
corresponding expressions for unboxed records. For other types Gencot provides its initialization functions 
described in Section~\ref{design-operations-init} and~\ref{design-operations-array},
or the initialization can be done by several applications of operation \code{set} (see Section~\ref{design-operations-parts}).

For a declaration without an initializer Gencot inserts an assignment with a default value. If the variable has a regular 
type, the Gencot operation \code{defaultVal} (see Section~\ref{design-operations-default}) is used. If the variable has 
linear type (is a pointer or directly contains pointers) the contained pointers are translated to \code{MayNull} types 
(see Section~\ref{design-types-pointer}) and are initialized to value \code{null()} (see Section~\ref{design-operations-null}).
If a Not-Null property (see Section~\ref{design-types-itemprops}) is used to avoid the \code{MayNull} type Gencot 
will not translate the declaration if it has no initializer and will generate a dummy expression instead.

Together, this way all declarations are replaced by statements and the declarations need not be translated to Cogent, 
since in Cogent a new variable is introduced whenever the C variable is modified by a statement. Therefore, only the C statements 
need to be translated to Cogent (including pseudo assignment statements corresponding to record and array initializers).

\subsection{Expressions}
\label{design-cstats-expr}

C statements usually have C expressions as syntactic parts. For translating C statements the contained C expressions must be 
translated. C expressions have a value but may also cause modifications as side effects. Especially, in C an assignment
is syntactically an expression. C expressions are translated depending on whether they have side effects or not.

\subsubsection{Expressions without Side Effects}

C expressions without side effects are literals, variable references, member accesses of the form \code{s.m} or 
\code{s->m}, index
expressions of the form \code{a[e]}, applications of binary operators of the form \code{e1 op e2}, 
applications of the unary operators \code{+,-,!,~} of the form \code{op e}, and function 
call expressions of the form \code{f(e1,...,en)}, if all subexpressions
\code{s,a,e,f,e1,...,en} have no side effects and function \code{f} does not modify its parameter values. 

These expressions are translated to Cogent expressions in a straightforward way with the same or a similar syntax.
A member access \code{s->m} is translated as \code{s.m}. An index expression \code{a[e]} is translated as function 
call \code{getArr(a,e)}. Note, that some C operators have a different form in Cogent:
\begin{verbatim}
  C  Cogent
  !=   /=
  ^    .^.
  &    .&.
  |    .|.
  !    not
  ~    complement
\end{verbatim}

Member accesses of the form \code{s->m} (or written \code{(*s).m}) and index expressions \code{a[e]} can only be 
translated in this way if the
container value \code{s} or \code{a}, respectively, is translated to a readonly value in Cogent. If it is a parameter,
variable, or record field, it may have been defined as readonly. Otherwise, it can be made readonly in the expression's
context by applying the bang operator \code{!} to it at the end of the context:
\begin{verbatim}
  ... s.m ...  !s
\end{verbatim}
The resulting expressions are also of readonly type. For expressions of nonlinear type this is irrelevant, for expressions
of linear type it implies that they cannot be modified. If they are used in C by modifying them, they must be translated
in a different way. For example in the C code fragment
\begin{verbatim}
  *(s->p) = 5
\end{verbatim}
the expression \code{s->p} denotes a pointer which is modified, therefore it cannot be translated to \code{s.m} where 
s is readonly.

For resultig expressions of linear types, using the bang operator \code{!} also means that the readonly result cannot
escape from the banged context, it can only be used inside the context.

Translation of expressions using the address operator \code{\&} are described in Section~\ref{app-transfunction-addrop}.

Dereferencing (application of the indirection operator \code{*} to) a pointer is translated depending on the type of value referenced by 
the pointer. If it points to a function, \code{*p} is translated as \code{fromFunPtr(p)} 
(See Section~\ref{design-operations-function}). 

If it points to a primitive type, an enum type, or again a pointer type, 
\code{*p} is translated as \code{getPtr(p)} (See Section~\ref{design-operations-pointer}). In this case \code{p} 
must be readonly as above, and the result is readonly. If it should be modified it must be translated differently.

Otherwise it points to a type which is mapped to Cogent as a record or abstract type. Then \code{*p} is translated as \code{p}. 

Expressions using the C operators \code{sizeof} or \code{\_Alignof} cannot be translated to Cogent. Usually, an abstract
function implemented in C is required here.

\subsubsection{Expressions with Side Effects}

We translate an expression with side effects to a Cogent binding of the form \code{pattern = expr}. Here, the
\code{pattern} is a tuple of variables \code{(v,v1,...,vn)}. The variable \code{v} is a
new variable which is not already bound in the context of the expression, it is used to bind the result value of the
expression. The other variables are the identifiers of all parameters and local 
variables modified by the expression. Since we presume that the C expression has side effects, there is at least one
such variable.
The \code{expr} is a Cogent expression for a corresponding tuple consisting of the result value of the C expression and the new values
of the identifiers modified by the C expression.

Expressions with side effects are applications of the increment and decrement operators \code{++,-{}-}, assignments,
and invocations of functions which modify one or more parameter values.

Applications of increment and decrement prefix operators must be rewritten in C using assignments. Assignments using assignment
operators other than \code{=} must be rewritten in C using the \code{=} operator. After these steps the only 
remaining expressions with side effects are assignments using \code{=}, increment and decrement postfix operators, 
and invocations of functions which modify parameter values.

If for such an expression all subexpressions are without side effects, they are translated as follows. 

The translation of an assignment expression of the form \code{lhs = e} depends on the form of \code{lhs}. If it is
a single identifier \code{v1} (name of a parameter or local variable), it is translated as
\begin{verbatim}
  (v,v1) = let v = expr in (v,v)
\end{verbatim}
where \code{expr} is the translation of \code{e}.

Note that this code is illegal in Cogent, if \code{expr} has a linear type, since it uses the result value twice. However, this
is a natural property of C code, where an assignment is an expression and the assigned value can be used in the context. For example,
the C code fragment \code{f(p = q)} assigns the value of \code{q} to \code{p} and also passes it as argument to function \code{f}.

There are several ways how to cope with this situation. In the simplest case, the outer variable \code{v} is never used in its
scope, then the double use of the value can be eliminated by simplifying the Cogent binding to the form
\begin{verbatim}
  v1 = expr
\end{verbatim}
This is typically the case if the assignment is used as a simple statement, where its result value is discarded. Most assignments
in C are used in this way.

Otherwise it depends on how the variable \code{v} is used in its scope. In some cases it may be possible to replace its use by 
using \code{v1} instead, then it can be eliminated in the same way as above. If that is not possible, the C program uses true
sharing of pointers, then the code cannot be translated to Cogent and must be translated using abstract functions.

If \code{lhs} is a logical chain of \code{n} member access, index, and dereferencing expressions starting with identifier \code{v1} 
(name of a parameter or local variable), the most general translation is
\begin{verbatim}
  (v,v1) =
    let v = expr
    in (v,fst(modify1(v1,(modify2,(...(modifyn-1,(set,v))...)))))
\end{verbatim}
where \code{expr} is the translation of \code{e} and the sequence of \code{modifyi} functions is determined by the chain of access
expressions and \code{set} is an 
instance of the operation \code{set} (see Section~\ref{design-operations-parts}). Again, the result of \code{expr} is used twice,
the same considerations as above apply if it has a linear type.

For example the C assignment expression
\begin{verbatim}
  s->a[i]->x = 5
\end{verbatim}
is translated according to this rule to the Cogent binding
\begin{verbatim}
  (v,s) = let v = 5
    in (v,fst(modifyFld_a(s,(modifyArr,(i,setFld_x,v)))))
\end{verbatim}
where \code{modifyFld\_a} and \code{setFld\_x} are abstract modification functions for the fields \code{a} and \code{x},
respectively, as described in Section~\ref{design-operations-record}. 

This translation is correct, but it is inefficient because \code{modifyFld\_a} will copy the whole array \code{a} only to access
one element of it by applying \code{modifyArr} to the unboxed array. Using \code{modref} (see Section~\ref{design-operations-modify}) 
this can be improved to
\begin{verbatim}
  (v,s) = let v = 5
    in (v,fst(modrefFld_a(s,(modifyArr,(i,setFld_x,v)))))
\end{verbatim}
here \code{modrefFld\_a} retrieves only the pointer to \code{a} and applies \code{modifyArr} to the boxed array. If the array element
would be a larger data structure it could be further improved by using \code{modrefArr} instead of \code{modifyArr}, here it does
not pay because the array element is already a single pointer.

No other cases for \code{lhs} are valid in a C assignment.

If \code{lhs} logically starts with a chain of member accesses \code{v1->m1->...->mn...} an alternative translation using the Cogent
take and put operations is the binding
\begin{verbatim}
  (v,v1) = 
    let v1{m1=m1{m2=...mn-1{mn}...}}
    and (v,mn) = expr
    in (v,v1{m1=m1{m2=...mn-1{mn=mn}...}})
\end{verbatim}
where \code{(v,mn) = expr} is the binding to which \code{mn... = e} is translated, if \code{mn} is assumed to be a local variable. 
For example, a corresponding translation of \code{*(s->m1->m2) = 5} is
\begin{verbatim}
  (v,s) = 
    let s{m1=m1{m2}}
    and (v,m2) = let v = 5 in (v,setPtr(m2,v))
    in (v,s{m1=m1{m2=m2}})
\end{verbatim}
This approach avoids the need to manually define and implement the functions \code{modifyFld\_mi}, however it prevents the improvement
described above by using \code{modref}.

An application of an increment/decrement postfix operator \code{ss} where \code{s} is \code{+} or \code{-} has the form
\code{lhs ss}. If \code{lhs} is a single identifier \code{v1} this identifier must have a numerical type and the expression 
is translated to the binding
\begin{verbatim}
  (v,v1) = (v1,v1 s 1)
\end{verbatim}
Using \code{v1} twice is always possible here since it has nonlinear type. As an example, \code{i++} is translated to
\begin{verbatim}
  (v,i) = (i,i+1)
\end{verbatim}

If \code{lhs} is a logical chain of \code{n} member access, index, and dereferencing expressions starting with identifier \code{v1}
this identifier must have linear type and the most general translation is the binding
\begin{verbatim}
  (v,v1) = 
    let v = tlhs !v1
    in (v, fst(modify1(v1,(modify2,(...(modifyn-1,(set,v s 1))...))))
\end{verbatim}
where \code{tlhs} is the translation of \code{lhs} when \code{v1} is readonly and \code{modifyi} and \code{set} are as 
for the assignment. Here \code{v1} is needed twice, first for retrieving the old
numerical value \code{v} and afterwards to set it to the incremented/decremented value. Since \code{v1} is linear the old
value must be retrieved in a readonly context and the modification must be done seperately. The double use of \code{v}
is always possible since it has numeraical type.

Again, this translation may be improved by using \code{modref} instead of \code{modify} where appropriate.

If \code{lhs} logically starts with a chain of member accesses \code{v1->m1->...->mn...} an alternative translation using the Cogent
take and put operations is the binding
\begin{verbatim}
  (v,v1) =
    let v = tlhs !v1
    and v1{m1=m1{m2=...mn-1{mn}...}} 
    in (v, v1{m1=m1{m2=...mn-1{mn=expr}...}})
\end{verbatim}
where \code{expr} is as above for the assignment using \code{(v s 1)} as the new value.

For example, a corresponding translation of \code{(*(s->m1->m2))++} is
\begin{verbatim}
  (v,s) =
    let v = getPtr(s.m1.m2) !s
    and s{m1=m1{m2}}
    in (v,s{m1=m1{m2=setPtr(m2,v+1)}})
\end{verbatim}

A C function which modifies parameter values is translated by Gencot to a Cogent function returning the tuple \code{(y,p1,...,pn)}
of the original function result \code{y} and the modified parameter values \code{p1,...,pn} (which must be pointers). 
A C function call \code{f(...)} is
translated depending on the form of the actual arguments passed to \code{f} for the modified parameters.

If all such arguments are identifiers (names of parameters and local variables) the function call is translated to
the binding
\begin{verbatim}
  (v,v1,...,vn) = f(...)
\end{verbatim}
where \code{v1,...,vn} are the identifiers passed to the parameters modified by \code{f} in the order returned by \code{f}.

If some of the arguments are member access chains of the form \code{vi->mi1->...->miki} they must be translated using the 
Cogent take and put operations as above to a binding of the form
\begin{verbatim}
  (v,v1,...,vn) = 
  let v1{m11=...{m1k1}...} = v1
  and ... 
  and vn{mn1=...{mnkn}...} = vn
  and (v,m1k1,...,mnkn)=f(...)
  in (v,v1{m11=...{m1k1=m1k1}...},
       ...
        vn{mn1=...{mnkn=mnkn}...})
\end{verbatim}
where in the arguments of \code{f} the chains are replaced by their last member name \code{miki}.

For example, the function call \code{f(5,s->m,t->n,z)} where \code{f} modifies its second and third parameter, is translated
to the binding
\begin{verbatim}
  (v,s,t) = 
  let s{m} = s
  and t{n} = t
  and (v,m,n)=f(5,m,n,z)
  in (v,s{m=m},t{n=n})
\end{verbatim}

If function \code{f} modifies only one parameter \code{p} of Cogent type \code{P} its standard type generated by
Gencot can be changed to the form of a modification function
\begin{verbatim}
  f: ModFun P (...) Res
\end{verbatim}
where \code{(...)} is the tuple of types of the other parameters and \code{Res} is the result type. Then for an arbitrary
chain of member access, index, and dereferencing expressions used as actual argument for \code{p} the function call 
can be translated to a binding of the form
\begin{verbatim}
  (v,v1) = let (v1,v) = 
    modify1(v1,(modify2,(...(modifyn,(f,(a1,...,an)))...)))
    in (v,v1)
\end{verbatim}
where the sequence of \code{modifyi} functions is determined by the chain of access expressions. Note
that the order of the variables must be exchanged since the original result of \code{f} is the second component in 
the inner tuple pattern due to the way \code{ModFun} is defined.

For example, the function call \code{f(5,s->a[i]->x,z)} can be translated by first modifying the translation of 
\code{f} so that it takes as parameters instead of the tuple \code{(a,b,c)} the pair \code{(b,(a,c))}. Then a translation
for the function call is
\begin{verbatim}
  (v,s) = let (s,v) = 
    modifyFld_a(s,(modifyArr,(i,modifyFld_x,(f,(5,z)))))
    in (v,s)
\end{verbatim}
If index \code{i} is invalid, according to the definition of \code{modifyArr} the function call \code{f(5,s->a[0]->x,z)} 
is executed instead.

Again, if the chain logically starts with member accesses, that part of the chain can be translated using take and put operations.

In all other cases the function \code{f} must be modified so that it takes the starting identifiers of the chains as
arguments instead of the chains, then it can be translated as in the first case where all actual arguments are identifiers.
Note that different translations of the function to Cogent may be required for translating different function calls.

\subsubsection{Nested Expressions with Side Effects}

If an expression contains subexpressions with side effects, these must be translated separately. 

Let \code{e1,...,en} be the expressions with side effects directly contained in the expression \code{e} and
let \code{p1 = expr1, ..., pn = exprn} their translations to Cogent bindings. Since in C the order of evaluation 
of the \code{e1,...,en} is undefined, we can only translate \code{e} if the subexpressions
modify pairwise different sets of identifiers, i.e., the \code{p1,...,pn} are pairwise disjunct tuples.
Let \code{x1,...,xn} be the
first variables of the patterns \code{p1,...,pn} and let \code{w1,...,wm} be the union of all other variables
in \code{p1,...,pn} in some arbitrary order. Let \code{e'} be \code{e} with every \code{ei}
substituted by \code{xi}. Then \code{e'} contains no nested expressions with side effects and can be translated
to Cogent according to the previous sections.

If \code{e'} has no side effects, let \code{expr} be its translation to a Cogent expression. Then the translation 
of \code{e} is the binding
\begin{verbatim}
  (v,w1,...,wm) = 
    let p1 = expr1 and ...
    and pn = exprn 
    in (expr,w1,...,wm)
\end{verbatim}
For example the expression \code{a[i++]} is translated according to this rule to
\begin{verbatim}
  (v,i) = 
    let (v,i) = (i,i+1)
    in (getArr(a,v),i)
\end{verbatim}

If \code{e'} has side effects, let \code{(v,v1,...,vk) = expr} be its translation to a Cogent binding. Then
the translation of \code{e} is the binding
\begin{verbatim}
  (v,v1,...,vk,w1,...,wm) = 
    let p1 = expr1 and ...
    and pn = exprn 
    and (v,v1,...,vk) = expr
    in (v,v1,...,vk,w1,...,wm)
\end{verbatim}
For example the expression \code{a[i++] = 5} is translated according to this rule to
\begin{verbatim}
  (v,a,i) = 
    let (v,i) = (i,i+1)
    and (w,a) = let w = 5 in (w,setArr(a,v,w))
    in (w,a,i)
\end{verbatim}
which can be simplified in Cogent to the binding
\begin{verbatim}
  (v,a,i) = (5,setArr(a,i,5),i+1)
\end{verbatim}

\subsubsection{Readonly Access and Modification of the Same Value}

If a C expression uses and modifies parts of a linear value at the same time, a special translation approach is required.
An example is the expression \code{r->sum = r->n1 + r->n2}. According to the rules above, translating the right hand side
requires to make \code{r} readonly by applying the bang operator \code{!} which then prevents translating the expression
as a whole, modifying \code{r}. There are three cases how to deal with this situation.

If all used parts of the value are nonlinear, they can be retrieved in a separate step using \code{!}, bound to Cogent 
variables and then used in the modification step. The translation of an expression \code{e} then has the general form
\begin{verbatim}
  (v,v1,...,vn) = 
    let (w1,...,wm) = ... !r1 ... !rk
    in expr
\end{verbatim}
where \code{w1,...,wm} are auxiliary Cogent variables for binding the used values, \code{r1,...,rk} are all linear values
from which parts are used in \code{e} and \code{(v,v1,...,vn) = expr} is the 
normal translation of \code{e} with all used parts replaced by the corresponding \code{wi}.

We used this approach for translating applications of postfix increment/decrement operations to complex expression,
such as \code{(*(s->m1->m2))++}.

If the used values are themselves linear, such as in \code{r->p = f(r->p)} where \code{p} is a pointer, this approach
cannot be used since in Cogent the binding \code{w = r.p !r} is illegal, \code{r.p} has a readonly linear type and
is not allowed to escape from the banged context so that it can be bound to \code{w}.

If the used linear value is replaced by the modification, as in the example, it is possible to use the take and put
operations. In a first step the used values are taken from the linear containers \code{r1,...,rk}, in the modification
step they are put back in. The translation of \code{e} then has the general form
\begin{verbatim}
  (v,v1,...,vn) = 
    let r1{... = w1 ...} ...
    and rk{... = wk ...}
    in expr
\end{verbatim}
where \code{expr} contains the necessary put operations for all \code{r1,...,rk}.

It may also be the case that the modification cannot be implemented by a combination of take and put operations, either 
because data types like arrays and pointers are involved, which are not translated to Cogent records, or because
the modification causes sharing or discarding linear values. In both cases the modification cannot be implemented in
Cogent, it must be translated by introducing an abstract function \code{fexpr} which implements the expression \code{e}
as a whole. Then the translation has the form
\begin{verbatim}
  (v,v1,...,vn) = 
    fexpr(v1,...,vn,x1,...,xm)
\end{verbatim}
where \code{x1,...,xm} are additional nonlinear values used by the expression.

\subsubsection{Comma Operator}

In C the expression \code{e1,e2} first evaluates \code{e1}, discarding its result and then evaluates \code{e2} for which
the result is the result of the expression as a whole. Thus, the comma operator only makes sense if \code{e1} has side
effects. Let \code{(v,v1,...,vn) = expr1} be the translation of \code{e1} to a Cogent binding.

If \code{e2} has no side effects and is translated to the Coent expression \code{expr2}, the expression \code{e1,e2}
is translated to the Cogent binding
\begin{verbatim}
  (v,v1,...,vn) = 
    let (_,v1,...,vn) = expr1
    in (expr2,v1,...,vn) 
\end{verbatim}
Note that the translation is only valid if the result value of \code{e1} is not linear, since it is discarded. If 
\code{expr1} is a tuple, this can be simplified by omitting the first component. This avoids the discarding and 
may even remove a double use of a linear value in \code{expr2}.

If \code{e2} has side effects let \code{(w,w1,...,wm) = expr2} be the translations of \code{e2}. Then the translation of 
\code{e1,e2} is 
\begin{verbatim}
  (v,u1,...,uk) = 
    let (_,v1,...,vn) = expr1
    and (w,w1,...,wm) = expr2
    in (w,u1,...,uk) 
\end{verbatim}
where \code{u1,...,uk} is the union of \code{v1,...,vn} and \code{w1,...,wm}. 

\subsubsection{Conditional Expression}

A conditional expression \code{e0 ? e1 : e2} is translated as follows. Let \code{(x,x1,...,xn) = expr0}, 
\code{(y,y1,...,ym) = expr1}, \code{(z,z1,...,zp) = expr2} be the translations of 
\code{e0}, \code{e1}, \code{e2}, respectively.
If \code{e0} evaluates to a value of numerical type the translation of the conditional expression is 
\begin{verbatim}
  (v,v1,...,vk) = 
    let (x,x1,...,xn) = expr0
    and (v,u1,...,uq) =
      if x /= 0 then 
        let (y,y1,...,ym) = expr1 
        in (y,u1,...,uq)
      else
        let (z,z1,...,zp) = expr2
        in (z,u1,...,uq)
    in (v,v1,...,vk) 
\end{verbatim}
where \code{u1,...,uq} is the union of \code{y1,...,ym} and \code{z1,...,zp} in some order, and 
\code{v1,...,vk} is the union of \code{y1,...,ym} and \code{x1,...,xn} in some order.

If the outermost operator of \code{e0} is a ``logical'' operator (one of \code{<, <=, >, >=, ==, !=, \&\&, ||})
it can be translated in a way that the first component in the result of \code{expr0} is of type Bool.
Then, instead of \code{if x /= 0 then} the condition is written as \code{if x then}.

In C the value used for testing the condition may be of any scalar type, i.e., a numerical or pointer type.
In case of a pointer type the condition corresponds to testing the pointer for being not \code{NULL}.
The translated expression \code{expr0} will return a value of linear type as first component. Since it
may be null (otherwise the conditional expression could be replaced by \code{e1}) it must be translated
so that \code{expr0} returns a value of type \code{MayNull a} (see Section~\ref{design-operations-null})
as its first component. Since the use as condition in Cogent would discard the value, it must be readonly,
i.e., of type \code{(MayNull a)!}. Then, instead of \code{if x /= 0 then} the condition must be written as
\begin{verbatim}
      if not isNull(x) then 
\end{verbatim}
using function \code{isNull} as described in Section~\ref{design-operations-null}. If the first component
of the result of \code{expr0} is not already of readonly type it must be made readonly in the condition:
\begin{verbatim}
      if not isNull(x) !x then 
\end{verbatim}

Often in C, after testing x successfully for not being \code{NULL}, the pointer \code{x} is used in 
\code{e1} by dereferencing it. In Cogent this is not possible, since a value of type \code{MayNull a} cannot
be dereferenced. Instead, a value of type \code{a} is required. This is made accessible by the functions
\code{notNull} and \code{roNotNull} (see Section~\ref{design-operations-null}). Then the translation of the
conditional expression has the form
\begin{verbatim}
  (v,v1,...,vk) = 
    let (x,x1,...,xn) = expr0
    and (v,u1,...,uq) =
      notNull(x)
      | Some x -> 
        let (y,y1,...,ym) = expr1 
        in (y,u1,...,uq)
      | None ->
        let (z,z1,...,zp) = expr2
        in (z,u1,...,uq)
    in (v,v1,...,vk) 
\end{verbatim}
Here the \code{x} introduced by matching \code{Some x} is of type \code{a} and can be used in \code{expr1}
to access and modify it. If \code{expr1} only reads \code{x} and should not consume it, it must be introduced
as readonly by replacing the line \code{notNull(x)} by
\begin{verbatim}
      roNotNull(x) !x
\end{verbatim}
Then the \code{x} introduced by matching \code{Some x} is of type \code{a!}. Function \code{roNotNull} must also
be used when the outer \code{x} is already of the readonly type \code{(MayNull a)!}.

In the following patterns we only show the form \code{if x /= 0 then}, it must be replaced as needed.

Usually, the condition \code{e0} has no side effects, the the translation can be simplified to
\begin{verbatim}
  (v,u1,...,uq) = 
    if expr0 /= 0 then 
      let (y,y1,...,ym) = expr1 
      in (y,u1,...,uq)
    else
      let (z,z1,...,zp) = expr2
      in (z,u1,...,uq)
\end{verbatim}

If also \code{e1} and \code{e2} have no side effects the translation can be further simplified to
the Cogent expression
\begin{verbatim}
  if expr0 /= 0 then expr1 else expr2
\end{verbatim}

As an example the C expression \code{i == 0? a = 5 : b++} is translated to
\begin{verbatim}
  (v,a,b) = 
    if i == 0 then 
      let a = 5
      in (a,a,b)
    else
      let (v,b) = (b,b+1)
      in (v,a,b)
\end{verbatim}

For these translations the way how the Cogent compiler is used is relevant. When it translates the Cogent code back to 
C it translates an expression of the form
\begin{verbatim}
  let v = if a then b else c
  in rest
\end{verbatim}
to a C statement of the form
\begin{verbatim}
  if a { v = b; rest}
  else { v = c; rest}
\end{verbatim}
duplicating the translated code for \code{rest} (which is called ``a-normal form'' in Cogent). 
The same happens for all bindings and expressions after
the binding of \code{v} in the context of the \code{let} expression. If several such bindings to a 
conditional expressions are used this leads to an exponential growth in size of the C code.

To prevent this the Cogent compiler must be used with the flag \code{--fnormalisation=knf}
which translates to ``k-normal form''
\begin{verbatim}
  if a { v = b}
  else { v = c}
  rest
\end{verbatim}

Alternatively, the conditional expression can be wrapped in a lambda expression in the form
\begin{verbatim}
  let v = (\(x1,...,xn) => if a then b else c) 
    (x1,...,xn)
  in rest
\end{verbatim}
where \code{x1,...,xn} are all Cogent variables used in the expressions \code{a, b, c}.
Cogent translates this code to C without duplicating \code{rest}, even if a-normal form is used.

\subsection{Statements}
\label{design-cstats-stat}

A C statement has no result value, it is only used for its side effects. Therefore we basically translate every
statement to a Cogent binding of the form \code{pattern = expr} where pattern is a tuple of all identifiers
modified by the statement.

However, a C statement can also alter the control flow in its environment, which is the case for 
\code{return} statements, \code{goto} statements etc. We treat this by adding a 
component to the \code{pattern}, so that the translation becomes
\begin{verbatim}
  (c,v1,...,vn) = expr
\end{verbatim}
The variable \code{c} is a
new variable which is not already bound in the context of the expression, it is used to bind the information about 
control flow modification by the statement. The other variables are the identifiers of all parameters and local 
variables modified by the statement. Since a statement need not modify variables, the pattern may also be
a single variable \code{c}. In the following descriptions we always use a pattern of the form \code{(c,v1,...,vn)}
with the meaning that for \code{n=0} it is the single variable \code{c}.

The value bound to \code{c} is a tuple of type 
\begin{verbatim}
  (Bool, Bool, Option T)
\end{verbatim}
where \code{T} is the original result type of the surrounding function. In a value \code{(cc,cb,res)} the component
\code{cc} is true if the statement contains a \code{return}, \code{break}, or \code{continue} statement outside
of a \code{switch} or loop statement. The component \code{cb} is true if the statement contains a \code{return}
or \code{break} statement outside of a \code{switch} or loop statement. The component \code{res} is the value 
\code{None} if the statement contains no \code{return} statement, otherwise it is the value \code{Some v} where
\code{v} is the value to be returned by the function.

The translation of \code{goto} statement and labels is not supported, this must be done manually.

\subsubsection{Simple Statements}

A simple C statement consists of a C expression, where the result is discarded. Therefore it makes only sense
if the C expression has side effects. Simple statements cannot modify the control flow, they always bind the 
variable \code{c} to the value \code{(False,False,None)} which we abbreviate as \code{ffn} in the following.

The binding in the translation of a simple C statement \code{e;} is mainly the binding resulting from the translation 
of the expression \code{e}, as described in Section~\ref{design-cstats-expr}. Let its translation
be \code{(v,v1,...,vn) = expr}.
Then the binding for the statement \code{e;} is
\begin{verbatim}
  (c,v1,...,vn) = 
    let (v,v1,...,vn) = expr
    in (ffn,v1,...,vn)
\end{verbatim}

The result \code{v} of \code{e} is discarded, the same considerations apply here as described for expressions using
the comma operator in Section~\ref{design-cstats-expr}. 

If \code{expr} is a tuple 
\code{(e0,e1,...,en)} the binding can further be simplified to
\begin{verbatim}
  (c,v1,...,vn) = (ffn,e1,...,en)
\end{verbatim}

As an example, the translation of the simple statement \code{i++;} is
\begin{verbatim}
  (c,i) = let (v,i) = (i,i+1) in (ffn,i)
\end{verbatim}
which can be simplified to
\begin{verbatim}
  (c,i) = (ffn,i+1)
\end{verbatim}

The empty C statement \code{;} is translated as
\begin{verbatim}
  c = ffn
\end{verbatim}

\subsubsection{Jump Statements}

A jump statement is a \code{return} statement, \code{break} statement, or \code{continue} statement.

A C return statement has the form \code{return;} or \code{return e;}. It always ends the control flow in the surrounding
function body. The first form can only be used in functions returning \code{void}, these are translated to Cogent as returning 
\code{()} or a tuple with \code{()} as first component.

The translation of a return statement of the form \code{return;} is the binding
\begin{verbatim}
  c = (True,True,Some ())
\end{verbatim}
If \code{e} has no
side effects, the translation of statement \code{return e;} is
\begin{verbatim}
  c = (True,True,Some expr)
\end{verbatim}
where \code{expr} is
the translation of \code{e}. 

Otherwise, let \code{(v,v1,...,vn) = expr} be the translation of \code{e}. Then 
the translation of \code{return e;} is
\begin{verbatim}
  (c,v1,...,vn) = 
    let (v,v1,...,vn) = expr
    in ((True,True,Some v), v1,...,vn)
\end{verbatim}
which can be simplified if \code{expr} is the tuple \code{(e0,e1,...,en)} to
\begin{verbatim}
  (c,v1,...,vn) = ((True,True,Some e0),e1,...,en)
\end{verbatim}

A C \code{break} statement has the form \code{break;}. It ends the next surrounding \code{switch} or loop statement,
otherwise it may not be used. It is translated to
\begin{verbatim}
  c = (True,True,None)
\end{verbatim}

A C \code{continue} statement has the form \code{continue;}. It ends the body of the next surrounding loop statement,
otherwise it may not be used. It is translated to
\begin{verbatim}
  c = (True,False,None)
\end{verbatim}

\subsubsection{Conditional Statements}

A conditional C statement has the form \code{if (e) s1} or \code{if (e) s1 else s2} where \code{e} is an expression
and \code{s1} and \code{s2} are statements. Let \code{(c1,v1,...,vn) = expr1} be the translation 
of \code{s1} and \code{(c2,w1,...,wm) = expr2} be the translation of \code{s2}.
If \code{e} has no side effects and translates to the Cogent expression \code{expr} we translate
the second form of the conditional statement to the binding
\begin{verbatim}
  (c,u1,...,uk) =
    if expr /= 0 then 
      let (c1,v1,...,vn) = expr1
      in (c1,u1,...,uk)
    else
      let (c2,w1,...,wm) = expr2
      in (c2,u1,...,uk)
\end{verbatim}
where \code{u1,...,uk} is the union of \code{v1,...,vn} and \code{w1,...,wm} in some order. 
Note that this closely corresponds to the translation 
of the conditional expression in Section~\ref{design-cstats-expr}. In particular, the condition test 
\code{if expr /= 0 then} must be replaced by \code{if expr then} or \code{if not isNull(expr) then} or 
the forms with \code{notNull} or \code{roNotNull} as needed.

The first form of the conditional statement is translated as
\begin{verbatim}
  (c,v1,...vn) =
    if expr /= 0 then expr1
    else ((False,False,None),v1,...vn)
\end{verbatim}

If expression \code{e} has side effects the translation is extended as for the conditional expression.

As an example the translation of the conditional statement \code{if (i==0) return a; else a++;} is the binding
\begin{verbatim}
  (c,a) =
    if i==0 then 
      let c = (True,True,Some a)
      in (c,a)
    else
      let (c,a) = ((False,False,None),a+1)
      in (c,a)
\end{verbatim}
which can be simplified to
\begin{verbatim}
  (c,a) =
    if i==0 then ((True,True,Some a),a)
    else ((False,False,None),a+1)
\end{verbatim}

\subsubsection{Compound Statements}

A compound statement is a block of the form \code{\{ s1 ... sn \}} where every \code{si} is a statement or a declaration.
Declarations of local variables are treated as described in Section~\ref{design-cstats-general}: if they contain an
initializer they are translated as a statement, otherwise they are omitted. This reduces the translation of a compound
statement to the translation of a sequence of statements.

We provide the translation for the simplest case \code{\{ s1 s2 \}} where \code{s1} and \code{s2} are statements.
The general case can be translated by rewriting the compound statement as a sequence of nested blocks, associating
from the right.

Let \code{(c,v1,...,vn) = expr1} be the translation of statement \code{s1} and \code{(c,w1,...,wm) = expr2} 
be the translation of statement \code{s2}. Then the translation of
\code{\{ s1 s2 \}} is
\begin{verbatim}
  (c,u1,...,uk) =
    let ((cc,cb,res),v1,...,vn) = expr1
    in if cc then ((cc,cb,res),u1,...,uk)
    else let (c,w1,...,wm) = expr2
    in (c,u1,...,uk)
\end{verbatim}
where \code{u1,...,uk} is the union of \code{v1,...,vn} and \code{w1,...,wm} in some order without the local variables 
declared in the block.

The values of local variables declared in the block are discarded. If such variables have linear type, they must
be allocated on the heap and disposed at the end of the block, as described in Section~\ref{app-transfunction-addrop}.

As an example, the compound statement \code{\{ if (i==0) return a; else a++; b = a; \}} is translated to the binding
\begin{verbatim}
  (c,a,b) =
    let ((cc,cb,res),a) = 
      if i==0 then ((True,True,Some a),a)
      else ((False,False,None),a+1)
    in if cc then ((cc,cb,res),a,b)
    else let (c,b) = ((False,False,None),a)
    in (c,a,b)
\end{verbatim}

If \code{s1} contains no jump statement the variable \code{cc} is bound to \code{False} and the translation
can be simplified to the binding
\begin{verbatim}
  (c,u1,...,uk) =
    let (c,v1,...,vn) = expr1
    and (c,w1,...,wm) = expr2
    in (c,u1,...,uk)
\end{verbatim}

As an example, the compound statement \code{\{ int i = 0; a[i++] = 5; b = i + 3; \}} is translated to the 
simplified binding
\begin{verbatim}
  (c,a,b) =
    let (c,i) = (ffn,0)
    and (c,i,a) = (ffn,i+1,setArr(a,i,5))
    and (c,b) = (ffn,i+3)
    in (c,a,b)
\end{verbatim}
which can be further simplified to
\begin{verbatim}
  (c,a,b) =
    let i = 0
    and (i,a) = (i+1,setArr(a,i,5))
    and b = i+3
    in (ffn,a,b)
\end{verbatim}

A compound statement used as the body of a C function is translated to a Cogent expression instead of a Cogent binding.
If the C function has result type \code{void} and modifies its parameters \code{pm1,...,pmn}, its translation is 
a Cogent function returning the tuple \code{((),pm1,...,pmk)}. If the translation of the compound statement
used as function body is the binding \code{(c,v1,...,vn) = expr}, the body is translated to the expression
\begin{verbatim}
  let (c,v1,...,vn) = expr
  in ((),pm1,...,pmn)
\end{verbatim}
If the function modifies no parameters
it returns the unit value \code{()}. Note that if the C function is correct, \code{c} contains the value \code{Some ()}.

Here all local variables and the unmodified parameters are discarded. If local variables have linear type, they must
be allocated on the heap and disposed at the end of the function, as described in Section~\ref{app-transfunction-addrop}.
If an unmodified parameter has linear type, it must be disposed at the end of the function.

If the C function returns a value its translation is a Cogent function returning the tuple \code{(v,pm1,...,pmk)}
where \code{v} is the original result value of the C function. Then the body is translated to the expression
\begin{verbatim}
  let ((_,_,res),v1,...,vn) = expr
  in res | None -> (defaultVal(),pm1,...,pmn)
         | Some v -> (v,pm1,...,pmn)
\end{verbatim}
where \code{defaultVal()} is as defined in Section~\ref{design-operations-default}. If the type of the original function
result \code{v} is not regular (in C: contains a pointer) \code{defaultVal} cannot be used, in this case a different
expression for the \code{None} alternative must be specified manually. Note that in this case the \code{None} alternative
is never used, if the C program is correct. Nevertheless, for Cogent to be correct some result value must be specified 
also for this case, it is irrelevant which value is used.

\subsubsection{Switch Statements}

A \code{switch} statement in C specifies a ``control expression'' of integer type and a body statement. Control jumps to a 
\code{case} statement in the body marked with the value of the control expression or to a \code{default} statement. If neither
is present in the body it is not executed.

The \code{case} and \code{default} statements may occur anywhere in the body other than in a nested \code{switch} statement,
in particular they may occur in nested conditional statements and in loop statements. In these cases the jump is past the 
condition or loop initialization. The resulting behavior is clearly defined in the C standard, however, it is difficult to 
be transferred to a functional specification and usually involves code copying. The jump targets may also occur in a nested 
compound statement so that the jump may be past some declarations into an inner scope, this is also difficult to transfer.

However, there is a simple form of \code{switch} statements which can be transferred in a rather straightforward way. It is
the case where the body is a compound statement and all \code{case} and \code{default} statements occur directly in the 
sequence of contained statements. These \code{switch} statements have the form
\begin{verbatim}
  switch (e) {
    ...
  case c1: s11 ... s1k1
  ...
  case cn: sn1 ... snkn
  default: s01 ... s0k0
  }
\end{verbatim}
where the \code{default} statement may have any other position or may be omitted. The C standard requires that all \code{ci}
constants are pairwise distinct. In practical applications most \code{switch} statements are of this form. 

If there are declarations among the \code{sij} or before the first \code{case} statement the \code{switch} statement may cause
a jump past the declaration which means the declared object exists but is not initialized, which is also difficult to transfer.
Therefore we assume that there are no declarations in the body.

Then the \code{switch} statement of the form shown above can be rewritten to the following statement
\begin{verbatim}
  { t v = e;
    if (v == c1) {s11 ... s1k1}
    ...
    if (v == c1 || ... || v == cn) {sn1 ... snkn}
    {s01 ... s0k0}
  }
\end{verbatim}
and can be translated as described in the corresponding sections. Here \code{v} is an otherwise unused variable and \code{t} is
the type of \code{e}. It is needed even if \code{e} is a single variable, because this variable could be modified in some cases
but in the following case conditions the original value must be used.

Note that the additional disjunctions and the omitted condition for the default part cover the feature of ``fall through'' 
in a \code{switch} statement: the execution of following cases is only prevented by an explicit \code{break} or \code{return}
statement. If the \code{default} statement is placed before other \code{case} statements it also needs a condition which prevents
its execution for the cases following it.

Since \code{break} statements which are not enclosed by a nested loop or \code{switch} statement are consumed by the \code{switch}
statement and not propagated to the context, the control flow tuple must be reset by the switch statement. If it is bound to
\code{(cc,cb,res)} by the body, it must be rebound by the \code{switch} statement to
\begin{verbatim}
  (res /= None || (cc && not cb),res /= None, res)
\end{verbatim}
which will only preserve the effect of \code{return} and \code{continue} statements. A \code{continue} statement does not 
affect the \code{switch} statement but may be present and affect a surrounding loop.

As an example, the \code{switch} statement
\begin{verbatim}
  switch (i) {
    case 1: return 0;
    case 2: i++;
    case 3: return i+10;
    default: return i;
  }
\end{verbatim}
can be rewritten to
\begin{verbatim}
  {
    if (v==1) return 0;
    if (v==1 || v==2) i++;
    if (v==1 || v==2 || v==3) return i+10;
    return i;
  }
\end{verbatim}
which is translated to
\begin{verbatim}
  (c,i) = let v = i in
    let (cc,cb,res) = 
      if v==1 then (True,True,Some 0)
              else (False,False,None)
    in if cc then ((cc,cb,res),i) else 
      let ((cc,cb,res),i) = 
        if v==1 || v==2 then ((False,False,None),i+1)
                        else ((False,False,None),i)
      in if cc then ((cc,cb,res),i) else 
        let c = 
          let (cc,cb,res) = 
            if v==1 || v==2 || v==3 
            then (True,True,Some i+10)
            else (False,False,None)
          in if cc then (cc,cb,res)
                   else (True,True,Some i)
        in (c,i)
\end{verbatim}
which could be simplified to
\begin{verbatim}
  (c,i) = 
    ((True,True,
      if i==1 then Some 0
      else if i==2 then Some 13
      else if i==3 then Some 13
      else Some i),
     if i==2 then 3 else i)
\end{verbatim}

\subsubsection{For Loops}

A \code{for} loop has the form \code{for (ee1; ee2; ee3) s} with optional expressions \code{eei} and a statement \code{s}. 
The first expression \code{ee1} can also be a declaration. The \code{for} loop can always be transformed to a \code{while}
loop of the form
\begin{verbatim}
  ee1;
  while (ee2) { s ee3; }
\end{verbatim}
therefore \code{ee2} must evaluate to a scalar value, as for the condition in a conditional expression or statement.

Since \code{while} loops are not directly supported by Cogent and need additional work for proofs when translated by using
an abstract function, we do not use this transformation in general, instead, for certain forms of \code{for} loops we
translate them using functions from the Cogent standard library. This is possible, whenever a \code{for} loop is a 
``counted'' loop of the form
\begin{verbatim}
  for ( w=e1; w<e2; w=e3) s
\end{verbatim}
with a counting variable \code{w} of an unsigned integer type. The only free variable in \code{e3} must be \code{w} and
\code{w} must not occur free in \code{e2}. The \code{e1, e2, e3} must not have side effects.

If a \code{for} loop is not of this form it may be possible to rewrite it in C to that form. Specifically, this can be done 
if it is possible to rewrite the expressions \code{eei} as follows:
\begin{verbatim}
  ee1: ee1', w=e1
  ee2: w<e2 && ee2'
  ee3: ee3', w=e3
\end{verbatim}
Then the loop \code{for (ee1; ee2; ee3) s} can be rewritten as the counted loop
\begin{verbatim}
  ee1';
  for (w=e1; w<e2; w=e3) {
    if (!ee2') break;
    s ee3'
  }
\end{verbatim}
In other cases, especially if some of the \code{eei} are empty, it may still be possible to rewrite the loop as counted by
moving parts from before the loop or from the loop body into the \code{ei}.

Let \code{expri} be the translation of the \code{ei} and let \code{(c,v1,...,vn) = expr} be the translation of 
statement \code{s}. Let \code{w1,...,wm} be the variables occurring free in \code{expr} other than \code{w} and the \code{vi}.
Then the translation of the counted loop 
\begin{verbatim}
  for (w=e1; w<e2; w=e3) s
\end{verbatim}
is
\begin{verbatim}
  (c, v1,...,vn) = 
  let ((v1,...,vn),lr) = 
    seq32_stepf #{
      frm = expr1, to = expr2, 
      stepf = \w => expr3, 
      acc = (v1,...,vn), obsv = (w1,...,wm), 
      f = \#{acc = (v1,...,vn), obsv = (w1,...,wm), 
             idx = w} => 
        let ((cc,cb,res),v1,...,vn) = expr
        in ((v1,...,vn),
            if cb then Break (res) else Iterate ())
      }
  in lr | Iterate () -> ((False,False,None), v1,...,vn)
        | Break (None) -> ((False,False,None), v1,...,vn)
        | Break (res) -> ((True,True,res), v1,...,vn)
\end{verbatim}

The function \code{seq32\_stepf} is defined in the Cogent standard library in \code{gum/common/common.cogent}.

The variables \code{w,w1,...,wm,v1,...,vn} either occur free in the loop body or are modified in the loop body.
In both cases they must be declared in C before the counted loop, so they are defined in Cogent and can be used
when \code{acc} and \code{obsv} are set. The \code{w1,...,wm} must not be of linear type. Since they are used
in the loop body, depending on how often the body is executed they could either be used several times or not at all.
If a \code{wi} is of linear type the loop cannot be translated in this way and must be handled manually.

Since all \code{w1,...,wm} have a regular or readonly type the tuple \code{(w1,...,wm)} is a valid value for 
both \code{obsv} fields which have the readonly type \code{obsv!}.

The translation correctly handles all \code{return}, \code{break}, and \code{continue} statements which may occur 
in the body. 

Note that the translation is only possible if the type of the counting variable \code{w} can be represented in 
Cogent by the type \code{U32}.

Since \code{seq32\_stepf} is an abstract function the Cogent compiler does not generate a refinement proof for it.
The refinement proof for \code{seq32\_stepf} can only be successful if the loop terminates. This depends on the 
expression \code{e3}. As usual, if it strictly counts the variable \code{w} up, and some additional assumptions 
about the modulo calculation for \code{U32} are valid, termination can be proved.

If the expression \code{w=e3} can be rewritten as \code{w+=e3'} where \code{w} does not occur free in \code{e3'}
the counted loop can be translated using 
function \code{seq32} instead of \code{seq32\_stepf}. The translation has the same form as above, with the line 
setting \code{stepf} replaced by
\begin{verbatim}
      step = expr3', 
\end{verbatim}
where \code{expr3'} is the translation of \code{e3'}. Here, arbitrary other variables may occur free in \code{expr3'}.

In this case the counting variable \code{w} may also have a type which can be represented in Cogent by the type \code{U64}.
Then the translation has the same form but uses the function \code{seq64} instead of \code{seq32}.

If the expression \code{w=e3} can be rewritten as \code{w-=e3'} where \code{w} does not occur free in \code{e3'}
and the expression \code{ee2} can be rewritten as \code{w>e2 \&\& ee2'}
the counted loop can be translated as for \code{w+=e3'} using the function \code{seq32\_rev} instead of \code{seq32}.

If the translation can be done using \code{seq32} and the body does not contain any \code{break} or \code{return} statement 
the translation can be simplified using the function \code{seq32\_simpl} in the form
\begin{verbatim}
  (c, v1,...,vn) = 
  let w = expr1
  and (v1,...,vn,w1,...,wm,w) = 
    seq32_simple #{
      frm = w, to = expr2, step = expr3', 
      acc = (v1,...,vn,w1,...,wm,w), 
      f = \(v1,...,vn,w1,...,wm,w) => 
        let (_,v1,...,vn) = expr
        in (v1,...,vn,w1,...,wm,w+expr3')
      }
  in ((False,False,None), v1,...,vn)
\end{verbatim}
Note that \code{seq32\_simple} does not pass the counting variable \code{w} to the body, so we must add it to \code{acc}
and count it explicitly in \code{f}.
If the counting variable \code{w} is not used in the body (does not occur free in \code{expr}) this can further be simplified to
\begin{verbatim}
  (c, v1,...,vn) = 
  let (v1,...,vn,w1,...,wm) = 
    seq32_simple #{
      frm = expr1, to = expr2, step = expr3', 
      acc = (v1,...,vn,w1,...,wm), 
      f = \(v1,...,vn,w1,...,wm) => 
        let (_,v1,...,vn) = expr
        in (v1,...,vn,w1,...,wm)
      }
  in ((False,False,None), v1,...,vn)
\end{verbatim}

As an example, the counted loop
\begin{verbatim}
  for (i=0; i<size; i++) { 
    sum += a[i];
  }
\end{verbatim}
is translated to
\begin{verbatim}
  (c, sum) = 
  let i = 0
  and (sum,a,i) = 
    seq32_simple #{
      frm = i, to = size, step = 1, 
      acc = (sum,a,i), 
      f = \(sum,a,i) => 
        let sum = sum + getArr(a,i)
        in (sum,a,i+1)
      }
  in ((False,False,None), sum)
\end{verbatim}
where \code{a} must have a readonly Gencot array type. 

For the counted loop
\begin{verbatim}
  for (i=0; i<size; i++) { 
    if (a[i] == trg) return i;
  }
\end{verbatim}
the body translates to 
\begin{verbatim}
  c = if getArr(a,i) == trg 
      then (True,True,Some i)
      else (False,False,None)
\end{verbatim}
where again \code{a} must be of a readonly Gencot array type. The translation of the loop is
\begin{verbatim}
  c = 
  let ((),lr) = 
    seq32 #{
      frm = 0, to = size, step = 1, 
      acc = (), obsv = (a,trg), 
      f = \#{acc = (), obsv = (a,trg), idx = i} => 
        let (cc,cb,res) =
          if getArr(a,i) == trg 
          then (True,True,Some i)
          else (False,False,None)
        in ((),if cb then Break (res) else Iterate ())
      }
  in lr | Iterate () -> (False,False,None)
        | Break (None) -> (False,False,None)
        | Break (res) -> (True,True,res)
\end{verbatim}
which can be simplified to
\begin{verbatim}
  c = 
  let ((),lr) = 
    seq32 #{
      frm = 0, to = size, step = 1, 
      acc = (), obsv = (a,trg), 
      f = \#{obsv = (a,trg), idx = i} => 
        ((),if getArr(a,i) == trg 
            then Break (Some i) 
            else Iterate ())
      }
  in lr | Iterate () -> (False,False,None)
        | Break (None) -> (False,False,None)
        | Break (res) -> (True,True,res)
\end{verbatim}

Since end of 2021 there is a new abstract function \code{repeat} for loops in Cogent for which a general refinement proof 
has been developed. It is intended to be added to the Cogent standard library. It is even more general than \code{seq32\_stepf} 
and can be used to translate all \code{for} loops for which an upper limit of the number of iterations can be calculated.
This includes the terminating counted \code{for} loops described above.

A general \code{for} loop \code{for (ee1; ee2; ee3) s} where \code{ee2} has no side effects is first rewritten to 
\begin{verbatim}
  ee1;
  for (; ee2; ee3) s
\end{verbatim}
by moving the initialization expression or declaration before the loop.

Let \code{expr2} be the translation of \code{ee2} and let \code{(c,v1,...,vn) = expr} be the translation of 
the statement sequence \code{\{s ee3\}}. Let \code{w1,...,wm} be the variables occurring free in \code{expr} other than 
\code{c} and the \code{vi}.
Then the translation of the remaining loop 
\begin{verbatim}
  for (; ee2; ee3) s
\end{verbatim}
is
\begin{verbatim}
  (c, v1,...,vn) =
  let ((_,_,res),v1,...,vn) = repeat #{
    n = exprmax,
    stop = \#{acc = ((_,cb,_),v1,...,vn), 
              obsv = (w1,...,wm)} => cb || not expr2,
    step = \#{acc = (_,v1,...,vn), 
              obsv = (w1,...,wm)} => expr
    acc = ((False,False,None),v1,...,vn), obsv = (w1,...,wm)
    }
  in ((res /= None,res /= None,res),v1,...,vn)
\end{verbatim}
where \code{exprmax} is an expression for calculating the upper limit of the number of iterations. The \code{repeat}
function is defined in \code{C} using a \code{for} loop of the form \code{for (i=0; i<n; i++)}, so it iterates atmost
\code{n} times. If the loop has the form \code{for (w=e1;w<e2;w+=e3') s} where \code{w} does not occur free in \code{e1},
\code{e2}, and \code{e3'} the expression \code{exprmax} can be constructed by translating the \code{C} expression
\begin{verbatim}
  (e2 - e1) / e3'
\end{verbatim}
In other cases the expression for the upper limit must be determined manually.

The \code{repeat} function passes an ``accumulator'' \code{acc} and an ``observed object'' \code{obsv} through all
iterations, were \code{acc} may be modified and \code{obsv} not. The \code{repeat} function takes as arguments the 
initial values of \code{acc} and \code{obsv} and two functions \code{stop} and \code{step} which both have \code{acc}
and \code{obsv} as arguments. The \code{stop} function is used before each iteration to determine whether a next iteration
should be executed, it returns a boolean value whether to stop.  The \code{step} function covers the effect of one loop iteration and returns the 
new \code{acc} value. In the translation the \code{acc} value includes the control variable \code{c}. The \code{stop}
function inspects the value of its \code{cb} component to detect whether a \code{break} or \code{return} statement
has been executed in the body and terminates the loop in that case.

Note that this translation may not be fully equivalent to the original C code. If the expression \code{e3'} evaluates 
to a value greater 1, the increment of \code{w} may cause an overflow of the word arithmetics, even if \code{w<e2}
holds before. Then the loop will not terminate in that step and possibly never. The translation described above will
instead terminate safely after \code{n} iterations. Since this situation is normally a programming error in C
this is not viewed as a problem.

\subsubsection{While Loops}

A \code{while} loop has the form \code{while (e) s} or \code{do s while (e);}. Since Cogent does not support recursive 
function definitions these loops cannot be translated in a natural way to Cogent.

To translate a \code{while} loop it may be rewritten in C as a counted \code{for} loop which can then be translated 
as described in the previous section. If that is not possible the loop must be put in an abstract function which is 
defined in C using the original \code{while} loop.

Alternatively, if an upper limit of the number of iterations can be determined, the loop can be translated using the 
abstract \code{repeat} function described above as follows.

Let \code{expr1} be the translation of \code{e} and let \code{(c,v1,...,vn) = expr2} be the translation of 
statement \code{s}. Let \code{w1,...,wm} be the variables occurring free in \code{expr1} and \code{expr2} other than 
\code{c} and the \code{vi}.
Then the translation of the loop \code{while (e) s} is
\begin{verbatim}
  (c, v1,...,vn) = 
  let ((_,cb,_),v1,...,vn) = repeat #{
    n = exprmax,
    stop = \#{acc = ((_,cb,_),v1,...,vn), 
              obsv = (w1,...,wm)} => cb || not expr1,
    step = \#{acc = (_,v1,...,vn), 
              obsv = (w1,...,wm)} => expr2
    acc = (c,v1,...,vn), obsv = (w1,...,wm)
    }
  in ((res /= None,res /= None,res),v1,...,vn)
\end{verbatim}
where \code{exprmax} is an expression which calculates an upper limit for the number of loop iterations.

\subsection{Dummy Expressions}
\label{design-cstats-dummy}

Dummy expressions are generated when Gencot cannot translate a C declaration, statement, or expression. It consist of 
a Cogent expression of the form
\begin{verbatim}
  gencotDummy "... reason for not translating ..."
  {- ... untranslated C code ... -}
\end{verbatim}
(see Section~\ref{design-operations-dummy}) with the C code contained as a comment.

The C code is intended for the programmer who has to deal with it manually. Therefore it should
be readable, even if it is a larger C fragment. Therefore Gencot provides the following features for it:
\begin{itemize}
\item generate origin markers so that comments and preprocessor directives are inserted,
\item map C names to Cogent names so that they can be related to the surrounding Cogent code.
\end{itemize}

\subsubsection{Origin Markers}

To preserve comments embedded in the embedded C code it is also considered as a structured source code part
and origin markers are inserted around its subparts. Its 
subpart structure corresponds to the syntactic structure of the C AST. Since in the target code only identifiers 
are substituted, the target code
structure is the same as that of the source code. The structure is only used for identifying and re-inserting
the transferrable comments and preprocessor directives. Note that this works only if the conditional directive 
structure is compatible with the syntactic structure, i.e., a group must always contain a complete syntactical
unit such as a statement, expression or declaration, which is the usual case in C code in practice.

An alternative approach would be to treat all nonempty source code lines as subparts of a function body, resulting
in a flat sequence structure of single lines. The advantage is that it is always compatible to the conditional 
directive structure and
all comment units would be transferred. However, generating the corresponding origin markers in an abstract syntax
tree is much more complex than generating them for syntactical units for which the origin information is present
in the syntax tree. Since the Gencot implementation generates the
target code as an abstract syntax tree, the syntactical statement structure is preferred. 

\subsubsection{Mapping Names}

To relate the names in the C code to the Cogent code, Gencot substitutes names occurring free in the C code. These may
be names with global scope (for types, functions, tags, global variables, enum constants or preprocessor constants)
or names of parameters and local variables. For all names with global scope Gencot has generated a Cogent definition using a mapped name.
These names are substituted in the C code of the function body by the corresponding mapped names so that the 
mapping need not be done manually by the programmer.

Names with local scope (either local variables or types/tags) are also mapped by Gencot, both in their (local)
definition and whenever they are used. 

The mapping does not include the mapping of derived types. Type derivation in C is done in declarators which refer
a common type specification in a declaration. In Cogent there is no similar concept, every declarator must be 
translated to a separate declaration. This is not done for embedded C code. As result, an embedded
local declaration may have the form
\begin{verbatim}
  Struct_s1 a, *b, c[5];
\end{verbatim}
although the Cogent types for \code{a, b, c} would be \code{\#Struct\_s1}, \code{Struct\_s1}, and 
\code{\#(CArr5 Struct\_s1)}, respectively. This translation for the derived types must be done manually.

Global variables referenced in a function body are treated in a specific way. If the global variable has a 
Global-State property (see Section~\ref{design-types-itemprops}) and a virtual parameter item with the same
property has been declared for the function, the identifier of the global variable is replaced by the 
mapped parameter name with the dereference operator applied. This corresponds to the intended purpose of the 
Global-State properties: instead of accessing global variables directly from within a function, a pointer is 
passed as parameter and the variable is accessed by dereferencing this pointer.

If a global variable has the Const-Val property (see Section~\ref{design-types-itemprops}) its access is
replaced by an invocation of the corresponding parameterless access function. If the global variable has 
neither of these properties declared, Gencot simply uses its mapped name, this case must be handled manually.



\section{Function Parameter Modifications}
\label{design-parmod}
 


\section{Extending the Isabelle Code}
\label{design-isabelle}
After Gencot has translated C code to Cogent, the Cogent compiler will generate a logic
based representation of the code and a proof that the generated C code refines that representation.
Both the representation (``shallow embedding'') and the proofs are generated in Isabelle notation. 

However, the abstract types and functions used in the Cogent code are not covered by this Isabelle 
code. Since Gencot uses several specific abstract types and functions to represent the C types (as 
described in Sections~\ref{design-types} and~\ref{design-operations}), the corresponding representations
and refinement proofs must be added to the Isabelle code to fully cover a C program translated by 
Gencot. Since the semantics of these types and functions is defined by Gencot, Gencot is able to 
automatically generate these parts of the Isabelle code.

\subsection{Representing Gencot Types}
\label{design-isabelle-types}

Specific treatment is necessary whenever Gencot uses abstract types to represent a C type in Cogent.
Most C types are mapped to Cogent using the normal Cogent types in specific ways. In these cases the 
Cogent compiler already generates corresponding Isabelle code. However, also in some of these cases 
Gencot modifies that code.

\subsubsection{Isabelle Types Generated by Cogent}

The Isabelle code for representing all Cogent types is generated by Cogent in a file \code{X\_ShallowShared\_Tuples.thy}
where \code{X} is a program specific prefix. The code consists of
\begin{itemize}

\item a type declaration for every abstract Cogent type, using the same type name as in Cogent.

\item a record definition for every equivalence class of record types occurring in the Cogent program. Two record
types are equivalent, if they have the same field names in the same order. If there is a type synonym defined
for the record type in Cogent (one of) the synonym name is used for the record, otherwise a generated name of
the form \code{T<nr>} is used. For every Cogent field name \code{x} the name $\code{x}_\code{\small{f}}$ is used in Isabelle.

Additionally, Cogent tuple types are represented by records with field names
\code{p1}, \code{p2}, ... and the corresponding records are defined. The record for pairs always has the name
\code{RR}, the names for the other tuple records are determined as described above.

\item a datatype definition for every variant type occurring in the Cogent program. The type name is determined 
as for the records. As names for the alternative constructors the Cogent tag names are used.

\item a type synonym definition for every Cogent type definition. For every Cogent type name \code{X} the
type synonym $\code{X}_\code{\small{T}}$ is defined in Isabelle.

\end{itemize}

Since Gencot uses tuple types in its predefined operations, it is necessary to know the names of the records for
representing tuples in Isabelle in advance, so that Isabelle specifications can be defined by Gencot for these 
operations. Gencot uses the include file \code{Tuples.cogent} which contains type synonyms \code{Tup<nr>} for 
the tuple types from 3 to 9 components. Whenever another Gencot include file uses tuple types with more than 
two components it includes this file, so that the tuple records in Isabelle are named accordingly.

\subsubsection{Gencot Array Types}

\subsubsection{Type \code{MayNull}}

\subsection{Specifying Gencot Operations}
\label{design-isabelle-operations}

\subsubsection{Gencot Array Operations}

\subsubsection{Operations for Type \code{MayNull}}

\subsection{Specifying Cogent Operations}
\label{design-isabelle-cogentops}

Cogent defines several abstract function in its standard library. However, as of February 2021 the Cogent distribution
does not contain Isabelle specifications for them. Therefore Gencot provides Isabelle specifications for all such
functions which are used by Gencot.

\subsubsection{Iteration Funktions}

\subsection{Refinement Proof for Gencot Operations}
\label{design-isabelle-gencotprf}

\subsection{Refinement Proof for Cogent Operations}
\label{design-isabelle-cogentprf}


\chapter{Implementation}

Gencot is implemented by a collection of unix shell scripts using the unix tools \code{sed}, \code{awk}, and the 
C preprocessor \code{cpp} and by Haskell programs using the C parser \code{language-c}. 

Many steps are implemented as Unix filters, reading from standard input and writing to standard output. A filter
may read additional files when it merges information from several steps. The filters
can be used manually or they can be combined in scripts or makefiles. Gencot provides predefined scripts
for filter combinations.

\section{Origin Positions}
\label{impl-origin}

Since the \code{language-c} parser does not support parsing preprocessor directives and C comments, the general approach
is to remove both from the source file, process them separately, and re-insert them into the generated files.

For re-inserting it must be possible to relate comments and preprocessor directives to the generated target code parts. As 
described in Sections~\ref{design-comments} and~\ref{design-preprocessor}, comments and preprocessor directives are associated
to the C source code via line numbers. Whenever a target code part is generated, it is annotated with the line numbers of its
corresponding source code part. Based on these line number annotations the comments and preprocessor directives can be
positioned at the correct places.

The line number annotations are markers of one of the following forms, each in a single separate line:
\begin{verbatim}
  #ORIGIN <bline>
  #ENDORIG <aline>
\end{verbatim}
where \code{<bline>} and \code{<aline>} are line numbers.

An \code{\#ORIGIN} marker specifies that the next code line starts a target code part which was generated from a source code
part starting in line \code{<bline>}. An \code{\#ENDORIG} marker specifies that the previous code line ends a target code part
which was generated from a source code part ending in line \code{<aline>}. Thus, by surrounding a target code part with an 
\code{\#ORIGIN} and \code{\#ENDORIG} marker the position and extension of the corresponding source code part can be derived.

In the case of a structured source code part the origin marker pairs are nested, if the target code part generated from a subpart 
is nested in the target code part generated from the main part. If there is no code generated for the main part, the 
\code{\#ORIGIN} marker for the first subpart immediately follows the \code{\#ORIGIN} marker for the main part and the 
\code{\#ENDORIG} marker for the last subpart is immediately followed by the \code{\#ENDORIG} marker for the main part.

If no target code is generated from a source code part, the origin markers are not present. This implies, that an 
\code{\#ORIGIN} marker is never immediately followed by an \code{\#ENDORIG} marker.

It may be the case that several source code parts follow each other on the same line, but the corresponding target code
parts are positioned on different lines. Or from a single source code part several target code parts on different lines 
are generated. In both cases there are several origin markers with the same line number. Conditional preprocessor directives
associated with that line must be duplicated to all these target code parts. For comments, however, duplication is not
adequate, they should only be associated to one of the target code parts. This is implemented by appending an additional 
``+'' sign to an origin marker, as in 
\begin{verbatim}
  #ORIGIN <bline> +
  #ENDORIG <aline> +
\end{verbatim}
Comments are only associated with markers where the ``+'' sign is present, all other markers are ignored. In this way,
the target code generation can decide where to associate comments, if a position is not unique.

Gencot uses the filter \code{gencot-reporigs} for removing repeated origin markers from the generated target code, as 
described in Section~\ref{impl-ccode-origin}.


\section{Item Properties}
\label{impl-itemprops}
 


\section{Parameter Modification Descriptions}
\label{impl-parmod}
As described in Section~\ref{design-parmod} Gencot uses a parameter modification description in JSON format 
to be filled in collaboration with the developer to determine which function parameters may be modified during
a function call. 

\subsection{Description Structure}
\label{impl-parmod-struct}

This
description is structured as follows. It is a list of JSON objects, where each object is an entry which describes a function using
the following attributes:
\begin{verbatim}
    { "f_name" : <string>
    , "f_comments" : <string>
    , "f_def_loc" : <string>
    , "f_num_params" : <int> or <string>
    , "f_result" : <string>
       <parameter descriptions>
    , "f_num_invocations" : <int>
    , "f_invocations" : <list of invocation descriptions>
    }
\end{verbatim}

The attribute \code{f\_name} specifies a unique identifier for the function, as described in Section~\ref{impl-parmod-ids}.
The attribute \code{f\_comments} is not used by Gencot, it can be used by the developer to add arbitrary textual
descriptions to the function entry. The attribute \code{f\_def\_loc} specifies the name of the C source file containing
the definition of the function or functionn pointer (or its declaration, when the function entry has been 
generated for closing the JSON description). 

The attribute \code{f\_num\_params} specifies the number of parameters. In the case of a variadic function or a function
with incomplete type (which may be the case if the entry has been generated from a declaration), it is specified
as \code{"variadic"} or \code{"unknown"}, respectively. The attribute \code{f\_result} specifies the identifier of
the parameter which is returned as function result (typically afer modifying it). If the result is not one of the 
function parameters, the attribute is not present. This attribute is never generated by Gencot, it must be added manually
by the developer. 

All known parameters of the function are described in the \code{<parameter descriptions>}. 
Every parameter description consists of a single attribute where the parameter identifier (see Section~\ref{impl-parmod-ids})
is the attribute name. The value is one of the following strings:
\begin{description}
\item[\code{"nonlinear"}] According to its type the parameter is not a pointer and its value does not contain pointers directly or
indirectly.
\item[\code{"readonly"}] The parameter is not \code{"nonlinear"} but according to its type all pointers in the parameter have a \code{const} qualified referenced type.
\item[\code{"yes"}] The parameter is neither \code{"nonlinear"} nor \code{"readonly"} and it is directly modified by the function.
\item[\code{"discarded"}] The parameter is neither \code{"nonlinear"} nor \code{"readonly"} and it is directly discarded 
(``freed'') by the function. 
\item[\code{"depends"}] The parameter is neither \code{"nonlinear"} nor \code{"readonly"}, it is neither modified 
or discarded directly, but it may be modified by an invoked function. 
\item[\code{"no"}] None of the previous cases applies to the parameter.
\item[\code{"?"}] The parameter is neither \code{"nonlinear"} nor \code{"readonly"}, but the remaining properties are unconfirmed.
\item[\code{"?depends"}] The parameter is neither \code{"nonlinear"} nor \code{"readonly"} and it may be modified by an invoked function,
but the remaining properties are unconfirmed.
\end{description}

Gencot only generates parameter descriptions with the values \code{"nonlinear"}, \code{"readonly"}, \code{"yes"}, \code{"?"}, and \code{"?depends"}.
The first two can be safely determined from the C type. If Gencot finds a direct modification, it sets the description to 
\code{"yes"}. Otherwise, if it finds a dependency on an invocation, it sets the description to \code{"?depends"}. Otherwise
it sets it to \code{"?"}.

The task for the developer is to check all unconfirmed parameter descriptions by inspecting the source code. If a local modification
is found, the value must be changed to \code{"yes"}. Otherwise, if the description was \code{"?depends"} it must be confirmed 
by changing it to \code{"depends"}. Otherwise, if a dependency is found, the value \code{"?"} must be changed to \code{"depends"},
otherwise it must be set to \code{"no"}.

Discarding a parameter is normally only possible by invoking the function ``free'' in the C standard library. No other function
can directly discard one of its parameters. However, a parameter may be discarded by invoking an external function which indirectly
invokes free. The task for the developer is to identify all these cases where an external function (where the entry has been
generated during closing the JSON description) discards one of its parameters and set its parameter description to \code{"discarded"}.

It may be the case that a C function modifies a value referenced by a parameter and returns the parameter as its result, such as 
the C standard function \code{memcpy}. In this case the parameter need not be added to the Cogent result tuple, since it already
is a part of it. To inform Gencot about this case the developer has to add a \code{f\_result} attribute to the function
description. As an example, the description for the C function \code{memcpy} should be
\begin{verbatim}
    { "f_name" : "memcpy"
    , "f_comments" : ""
    , "f_def_loc" : "string.h"
    , "f_num_params" :3
    , "f_result" :"1-__dest"
    , "1-__dest" :"yes" 
    , "2-__src" :"readonly" 
    , "3-__n" :"nonlinear" 
    }
\end{verbatim}
since it always returns its first parameter as result.

The attribute \code{f\_num\_invocations} specifies the number of function invocations found in the body of the described function.
If the same function is invoked several times, it is only counted once. The attribute \code{f\_invocations} specifies a list
of JSON objects where each object describes an invocation. If the function entry describes a function pointer or an external
function, no body is available, so no invocations can be found and both attributes are omitted. If no parameter depends on
any invocation, the second attribute (invocation list) is omitted, only the number of invocations is specified.

An entry in the list of invocations describes an invocation using the following attributes:
\begin{verbatim}
    { "name" : <string> 
    , "num_params" : <int> or <string>
      <argument descriptions>
    }
\end{verbatim}

The attribute \code{"name"} specifies the function identifier of the invoked function. The attribute \code{"num\_params"}
specifies the number of parameters according to the type of the invoked function, in the same way as the attribute
\code{"f\_num\_params}. If an invoked function has no parameters, no entry for it is added to the invocation list.

The \code{<argument descriptions>} describe all known arguments of invocations of the function. When Gencot creates an
invocation description, it inserts argument descriptions according to the maximal number of arguments found in an 
invocation of this function. Thus, also for invocations of incompletely defined or variadic functions, an argument
description is present for every actual argument used in an invocation.

Every argument description consists of a single attribute where the attribute name is the parameter identifier of the 
parameter corresponding to the argument. The value is one of the strings \code{"nonlinear"} or \code{"readonly"}
(according to the type of the parameter of the invoked function), or a list of parameter identifiers.

If the value is \code{"nonlinear"} or \code{"readonly"} parameter dependencies on this argument are irrelevant, since
the invoked function cannot modify or discard it. Otherwise, the list specifies parameters of the \textit{invoking}
function for which the modification or discarding depends on whether the invoked function modifies or discards this
argument.

The task for the developer is for all unconfirmed parameter descriptions of the invoking function to check whether
there are (additional) dependencies on arguments of invoked functions and add these dependencies to the argument 
descriptions.

\subsection{Identifiers for Functions and Parameters}
\label{impl-parmod-ids}

In the JSON description unique identifiers are needed for all described functions, so that they can be referenced by
invocations. Gencot uses the item identifiers as defined in Section~\ref{impl-itemprops-ids} as function identifiers.
An item identifier is a function identifier, if the item has a derived function type.

In the JSON description unique identifiers are also needed for all described function parameters, so that they can 
be referenced by argument descriptions. Since these references are always local in a function description, parameter
identifiers need only be unique per function. Therefore the C parameter name is usually sufficient as id in the 
JSON description.

However, the parameter name is not always available: If the function has an incomplete type no parameter names are
specified, if the function is variadic, only the names of the non-variadic parameters are specified. Therefore
Gencot always uses the position number as parameter id, where the first parameter has position 1. To make the JSON
description more readable for the developer, Gencot appends the parameter name whenever it is available. Together,
a parameter id is a string with one of the forms
\begin{verbatim}
    <pos>
    <pos>-<name>
\end{verbatim}
where \code{<pos>} is the position number and \code{<name>} is the declared parameter name.

Since all parameter identifiers are strings they can be used as JSON attribute names and since they always
start with a digit they can be recognized and do not conflict with other JSON attribute names. 

When Gencot reads and processes a JSON parameter modification description it removes the optional name from 
all parameter identifiers and uses only the position. Hence parameter identifiers are considered equal if they
begin with the same position.

\subsection{Example}

The following example illustrates the format of the parameter modification descriptions. It consists of two
function descriptions, one for a defined function with internal linkage and one for a function pointer parameter
invoked by that function.
\begin{verbatim}
[
    { "f_name" :"app:somefun" 
    , "f_comments" :"" 
    , "f_def_loc" :"app.c" 
    , "f_num_params" :5
    , "1-f_sort" :"nonlinear" 
    , "2-desc" :"readonly" 
    , "3-input" :"?depends" 
    , "4-size" :"nonlinear" 
    , "5-result" :"yes" 
    , "f_num_invocations" :2
    , "f_invocations" :
        [ 
            { "name" :"memcpy" 
            , "num_params" :3
            , "1-__dest" :[ "3-input" ]
            , "2-__src" :"readonly" 
            , "3-__n" :"nonlinear" 
            } 
        , 
            { "name" :"app:somefun/f_sort/*" 
            , "num_params" :3
            , "1-arr" :[ "3-input" ]
            , "2-h" :[]
            , "3-len" :"nonlinear" 
            } 
        ] 
    } 
, 
    { "f_name" :"app:somefun/f_sort/*"
    , "f_comments" :"" 
    , "f_def_loc" :"app.c" 
    , "f_num_params" :3
    , "1-arr" :"?" 
    , "2-h" :"?" 
    , "3-len" :"nonlinear" 
    } 
]
\end{verbatim}
The description is not closed, since it does not contain an entry for the invocation \code{memcpy}. This
invocation is required, since parameter \code{input} of \code{somefun} depends on its first argument
which is neither nonlinear nor readonly. If the description of parameter \code{input} is changed
to \code{"yes"}, the description is closed, since now the invocation \code{memcpy} is not required
anymore.

\subsection{Reading and Writing Json}
\label{impl-parmod-json}

Input and output of JSON data is done using the package \code{Text.Json}. There is another package \code{Data.Aeson} for
Json processing which supports a much more flexible conversion between JSON and Haskell types, but this is not
needed here.

The package \code{Text.Json} uses the type \code{JSValue} to represent an arbitrary JSON value. It provides the functions
\code{encode} and \code{decode} to convert between \code{JSValue} and the JSON string representation. Depending
on the kind of actual value, a \code{JSValue} can be converted from and to corresponding Haskell types using
the functions \code{readJSON} and \code{showJSON}. A JSON object is converted to the type \code{JSObject a} where
\code{a} is the type of attribute values, usually this is again \code{JSValue}. A \code{JSObject a} can be further
processed by converting it from and to an attribute-value list of type \code{[(String,a)]} using the functions
\code{fromJSobject} and \code{toJSObject}.

The Gencot parameter modification description is represented as a list of Json objects of type \code{[JSObject JSValue]}.
The same representatiom is used for the contained invocation description lists. For internal processing the objects
are converted to attribute-value lists. All processing of JSON data is directly performed on these lists.

A parameter modification description is read by applying the \code{decode} function to the input to yield a
list of \code{JSObject JSValue}. It is output by applying the \code{encode} function to such a list. 

Additionally,
the resulting string representation is formatted using the general prettyprint package \code{Text.Pretty.Simple}.
The function \code{pStringNoColor} is used since it does not insert control sequences for colored represenation
and produces a plainly formatted text representation. Its result is of type \code{Text} and must be converted to a
string using the function \code{unpack} from package \code{Data.Text}.

\subsection{Evaluating a Description}
\label{impl-parmod-eval}

A parameter modification description is complete, if all paremeter descriptions are confirmed and the description 
is ``closed'', i.e., has no required invocations. This implies that whenever a parameter is dependent on an invocation argument,
there is a function description present for the invoked function where the corresponding parameter is described.
This allows to eliminate all parameter dependencies by following them until an independent parameter description is found.
This process is called ``evaluation'' of the parameter modification description.

The result of evaluation is a simplified parameter modification description where the value \code{"depends"} does not 
occur anymore as parameter description. Additionally, all information about function invocations is removed, since it is
no more needed.

Evaluation only terminates, if there are no cyclic parameter dependencies. This property is checked by Gencot. If there
are cyclic parameter dependencies in the C code they must be eliminated manually by the developer by removing enough dependencies 
to break all cycles.

A parameter may have several dependencies, which may result in different description values. This cannot be the case for
the values \code{"nonlinear"} and \code{"readonly"}: If the parameter of the invoked function has nonlinear type, this
also holds for the passed parameter, no dependency can exist in this case. If the parameter of the invoked function has
a readonly type, the passed parameter can still have a linear type which is not readonly. But it cannot be modified by the
function invocation, hence no dependency can exist in this case either.

Thus, after evaluation, a parameter may have any subset of the values \code{"yes"}, \code{"discarded"}, and \code{"no"}, meaning 
that it may be modified by some invocations, discarded by some invocations, and not modified by others. This is reduced to
a single value as follows. The value \code{"no"} is only used if none of the other two is present. If both \code{"yes"}
and \code{"discarded"} are present, Gencot cannot decide whether the parameter is always discarded (perhaps after modification),
or always modified (perhaps after discarding and reallocating it). In this case it always assumes a modification and
uses the single value \code{"yes"}. If this is not correct it must be handled manually by the developer.

The reason for this treatment is that before evaluation Gencot gives a local modification a higher priority than a dependency,
to reduce the number of required invocations. This may hide a dependency which results in discarding the parameter. To
be consistent, Gencot also prefers modifications resulting from dependencies after evaluation over discarding. 

Since in evaluated parameter modification descriptions all parameter descriptions are still confirmed, the only values
possible for a parameter description are \code{"nonlinear"}, \code{"readonly"}, \code{"yes"}, \code{"discarded"}, and \code{"no"}.
This information is used for declaring the properties Read-Only and Add-Result for the function parameters.

\subsection{Haskell Modules}
\label{impl-parmod-modules}

Parameter modification descriptions are implemented in the following three Haskell modules.

Module \code{Gencot.Json.Parmod} defines the following types for representing parameter modification descriptions:
\begin{verbatim}
  type Parmod = JSObject JSValue
  type Parmods = [Parmod]
\end{verbatim}
They are used for constructing and processing the JSON representation. Type \code{Parmod} represents a single
function description as a JSON object with arbitrary JSON values. Type \code{Parmods} represents a full parameter modification
description consisting of a sequence of function descriptions.

Module \code{Gencot.Json.Translate} defines the translation from C source code to parameter modification descriptions
in JSON format. The monadic action
\begin{verbatim}
  transGlobals :: [DeclEvent] -> CTrav Parmods
\end{verbatim}
translates a sequence of global declarations and definitions (see Section~\ref{impl-ccode-read}) to a sequence of
function descriptions. It translates all definitions and declarations of functions and of objects with function pointer
type or function pointer array type. It translates all type name definitions for a function type, a function pointer type, 
or a function pointer array type.
For every definition of a compound struct or union type it translates all members with function pointer type or function 
pointer array type.
All other \code{DeclEvent}s are ignored.

Module \code{Gencot.Json.Process} defines functions for reading and processing parameter modification descriptions in JSON format.
The main functions are 
\begin{verbatim}
  showRemainingPars :: Parmods -> [String]
  getRequired :: Parmods -> [String]
  filterParmods :: Parmods -> [String] -> Parmods
  mergeParmods :: Parmods -> Parmods -> Parmods
  sortParmods :: Parmods -> [String] -> Parmods
  addParsFromInvokes :: Parmods -> Parmods
  evaluateParmods :: Parmods -> Parmods
  convertParmods :: Parmods -> ItemProperties
\end{verbatim}
The first function retrieves the list of all unconfirmed parameters in the form
\begin{verbatim}
  <function identifier> / <parameter identifier>
\end{verbatim}
The function \code{getRequired} retrieves the function identifiers of all invocations on which at least one parameter depends.
The function \code{filterParmods} restricts a parameter modification description to the descriptions of all function
where the function identifier belongs to the string list. 
The function \code{mergeParmods} merges two parameter modification description. If a function is described in both, the description
with less unconfirmed parameter descriptions is used. If the same number of parameter descriptions are confirmed 
always the description in the first sequence is selected. The function \code{sortParmods} sorts the function descriptions
in a parameter modification description according to the order specified by a list of function identifiers. The parameter
modification description is reduced to the functions specified in the list. If a function occurs in the list more than once, 
the position of its first occurrence is used for the ordering.
The function \code{evaluateParmods} evaluates a parameter modification description as described in 
Section~\ref{impl-parmod-eval}. 

The function \code{convertParmods} converts the JSON representation to an item
property map (see Section~\ref{impl-itemprops-internal} with declarations of the properties Read-Only and Add-Result.
The parameter modification description must have been evaluated as described in Section~\ref{impl-parmod-eval}. The 
resulting single parameter description values are translated to properties as follows: the description \code{"yes"}
is translated to the Add-Result property, the descriptions \code{"readonly"} and \code{"no"} are translated to the 
Read-Only Property. For the remaining descriptions \code{"nonlinear"} and \code{"discarded"} and the \code{f\_result}
attribute no property is declared.


\section{Comments}
\label{impl-comments}

In a first step all comments are removed from the C source file and are written to a separate file. The remaining 
C code is processed by Gencot. In a final step the comments are reinserted into the generated code.

Additional steps are used to move comments from declarations to definitions.

The filter \code{gencot-selcomments} selects all comments from the input, translates them to Cogent comments and writes 
them to the output.
The filter \code{gencot-remcomments} removes all comments from the input and writes the remaining code to the output.
The filter \code{gencot-mrgcomments <file>} merges the comments in \code{<file>} into the input and writes the
merged code to the output. <file> must contain the output of \code{gencot-selcomments} applied to a code X, the 
input must have been generated from the output of \code{gencot-remcomments} applied to the same code X.

\subsection{Filter \code{gencot-remcomments}}

The filter for removing comments is implemented using the C preprocessor with the option \code{-fpreprocessed}. With this
option it removes all comments, however, it also processes and removes \code{\#define} directives. To prevent this, a sed
script is used to insert an underscore \code{\_} before every \code{\#define} directive which is only preceded 
by whitespace in its line. Then it is not recognized by the preprocessor. Afterwards, a second sed script removes the underscores.

Instead of an underscore an empty block comment could have been used. This would have the advantage that the second sed script
is not required, since the empty comments are removed by the preprocessor. The disadvantage is that the empty comment
is replaced by blanks. The resulting indentation does not modify the semantics of the \code{\#define} statenments but it
looks unusual in the Cogent code.

The preprocessor also removes \code{\#undef} directives, hence they are treated in the same way.

The preprocessor preserves all information about the original source line numbers, to do so it may insert 
line directives of the form \code{\# <linenumber> <filename>}. They must be processed by all following filters. The Haskell
C parser \code{language-c} processes these line directives.

\subsection{Filter \code{gencot-selcomments}}

The filter for selecting comments is implemented as an awk script. It scans through the input for the comment start
sequences \code{/*} and \code{//} to identify comments. It translates C comments to Cogent comments in the output.
The translation is done here since the filter must identify the start and end sequences of comments, so it can 
translate them specifically. Start and end sequences which occur as part of comment content are not translated.

To keep it simple, the cases when the comment start sequences can occur in C code parts are ignored. This may lead to
additional or extended comments, which must be corrected manually. It never leads to omitted comments or missing 
comment parts. Note that \code{gencot-remcomments} always identifies comments correctly, since there comment detection
it is implemented by the C preprocessor.

To distinguish before-units and after-units, \code{gencot-selcomments} inserts a separator between them. The separator
consists of a newline followed by \code{-\}\_}. It is constructed in a way that it cannot be a part of or overlap with a 
comment and to be easy to detect when processing the output of \code{gencot-selcomments} line by line. The newline and
the \code{-\}} would end any comment. The underscore (any other character could have been used instead) distinguishes the
separator from a normal end-of-comment, since \code{gencot-selcomments} never inserts an underscore immediately after a comment. 

The separator is inserted after every unit, even if the unit is empty. The first unit in the output of \code{gencot-selcomments}
is always a before-unit.

When in an input line code is found outside of comments all this code with all embedded comments is replaced by the
separator. Only the comments before and after the code are translated to the output, if present.
Note, that the separator includes a newline, hence every source line with code outside of
comments produces two output lines. 

An after-unit consists of all comments after code in a line. The last comment is either a line comment or
it may be a block comment which may include following lines. After this last comment the after-unit ends and
a separator is inserted.

All whitespace in and between comments and before the first comment in a before-unit is preserved in the output, 
including empty lines. After a before-unit only empty lines are preserved. Whitespace around code is typically used
to align code and comments, this must be adapted manually for the generated target code. Whitespace after an after-unit
is not preserved since the last comment in an after unit in the target code is always followed by a newline.

The filter never deletes lines, hence in its output the original line numbers can still be determined by counting lines, 
if the newlines belonging to the separators are ignored.

\subsubsection{State Machine}

The implementation processes the input line by line using a finite state machine. It uses the variables \code{before} and 
\code{after} to collect block comments at the beginning and end of the current line, initially both are empty. 
The collect action appends the input from the current position up to and including the next found item in 
the specified variable. The separate action appends the separator to the specified variable. The output action 
writes the specified content to the output, replacing C comment start and end sequences by their Cogent counterpart.
The newline action advances to the beginning of the next line and clears \code{before}.

The state machine has the following states, nocode is the initial state:

\begin{description}
\item[nocode] If next is
  \begin{description}
  \item[end-of-line] output(\code{before}); newline; goto nocode
  \item[block-comment-start] collect(\code{before}); goto nocode-inblock
  \item[line-comment-start] collect(\code{before}); output(\code{before} + line-comment); newline; goto nocode
  \item[other-code] separate(\code{before}); clear(\code{after}); goto code
  \end{description}
\item[nocode-inblock] If next is
  \begin{description}
  \item[end-of-line] collect(\code{before}); output(\code{before}); newline; goto nocode-inblock
  \item[block-comment-end] collect(\code{before}); goto nocode
  \end{description}
\item[code] If next is
  \begin{description}
  \item[end-of-line] output(\code{before} + separator); newline; goto nocode
  \item[block-comment-start] append comment-start to \code{after}; goto code-inblock
  \item[line-comment-start] output(\code{before} + line-comment + separator); newline; goto nocode
  \end{description}
\item[code-inblock] If next is
  \begin{description}
  \item[end-of-line] collect(\code{after}); output(\code{before} + \code{after}); newline; goto aftercode-inblock
  \item[block-comment-end] collect(\code{after}); goto code-afterblock
  \end{description}
\item[code-afterblock] If next is
  \begin{description}
  \item[end-of-line] output(\code{before} + \code{after} + separator); newline; goto nocode
  \item[block-comment-start] collect(\code{after}); goto code-inblock
  \item[line-comment-start] collect(\code{after}); output(\code{before} + \code{after} + line-comment + separator); newline; goto nocode
  \item[other-code] clear(\code{after}); goto code
  \end{description}
\item[aftercode-inblock] If next is
  \begin{description}
  \item[end-of-line] collect(\code{before}); output(\code{before}); newline; goto aftercode-inblock
  \item[block-comment-end] collect(\code{before}); separate(\code{before}); goto nocode
  \end{description}
\end{description}

\subsection{Filter \code{gencot-mrgcomments}}

The filter for merging comments into the target code is implemented as an awk script. It consists of a function 
\code{flushbufs}, a BEGIN rule, a line rule, and an END rule. 

The BEGIN rule reads the <file> line by line and collects before- and after-units as strings in the arrays \code{before} and \code{after}. The arrays are indexed with the (original) line number of the separator between before- and after-unit.

The line rule uses a buffer for its output. It is used to process all \code{\#ORIGIN} and \code{\#ENDORIG} markers around
a nonempty line and collect the associated comment units and content. The END rule is used to flush the buffer.

For every consecutive sequence of \code{\#ORIGIN} markers (i.e., separated by a single line containing only whitespace),
the before units associated with the line numbers of all markers with 
a ``+'' sign are collected in a buffer. The following nonempty code line is put in a second buffer. 
For every consecutive sequence of \code{\#ENDORIG} markers, the after units associated with the line numbers of all 
markers with a ``+'' sign are collected in a third buffer. Whenever a code line or an \code{\#ENDORIG} marker is 
followed by a line which is no \code{\#ENDORIG} marker, the content of all three buffers is processed by the 
function \code{flushbufs}.

Normally, the buffers are output and reset. In the case that the third buffer is empty (there are no after 
units for the code) the code line is retained in a separate buffer. If in the next group the first buffer is 
empty (there are no before units for the code) the code lines of both groups are concatenated without a newline 
in between (also removing the indentation for the second line). This has the effect of completely removing all 
markers which are not used for inserting comments at their position, thus removing all unnecessary line breaks.

Markers are also completely removed, if a unit exists but contains only whitespace (such as empty lines for
formatting). This is done because the target code is already formatted during generation and inserting additional
whitespace for formatting is not useful.

In the buffer, before-units are concatenated without any separator. After-units are separated by a newline to end possibly 
trailing line comments.

\subsection{Declaration Comments}
\label{impl-comments-decl}

To safely detect C declarations and C definitions Gencot uses the language-c parser. 

Only comments associated with declarations with external linkage are transferred to their definitions. For declarations
with internal linkage the approach for transferring the comments does not work, since the declared names need not be
unique in the <package>.

Processing the decalaration comments is implemented by the following filter steps. 

\subsubsection{Filter \code{gencot-deccomments}}

The filter \code{gencot-deccomments} parses the input. For every declaration it outputs a line
\begin{verbatim}
  #DECL <name> <bline>
\end{verbatim}
where \code{<name>} is the name of the declared item, and \code{<bline>} is the original source line 
number where the declaration begins.

The filter implementation is described in Section~\ref{impl-comps-decls}.

\subsubsection{Filter \code{gencot-movcomments}}

The filter \code{gencot-movcomments <file>} processes the output of \code{gencot-deccomments} as input. For every
line as above, it retrieves the before-unit of \code{<bline>} from \code{<file>}
and stores it in the file \code{<name>.comment} in the
current directory. The content of \code{<file>} must be the output of \code{gencot-selcomments} applied to the same
original source from which the input of \code{gencot-deccomments} has been derived.

The filter is implemented as an awk script. It consists of a 
BEGIN rule reading \code{<file>} in the same way as \code{gencot-mrgcomments}, and a rule for lines starting with 
\code{\#DECL}. For every such line it writes the associated comment unit to the comment file. A comment
file is even written if the comment unit is empty.

\subsubsection{Filter \code{gencot-defcomments}}

For inserting the comments before the target code parts generated from a definition, Gencot uses the marker
\begin{verbatim}
  #DEF <name>
\end{verbatim}
The marker must be inserted by the filter which generates the definition target code.

The filter \code{gencot-defcomments <dir>} replaces every marker line in its input by the content of the corresponding
\code{.comment} file in directory \code{<dir>} and writes the result to its output.

It is implemented as an awk script with a single rule for all lines. If the line starts with \code{\#DEF} it is
replaced by the content of the corresponding file in the output. All other lines are copied to the output without 
modification.


\section{Preprocessor Directives}
\label{impl-preprocessor}
\subsection{Filters for Processing Steps}

Directive processing is done for the output of \code{gencot-remcomments}. All comments have been removed. 
However, there may be line directives present. 

The filter \code{gencot-selpp} selects all preprocessor directives and
copies them to the output without changes. All other lines are replaced by empty lines, so that the original
line numbers for all directives can still be determined. 

The filter \code{gencot-rempp <file>} removes all preprocessor directives from its input, replacing them by empty
lines. All other lines are copied to the output without modification. If \code{<file>} is specified it must 
contain a list of regular expressions for directives which shall be retained.

How the directives are processed depends on the kind of directives (see Section~\ref{design-preprocessor}).
Gencot provides the processing
filters \code{gencot-prcppconst, gencot-prcppmacro, gencot-prcppincl}. Conditional directives are not processed, they are inserted
without changes. However, they are merged into the target code in a different way, therefore Gencot provides 
the specific merging filter \code{gencot-mrgppcond <file>} for them. All other directives are merged into the 
target code by filter \code{gencot-mrgpp <file>}. Both filters merge the content in \code{<file>} 
into the input and write the merged code to the output. \code{<file>} must contain the directives to be merged.

\subsection{Separating Directives}

Gencot supports to keep some directives in the output of \code{gencot-rempp} to handle cases where
the C code of different groups in a section causes conflicts. These conditional directives are still selected
by \code{gencot-selpp} and re-inserted by \code{gencot-mrgcond}. 

\subsubsection{Filter \code{gencot-selpp}}

The filter for selecting preprocessor directives from the input for separate processing and insertion into
the generated target code is implemented as an awk script.

It detects all kinds of preprocessor directives, which always begin at the beginning of a separate line.
A directive always ends at the next newline which is not preceded by a backslash \code{\\}. All corresponding
lines are copied to the output without modifications with the exception of line directives.

Line directives in the input are expanded to the required number of empty lines
which have the same effect. This is done to simplify reading the input for all \code{gencot-prcX} filters.

Every other input line is replaced by an empty line in the output.

\subsubsection{Filter \code{gencot-rempp}}

The filter for removing preprocessor directives from its input is implemented as an awk script.
Basically, it replaces lines which are a part of a directive by empty lines. However, there are the following
exceptions:
\begin{itemize}
\item line directives are never removed, they are required to identify the position in the original source
during code processing.
\item system include directives are never removed, they are inteded to be interpreted by the language-c
preprocessor to make the corresponding information available during code processing. Since it is assumed that
all quoted include directives have already been processed in an initial step, simply all include
directives are retained.
\item directives which match a regular expression from a specified list are not removed, they are intended
to be interpreted by the language-c preprocessor to suppress information which causes conflicts during code
processing.
\end{itemize}

For conditional directives always all directives belonging to the same section are treated in the same way.
To retain them the first directive (\code{\#if}, \code{\#ifdef}, \code{\#ifndef}) must match a regular expression
in the list. For all other directives of a section (\code{\#else}, \code{\#elif}, \code{\#endif}) the 
regular expressions are ignored.

The regular expressions are specified in the argument file line by line. An example file content is
\begin{verbatim}
  ^[[:blank:]]*#[[:blank:]]*if[[:blank:]]+!?[[:blank:]]*defined\(SUPPORT_X\)
  ^[[:blank:]]*#[[:blank:]]*define[[:blank:]]+SUPPORT_X
  ^[[:blank:]]*#[[:blank:]]*undef[[:blank:]]+SUPPORT_X
\end{verbatim}
It retains all directives which define the macro \code{SUPPORT\_X} or depend on its definition.

\subsection{Processing Directives}

\subsubsection{Processing Constants Defined as Preprocessor Macros}

We provide the script \code{convert-const.csh} for automating this task. If comments should be also be converted to Cogent 
the script can be used together with the script \code{convert-comment.csh}.

\subsubsection{Processing Other Preprocessor Directives}

The line numbers for positions count actual lines. Therefore the position of a preprocessor directive is specified by its starting line and its ending line. 

\subsection{Filter \code{gencot-prcconst}}

\subsection{Merging Directive Processing Results}

When the conditional directives are merged into the target code, the other 
directives must have already been merged in, since the conditional directives are inserted depending on the content 
of groups and their positions. Therefore, first all other directives must be merged using the filter \code{gencot-mrgpp},
then the conditional directives must be merged using the filter \code{gencot-mrgppcond}.

\subsection{Filter \code{gencot-mrgppcond}}

The filter for merging the conditional directives into the target code is implemented as an awk script. As argument
it takes the name of a file containing the directives to be merged. Since conditional directives need not be
processed it is intended that this file contains the output of filter \code{gencot-selpp}, i.e., all directives
selected from the source. \code{gencot-mrgppcond} selects only the conditional directives and merges them into
the filter input, additionally generating origin markers for every merged directive. 

The filter input must contain the generated Cogent target code and the other preprocessor directives. 
All content in the input must have been marked by origin markers. 

In its BEGIN rule the filter reads the
conditional directives from the argument \code{<file>} and associates them with their line numbers, building the
list of all sections, ordered according to the line number of their first directive (\code{\#if, \#ifdef, \#ifndef}).
Every section is represented by its list of directives.

While processing the input the program maintains a stack of active sections and for every section in the stack the 
active group. A section is active if some of its directives have been output but not all. The active group is that
corresponding to the last directive which has been output for the section, i.e. it is the group which will contain
all target code which is currently output.

The filter only uses the \code{\#ORIGIN} markers to position conditional directives. Gencot assumes that for every 
\code{#ENDORIG} marker a previous \code{\#ORIGIN} marker exists for the same line number. Since a condition group
always contains a sequence of complete lines, the information about the origin lines in the input is fully specified
by the \code{\#ORIGIN} markers, the \code{\#ENDORIG} markers are not relevant for placing the conditional directives.

For every \code{\#ORIGIN} marker in the input the following steps are performed:
\begin{itemize}
\item if the line belongs to a group of an active section which is after the active group, all directives of the active
section are output until the group containing the line is reached, this group is the new active group of the section.
\item if for an active section the line does not belong to the active group or any of its following groups, 
the \code{\#endif} directive for the 
section is output and the section becomes inactive (is removed from the stack of active sections).
\item if the line belongs to a group of a section which is (after the previous step) not active, the section is set 
active (put on the stack) and all its directives are output until the group containing the line is reached.
\item finally the \code{\#ORIGIN} directive and all following lines which are not an \code{\#ORIGIN} marker
are output without any changes.
\end{itemize}

Note that due to the semantics of the \code{\#elif} and \code{\#else} directives, for every group in a section all
directives of preceding groups are relevant and must be output when the active group changes, even if this produces 
empty groups in between.

Due to the nesting structure, when newly active sections are pushed on the stack in the order of the position of their
first directive, the stack reflects the section nesting and the sections will be inactivated in the reverse order and
can be removed from the stack accordingly.

If an \code{\#ORIGIN} marker idicates, that the next line belongs to a group \textit{before} the active group, the
steps described above imply that the current section is ended and restarted from the beginning.

\subsection{Filter \code{gencot-mrgpp}}



\section{Parsing and Processing C Code}
\label{impl-ccode}

Parsing and processing C code in Gencot is always implemented in Haskell, to be able to use an existing
C parser. There are at least two choices for a C parser in Haskell:
\begin{itemize}
\item the package ``language-c'' by Benedikt Huber and others,
\item the package ``language-c-quote'' by Geoffrey Mainland and others.
\end{itemize}

The Cogent compiler uses the package language-c-quote for outputting the generated C code and for parsing the antiquoted
C source files. The reason is its support for quasiquotation (embedding C code in Haskell code) and antiquotation
(embedding Haskell code in the embedded C code). The antiquotation support is used for parsing the antiquoted C sources.

Gencot performs three tasks related to C code:
\begin{itemize}
\item read the original C code to be translated,
\item generate antiquoted C code for the function wrapper implementations,
\item output normal C code for the C function bodies as placeholder in the generated Cogent function definitions.
\end{itemize}

The first task is supported by both packages: a C parser reads the source text and creates an internal abstract syntax tree (AST).
Every package uses its own data structures for representing the AST. However, the language-c package provides an additional
``analysis'' module which processes the rather complicated syntax of C declarations and returns a ``symbol map'' mapping
every globally declared identifier to its declaration or definition. Since Gencot generates a single Cogent definition for
every single globally declared identifier, this is the ideal starting point for Gencot. For this reason Gencot uses
the language-c parser for the first task.

The second task is only supported by the package language-c-quote, therefore it is used by Gencot. 

The third task is supported by both packages, since both have a prettyprint function for outputting their AST. Since the 
function bodies have been read from the input and are output with only minor modifications, it is easiest to use
the language-c prettyprinter, since language-c has been used for parsing and the body is already represented by its 
AST data structures. However, the language-c prettyprinter cannot be extended to generate the ORIGIN markers, therefore
the AST is translated to the language-c-quote AST and the corresponding prettyprinter is used for the third task (see 
Section~\ref{impl-ccode-expr}).

Note that in both packages the main module is named \code{Language.C}. If both packages are exposed to the ghc Haskell
compiler, a package-qualified import must be used in the Haskell program, which must be enabled by a language pragma:
\begin{verbatim}
  {-# LANGUAGE PackageImports #-}
  ...
  import "language-c" Language.C
\end{verbatim}

\subsection{Including Files}
\label{impl-ccode-include}

The filter \code{gencot-include <dirlist> [<filename>]} processes all quoted include directives and replaces them (transitively) by the 
content of the included file. Line directives are inserted at the begin and end of an included file, so that
for all code in the output the original source file name and line number can be determined. The \code{<dirlist>}
specifies the directories to search for included files. The optional \code{<filename>}, if present, must be the name of
a file with a list of names of files which should be omitted from being included.

\subsubsection{Filter \code{gencot-include}}

The filter for expanding the include directives is implemented as an awk script, heavily inspired by the ``igawk''
example program in the gawk infofile, edition 4.2, in Section 11.3.9.

As argument it expects a directory list specified with ``:'' as separator. The list corresponds
to directories specified with the \code{-I} cpp option, it is used for searching included files.
All directories for searching included files must be specified in the arguments, there are no defaults.

Similar to cpp, a file included by a quoted directive is first searched in the directory of the including file. 
If not found there, the argument directory list is searched.

Since the input of \code{gencot-include} is read from standard input it is not associated with a directory. Hence
if files are included from the same directory, that directory must also be specified explicitly in an argument directory
list.

\subsubsection{Generating Line Directives}

Line directives are inserted into the output as follows.

If the first line of the input is a line directive, it is copied to the output. Otherwise 
the line directive
\begin{verbatim}
  # 1 "<stdin>"
\end{verbatim}
is prepended to the output.

If after a generated line directive with file name \code{"fff"} the input line \code{NNN} contains the 
directive 
\begin{verbatim}
  #include "filepath"
\end{verbatim}
the directive is replaced in the output by the lines 
\begin{verbatim}
  # 1 "dir/filepath" 1
  <content of file filepath>
  # NNN+1 "fff" 2
\end{verbatim}

The \code{"dir/"} prefix in the line directives for included files is determined as follows. 
If the included file has been found in the 
directory of its includer, the directory pathname is constructed from \code{"fff"} by taking the pathname 
up to and including the last ``/'' (if present, otherwise the prefix is empty).
If the included file has been found in a directory from the argument directory list
the directory pathname is used as specified in the list.

\subsubsection{Multiple Includes}

The C preprocessor does not prevent a file from being included multiple times. Usually, C include files use
an ifdef directive around all content to prevent multiple includes. The \code{gencot-include} filter does
not interprete ifdef directives, instead, it simply prevents multiple includes for all files independent 
from their contents, only based on their full file pathnames. To mimic the behavior of cpp, if a file is 
not include due to repeated include, the corresponding line directives are nevertheless generated in the form
\begin{verbatim}
  # 1 "dir/filepath" 1
  # NNN+1 "fff" 2
\end{verbatim}

\subsubsection{Omitted Includes}

A special case of multiple include is the recursive include of the main input file. However, since it is read from standard
input, its name is not known to \code{gencot-include}. If it may happen that it is recursively included, the corresponding
pathname, as it appears in an include directive, must be added to the list of includes to be omitted.

There are other reasons, why some include files should be omitted. One case is that an include file may or may not exists
which is configured through a preprocessor flag. Since \code{gencot-include} ignores all conditional directives,
this would not be detected and an error message would be caused if the file does not exist.

The list of files to be omitted from inclusion is specified in the optional file passed as second argument to 
\code{gencot-include}. Every file must be specified on a separate line by its pathname exactly in the form it occurs 
in a quoted include directive (without quotes). Since system includes and includes where the included file is specified 
as a macro call are not processed by \code{gencot-include} they need not be added to the list.

Includes for files listed to be omitted are simply ignored. No line
directives are generated for them.

\subsection{Preprocessing}
\label{impl-ccode-preproc}

The language-c parser supports an integrated invocation of an external preprocessor, the default is to use
the gcc preprocessor. However, the integrated invocation always reads the C code from a file (and checks
its file name extension) and not from standard input.

To implement C code processing as a filter, Gencot does not use the integrated preprocessor,
it invokes the preprocessor as an additional separate step. For consistency reasons it is wrapped in
the minimal filter script \code{gencot-cpp}. 

The preprocessor step only has the following purpose:
\begin{itemize}
\item process all system include directives by including the file contents,
\item process retained conditional directives to prevent conflicts in the C code.
\end{itemize}
All other preprocessing has already been done by previous steps.

\subsection{Reading the Input}
\label{impl-ccode-read}

\subsubsection{Parsing}

To apply the language-c parser to the standard input we invoke it using function \code{parseC}. It needs an \code{InputStream}
and an initial \code{Position} as arguments. 

The language-c parser defines \code{InputStream} to be the standard type \code{Data.ByteString}. To get the 
standard input as a \code{ByteString} the function \code{ByteString.getContents} can be used. 

The language-c parser uses type \code{Position} to describe a character position in a named file. It provides
the function \code{initPos} to create an initial position at the beginning of a file, taking a \code{FilePath}
as argument, which is a \code{String} containing the file name. Since Gencot and the C preprocessor create
line directives with the file name \code{<stdin>} for the standard input, this string is the correct argument
for \code{initPos}. 

The result of \code{parseC} is of type \code{(Either ParseError CTranslUnit)}. Hence it should be checked whether
an error occurred during parsing. If not, the value of type \code{CTranslUnit} is the abstract syntax tree for
the parsed C code.

Both \code{parseC} and \code{initPos} are exported by module \code{Language.C}. The function \code{ByteString.getContents}
is exported by the module \code{Data.Bytestring}. Hence to use the parser we need the following imports:
\begin{verbatim}
  import Data.ByteString (getContents)
  import "language-c" Language.C (parseC,initPos)
\end{verbatim}

Then the abstract syntax tree can be bound to variable \code{ast} using
\begin{verbatim}
  do
    input_stream <- Data.ByteString.getContents
    ast <- either (error . show) return $ parseC input_stream (initPos "<stdin>")
\end{verbatim}

\subsubsection{Analysis}

Although it is not complete and only processes toplevel declarations (including typedefs), and object definitions, the
language-c analysis module is very
useful for implementing Gencot translation. Function definition bodies are not covered by analysis, but they are
not covered by Gencot either.

The main result of the analysis module is the symbol table. Since at the end of traversing a correct C AST the toplevel
scope is reached, the symbol table only contains all globally defined identifiers. From this symbol table 
a map is created containing all toplevel declarations and object definitions, mapping the identifiers
to their semantics, which is mainly its declared type. Whereas in the abstract syntax tree there may be several declarators
in a declaration, declaring identifiers with different types derived from a common type, the map maps every identifier
to its fully derived type. 

Also, tags for structs, unions and enums are contained in the map. In C their definitions can be embedded in other declarations.
The analysis module collects all these possibly embedded declarations in the symbol table. The map also gives for
every defined type name its definition.

Together, the information in the map is much more appropriate for creating Cogent code, where all type definitions are on
toplevel. Therefore, Gencot uses the map resulting from the analysis step as starting point for its translation. 
Additionally, Gencot uses the symbol table built by the analysis module during its own processing to access the
types of globally defined identifiers and for managing local declarations when traversing function bodies, as described in
Section~\ref{impl-ccode-trav}.

Additionally, the analysis module provides a callback handler which is invoked for every declaration entered into the symbol 
table (with the exception of tag forward declarations and enumerator declarations). The callback handler can accumulate results 
in a user state which can be retrieved after analysis together with the
semantics map. Since the callback handler is also invoked for all local declarations it is useful when all declarations
shall be processed in some form.

To use the analysis module, the following import is needed:
\begin{verbatim}
  import Language.C.Analysis
\end{verbatim}

Then, if the abstract syntax tree has been bound to variable \code{ast}, it can be analysed by
\begin{verbatim}
  (table,state) <- either (error . show) return $ 
    runTrav uinit (withExtDeclHandler (analyseAST ast >> getDefTable) uhandler)
\end{verbatim}
which binds the resulting symbol table to variable \code{table} and the resulting state to \code{ustate}. \code{runTrav}
returns a result of type \code{Either [CError] (DefTable, TravState s)}, where \code{DefTable}
is the type of the symbol table and \code{s} is the type of the user state. The error list in the first alternative contains 
fatal errors which made the analysis fail. The state in the second alternative contains warnings about semantic inconsistencies, 
such as unknown identifiers, and it contains the user state. \code{uinit} is the initial user state and \code{uhandler}
is the callback handler of type
\begin{verbatim}
  DeclEvent -> Trav s ()
\end{verbatim}
It returns a monadic action without result.

The semantics map is created from the symbol table by the function \code{globalDefs}, its type is \code{GlobalDecls}.

On this basis, Gencot implements the following functions in the module \code{Gencot.Input} as utility for parsing and analysis:
\begin{verbatim}
  readFromInput :: s -> (DeclEvent -> Trav s ()) -> IO (DefTable, s)
  readFromFile :: FilePath -> s -> (DeclEvent -> Trav s ()) -> IO (DefTable, s)
\end{verbatim}
The first one takes as arguments an initial user state and a callback handler. It reads C code from standard input, parses
and analyses it and returns the symbol table and the user state accumulated by the callback handler. The second function
takes a file name as additional argument and does the same reading from the file.

All Gencot filters which read C code use one of these two functions.

\subsubsection{Source Code Origin}

The language-c parser adds information about the source code origin to the AST. For every syntactic construct represented
in the AST it includes the start origin of the first input token and the start origin and length of the last input token.
The start origin of a token is represented by the type \code{Position} and includes the original source file name and 
line number, affected by line directives if present in the input. It also includes the absolute character offset in the 
input stream. The latter can be used to determine the ordering of constructs which have been placed in the same line.
The type \code{Position} is declared as instance of class \code{ORD} by comparing the character offset, hence it can 
easily be used for comparing and sorting.

The origin information about the first and last token is contained in the type \code{NodeInfo}. All types for representing
a syntactic construct in the AST are parameterized with a type parameter. In the actual AST types this parameter is always 
substituted by the type \code{NodeInfo}. 

The analysis module carries the origin information over to its results, by including a \code{NodeInfo} in most of its
result structures. This information can be used to
\begin{itemize}
\item determine the origin file for a declared identifier,
\item filter declarations according to the source file containing them,
\item sort declarations according to the position of their first token in the source,
\item translate identifiers to file specific names to avoid conflicts.
\end{itemize}

For the last case the true name of the processed file is required, however, the parsed input is read from a pipe where
the name is always given as \code{<stdin>}. The true name is passed to the Haskell program as an additional 
argument, as described in Section~\ref{impl-ccomps-filters}. Since there is no easy way to replace the file name in
all \code{NodeInfo} values in the semantic map, Gencot adds the name to the monadic state used for processing
(see Section~\ref{impl-ccode-trav}).

\subsubsection{Preparing for Processing}

The main task for Gencot is to translate all declarations or definitions which are contained in a single source file, where
nested declarations are translated to a sequence of toplevel Cogent definitions. This is achieved by parsing and analysing
the content of the file and all included files, filtering the resulting set of declarations according to the source file name
\code{<stdin>}, removing all declarations which are not translated to Cogent, and sorting the remaining ones in a list. 
Translating every list entry to Cogent yields the resulting Cogent definitions in the correct ordering.

For syntactically nested constructs in C the analysis phase creates separate declarations. This corresponds to the 
Cogent form where every declaration becomes a seperate toplevel construct. However, for generating the origin information 
a seperate processing of these declarations would yield repeated origin ranges which may result in repeated comment units in
the target code. Therefore Gencot processes nested 
constructs as part of the containing constructs. Sorting the declarations according to their positions always puts the
nested declarations after their containing declarations. Processing them as part of the containing declaration will
always be done before the nested declaration occurs in the main list of declarations. By maintaining a list of declarations
already processed as nested, Gencot skips these declarations when it finds them in the main list.

The only C constructs which can be nested are tag definitions for struct, union, and enum types (all other cases of nesting
occurs only in C function bodies where the nested parts are not contained as seperate entries in the main list). Tag definitions
can occur as part of every type specification, the most important case is the occurrence in a typedef as in
\begin{verbatim}
  typedef struct s { ... } t
\end{verbatim}
Other relevant cases are the occurrence in the declaration of a global variable, in the result or parameter types of a 
function declaration and in the declaration of a struct or union member (which may result in arbitrarily deep nesting).

For these cases, if the type references a tag definition, Gencot inspects the position information of the tag definition.
If it is syntactically embedded, it processes it and marks it as processed so that it is skipped when it finds it in the
main list.

The type \code{GlobalDecls} consists of three separate maps, one for tag definitions, one for type definitions,
and one for all other declarations and definitions. Every map uses its own type for its range values, however, 
there is the wrapper type \code{DeclEvent} which has a variant for each of them. 

The language-c analysis module provides a filtering function for its resulting map of type \code{GlobalDecls}. The filter 
predicate is defined for values of type \code{DeclEvent}. If the map has been bound to the variable \code{gmap}
it can be filtered by
\begin{verbatim}
  filterGlobalDecls globalsFilter gmap
\end{verbatim}
where \code{globalsFilter} is the filter predicate.

Gencot uses a filter which reduces the declarations to those contained directly in the input file, removing all
content from included files. Since the input file is always associated with the name \code{<stdin>} in the \code{NodeInfo}
values, a corresponding filter function is
\begin{verbatim}
  (maybe False ((==) "<stdin>") . fileOfNode)
\end{verbatim}
Additionally, for a specific Gencot component, the declarations are reduced to those which are processed by the component. 

Every map range value, and hence every \code{DeclEvent} value contains the identifier which is mapped to it, 
hence the full information required for translating the definitions is contained in the range values. 
Gencot wraps every range value as a \code{DeclEvent}, and puts them in a common list for all three maps. This
is done by the function
\begin{verbatim}
  listGlobals :: GlobalDecls -> [DeclEvent]
\end{verbatim}

Finally, the declarations in the list are sorted according to the offset position of their first tokens, using the
compare function
\begin{verbatim}
  compEvent :: DeclEvent -> DeclEvent -> Ordering
  compEvent ci1 ci2 = compare (posOf ci1) (posOf ci2)
\end{verbatim}

Together, the list for processing the code is prepared from the symbol table \code{table} by
\begin{verbatim}
  sortBy compEvent $ listGlobals $ filterGlobalDecls globalsFilter $ globalDefs table
\end{verbatim}

All this preprocessing is implemented in module \code{Gencot.Input}. It provides the function
\begin{verbatim}
  getDeclEvents :: GlobalDecls -> (DeclEvent -> Bool) -> [DeclEvent]
\end{verbatim}
It performs the preprocessing and returns the list of \code{DeclEvent}s to be processed.
As its second argument it expects a predicate for filtering the content of \code{<stdin>} to the
\code{DeclEvent}s to be processed by the specific Gencot component.

\subsection{Reading Packages}
\label{impl-ccode-package}

In some cases several source files of the <package> must be processed together. The typical case is when the main
files for the Cogent compilation unit are generated (see Section~\ref{design-files}). For this it is necessary 
to determine and process the external name references in a set of
C source files. This set is the subset of C sources in the <package> which is translated to Cogent and together yields
the Cogent compilation unit. 

\subsubsection{General Approach}

There are different possible approaches how to read and process this set of source files.

The first approach is to use a single file which includes all files in the set. This file is processed as usual by
\code{gencot-include}, \code{gencot-remcomments}, and \code{gencot-rempp} which yields the union of all definitions
and declarations in all files in the set as input to the language-c parser. However, this input may contain conflicting
definitions. For an identifier with internal linkage different definitions may be present in different source files.
Also for identifiers with no linkage different definitions may be present, if, e.g., different \code{.c} files define
a type with the same name. The language-c parser ignores duplicate definitions for identifiers with internal linkage,
however, it treats duplicate definitions for identifiers without linkage as a fatal error. Hence Gencot does not use
this approach.

The second approach ist to process every file in the set separately and merge the generated target code. However, for
identifiers with external linkage (function definitions) the external references cannot be determined from the content
of a single file. A non-local reference is only external if it is not defined in any of the files in the set. It would
be possible to determine these external references in a separate processing step and using the result as additional input
for the main processing step. Since this means to additionally implement reading and writing a list of external references,
Gencot does not use this approach.

The third approach is to parse and analyse the content of every file separately, then merge the resulting semantic maps
discarding any duplicate definitions. This approach assumes that the external name references, which are relevant for
processing, are uniquely defined in all source files. If this is not the case, because conflicting definitions are used
inside the <package>, which are external to the processed file subset, this must be handled manually. 
This approach is used by Gencot.

\subsubsection{Specifying Input Files}

Due to the approach used, the Gencot filters for generating the files common to the Cogent compilation unit expect
as input a list of names of the files which comprise the Cogent compilation unit. The file names must be pathnames which are either 
absolute or relative to the current directory. Every file name must occur on a single line.

Like all other input to the language-c parser their content must have been processed by \code{gencot-include}, 
\code{gencot-remcomments}, and \code{gencot-rempp}. This implies that all included content is already present
and need not be specified separately, usually only \code{.c} files need to be specified as input, after they have
been processed as usual.

If a processor only processes items which are external to all input files, the original file name is not required 
for the files input to a Gencot processor. However, the processor \code{gencot-dvdtypes} also processes derived 
types which are not external. In a derived type the original file name is only used for constructing the Cogent name
and item identifier for tagless compound types. Currently the only case where tagless types are processed by 
\code{gencot-dvdtypes} is when they occur in a derived function type. If it occurs as parameter type, it is mostly
useless, because then the type has only prototype scope and is not available when the function is invoked. If, however,
it is used as result type, a typedef name can be defined for it in another declarator of the same declaration and
used anywhere in the program. Since this is a very unusual construction in a C program Gencot does not handle it.
This will possibly result in wrong type names in the result of \code{gencot-dvdtypes} which must be manually corrected.

It would be possible to provide a list of the original file names together with the list of files to be processed and
use them as described in Section~\ref{impl-ccode-trav}. However, then it is only available in the userstate during 
analysis for a single input file. After merging the semantic maps the content of the resulting map cannot be associated
with the original file name any more. It would be necessary to preserve this association as a separate data structure.

The utility function
\begin{verbatim}
  readPackageFromInput :: IO [DefTable]
\end{verbatim}
in module \code{Gencot.Package} reads the file name list from input and parses and analyses all files using 
\code{readFromFile}. It returns the list of resulting symbol tables.

\subsubsection{Combining Parser Results}

When the parser results are combined it is relevant, how they are structured, in particular, if the same file
is included by several of the \code{.c} files. Most of the information only depends on the parsed text.
However, the language-c parser also uses unique identifiers, which are counted integers starting at 1 for every
parsed file. This implies, that these identifiers are not unique anymore, if several files are parsed separately
and then the results are combined.

The unique identifiers are associated with most of the AST nodes and are part of the \code{NodeInfo} values.
In the analysis phase they are used to cache the relation between defining and referencing occurrences of C identifiers,
and the relation between C expressions and their types. The corresponding caches are part of the symbol table structure.
However, the first relation cache seems to be built but not used, the second relation cache is only used during 
type analysis for expressions. In both cases, after the analysis phase the caches are still present, but are not 
relevant for the further processing by Gencot.

As a consequence, it is not possible to combine the raw AST structures and then perform the language-c analysis
on the combined AST, since then the non-unique identifiers may cause problems. Instead, Gencot parses and analyses
every file separately and the combines the resulting symbol tables.

However, the unique identifiers are additionally used for identifying tagless struct/union/enum types in the symbol table.
Language-c uses the alternative type \code{SUERef} to identify struct/union/enum types, with the alternatives 
\code{NamedRef} and \code{AnonymousRef}, where the latter specifies the unique integer identifier of the AST node
of the type definition. The symbol table maps \code{SUERef} values to their type definitions. This implies, that
tagless types may be entered in different symbol tables under different keys. This must be detected and handled
when combining the symbol tables.

As described in Section~\ref{impl-ccode-read}, Gencot uses for its processing the list of \code{DeclEvent}s which is
derived from the \code{GlobalDecls} map. This means, the combination could be implemented on the \code{GlobalDecls}
maps or even on the \code{DeclEvent} lists. However, as described in Section~\ref{impl-ccode-trav}, Gencot also
uses the symbol table during processing, for looking up identifiers. For this reason the combination is implemented
on the symbol tables.

A language-c symbol table is implemented by the type 
\begin{verbatim}
  data DefTable = DefTable {
    identDecls   :: NameSpaceMap Ident IdentEntry,
    tagDecls   :: NameSpaceMap SUERef TagEntry,
    labelDefs  :: NameSpaceMap Ident Ident,
    memberDecls :: NameSpaceMap Ident MemberDecl,
    refTable   :: IntMap Name,
    typeTable  :: IntMap Type
  }
\end{verbatim}
where a \code{NameSpaceMap k v} is a mapping from \code{k} to \code{v} with nested scopes. The last two components are the 
relation caches as described above, they are ignored and combined to be empty. After the analysis phase, on the toplevel, 
the maps \code{labelDefs} and \code{memberDecls} are empty, since in C there are no global labels and \code{memberDecls}
contains the struct/union members only while processing the corresponding declaration. So only the first two maps must
be combined.

The map \code{identDecls} contains all identifiers with file scope. For them, the linkage is relevant. If an identifier has 
internal linkage, it is only valid in its symbol table and may denote a different object in another symbol table. These
identifiers are not relevant for Gencot when processing several source files together, thus they are omitted when the
symbol tables are combined. Only identifiers with external or no linkage are retained in the combined symbol table. Note
that this implies that identifiers with internal linkage cannot be looked up in the combined table anymore.

If an identifier has external linkage, it is assumed to denote the same object in every symbol table. However, it may be
declared in one symbol table and defined in another one. In this case, always the definition is used for the combined
table, the declaration is ignored. Only if it is declared in both symbol tables one of the declarations is put into the
combined table. Note that if the C program is correct, the identifier may be defined in at most one C compilation unit
and thus a definition for it occurs in at most one of the symbol tables. 

If both symbol tables contain a definition for the same identifier with external linkage Gencot compares their source 
file positions. If both positions are the same, the definition is in a file included by both sources and it is correct to
discard one of them. If the positions are not the same, Gencot signals an error. The typical case for such included 
definitions are object declarations where the \code{extern} specifier is omitted. This is interpreted as a ``tentative''
definition in C and is classified as definition by the language-c analysis step, if it is not followed by a definition.

The dummy declarations generated by Gencot for parameterless macro definitions (see Section~\ref{impl-ccode-dummydecl})
have the form of such tentative definitions. Since they are generated separately for every C source file, they cannot
be recognized by comparing their source file positions. To prevent them from being signaled as duplicate definitions they
are explicitly specified as \code{extern}, then they are classified as declarations by language-c (instead as tentative 
definitions) and no error is signaled by Gencot. Alternatively they could be generated with internal linkage (specified 
as \code{static}), then they are removed before combining the symbol tables. However, then their names are mapped differently,
as described in Section~\ref{design-names}. Note that (manually created) dummy declarations for
macros with parameters are never classified as definitions, since they have the form of a C function declaration without
a body. 

The typical cases for toplevel identifiers with no linkage are typedef names and struct/union/enum tags. Such an identifier
may occur in two symbol tables, if it is
defined in a file included by both corresponding \code{.c} files. In this case it names the same type and one of both
entries is put in the combined table. However, it may also be the case that the identifier is defined in both \code{.c}
files and used for different types. In this case the combination approach does not work. Gencot assumes that this case
does not occur and tests whether the identifier is mapped to the same semantics (determined from the position information
of the corresponding definition). If not, an error is signaled, this must be handled manually by the developer.

The map \code{tagDecls} contains all tags of struct/union/enum types, mapped to the type definition. For tagless types
the unique identifier is used as key, as described above. If a tag is present, it is treated in the same way as other 
identifiers with no linkage.

A tagless struct/union/enum type in C can be referenced only from a single place, since it must be syntactically embedded
there. This means, when the same tagless type occurs in two symbol tables using a different key, every symbol table
contains at most one reference to the key. When the symbol tables are combined, at most one of these references are transferred
to the combined table. Thus it is possible to use the same key for the transferred definition to yield a consistent
combined table. The simples way to do so would be to always use the entry of the same symbol table for the combination 
when an object occurs in both. However, this is not possible, since for identifiers with external linkage the definition
must be preferred over a declaration, independent where it occurs. Therefore Gencot first transfers both definitions to the 
combined table and afterwards removes all definitions which are not referenced there.

Finally, it may happen that the same unique identifier is used in different symbol tables to reference different types, 
as described for type names above. This is not a problem in the C sources, it is an internal collision of the parser-generated 
unique identifiers. The easiest way to solve this is to introduce a tag for at least one of both tagless types.

The combination is implemented by the function
\begin{verbatim}
  combineTables :: DefTable -> DefTable -> DefTable
\end{verbatim}
in module \code{Gencot.Package}. It should be applied to the tables built by \code{readFromFile} for two different
\code{.c} files of the same package. It can be iterated to combine the result of \code{readPackageFromInput}.
The result can then be processed mainly in the same way as described in 
Section~\ref{impl-ccode-read} for a table built from a single input file.

\subsection{Dummy Declarations for Preprocessor Macros}
\label{impl-ccode-dummydecl}

As described in Section~\ref{design-preprocessor-macros} macro calls in C code must either be syntactically correct
C code or they must be converted to syntactically correct C code. Due to the language-c analysis step this is not 
sufficient. The analysis step checks for additional properties. In particular, it requires that every identifier 
is either declared or defined.

Thus for every identifier which is part of a converted macro call a corresponding declaration must be added to the 
C code. They are called ``dummy declarations'' since they are only used for making the analysis step happy. 

For all preprocessor defined constants Gencot automatically generates the required dummy declarations. The corresponding
macro calls always have the form of a single identifier occurring at positions where a C expression is expected. The type
of the identifier is irrelevant, hence Gencot always uses type \code{int} for the dummy declarations. For every preprocessor
constant definition of the form 
\begin{verbatim}
  #define NNN XXX
\end{verbatim}
a dummy declaration of the form
\begin{verbatim}
  extern int NNN;
\end{verbatim}
is generated. This is implemented by the additional filter \code{gencot-gendummydecls}. It is applied to the result of 
\code{gencot-selppconst}. The resulting dummy declarations are prepended to the input of the language-c preprocessor
since this prevents the lines from being counted for the \code{<stdin>} part.

Dummy declarations are generated with explicit external linkage. This causes them to be classified by language-c as 
declarations instead as (tentative) definitions, hence they are not signaled as repeated definitions when they occur 
in several symbol tables (see Section~\ref{impl-ccode-package}).

Flag macro calls do not occur in C code, hence no dummy declarations are required for them.

For all other macros the required dummy declarations must be created manually and added to the Gencot macro call conversion.
Even if no macro call conversion is needed because the macro calls are already in C syntax, it may be necessary to
add dummy declarations to satisfy the requirements of the language-c analysis step.

\subsection{Generating Cogent Code}
\label{impl-ccode-gencog}

When Gencot generates its Cogent target code it uses the data structures defined by the Cogent compiler for representing
its AST after parsing Cogent code. The motivation to do so is twofold. First, the AST omits details such as using code layout
and parentheses for correct code structure and the Cogent compiler provides a prettyprint function for its AST which cares
about these details. Hence, it is much easier to generate the AST and use the prettyprinter for output, instead of generating
the final Cogent program text. Second, by using the Cogent AST the generated Cogent code is guaranteed to be syntactically correct and
current for the Cogent language version of the used compiler version. Whenever the Cogent language syntax is changed
in a newer version, this will be detected when Gencot is linked to the newer compiler version.

\subsubsection{Cogent Surface Syntax Tree}

The data structures for the Cogent surface syntax AST are defined in the module Cogent.Surface. It defines parameterized types
for the main Cogent syntax constructs (\code{TopLevel}, \code{Alt}, \code{Type}, \code{Polytype}, \code{Pattern}, 
\code{IrrefutablePattern}, \code{Expr}, and \code{Binding}), where the type parameters determine the types of the 
sub-structures. Hence the AST types
can easily be extended by wrapping the existing types in own extensions which are then also used as actual type parameters.

Cogent itself defines two such wrapper type families: The basic unextended types \code{RawXXX} and the types \code{LocXXX}
where every construct is extended by a representation of its source location. 

All parameterized types for syntax constructs and the \code{RawXXX} and \code{LocXXX} types are defined as instances of 
class \code{Pretty} from
module \code{Text.PrettyPrint.ANSI.Leijen}. This prettyprinter functionality is used by the Cogent compiler for outputting
the parsed Cogent source code after some processing steps, if requested by the user.

As source location representation in the \code{LocXXX} types Cogent uses the type \code{SourcePos} from Module 
\code{Text.Parsec.Pos} in package \code{parsec}.
It contains a file name and a row and column number. This information is ignored by the prettyprinter.

\subsubsection{Extending the Cogent Surface Syntax}

Gencot needs to extend the Cogent surface syntax for its generated code in two ways:
\begin{itemize}
\item origin markers must be supported, as described in Section~\ref{impl-origin},
\item C function bodies must be supported in Cogent function definitions, as described in Section~\ref{design-fundefs-body}.
\end{itemize}

\paragraph{Origin Markers}

The origin markers are used to optionally surround the generated target code parts, which may be arbitrary syntactic constructs
or groups of them. Hence it would be necessary to massively extend the Cogent surface syntax, if they are added as explicit 
syntactic constructs. Instead, Gencot optionally adds the information about the range of source lines to the syntactic
constructs in the AST and generates the actual origin markers when the AST is output. 

Although the \code{LocXXX} types already support a source position in every syntactic construct, it cannot be used by Gencot,
since it represents only a single position instead of a line range. Gencot uses the \code{NodeInfo} values, since they represent
a line range and they are already present in the C source code AST, as described in Section~\ref{impl-ccode-read}. Hence, they
can simply be transferred from the source code part to the corresponding target code part. For the case that there is no
source code part in the input file (such as for code generated for external name references), or there is no position 
information available for the source code part, the \code{NodeInfo} is optional.

It may be the case that a target AST node is generated from a source code part which is not a single source AST node. Then
there is no single \code{NodeInfo} to represent the origin markers for the target AST node. Instead, Gencot uses the 
\code{NodeInfo} values of the first and last AST nodes in the source code part.

It may also be the case that a structured source code part is translated to a sequence of sub-part translations without target
code for the main part. In this case the \code{\#ORIGIN} marker for the main part must be added before the \code{\#ORIGIN} 
marker of the first target code part and the \code{\#ENDORIG} marker for the main part must be added after the \code{\#ENDORIG} 
marker of the last target code part. 

To represent all these cases, the origin information for a construct in the target AST consists of two lists of \code{NodeInfo}
values. The first list represents the sequence of \code{\#ORIGIN} markers to be inserted before the construct, here only the
start line numbers in the \code{NodeInfo} values are used. The second list represents the sequence of \code{\#ENDORIG} markers 
to be inserted after the construct, here only the end line numbers in the \code{NodeInfo} values are used. If no marker of
one of the kinds shall be present, the corresponding list is empty.

Additional information must be added to represent the marker extensions for placing the comments (the trailing ``+'' signs).
Therefore, a boolean value is added to all list elements.

Together, Gencot defines the type \code{Origin} for representing the origin information, with the value \code{noOrigin}
for the case that no markers will be generated:
\begin{verbatim}
  data Origin = Origin { 
    sOfOrig :: [(NodeInfo,Bool)], 
    eOfOrig :: [(NodeInfo,Bool)] } 
  noOrigin = Origin [] []
\end{verbatim}
Gencot adds an \code{Origin} value to every Cogent AST element. The type \code{Origin} is defined in the module 
\code{Gencot.Origin}

\paragraph{Embedded C Code}

Cogent function definitions are represented by the \code{FunDef} alternative of the type for toplevel syntactic constructs:

\begin{verbatim}
  data TopLevel t p e = 
    ... | FunDef VarName (Polytype t) [Alt p e] | ...
\end{verbatim}
The type parameter \code{e} for representing syntactic expressions is only used in this alternative and in the alternative
for constant definitions. Cogent constant definitions are generated by Gencot only from C enum constants (preprocessor
constants are processed by \code{gencot-prcconst} which is not implemented in Haskell). The defined value for a C enum
constant is represented in the C AST by the type for expressions. Together, instead of Cogent expressions, Gencot always
uses either a C expression or a Cogent expression together with a C function body (which syntactically is a statement) 
in the Cogent AST. 

To modify the Cogent syntax in this way, Gencot defines an own expression type with two alternatives for a C expression 
and for a Cogent expression together with a C statement:
\begin{verbatim}
  data GenExpr = ConstExpr Exp
               | FunBody RawExpr Stm
\end{verbatim}
where \code{Exp} and \code{Stm} are the types for C expressions and statements as defined by the language-c-quote AST 
(see Section~\ref{impl-ccode-expr}). Note that no \code{Origin} components are added, since the types \code{Exp} and 
\code{Stm} already contain \code{Origin} information. The type \code{RawExpr} is used for the dummy result expression.
It has no origin in the C source, therefore the raw type without origin information is sufficient. 

Other than for the dummy result expression, the Cogent AST expression type is not used by Gencot. 
Since bindings only occur in expressions, the AST type for Cogent bindings is not used either.

For the type parameters \code{t} and \code{p} for representing types and patterns, respectively, the normal types for 
the Cogent constructs are used, since Gencot generates both in Cogent syntax. The pattern generated for a function
definition is always a tuple pattern, which is irrefutable. Gencot never generates other patterns, hence the AST
type for irrefutable patterns is sufficient. 

Together, Gencot uses the following types to represent its extended Cogent surface AST:
\begin{verbatim}
  data GenToplv =
    GenToplv Origin (TopLevel GenType GenIrrefPatn GenExpr)
  data GenAlt =
    GenAlt Origin (Alt GenIrrefPatn GenExpr)
  data GenIrrefPatn = 
    GenIrrefPatn Origin (IrrefutablePattern VarName GenIrrefPatn)
  data GenType = 
    GenType Origin (Type GenExpr GenType)
  data GenPolytype = 
    GenPolytype Origin (Polytype GenType)
\end{verbatim}
The first parameter of \code{Type} for expressions is only used for Cogent array types, which are currently 
not generated by Gencot.

All five wrapper types are defined as instances of class \code{Pretty}, basically by applying the Cogent prettyprint
functionality to the wrapped Cogent AST type.

\subsection{Mapping Names}
\label{impl-ccode-names}

Names used in the target code are either mapped from a C identifier or introduced, as described in 
Section~\ref{design-names}. Different schemas are used depending on the kind of name to be generated.
The schemas require different information as input.

\subsubsection{General Name Mapping}

The general mapping scheme is applied whenever a Cogent name is generated from an existing C identifier.
Its purpose is to adjust the case, if necessary and to avoid conflicts between the Cogent name and
the C identifier.

As input this scheme only needs the C identifier and the required case for the Cogent name.
It is implemented by the function
\begin{verbatim}
  mapName :: Bool -> Ident -> String
\end{verbatim}
where the first argument specifies whether the name must be uppercase.

\subsubsection{Cogent Type Names}

A Cogent type name (including the names of primitive types) may be generated as translation of a C 
primitive type, a C typedef name, or a C struct/union/enum type reference. 

A C primitive type is translated according to the description in Section~\ref{design-types}. Only the
type specifiers for the C type are required for that.

A C typedef name is translated by simply mapping it with the help of \code{mapName} to an uppercase name.
Only the C typedef name is required for that.

A C struct/union/enum type reference may be tagged or tagless. If it is tagged, the Cogent type name is
constructed from the tag as described in Section~\ref{design-names}: the tag is mapped with the help of
\code{mapName} to an uppercase name, then a prefix \code{Struct\_}, \code{Union\_} or \code{Enum\_} is 
prepended. For this mapping the tag and the kind (struct/union/enum) are required. Both are contained
in the language-c type \code{TypeName} which is used to represent a reference to a struct/union/enum.

If the reference is untagged, Gencot nevertheless generates a type name, as motivated and described 
in Section~\ref{design-names}. As input it needs the kind and the position of the struct/union/enum 
definition. The latter is not contained in the \code{TypeName}, it contains the position of the reference
itself. To access the position of the definition, the definition must be retrieved from the symbol table
in the monadic state. To access the real name of the input file it must be retrieved from the user
state (see Section~\ref{impl-ccode-trav}). Hence, the mapping function is defined as a monadic action. 

Together the function for translating struct/union/enum type references is
\begin{verbatim}
  transTagName :: TypeName -> f String
\end{verbatim}
where \code{f} is a \code{FileNameTrav} and \code{MonadTrav} (see Section~\ref{impl-ccode-trav}).

If the definition itself is translated, it is already available and need not be retrieved from the map. 
However, the user state is still needed to map the generic name \code{<stdin>} to the true source file 
name. Therefore Gencot uses function \code{transTagName} also when translating the definition.

\subsubsection{Cogent Function Names}

Cogent function names are generated from C function names. A C function may have external or internal
linkage, according to the linkage the Cogent name is constructed either as a global name or as a name specific
to the file where the function is defined. For deciding which variant to use for a function name reference,
its linkage must be determined. It is available in the definition or in a declaration for the function name,
either of which must be present in the symbol table. The language-c analysis module replaces all 
declarations in the tyble by the
definition, if that is present in the parsed input, otherwise it retains a declaration. 

A global function name is generated by mapping the C function name with the help of \code{mapName} to
a lowercase Cogent name. No additional information is required for that.

For generating a file specific function name, the file name of the definition is required. Note that 
this is only done for a function with internal linkage, where the definitions must be present in
the input whenever the function is referenced. The definition contains the position information
which includes the file name. Hence, the symbol table together with the real name of the input file 
is sufficient for translating the name. To make both available the translation function is defined as 
a monadic action.

In C bodies function names cannot be syntactically distinguished from variable names. Therefore, Gencot
uses a common function for translating function and variable names. For a description how variable
names are translated see Section~\ref{impl-ccode-expr}.
\begin{verbatim}
  transObjName :: Ident -> f String
\end{verbatim}
where \code{f} is a \code{FileNameTrav} and \code{MonadTrav} (see Section~\ref{impl-ccode-trav}).

Similar as for tags, the function is also used when translating a function definition, although the 
definition is already available.

\subsubsection{Cogent Constant Names}

Cogent constant names are only generated from C enum constant names. They are simply translated
with the help of \code{mapName} to a lowercase Cogent name. No additional information is required.

\subsubsection{Cogent Field Names}

C member names and parameter names are translated to Cogent field names. Only if the C name is
uppercase, the name is mapped to a lowercase Cogent name with the help of \code{mapName}, 
otherwise it is used without change. Only the C name is required for that, in both cases it is
available as a value of type \code{Ident}. The translation is implemented by the function
\begin{verbatim}
  mapIfUpper :: Ident -> String
\end{verbatim}

\subsection{Generating Origin Markers}
\label{impl-ccode-origin}

For outputting origin markers in the target code, the AST prettyprint functionality must be extended.

The class \code{Pretty} used by the Cogent prettyprinter defines the methods
\begin{verbatim}
  pretty :: a -> Doc
  prettyList :: [a] -> Doc
\end{verbatim}
but the method \code{prettyList} is not used by Cogent. Hence, only the method \code{pretty} needs to be defined
for instances. The type \code{Doc} is that from module \code{Text.PrettyPrint.ANSI.Leijen}.

The basic approach is to wrap every syntactic construct in a sequence of \code{\#ORIGIN} markers and 
a sequence of \code{\#ENDORIG} markers according to the origin information for the construct in the extended AST. 
This is done by an instance definition of the form
\begin{verbatim}
  instance Pretty GenToplv where
    pretty (GenToplv org t) = addOrig org $ pretty t
\end{verbatim}
for \code{GenToplv} and analogous for the other types. The function \code{addOrig} has the type
\begin{verbatim}
  addOrig :: Origin -> Doc -> Doc
\end{verbatim}
and wraps its second argument in the origin markers according to its first argument.

The Cogent prettyprinter uses indentation for subexpressions. Indentation is implemented by the \code{Doc} type, 
where it is called ``nesting''. The prettyprinter maintains a
current nesting level and inserts that amount of spaces whenever a new line starts. 

The origin markers must be positioned in a separate line, hence \code{addOrig} outputs a newline before and after
each marker. This is done even at the beginning of a line, since due to indentation it cannot safely be determined
whether the current position is at the beginning of a line. Cogent may change the nesting of the next line after \code{addOrig}
has output a marker (typically after an \code{\#ENDORIG} marker). The newline at the end of the previous marker 
still inserts spaces according to the old nesting level, which determines the current position at the begin of
the following marker. This is not related to the new nesting level. 

This way many additional newlines are generated, in
particular an empty line is inserted between all consecutive origin markers. The additional newlines are later removed
together with the markers, when the markers are processed. Note that, if a syntactic construct
is nested, the indentation also applies to the origin markers and the line after it. To completely remove an
origin marker from the target code it must be removed together with the newline before it and with the newline 
after it and the following indentation. The following indentation can be determined since it is the same as that 
for the marker itself (a sequence of blanks of the same length). 

\subsubsection{Repeated Origin Markers}

Normally, target code is positioned in the same order as the corresponding source code. This implies, that
origin markers are monotonic. A repeated origin marker is a marker with the same line number as its previous marker.
Repeated origin markers of the same kind must be avoided, since they would result in duplicated comments or 
misplaced directives.
Repeated origin markers of the same kind occur, if a subpart of a structured source code part begins or ends 
in the same line as its main part. In this case only the outermost markers must be retained.

An \code{\#ENDORIG} marker repeating an \code{\#ORIGIN} marker means that the source code
part occupies only one single line (or a part of it), this is a valid case. 
An \code{\#ORIGIN} marker repeating an \code{\#ENDORIG} marker means that the previous source code
part ends in the same line where the following source code part begins. In this case the markers are
irrelevant, since no comments or directives can be associated with them. However, if they are
present they introduce unwanted line breaks, hence they also are avoided by removing both of them.

Together, the following rules result. In a sequence of repeated \code{\#ORIGIN} markers, only the first one 
is generated. In a sequence of repeated \code{\#ENDORIG} markers only the last one is generated.
If an \code{\#ORIGIN} marker repeats an \code{\#ENDORIG} marker, both are omitted.

There are several possible approaches for omitting repeated origin markers:
\begin{itemize}
\item omit repeated markers when building the Cogent AST
\item traverse the Cogent AST and remove markers to be omitted
\item output repeated markers and remove them in a postprocessing step
\end{itemize}
Note, that it is not possible to remove repeated markers already in the language-c AST, since there a \code{NodeInfo}
value always corresponds to two combined markers.

Handling repeated markers in the Cogent AST is difficult, because for an \code{\#ORIGIN} marker the context
before it is relevant whereas for an \code{\#ENDORIG} marker the context after it is relevant. An additional
AST traversal would be required to determine the context information. The first approach is even more complex
since the context information must be determined from the source code AST where the origin markers are not
yet present. 

For this reason Gencot uses the third approach and processes repeated markers in the generated target code text,
independent from the syntactical structure.

\subsubsection{Filter for Repeated Origin Marker Elimination}

The filter \code{gencot-reporigs} is used for removing repeated origin markers. It is implemented as an awk script.

It uses five string buffers: two for the previous two origin markers read, and three for the code before,
between, and after both markers. Whenever all buffers are filled (the buffer after both markers with a 
single text line; this line exists, since consecutive markers are always separated by an empty line),
the markers are processed as follows, if they have the same line number: in the case of two 
\code{\#ORIGIN} markers the second is deleted, in the
case of two \code{\#ENDORIG} markers the first is deleted, and in the case of an \code{\#ORIGIN}
marker after an \code{\#ENDORIG} marker both are deleted. In the latter case the line number of the
\code{\#ORIGIN} marker is remembered and subsequent \code{\#ORIGIN} markers with the same line number
are also deleted.

When both markers have different line numbers or if an \code{\#ENDORIG} marker follows an \code{\#ORIGIN}
marker the first marker and the code before it are output and the buffers are filled until the next marker
has been read.

\subsection{Generating Expressions}
\label{impl-ccode-expr}

For outputting the Cogent AST the prettyprint functionality must be extended to 
output C function bodies and the C expressions used for constant definitions. Additionally, at least in function bodies,
origin markers must be generated to be able to re-insert comments and preprocessor directives. Finally, all names
occurring free in a function body or a constant expression must be mapped to Cogent names.

The language-c prettyprinter is defined in module \code{Language.C.Pretty}. It defines an own class \code{Pretty} with 
method \code{pretty} to convert the AST types to a \code{Doc}. However, other than the Cogent prettyprinter, it uses 
the type \code{Doc} from module \code{Text.PrettyPrint.HughesPJ} instead of module \code{Text.PrettyPrint.ANSI.Leijen}.
This could be adapted by rendering the \code{Doc} as a string and then prettyprinting this string to a \code{Doc}
from the latter module. This way, a prettyprinted function body could be inserted in the document created by the
Cogent prettyprinter.

\subsubsection{Origin Markers}

For generating origin markers, a similar approach is not possible, since they must be inserted between single statements,
hence, the function \code{pretty} must be extended. Although it does not use the \code{NodeInfo}, it is only defined for
the AST type instances with a \code{NodeInfo} parameter and has no genericity which could be exploited for extending it.
Therefore, Gencot has to fully reimplement it. 

In the prettyprint reimplementation the target code parts must be wrapped by origin markers
in the same way as for the Cogent AST. However, for the type \code{Doc} from module \code{Text.PrettyPrint.HughesPJ} 
this is not possible, since newlines are only
available as separators between documents and cannot be inserted before or after a document. An alternative choice
would be to use the type \code{Doc} from \code{Text.PrettyPrint.ANSI.Leijen}, as the Cogent prettyprinter does.
However, the approach of both modules is quite different so that it would be necessary to write a new C 
prettyprint implementation nearly from scratch. 

It has been decided to use another approach which is expected to be simpler. The alternative C parser language-c-quote 
also has a prettyprinter. It generates a type \code{Doc} defined by a third module \code{Text.PrettyPrint.Mainland}.
It is similar to \code{Text.PrettyPrint.ANSI.Leijen} and also supports adding newlines before and after a document.
The language-c-quote prettyprinter is defined in the module \code{Language.C.Pretty} of language-c-quote and consists
of the method \code{ppr} of the class \code{Pretty} defined in module \code{Text.PrettyPrint.Mainland.Class.Pretty}.
This method is not generic at all, hence it must be completely reimplemented to extend it for generating origin 
markers. However, this reimplementation is straightforward and can be done by copying the original implementation
and only adding the origin marker wrappings. The resulting Gencot module is \code{Gencot.C.Output}.

Whereas the type \code{Doc} from \code{Text.PrettyPrint.ANSI.Leijen} provides a \code{hardline} document which always
causes a newline in the output, the type \code{Doc} from \code{Text.PrettyPrint.Mainland} does not. Normal line breaks
are ignored in certain contexts, if there is enough room. Using normal line breaks around origin markers could result
in origin markers with other code in the same line before or after the marker.

For the reimplemented language-c-quote prettyprinter Gencot defines its own \code{hardline} by using a newline 
which is hidden for type \code{Doc}. This could be implemented without nesting the marker and the subsequent line.
However, if at the marker position a comment is inserted, the subsequent line should be correctly indented.
To achieve this, the \code{hardline} implementation also adds the current nesting after the newline.

Hiding the newline from \code{Doc} implies that the ``current column'' maintained by \code{Doc} is not
correct anymore, since it is not reset by the \code{hardline}. Every \code{hardline} will instead advance the current
column by the width of the marker and twice the current nesting. This has two consequences.

First, in some places the language-c-quote prettyprinter uses ``alignment'' which means an indentation of subsequent lines
to the current column. This indentation will be too large after inserted markers. Gencot handles this by replacing 
alignment everywhere in the prettyprint implementation by a nesting of two additional columns. 

Second, the language-c-quote prettyprinter is parameterized by a ``document width''. It automatically breaks lines 
when the current column exceeds the document width. The incorrect column calculation causes many additional such
line breaks, since the current column increases much faster than normal. Gencot handles this by setting the document
width to a very large value (such as 2000 instead of 80) to compensate for the fast column increase.

\subsubsection{Using the language-c-quote AST}

Language-c-quote uses a different C AST implementation than language-c. To use its reimplemented prettyprinter, the 
language-c AST must be translated to a language-c-quote AST. This is not trivial, since the structures are somewhat
different, but it seems to be simpler than implementing a new C prettyprinter. The translation is implemented in
the module \code{Gencot.C.Translate}. 

Additionally the language-c-quote AST must be extended by \code{Origin} values. The language-c-quote AST already 
contains \code{SrcLoc} values which are similar to the \code{NodeInfo} values in language-c. Like these they cannot
be used as origin marker information since they cannot represent begin and end markers independently. Therefore
Gencot also reimplements the language-c-quote AST by copying its data types and replacing the \code{SrcLoc}
values by \code{Origin} values. This is implemented in module \code{Gencot.C.Ast}.

Together, this approach yields a similar structure as for the translation to Cogent: The Cogent AST is extended 
by the structures in \code{Gencot.C.Ast} to represent function bodies and constant expressions. The function for
translating from language-c AST to the Cogent AST is extended by the functions in \code{Gencot.C.Translate} to
translate function bodies and constant expressions from the language-c AST to the reimplemented language-c-quote 
AST, and the Cogent prettyprinter is extended by the prettyprinter
in \code{Gencot.C.Output} to print function bodies and constant expressions with origin markers.

In addition to translating the C AST structures from language-c to those of language-c-quote, the translation
function in \code{Gencot.C.Translate} implements the following functionality:
\begin{itemize}
\item generate \code{Origin} values from \code{NodeInfo} values,
\item map C names to Cogent names.
\end{itemize}

\subsubsection{Name Mapping}

Name mapping depends on the kind of name and may additionally depend on its type. Both information is
available in the symbol table (see Section~\ref{impl-ccode-trav}). However, the scope cannot be queried
from the symbol table. Hence it is not possible to map names depending on whether they are locally defined
or globally.

The following kinds of names may occur in a function body: primitive types, typedefs, tags, members, 
functions, global variables, enum constants, preprocessor constants, parameters and local variables.

Primitive type names and typedef names can only occur as name of a base type in a declaration. Primitive
type names are mapped to Cogent primitive type names as described in Section~\ref{design-types-prim}.

A typedef name may also occur in a declarator of a local typedef which defines the name. 
In both cases, as described in Section~\ref{design-fundefs-body}, Gencot
only maps the plain typedef names, not the derived types. The typedef names are mapped according to
Section~\ref{design-types-typedef}: If they ultimately resolve to a struct, union, or array type they
are mapped with an unbox operator applied, otherwise they are mapped without.

A tag name can only occur as base type in a declaration. It is always mapped to a name with a prefix 
of \code{Struct\_}, \code{Union\_}, or \code{Enum\_}. Tagless structs/unions/enums are not mapped at all.
Tag names are mapped according to Sections~\ref{design-types-enum} and~\ref{design-types-struct}: struct
and union tags are mapped with an unbox operator applied, enum tags are mapped without.

Gencot also maps defining tag occurrences. Thus an occurrence of the form 
\begin{verbatim}
  struct s { ... }
\end{verbatim}
is translated to
\begin{verbatim}
  struct #Struct_s { ... }
\end{verbatim}

Every occurrence of a field name can be syntactically distinguished. It is mapped according to 
Section~\ref{design-names} to a lowercase Cogent name if it is uppercase, otherwise it is unchanged.
Field names are also mapped in member declarations in locally defined structures and unions.

All other names syntactically occur as a primary expression. They are mapped depending on their semantic
information retrieved from the symbol table. In a first step it distinguishes objects, functions, 
and enumerators.

An object identifier may be a global variable, parameter, or local variable. It may also be a preprocessor 
constant since for them dummy declarations have been introduced which makes them appear as a global variable
for the C analysis. For the mapping the linkage is relevant, this is also available from the symbol table.

Identifiers for global variables may have external or internal linkage and are mapped depending on the
linkage. Identifiers for parameters always have no linkage and are always mapped like field names. Identifiers
for local variables either have no linkage or external linkage. In the first case they are mapped like
field names. In the second case they cannot be distinguished from global variables with external linkage,
and are mapped to lowercase. The dummy declarations introduced for preprocessor constants
always have external linkage, the identifiers are mapped to lowercase. Together, object identifiers with 
internal linkage are mapped as described in Section~\ref{design-names}, object identifiers with external
linkage are mapped to lowercase, and object identifiers with no linkage are mapped to lowercase if they are
uppercase and remain unchanged otherwise.

An identifier for a function has either internal or external linkage and is mapped depending on its linkage.
An identifier for an enumerator is always mapped to lowercase, like preprocessor constants.

Identifiers for local variables may also occur in a declarator of a local object definition which defines 
the name. They are also mapped depending on their linkage, as described above.

\subsection{Traversing the C AST}
\label{impl-ccode-trav}

The package language-c uses a monad for traversing and analysing the C AST. The monad is defined in module 
\code{Language.C.Analysis.TravMonad} and mainly provides the symbol table and user state during the traversal.
The traversal itself is implemented by a recursive descent according to the C AST using a separate function
for analysing every syntactic construct. 

When processing the semantic map resulting from the language-c analysis Gencot implements similar recursive 
descents using a processing function for every syntactic construct. For this it uses the same monad for two reasons.
\begin{itemize}
\item the definitions and declarations of the global identifiers are needed for accessing their types and for
mapping the identifiers to Cogent names,
\item additionally, the definitions and declarations of locally defined identifiers are needed in C function
bodies for the same purpose.
\end{itemize}

The global definitions and declarations in the symbol table correspond to the semantics map which is the result 
of the language-c analysis step. It is created from the symbol table after the initial traversal of the C AST. Although Gencot 
processes the content of the semantics map, it is not available as a whole in the processing functions. Instead
of passing the semantics map as an explicit parameter to all processing functions, Gencot uses monadic traversals
through the relevant parts of the semantics map, which implicitly make the symbol table available to all 
processing functions. This is achieved by reusing the symbol table after the analysis phase for the traversals
of the semantics map.

Additionally, when processing the C function bodies, the symbol table is used for managing the local declarations. 
This is possible because although the analysis phase translates global declarations and definitions to a 
semantic representation, it does not modify function bodies and returns them as the original C AST. Since
the information about local declarations is discarded at the end of its scope, the information is not 
present anymore in the symbol table after the analysis phase. Gencot uses the symbol table functionality
to rebuild this information during its own traversals.

The user state is used by Gencot to provide additional information, depending on the purpose of the traversal.
A common case is to make the actual name of the processed file available during processing. In the \code{NodeInfo}
values in the AST it is always specified as \code{<stdin>} since the input is read from a pipe. All C processing 
filters take the name of the original C source file as an additional argument. It is added to the user state 
of traversal monads so that it can be used during traversal.

This is supported by defining in module \code{Gencot.Name} the class \code{FileNameTrav} as
\begin{verbatim}
  class (Monad m) => FileNameTrav m where
    getFileName :: m String
\end{verbatim}
so that the method \code{getFileName} can be used to retrieve the source file name from all traversal monads of 
this class. 

A second information is the set of external type names directly used in the Cogent compilation unit. All other 
external type names are resolved during translation, as described in Section~\ref{design-modular}.

This is supported by defining in module \code{Gencot.Tranversal} the class \code{TypeNamesTrav} as
\begin{verbatim}
  class (Monad m) => TypeNamesTrav m where
    stopResolvTypeName :: Ident -> m Bool
\end{verbatim}
so that the method \code{stopResolvTypeName} can be used to test a type name whether it should be resolved or not.
The list can be deactivated using a flag in the user state. This is useful, if only code is processed which belongs
to the Cogent compilation unit. Then all referenced external type names are directly used and no additional information about
the external type names is required, the list can be deactivated and be empty.

Another information needed during C AST traversal is the item property map (see Section~\ref{impl-itemprops-internal}).
It is used for translating all types, as described in Section~\ref{design-types-itemprops}.

Finally, Gencot maintains a list of tag definitions which are processed in advance as nested, as described in 
Section~\ref{impl-ccode-read}. This list is implemented as a list of elements of type \code{SUERef} which is used
by language-c to identify both tagged and untagged struct/union/enum types.

The utilities for the monadic traversal of the semantics map are defined in module \code{Gencot.Traversal}. 
The main monadic type is defined as
\begin{verbatim}
  type FTrav = Trav (String,[SUERef],ItemProperties,(Bool,[String]))
\end{verbatim}
where \code{String} is the type used for storing the original C source file name in the user state,
\code{ItemProperties} is the type for storing the item property map, the second component is the list for
maintaining tag definitions processed as nested, and the last component is the list of directly used external type names 
together with the flag for activating or deactivating the list. \code{FTrav} is an
instance of \code{FileNameTrav} and \code{TypeNamesTrav}. As execution function for the monadic actions the functions
\begin{verbatim}
  runFTrav :: DefTable -> (String,ItemProperties,(Bool,[String])) -> FTrav a -> IO a
  runWithTable :: DefTable -> FTrav a -> IO a
\end{verbatim}
are defined. The first one takes the symbol table, the original C source file name, the item property map and
the list of directly used external type names as arguments to initialize the state. The tag definition list
is always initialized as empty. The second 
function leaves also the other three components empty and deactivated. The functions are themselves 
\code{IO} actions and print error messages generated during traversal to the standard output.

In the monadic actions the symbol table can be accessed by actions defined in the modules 
\code{Language.C.Analysis.TravMonad} and \code{Language.C.Analysis.DefTable}. An identifier can be
resolved using the actions
\begin{verbatim}
  lookupTypeDef :: Ident -> FTrav Type
  lookupObject :: Ident -> FTrav (Maybe IdentDecl)
\end{verbatim}
For resolving tag definitions the symbol table must be retrieved by
\begin{verbatim}
  getDefTable :: FTrav DefTable
\end{verbatim}
then the struct/union/enum reference can be resolved by
\begin{verbatim}
  lookupTag :: SUERef -> DefTable -> Maybe TagEntry
\end{verbatim}
To maintain the list of tag definitions processed as nested two monadic actions are defined:
\begin{verbatim}
  markTagAsNested :: SUERef -> FTrav ()
  isMarkedAsNested :: SUERef -> FTrav Bool
\end{verbatim}

Additionally, there are actions to enter and leave a scope and actions for inserting definitions.

An error can be recorded in the monad using the action
\begin{verbatim}
  recordError :: Language.C.Data.Error.Error e => e -> m () 
\end{verbatim}

The item property map can be accessed by the monadic actions
\begin{verbatim}
  getProperties :: String -> FTrav [String]
  hasProperty :: String -> String -> FTrav Bool
\end{verbatim}
where the last \code{String} argument is an item identifier. The first action returns the list of all properties declared 
for the item, the second action tests whether the property named as first argument is declared for the item.

\subsection{Creating and Using the C Call Graph}
\label{impl-ccode-callgraph}

In some Gencot components we use the C call graph. This is the mapping from functions to the functions
invoked in their body. Here we describe the module \code{Gencot.Util.CallGraph} which provides
utility functions for creating and using the call graph.

The set of invoked functions is determined by traversing the bodies of all function definitions after the analysis
phase. The callback handler is not used since it is only invoked for declarations and definitions and does not help
for processing function invocations.

Invocations can be identified purely syntactically as C function call expressions. The invoked function is usually 
specified by an identifier, however, it can be specified as an arbitrary C expression. We only support the cases
where the invoked function is specified as an identifier for a function or function pointer, by a chain of 
member access operations starting at an identifier for an object of struct or union type, or by an array index
expression where the array is specified as an identifier or member access chain and the element type is a function
pointer type. All other invocation where the invoked function is specified in a different way are ignored and not 
added to the call graph.

The starting identifier can be locally declared, such as a parameter of the function where the invocation occurs. The 
declaration information of these identifiers would not be available after the traversal which builds the call graph.
To make the full information about the invoked functions available, Gencot inserts the declarations into the call graph 
instead of the identifiers. In the case of a member access chain it uses the struct or union type which has the 
invoked function pointer or the indexed function pointer array as its direct member. This struct or union type
must have a declared tag, otherwise the invocation is ignored and not inserted into the call graph.

The information about such an invocation in a function body is represented by the following type:
\begin{verbatim}
  data CGInvoke =
      IdentInvoke IdentDecl Int
    | MemberTypeInvoke CompType MemberDecl Int
\end{verbatim}
The additional integer value specifies the number of actual arguments in this invocation.
Note that in a function definition
the parameters are represented in the symbol table by \code{IdentDecl}s, not by \code{ParamDecl}s. In the case
of an array element invocation the actual index is ignored, all array elements are treated in a common way.

The call graph has the form of a set of globally described invocations. These are triples consisting of the definition
of the invoking function, the invocation, and a boolean value telling whether the identifier in the case of an
\code{IdentInvoke} is locally defined in the invoking function:
\begin{verbatim}
  type CallGraph = Set CGFunInvoke
  type CGFunInvoke = (FunDef, CGInvoke, Bool)
\end{verbatim}
The equality relation for values of type \code{CGFunInvoke} is based on the location of the contained declarations
in the source file. This is correct since after the initial traversal every identifier has a unique declaration associated.

To access the declarations of locally declared identifiers, the symbol table with local declarations
must be available while building the call graph. Therefore we traverse the function bodies with the help of
the \code{FTrav} monad and \code{runWithTable} as described in Section~\ref{impl-ccode-trav}.

The call graph is constructed by the monadic action
\begin{verbatim}
  getCallGraph :: [DeclEvent] -> FTrav CallGraph
\end{verbatim}
It processes all function definitions in its argument list and ignores all other \code{DeclEvent}s.

The function 
\begin{verbatim}
  getIdentInvokes :: CallGraph -> Set LCA.IdentDecl
\end{verbatim}
returns the set of all invoked functions which are specified directly as an identifier. In particular, they include
all invoked functions which are no function pointers.

The declaration of an invoked function also tells 
whether the function or object is defined or only declared. Note that the traversal for collecting invocations is a ``second 
pass'' through the C source after the analysis phase of language-c. During analysis language-c replaces
declarations in the symbol table whenever it finds the corresponding definition.

To use the call graph the \code{CallGraph} module defines a traversal monad \code{CTrav} 
with the call graph in the user state. The corresponding execution function is
\begin{verbatim}
  runCTrav :: CallGraph -> DefTable -> (String,(Bool,[String])) -> CTrav a -> IO a
\end{verbatim}
The monadic action to access the call graph is
\begin{verbatim}
  lookupCallGraph :: Ident -> CTrav CallGraph
\end{verbatim}
It takes the identifier of an invoking function as argument and returns the part of the call graph for this function,
consisting of all invocations in its body.

The monad \code{CTrav} is an instance of classes \code{FileNameTrav} and \code{TypeNamesTrav}, so the own source file 
name can be accessed by \code{getFileName} and external type names can be tested for being directly used by 
\code{stopResolvTypeName} (see Section~\ref{impl-ccode-trav}). The corresponding information is passed as third argument
to \code{runCTrav}.


\section{Postprocessing Cogent Code}
\label{impl-post}
The Cogent code generated for C expressions and statements in the first translation phase as described in 
Section~\ref{impl-ccode-cstats} is in general neither correct nor efficient. Therefore it must be improved, 
which is done by postprocessing. The postprocessing is done directly in the Cogent AST. 

This approach taken by Gencot has several advantages over generating the Cogent code in a single phase.
First, the actual translation step ist rather simple and straightforward and even needs not take the C types
into account. Second, the postprocessing is done on a restricted subset of a purely functional language where 
there is no difference between statements and expressions, so it tends to be simpler. Third, it can be separated
into arbitrary many different processing steps which can be freely combined, since they all process the same
data structures (the Cogent AST). The drawback is that the code generation is not very efficient, because it
first builds a quite voluminous code which is then simplified by the postprocessing. However, the quality 
of the resulting code has been considered more important than the performance of the Gencot translation.

In the following sections the postprocessing steps are described independently of each other and the last
section describes how they are combined. Every postprocessing step corresponds to a transformation from
a Cogent expression to a Cogent expression and is implemented by a Haskell function of the form
\begin{verbatim}
  Xproc :: GenExpr -> GenExpr
\end{verbatim}
Note that no monadic actions are used, all information required for the processing must already be present in the 
expressions.

Postprocessing is applied to all expressions which occur in the generated Cogent program. These are function 
body expressions, the expressions in constant definitions, and the expressions in array type size specifications.

\subsection{Evaluating Constant Expressions}
\label{imp-post-const}

In several cases the original translation phase or postprocessing steps result in constant expressions built
from predefined operators. Such expressions can be statically evaluated and the resulting constant then may
enable other postprocessing steps. Therefore a constant expression evaluation is defined as auxiliary 
function for other postprocessing steps. It is implemented by the function
\begin{verbatim}
  evalproc :: GenExpr -> GenExpr
\end{verbatim}
defined in module \code{Gencot.Cogent.Simplify}.

The \code{evalproc} step is not applied on its own to arbitrary expressions. The reason is that a constant
expression may be intentionally present in the C program to show how a value is calculated. In that case the 
translation should also result in a constant expression in Cogent. Only if the evaluation is useful for other 
postprocessing steps it is applied.

The \code{evalproc} step only processes operator expressions. For them it recurses into the arguments. If all arguments
are constants it evaluates the operator and replaces the expression by the resulting constant. All other forms
of expressions are left unmodified, in particular, \code{evalproc} does not recurse into subexpressions which are
not operator applications.

\subsection{Simplifying Operator Application}
\label{imp-post-op}

If an expression cannot be completely evaluated statically, there are cases where it can be simplified.
The following cases are implemented by Gencot postprocessing.

If the first argument of a boolean operation evaluates to a constant the operation can be simplified according 
to the rules
\begin{verbatim}
  True  || e --> True
  False || e --> e
  True  && e --> e
  False && e --> False
\end{verbatim}
Currently only the first argument is treated this way, because only that case occurs in actual examples of translation
and postprocessing, mainly for the translation of \code{switch} statements.

Simplifying operator expressions using these rules is implemented by the function
\begin{verbatim}
  opproc :: GenExpr -> GenExpr
\end{verbatim}
defined in module \code{Gencot.Cogent.Simplify}.

\subsection{Simplifying \code{let}-Expressions}
\label{imp-post-let}

One of the simplest and most straightforward postprocessing steps is substitution of bound variables by the
expression bound to them. In Cogent variables can be bound by \code{let} expressions and by \code{match} and \code{lambda}
expressions. Variables bound in \code{match} and \code{lambda} expressions can usually only be substituted in special cases, 
therefore this processing step only substitutes variables bound in \code{let} expressions. 

The basic transformation is for an expression
\begin{verbatim}
  let v = expr1 in expr2
\end{verbatim}
to replace it by \code{expr2} where every free occurrence of \code{v} is substituted by \code{expr1}. This is only 
possible if after the substitution all free variables in \code{expr1} are still free in the resulting expression, i.e., 
they are not ``drawn under a binding'' in \code{expr2}. This could be avoided by consistent renaming of variables bound in 
\code{expr2}. Gencot never renames variables and does not substitute in this case.

This scheme can directly be extended to expressions of the form \code{let v1 = e1 and ... vn = en in e} using the
equivalence to an expression of the form \code{let v1 = e1 in let ... in let vn = en in e}.

As of February 2022, Cogent does not support closures for lambda expression. This means that a lambda expression must not
contain free variables, therefore lambda expressions in \code{expr2} are never inspected for substituting.

In a Cogent \code{let} expression instead of a variable \code{v} an (irrefutable) pattern \code{p} can be used:
\begin{verbatim}
  let p = expr1 in expr2
\end{verbatim}
An irrefutable pattern
is a variable or wildcard or it is a pattern for a tuple, record, array or the unit value were the components are again
irrefutable patterns. In other words, it is a complex structure of variables which is bound to an expression \code{expr1}
of a type for which the values have a corresponding structure. Every variable may occur only once in a pattern.

Currently, the translation phase only creates bindings with patterns which are either a single variable, or a flat tuple pattern where
all components are variables or wildcards, or a take pattern where all sub patterns are variables. The postprocessing does not 
introduce more complex patterns. However, to make the code more robust, these restrictions are not assumed for binding processing,
the processing is always implemented to work for arbitrary patterns.

If the pattern occurs as a whole in \code{expr2} it can be
substituted by \code{expr1} as described above. If only parts of the pattern occur (such as a single variable) it depends
on the structure of \code{expr2} whether such a part can be substituted. Gencot tries to substitute as much parts as possible 
and only retains those parts of the pattern for which a substitution is not possible.

If the substitution is successful the \code{let} expression is replaced by \code{expr2} which may again be a \code{let}
expression or any other kind of expression. Therefore the simplification may reduce the number of \code{let} expressions
and may replace a \code{let} expression by an expression of another kind.

Substitution of bound variables may lead to exponentially larger code, which must be avoided. Gencot uses an expression metrics
which roughly measures the size of the printed expression in the Cogent surface syntax. A binding is only substituted if the 
resulting expression is not much larger than the original \code{let} expression.

Simplifying a \code{let} expression by substitution can reduce the variables which occur free in it. This is the case if no
parts of the pattern \code{p} occur free in \code{expr2}, then \code{expr1} is removed and all variables which only occur free
in \code{expr1} are removed with it.

Simplifying \code{let} expressions is implemented by the function
\begin{verbatim}
  letproc :: GenExpr -> GenExpr
\end{verbatim}
defined in module \code{Gencot.Cogent.Simplify}.

\subsubsection{Processing Subexpressions}

When an expression \code{let p = expr1 in expr2} is simplified, first the subexpressions \code{expr1} and \code{expr2} are
simplified by processing all contained \code{let} expressions. Simplifying \code{expr1} has the following advantages for the 
substitution:
\begin{itemize}
\item The resulting expression usually is smaller. Then its substitution into \code{expr2} leads to a lower increase of
size and may be allowed whereas substitution of the original \code{expr1} would not be accepted.
\item Simplifying may reduce the free variables so that it may be possible to substitute it in more places than
the original expression without drawing free variables under a binding.
\item If \code{expr1} is again a \code{let} expression the pattern can only be substituted as a whole. After simplification 
it may have a form which corresponds more with the pattern so that also parts of the pattern can be substituted.
\end{itemize}
Simplifying \code{expr2} has the following advantages for the substitution:
\begin{itemize}
\item Simplifying may reduce the free variables so that there are fewer places for substituting the pattern
or parts of it. This may allow substitutions of patterns which where not possible in the original \code{expr2}.
It may also allow substitutions which would have lead to a too large growth of the original \code{expr2}.
\end{itemize}

Since an expression \code{let p1 = expr1 and p2 = expr2 in expr3} is equivalent to \code{let p1 = expr1 in (let p2 = expr2
in expr3)} this means that a sequence of bindings connected by \code{and} in a \code{let} expression is processed
from its end backwards.

\subsubsection{Pattern Substitution}

If (after its simplification) \code{expr1} has the same structure as the pattern \code{p} the binding could be split and 
the parts could be substituted independently. This would correspond to the transformation of the binding \code{p = expr1}
to the sequence 
\begin{verbatim}
  p1 = expr11 and ... pn = expr1n
\end{verbatim}
where the \code{pi} are the subpatterns of \code{p} and the \code{expr1i} are the corresponding subexpressions of \code{expr1}.
Note that the variables in the \code{pi} are pairwise disjunct since every variable may occur only once in \code{p}.
Then the sequence could be processed from its end, as described above.

However, the transformation is only correct, if no variable in \code{pi} occurs free in an expression \code{expr1j} with 
\code{j > i}, otherwise the transformation would draw it under the binding \code{pi = expr1i}. It could be tried to sort
the bindings to minimize this problem but in general it cannot be avoided. Additionally, there may be cases where \code{expr1}
even after its simplification has no structure corresponding with that of \code{p}, which also prevents the transformation.

For this reason, instead of transforming the binding and substituting it sequentially, Gencot deconstructs the pattern while
searching for matches. Whenever it searches the binding \code{p = expr1} in an expression (starting with \code{expr2}) 
it determines the variables occurring free in the expression and ``reduces'' the binding to these. Reducing a binding to
a set of variables is done by first replacing in \code{p} all variables which are not in the set by a wildcard (underscore) 
pattern. This is always possible. Then it is tried to remove as much wildcard parts from the binding, this is only possible if
the corresponding part can be removed from \code{expr1}. If \code{p} is a tuple pattern where some components are wildcards
they can be removed if \code{expr1} is a corresponding tuple expression. If \code{expr1} is an application of a function to
an argument the wildcard parts cannot be removed, since the tuple returned by the function is not available.

If no variable in the set occurs in the pattern all variables are replaced by wildcards and the binding is reduced to the 
empty binding, represented by the ``unit binding'' \code{() = ()}.

If after reducing it the binding is not empty, it is matched with the expression, which is successful if it has the same structure with the same
variables (which is only possible if the pattern contains no wildcards). If it matches the search stops, if not the binding 
is searched recursively in all subexpressions (by first reducing it to the subexpression). When the search reaches a single 
variable, the binding has been reduced to that variable. If the pattern consists of the same variable it matches there.

Otherwise the pattern is empty or it contains the variable together with other parts which means that it cannot be used to substitute the 
variable at that position in the expression. In this case at least the binding for that variable must be retained. Gencot 
determines all variables in \code{expr2} for which that is the case. Then it reduces the original binding \code{p = expr1}
to that set to determine the binding \code{p' = expr1'} which must be retained in the \code{let} expression. Only if no
such variable exists the binding can be completely removed from the \code{let} expression.

If \code{p} contains a \code{take} pattern for a record or array it can only match a \code{put} subexpression 
of exactly the same form in \code{expr2}. That is only present if the component is taken and put back without modifying it or the 
remaining record or array. In this case the \code{put} expression is replaced by the part of \code{expr1} corresponding to 
the \code{take} pattern.

Wildcards in a \code{take} pattern can only be removed if the corresponding part of \code{expr1} is a \code{put} expression. 
In the original translation phase Gencot always generates all put operations after all take operations in a sequence of bindings.
However, during postprocessing bindings may be rearranged, so that this case may occur.

Even if a part of the binding successfully matches in \code{expr2}, substitution may be prevented because it would draw a 
free variable in \code{expr1} under a binding. Such a binding may be the retained part of the original binding or it may 
be a binding in a \code{let} subexpression which has been retained during the simplification of \code{expr2}
or it may be a binding in a \code{match} expression which is not processed by the simplification. These cases are handled
by splitting the binding for which matches are searched whenever it is drawn under another binding in a (maximal) part allowed
to draw under the binding and a rest which must be retained. The rest is still searched under the binding to determine whether
the variables in it occur at all, if that is not the case it is not required in the expression and is not retained.

A binding is split according to a set of variables not to occur free by first determining all bound variables so that the 
corresponding part of the bound expression contains a free occurrence of a variable from the set. Then the binding is reduced
to this set and to its complement, yielding the binding to retain and the binding for substitution.

If \code{expr1} is a \code{let} or \code{if} expression or a match expression the binding cannot be reduced or split directly.
In these cases Gencot constructs the reduced or split binding by recursively reducing or splitting the subexpressions (the 
let-body, the if-branches, the match-alternatives).

The substitution and binding simplification is implemented in two phases. In the first phase the matches for the pattern are 
searched in \code{expr2}, resulting in the part to be retained and for every matching part the number of successful matches
The matching parts are actually substituted in \code{expr2} in the second phase. If the retained part is not empty it is 
prefixed to the result.

\subsubsection{Growth Restriction}

As size metrics for an expression the number of characters appearing in its surface representation is used. It could be determined
by actually prettyprinting the expression and measuring the size of the resulting string. However, it is assumed to be more
efficient to traverse the expression and calculate the size from the number of characters in the names and literals and in
the keywords, special characters and separating blanks needed for constructing composed expressions.

After the first phase the metrics of the \code{let} expression is calculated. Since for each matching part to be substituted it 
has been determined in the first phase how often it occurs in \code{expr2} the metrics for the simplified expression can be 
calculated and it is known how much each subpattern contributes to its size. If the size is larger than for the original 
expression and its growth exceeds a fixed limit factor additional binding parts are determined which are retained. 
Beginning with the binding part with the largest contribution, parts are retained until the growth is below the limit.

Instead of only taking its contribution to the size into account, binding parts could also be selected according to the kind of
variables they contain. 
Gencot uses different kinds of variables in its generated code (see Section~\ref{impl-ccode-cstats}): value variables, component and 
index variables, the control and result variables, and variables corresponding to C object names. These could be prioritized
as follows: first as many value 
variables are substituted as possible in a complete expression, then the control variables, then the component and index variables and
finally the C object names. In this way the most ``technical'' variables are substituted before the more ``semantical''.

The reference metrics is calculated for the expression after simplifying its subexpressions. This means that the growth limit factor
applies to every subexpression simplification step separately. This has two implications. First, a subexpression simplification
may strongly reduce the size of the subexpression and that may also reduce the size of the \code{let} expression, which becomes 
the reference for its own simplification. Thus it is not possible to tolerate a larger growth after strongly reducing the size for
subexpressions. Alternatively, the reference size could be measured before simplifying the subexpressions. In the code generated
by Gencot there are typically large nestings of \code{let} expressions with unnecessary ``chain bindings''. It is assumed that it
does not yield good results when these unnecessary large expressions are used as reference, therefore Gencot uses the first approach.

Second, in the worst case each simplification step grows the expression by the limit factor which still results in an overall exponential growth
relative to the number of subexpressions. Therefore the growth limit factor should not be much larger than 1. The effect of this
factor and a good selection for it must be determined by practical tests. Alternatively the factor could be specified as an input 
parameter for Gencot so that it can be selected specifically for every translated C program.

\subsection{Simplifying If-Expressions}
\label{imp-post-if}

The most straightforward simplification of a conditional expression is replacing it by one of the branches, if the condition can be
statically evaluated. Gencot tries this, by recursing into the condition and then using \code{evalproc} (see Section~\ref{imp-post-const})
for evaluating the condition, before it recurses into the branches. This avoids processing a branch which is removed afterwards. 
If the condition cannot be statically evaluated, both branches are processed recursively.

\subsubsection{Other Used Transformations}

Afterwards, the following additional simplification rules are applied, if possible.
\begin{itemize}
\item If both branches are the same expression, the conditional expression is replaced by this expression.
\item If both branches statically evaluate to a boolean constant, the conditional expression is replaced
by the condition or the negated condition according to the rules
\begin{verbatim}
  if c then True else False --> c
  if c then False else True --> not c
\end{verbatim}
\item The condition is substituted by \code{True} in the \code{then} branch and by \code{False} in the \code{else} branch. This 
may enable additional simplifications in subsequent iterations (see Section~\ref{imp-post-combine}).
\end{itemize}

The following rule is applied to operator expressions where the first argument is a conditional expression and the second argument
can be statically evaluated:
\begin{verbatim}
  (if c then e1 else e2) <op> e -->
  if c then e1 <op> e else e2 <op> e
\end{verbatim}
This transformation may enable the static evaluation of the resulting branches. 

All these rules have been selected, because they specifically apply to conditional expressions resulting from the Gencot translation
and postprocessing, in particular for the control variable.

Simplifying \code{if} expressions using these rules is implemented by the function
\begin{verbatim}
  ifproc :: GenExpr -> GenExpr
\end{verbatim}
defined in module \code{Gencot.Cogent.Simplify}.

\subsubsection{Unused Transformations}

Other rules would transform conditional expressions where one branch can be statically evaluated to a boolean value according to
\begin{verbatim}
  if c then True else e  --> c || e
  if c then False else e --> not c && e
  if c then e else True  --> not c || e
  if c then e else False --> c && e
\end{verbatim}
This is not used, because it removes boolean constants which may be useful for static evaluation after other processing steps.

Other rules for simplifying conditionals with boolean values are
\begin{verbatim}
  if c then e else not e --> c == e
  if c then not e else e --> c /= e
\end{verbatim}
These are not used because they turn conditional expressions into equations, which currently are not processed any further.

If a branch is again a conditional expression, the following transformation is possible:
\begin{verbatim}
  if c then (if c' then e1 else e2) else e -->
  if c' then (if c then e1 else e)
        else (if c then e2 else e)
\end{verbatim}
It is not used because it duplicates expression \code{e} and does not reduce the structure of conditional expressions,
so there is the danger of cyclic transformation.

Another rule would split conditional tuples into a tuple of conditionals according to
\begin{verbatim}
  if c then (t1,..,tn) else (e1,..,en) --> 
  (if c then t1 else e1, .., if c then tn else en)
\end{verbatim}
which would allow further simplification for all components with \code{ti = ei} and it could allow additional substitutions
by \code{letproc}, because the components can be substituted or omitted separately. However, if both are not applicable, it 
tends to enlarge the expression by copying the condition \code{c}. Therefore the effect on \code{letproc} has been implemented
there by the way how conditional expressions are split and the rule is not used here for \code{ifproc}.

The substitution of the condition in the branches could be generalized by substituting values for variables which can be
inferred from the condition, such as in
\begin{verbatim}
  if i == 0 then e1 else e2
\end{verbatim}
where it is possible to substitute \code{i} by \code{0} in \code{e1}. Even more general, the condition can be interpreted as
a set of equations for its free variables, if it can be solved for some of them they can be substituted by their solution
in the first branch. It has not yet been considered whether such substitutions would have an effect that would pay for the 
additional complexity.

If the condition is itself a conditional expression, it may be possible to derive substitutions, although the condition does
not occur as a whole in the branches. As an example, the following transformation could be applied:
\begin{verbatim}
  if (if c then x else y) 
     then (if c then (if x then e1 else e2) else e3) 
     else e4 
  -->
  if (if c then x else y) 
     then (if c then e1 else e3)
     else e4
\end{verbatim}
It has not yet been considered, whether such transformations would be useful for cases occurring during translations of C programs.

\subsection{Function Call Processing}
\label{imp-post-funcall}

\subsection{Take/Put Processing}
\label{imp-post-takeput}

\subsection{Combining the Steps}
\label{imp-post-combine}

Generally, the postprocessing steps recurse into the structure of the processed expression. Thus it would be possible to combine them
on every recursion level by combining the non-recursive steps to a common postprocessing function which recurses into the expression
structure. Alternatively each step recurses on its own and the overall simplification applies the steps sequentially to the toplevel
expressions.

Some steps depend on each other in a cyclic way. These are in particular the functions \code{evalproc}, \code{letproc}, and \code{ifproc}.
After \code{letproc} substitutes variables it may be possible to evaluate more expressions by \code{evalproc} than before. This may
allow additional simplifications by \code{ifproc} since conditions may evaluate to a constant. These simplifications may remove code parts 
with occurrences of free variables which in turn may allow additional simplification by \code{letproc}. Therefore these steps must 
be applied in a loop until the expression does not change any more.

The \code{letproc} step works bottom up and is applied to subexpressions before it is applied to an expression. The \code{ifproc} step 
for a subexpression will only benefit if variables are substituted from the context. Thus it does not help to iterate these steps for
the same subexpression. Instead, both steps recurse on their own and are iterated on the toplevel expressions.

The \code{evalproc} step is only used as auxiliary step in other postprocessing steps. Since the other steps recurse into the control
structures like \code{if} and \code{let}, it is not necessary for \code{evalproc} to do this. Therefore \code{evalproc} only recurses
into operator subexpressions.

The \code{opproc} step recurses into arbitrary subexpressions. It processes cases which are typically created by \code{ifproc}.
Therefore it is executed after every toplevel execution of \code{ifproc}.


\section{Reading C Types Generated by Cogent}
\label{impl-ctypinfo}
For postprocessing the output generated by Cogent some information about how the Cogent compiler translates the 
Cogent types to C types is required. The main reason is that Cogent does not preserve the Cogent type names,
instead it generates type names of the form \code{t<nn>} in the C program.

Cogent translates record types, tuple types and variant types to C struct types. The Gencot array and pointer types are 
always mapped to a wrapper record in Cogent, so they are also translated to a C struct type. Function types 
are translated to enumeration types with an index number for each known function of that type. Basic types are
translated using specific type names in C (\code{u8, u16, u32, u64, unit\_t, bool\_t}). Type \code{String} is
always translated to \code{char *}. Abstract types are translated to C types of exactly the same name as in Cogent.

So the relevant information is that for all generated struct types and for the enumeration types generated for 
the function types. For struct types the information consists of the names and types of all members. For the latter
the information consists of the parameter and result types and of the set of known functions.

This information is available in the main \code{.h} file written by the Cogent compiler. Gencot parses this file 
and provides the information in a textual representation which can be read and processed when needed.

\subsection{Textual Representation of Type Information}
\label{impl-ctypinfo-repr}

The textual representation consists of a sequence of lines. 

For the generated struct types every line describes a single struct type using the following format:
\begin{verbatim}
  <name> <memnam>:<memtype> ... <memnam>:<memtype>
\end{verbatim}
where \code{<name>} is the struct type name of the form \code{t<nn>} and each \code{<memnam>} is the 
member id. For Cogent record types the member ids are equal to the record component names. For Cogent
variant types the member ids are equal to the alternative tags, with a member with id \code{"tag"}
and type \code{"tag\_t"}
prepended. For Cogent tuple types the member ids are \code{p1, p2, ...}. For a Gencot array type
the id of the single member is \code{arr<size>} (see Section~\ref{design-types-array}). For a Gencot 
pointer type the id of the single member is \code{cont} (which cannot be distinguished from a record 
with a single component of that name).

The \code{<memtype>} specifies the member's type and may have three different forms. It may be a single 
identifier referencing a struct type (for tuples, unboxed records and variants), an enumeration 
(for function types), or a basic type. It may be an identifier followed by a single \code{"*"} 
(for boxed records and variants). Since in Cogent the case of pointer-to-pointer cannot be expressed
a single star is always sufficient. Since the boxed form in Cogent is only available for records and
variants, an identifier followed by a star normally has the form \code{t<nn>}. However, Gencot implements
the type \code{MayNull b} where \code{b} is a boxed record by defining the typedef name \code{MayNull\_tnn}
as an alias for \code{tnn}, where \code{tnn *} is the type translation used for \code{b}. Thus an 
identifier followed by a star may also have the form \code{MayNull\_t<nn>}.

The third form specifies a an array type as
\begin{verbatim}
  <eltype>[<size>]
\end{verbatim}
and is used for Cogent builtin array types. Since Cogent builtin array types are always wrapped in a
struct, the element type \code{<eltype>} cannot be an array or pointer type and is always a single identifier. 
The array size \code{<size>} is always specified by a literal positive number which is not zero (since 
Cogent builtin arrays cannot be empty).

The member specifications are separated by whitespace, every member specification is a single word
without whitespace. 

For the translated function types every line describes a single function type or a function belonging 
to a type. Every type description is immediately followed by the list of its functions descriptions.
A type description uses the following format:
\begin{verbatim}
  <name> <restype> <partype>
\end{verbatim}
where \code{<name>} is the enumeration type name of the form \code{t<nn>}. The name and types are
separated by whitespace. Note that in Cogent every function has exactly one parameter and one result. 

A function belonging to 
a type is simply specified by its name (which is the same in Cogent and in the generated C code).
So type and function specification lines can be distinguished by the number of their words.

\subsection{Reading the C Source}
\label{impl-ctypinfo-read}

The \code{.h} C source generated by Cogent uses C preprocessor directives to include files and to define macros.
The included files contain type definitions for the translated basic Cogent types and for all abstract
polymorphic types defined in the Cogent program. They are required for parsing the C code. The 
macros defined in the \code{.h} file are never used there and can be ignored. Thus Gencot processes the
\code{.h} file using the standard C preprocessor \code{cpp} as described in Section~\ref{impl-ccode-preproc}
to execute and remove all preprocessor directives.

Then the file is parsed by a Haskell program using the language-c parser as described in 
Section~\ref{impl-ccode-read}. The file is read from standard input. It is parsed and then processed
by the language-c analysis module using function \code{readFromInput} in module \code{Gencot.Input}.
The resulting symbol table contains all globally defined identifiers.

To determine the required type information only the following definitions must be processed:
\begin{itemize}
\item \code{struct} definitions to determine all relevant struct types,
\item dispatcher function definitions to determine the information about function types.
\end{itemize}
Definitions from included files are not needed, so their content is filtered from the symbol table
as described in Section~\ref{impl-ccode-read}. The remaining symbol table entries are converted to
a list of \code{DeclEvent} values and filtered according to the required entries. Together,
Gencot uses the function \code{getDeclEvents} from module \code{Gencot.Input} in the same way as
for reading an original C program for translating it to Cogent.

\subsection{Generating the Struct Type Information}
\label{impl-ctypinfo-struct}

The struct type information is generated by a separate processing step. The \code{DeclEvent} list is filtered
so that only the struct definitions remain. Since every struct definition can be processed independently
and the state information used for translatig a C source is not used here, no monadic traversal is required. 

The processing is implemented by the function
\begin{verbatim}
  procStruct :: DeclEvent -> String
\end{verbatim}
defined in module \code{Gencot.Text.CTypInfo}. It is mapped to the filtered list of \code{DeclEvent}s and the resulting
List of Strings is output as a sequence of lines.

\subsection{Generating the Function Type Information}
\label{impl-ctypinfo-func}

The function type information is generated by a separate processing step. 

In its default configuration, Cogent uses the single enumeration type \code{untyped\_func\_enum} for enumerating
the functions of all function types. They are defined in the form
\begin{verbatim}
  typedef untyped_func_enum t<nn>;
\end{verbatim}
This is not sufficient for retrieving the information about a function type.

Instead, the information is taken from the dispatcher functions. Cogent generates for every function type a dispatcher 
function of the form
\begin{verbatim}
  static inline <restype> dispatch_<funtype>(
      untyped_func_enum a2, <paramtype> a3) { 
    switch (a2) {
      case FUN_ENUM_<name>: return <name>(a3);
        ...
      default: return <name>(a3);
  }}
\end{verbatim}
If the type has only one known function the body only consists of the default return statement.
The dispatcher function definition contains the function type name \code{<funtype>} and the \code{<restype>} and \code{<paramtype>}.
The names of the functions can be read from the \code{return} statements in the body. The \code{DeclEvent} list is filtered
so that only the dispatcher function definitions remain. Again, every dispatcher function definition
can be processed independently, no monadic traversal is required.

The processing is implemented by the function
\begin{verbatim}
  procFunc :: DeclEvent -> [String]
\end{verbatim}
defined in module \code{Gencot.Text.CTypInfo}. It returns the list of strings consisting of the type description and all related 
function names. It is mapped to the filtered list of \code{DeclEvent}s and the resulting
List is flattened and output as a sequence of lines.


\section{Implementing Basic Operations}
\label{impl-operations}
Gencot supports basic operations either by providing a Cogent implementation, or by providing
a C implementation for functions defined as abstract in Cogent.

\subsection{Implementing Polymorphic Functions}
\label{impl-operations-poly}

If a basic function is implemented in Cogent, it is automatically available for all possible argument types, according
to the Cogent definition of the argument types. 

Basic functions which cannot be implemented in Cogent are defined as abstract in Cogent and are implemented in
antiquoted C. Antiquotation is used to specify Cogent types, in particular, the argument types of polymorphic
functions. Then the Cogent mechanism for creating all monomorphic instances used in a program.

However, for some of the basic functions defined by Gencot the argument types are restricted in ways which cannot
be expressed in Cogent. An example is the restriction to linear types. In Cogent the ``permissions'' \code{DS} express 
the restriction that an argument type must be sharable and discardable, i.e., \textit{not} linear. But the opposite
restriction cannot be expressed. 

In such cases Gencot only provides function implementation instances for the restricted argument types. If the function
is used with other argument types in a program this is not detected as error by the Cogent compiler. However, due to 
the missing C implementation the resulting C program will not compile.

\subsubsection{Genops}

Additionally, there are cases where the Cogent antiquotation mechanism is not sufficient for generating the 
monomorphic instances from a common polymorphic template. The antiquotation mechanism can replace Cogent type
and type argument specifications by their translation to a C type, but it cannot generate other C code parts 
depending on these types. This is required for several of the Gencot basic functions.

To provide instances also for these functions, Gencot extends the Cogent antiquotation mechanism by a postprocessing
mechanism called ``Genops''. It replaces specific code templates in the monomorphic function instances. The Genops
templates are specified in the antiquoted C definition of the polymorphic function and contain an (antiquoted) 
Cogent type. When Cogent generates the monomorphic instance it replaces the type in the template by the 
corresponding C type. Afterwards, Genops reads the template, and expands it depending on the contained C type. 
If the C type results from a Cogent type for which Gencot does not provide a function instance, Genops signals
an error. The required information about the generated C types is accessed by reading the struct and function type
information described in Section~\ref{impl-ctypinfo-repr}.

The Genops template syntax is defined in a way that its insertion results in syntactically valid C code. Thus, 
Cogent can read and process the antiquoted C code with embedded Genops templates as usual and the resulting 
monomorphic instances are syntactically valid (plain) C code. Genops is implemented by reading the monomorphic 
instances, detecting and expanding the templates, and outputting the result again as valid C code.

Genops templates have the form of a C function invocation
\begin{verbatim}
  genopsTemplate("XXX",(typ1) expr1, "genopsEnd")
\end{verbatim}
where \code{XXX} names the template and \code{typ1} is an arbitrary C type embedded in the template, and \code{expr1}
is an arbitrary C expression embedded in the template. Both \code{typ1} and \code{expr1} are read by Genops and are 
used to determine the template expansion. If more than one type and/or expression is required, additional function
arguments are used to specify them, each but the last followed by the string \code{"genopsNext"}. If only a type is 
required, \code{0} is used as expression. If only an expression is required, the cast with the type is omitted. The Genops 
templates always expand to a C expression, thus they are syntactically valid wherever the template may occur.

The syntax of the templates is selected for simple parsing, so that it is not required to use a parser for C types and 
expressions. It is assumed that the \code{typi} and \code{expri} in the template do not contain the strings \code{"genopsNext"}
and \code{"genopsEnd"}, so they can be separated by searching for these strings. All templates can be found by searching for 
\code{genopsTemplate}. The \code{typi} and \code{expri} can be separated because the \code{typi} generated from Cogent 
types never contain parantheses (see Section~\ref{impl-ctypinfo}).

Using a C parser is avoided because the result of processing an antiquoted C file by Cogent is usually only a fragment of
a C program which contains types but not their definitions. The C syntax is not context free and a C parser needs to 
know which identifiers name types and which name values, therefore the files could not be parsed without additional information
from the rest of the program.

\subsection{Implementing Default Values}
\label{impl-operations-default}

Default values are specified using the abstract polymorphic function \code{defaultVal} (see Section~\ref{design-operations-default}).
This function is implemented with the help of the Genops template
\begin{verbatim}
  genopsTemplate("DefaultVal",(typ) 0, "genopsEnd")
\end{verbatim}
where \code{typ} is the type for which to generate the default value. The Genops mechanism replaces this template by the value
as specified in Section~\ref{design-operations-default}.

The Cogent type is determined by its translation as C type \code{typ}. The primitive numerical types are translated to \code{u8},
\code{u16}, \code{u32}, \code{u64}, the unit type is translated to \code{unit\_t}, the type \code{String} is translated to \code{char*}.
Gencot function pointer types are represented by abstract types with names of the form \code{CFunPtr\_...} and have the same name when
translated to C.
Tuple types, unboxed record types, and variant types, are translated to types \code{tNN} which are described in the struct type 
information (see Section~\ref{impl-ctypinfo-repr}). Unboxed Gencot array types are implemented as specific unboxed record types and thus
have the same form. Tuple and record types are treated in the same way. Variant types are recognized by the first member having 
type \code{tag\_t}. Gencot array types are recognized as double wrapper records with a member of array type.

Default values for aggregate types (structs and arrays, including the unit type which is translated to a struct type \code{unit\_t})
are implemented with the help of C initializers. For these types the 
\code{"DefaultVal"} template expands to an initializer expression, it may thus only be used in an object definition. The initializer 
only specifies a value for the first member or element, according to the C standard this causes the remaining members or elements to
be initialized according to the C default initialization, which corresponds to the Gencot default values as specified in 
Section~\ref{design-operations-default}. This also applies to objects of automatic storage duration, which are not initialized when
no initializer is provided. So the template can be used to initialize a local stack-allocated variable in a function to the default 
value and use it to copy the value elsewhere.

\subsection{Implementing Create and Dispose Functions}
\label{impl-operations-create}

\subsection{Implementing Initialize and Clear Functions}
\label{impl-operations-init}

The functions \code{initFull} and \code{clearFull} are implemented in antiquoted C by moving the boxed value from stack
to heap and back in a single C assignment.

The functions \code{initHeap}, \code{clearHeap}, \code{initSimp}, and \code{clearSimp} must be implemented by treating
each component of a struct type and each element of an array type separately. This is done with the help of type specific
functions for initializing and clearing, which are generated automatically. They are invoked by two Genops templates:
\begin{verbatim}
  genopsTemplate("InitStruct",(typ) expr, "genopsEnd")
  genopsTemplate("ClearStruct",(typ) expr, "genopsEnd")
\end{verbatim}
In the first case \code{typ} must be the translation of a Gencot empty-value type, in the second case it must be the translation 
of a Gencot valid-value type. In both cases \code{expr} must be an expression for a value of type \code{typ}. The templates
expand to an invocation of a generated function which initializes or clears this value as described in 
Section~\ref{design-operations-init}.

For every type \code{t} occurring as \code{typ} in an \code{"InitStruct"} template, and also for every pointer type \code{t} which 
transitively occurs in a component or element of \code{typ}, Genops generates a C function \code{gencotInitStruct\_t} which implements
the initialization. Analogously Genops generates C functions \code{gencotClearStruct\_t} for the \code{"ClearStruct"} templates.

The generated function definitions must be placed at toplevel before the first occurrence of the corresponding template. Since
Genops does not parse the C file it uses a marker, which must be manually inserted, to determine that position. The marker has 
the form
\begin{verbatim}
  int genopsInitClearDefinitions;
\end{verbatim}
and is replaced by the sequence of all generated function definitions. The function definitions are ordered so that functions
for components or elements of a type precede the function for the type and can be invoked there.

\subsection{Implementing Pointer Types}
\label{impl-operations-pointer}

\subsection{Implementing Function Pointer Types}
\label{impl-operations-function}

\subsection{Implementing MayNull}
\label{impl-operations-null}

The functions of the abstract data type defined in Section~\ref{design-operations-null} are implemented in antiquoted
C, no Genops templates are required. 

The function \code{isNull} and the part access operation functions could be implemented in Cogent, but they are also 
defined as abstract and implemented in antiquoted C, for efficiency reasons. The implementation of \code{modifyNullDflt}
uses the Genops template \code{"DefaultVal"} (see Section~\ref{impl-operations-default}) to generate the default value
for type \code{out}.

\subsection{Implementing Record Types}
\label{impl-operations-record}

\subsection{Implementing Array Types}
\label{impl-operations-array}

\subsubsection{Creating and Disposing Arrays}

The \code{create} instance is implemented by simply allocating the
required space on the heap, internally using the translation of type \code{\#(CArr<size> El)} to specify
the amount of space. 

\subsubsection{Initializing and Accessing Arrays}

All abstract functions for initializing and accessing arrays defined in Section~\ref{design-operations-array} are implemented
in antiquoted C using the following two Genops templates:
\begin{verbatim}
  genopsTemplate("ArrayPointer",(typ) expr,"genopsEnd")
  genopsTemplate("ArraySize",(typ) expr,"genopsEnd")
\end{verbatim}
In both templates the \code{typ} must be the translation of a Gencot array type \code{CArr<size> El}. In the first template \code{expr}
must be an expression of type \code{typ}, which denotes a wrapper record according to Section~\ref{design-types-array}.
The template is expanded to an expression for the corresponding C array (pointer to the first element), which is constructed by 
accessing the array in the wrapper record. In the second template \code{expr} is arbitrary and is ignored by Genops, by convention it 
should also be an expression of type \code{typ} as in the first case. The template is expanded to the size of all arrays of type 
\code{typ}, specified as a numerical literal.

\subsection{Implementing Explicitly Sized Array Types}
\label{impl-operations-esarray}



\section{Generating Isabelle Code}
\label{impl-isabelle}
As described in Section~\ref{design-isabelle} Gencot extends the shallow embedding and 
the refinement proofs generated by Cogent. This is implemented by
\begin{itemize}
\item providing predefined Isabelle code as theory files. These files are located 
in the Gencot distribution in the subfolder \code{isa}.
\item generating additional Isabelle code in separate theory files, if it depends 
on the translated C program.
\item modifying Isabelle code which has been generated by Cogent. This is only done
in cases where there is no other solution possible.
\end{itemize}

\subsection{Generating and Processing Isabelle Code}
\label{impl-isabelle-code}

\subsection{Extending the Shallow Embedding}
\label{impl-isabelle-shallow}

\subsubsection{Theory File Structure}

The shallow embedding generated by Cogent for a program \code{X} consists of two Isabelle theory files. The file 
\code{X\_ShallowShared.thy} contains definitions for all types occurring in the Cogent program and declarations
for all abstract functions in the Cogent program. The file \code{X\_Shallow\_Desugar.thy} contains definitions 
for all non-abstract functions defined in the Cogent program.

This shallow embedding is transformed by Gencot to the following set of files:
\begin{description}
\item[\code{GencotTypes.thy}:] this file is predefined and contains definitions used for implementing the Gencot 
types in Isabelle.
\item[\code{X\_ShallowShared.thy}:] this is a modified form of the file generated by Cogent, where some type 
declarations and type synonym definitions have been replaced. The file additionally imports \code{GencotTypes.thy}
since its content may be used in the replaced type specifications.
\item[\code{X\_Shallow\_Desugar.thy}:] this is the original file as generated by Cogent.
\item[\code{ShallowShared.thy}:] this is a program specific file generated by Gencot. It only contains an import of 
\code{X\_ShallowShared.thy}. It serves as an interface so that the content of \code{X\_ShallowShared.thy} can be
imported in predefined files without the need to know the program specific prefix \code{X}.
\item[\code{Gencot\_TTT.thy}:] these are predefined files for every Gencot include file \code{TTT.cogent} which defines
abstract Cogent functions. They import \code{GencotTypes.thy} and \code{ShallowShared.thy} and define Isabelle
specifications for the abstract Cogent functions defined in \code{TTT.cogent}.
\item[\code{CogentCommon\_ttt.thy}:] these are predefined files for some Cogent include files \code{ttt.cogent} in the 
Cogent standard library \code{gum/common} which define abstract Cogent functions. They import \code{ShallowShared.thy}
and define Isabelle specifications for the abstract Cogent functions defined in \code{ttt.cogent}
\item[\code{X\_Shallow\_Embedding.thy}:] this is a program specific file generated by Gencot. It imports 
\code{X\_Shallow\_Desugar.thy} and all \code{Gencot\_TTT.thy} and \code{CogentCommon\_ttt.thy} for which \code{TTT.cogent}
or \code{ttt.cogent} is used in the Cogent program, respectively. It may contain additional program specific parts
of the shallow embedding. 
\end{description}

Together, the Gencot shallow embedding for a program \code{X} is made available in an Isabelle theory file by importing
\code{X\_Shallow\_Embedding.thy}. This file structure is designed with the goal to reduce modifications and program 
specific parts, which must be generated, to an absolute minimum.

The predefined files \code{GencotTypes.thy}, \code{Gencot\_TTT.thy} and \code{CogentCommon\_ttt.thy} are provided in 
directory \code{isa} in the Gencot distribution. However, they cannot be loaded directly from there by Isabelle, 
since they depend on the program specific file \code{ShallowShared.thy} and no complete session can be defined for 
them in the Gencot distribution. For a complete shallow embedding which can be loaded by Isabelle the predefined
files must be copied from the Gencot distribution to the directory where the other files have been generated. Then 
a session can be defined for them in a \code{ROOT} file there.

It would be possible to determine from the imports in \code{X\_Shallow\_Embedding.thy} exactly those predefined
files which are needed for a specific shallow embedding. However, transitive imports among the predefined files
must be taken into account. Therefore a simpler solution is to always copy all predefined theory files to a 
specific shallow embedding, independent of which are actually needed.

\subsection{Generating Refinement Proofs}
\label{impl-isabelle-refine}



\section{C Processing Components}
\label{impl-ccomps}
As described in Section~\ref{design-modular} there are several different Gencot components which process C code and generate 
target code.

\subsection{Filters for C Code Processing}
\label{impl-ccomps-filters}

All filters which parse and process C code are implemented in Haskell and read the
C code as described in Section~\ref{impl-ccode-read}.

The following filters always process the content of a single C source file and produce the content for a single 
target file.
\begin{description}
\item[\code{gencot-translate}] translates a single file \code{x.c} or \code{x.h} to the Cogent code to be put in file
\code{x.cogent} or \code{x-incl.cogent}. It processes typedefs, struct/union/enum definitions, and function
definitions. 
\item[\code{gencot-entries}] translates a single file \code{x.c} to the antiquoted C entry wrappers to be put in
file \code{x-entry.ac}. It processes all function definitions with external linkage.
\item[\code{gencot-remfundef}] processes a single file \code{x.c} by removing all function definitions. The output
is intended to be put in file \code{x-globals.c}
\item[\code{gencot-deccomments}] processes a single file \code{x.c} or \code{x.h} to generate the list of
all declaration positions.
\item[\code{parmod-gen}] processes a single file \code{x.c} and generates the 
function parameter modification descriptions (see Section~\ref{impl-parmod}).
\item[\code{items-gen}] processes a single file \code{x.c} and generates a list of all items with internal linkage 
for which properties may be declared.
\end{description}

All these filters take the name of the original source file as additional first
argument, since they need it to generate Cogent names for C names with internal linkage and for tagless C struct/union
types.

There are other target files which are generated for the whole Cogent compilation unit. The filters for generating these target files 
take as input the list of file names to be processed (see Section~\ref{impl-ccode-package}). Filters of this kind 
are called ``processors'' in the following.

Usually only \code{.c} files need to be specified as input to processors.
In contrast to the single-file filters, the original file name is not required for the files input to a 
Gencot processor. A processor only processes items which are external to all input files, whereas the original
file name is needed for items which are defined in the input files. Therefore the list of input file names is
sufficient and the input files may have arbitrary names which need not be related to the names of the original 
\code{.c} files.

Gencot uses the following processors of this kind:
\begin{description}
\item[\code{gencot-exttypes}] generates the content to be put in the file \code{<package>-exttypes.cogent}. It 
processes externally referenced typedefs, tag definitions and enum constant definitions.
\item[\code{gencot-dvdtypes}] generates the content to be put in the file \code{<package>-dvdtypes.cogent}. It 
processes derived types.
\item[\code{gencot-dvdtypesah}] generates the content to be put in the file \code{<package>-dvdtypes.ah}. It 
processes derived types.
\item[\code{gencot-externs}] generates the abstract function definitions of the exit wrappers to be put in the file 
\code{<package>-externs.cogent}. It processes the declarations of externally referenced functions and variables.
\item[\code{gencot-externsac}] generates the exit wrappers to be put in the file \code{<package>-externs.ac}. It processes
the declarations of externally referenced functions.
\item[\code{parmod-externs}] generates the list of function identifiers of all externally referenced functions and
function pointer (array)s.
\item[\code{items-used}] determines a list of all declared external items which are actually used by the Cogent
compilation unit.
\item[\code{items-externs}] generates a list of all items with external or no linkage 
for which properties may be declared.
\end{description}

\subsection{Main Translation to Cogent}
\label{impl-ccomps-main}

The main translation from C to Cogent is implemented by the filter \code{gencot-translate}. It translates \code{DeclEvent}s
of the following kinds:
\begin{itemize}
\item struct/union definitions
\item enum definitions with a tag
\item enum constant definitions
\item function definitions
\item object definitions
\item type definitions
\end{itemize}
The remaining global items are removed by the predicate passed to \code{Gencot.Input.getDeclEvents}: all declarations, 
and all tagless enum definitions. No Cogent type name is generated for a tagless enum definition,
references to it are always directly replaced by type \code{U32}.

The translation does not use the callback handler to collect information during analysis. It only uses the semantics map
created by the analysis and processes the \code{DeclEvent} sequence created by preprocessing as described in 
Section~\ref{impl-ccode-read}.

\subsubsection{Translating Toplevel Definitions}

The translation of the \code{DeclEvent} sequence is implemented by the monadic action
\begin{verbatim}
  transGlobals :: [DeclEvent] -> FTrav [GenToplv]
\end{verbatim}
in module \code{Gencot.Cogent.Translate}.
It performs the monadic traversal as described in Section~\ref{impl-ccode-trav} and returns the list of toplevel
Cogent definitions of type \code{GenToplv}. It is implemented by mapping the function \code{transGlobal} to the 
\code{DeclEvent} sequence.

A struct/union/enum definition corresponds to a full specifier, as described in Section~\ref{design-decls-tags}.
The language-c analysis already implements moving all nested full specifiers to separate global definitions and the
sorting step done by \code{Gencot.Input.getDeclEvents} creates the desired ordering. Therefore, basically the 
\code{DeclEvent}s in the list could be processed by \code{transGlobal} independently from each other.

However, this approach would create inappropriate origin markers which could result in duplicated conditional preprocessor
directives and comments. If a struct definition in lines $b1 < e1$ contains a nested struct definition in lines $b2 < e2$
independent translation would wrap the translated nested struct always in origin markers for $b2$ and $e2$. If $e1 = e2$
the corresponding origin marker would occur twice. A better approach is to use origin markers for $b1, e1$ to wrap the 
whole sequence consisting of the translated struct together with the translations of all nested structs. Then, the translated
nested struct would be followed by the two origin markers for $e2$ and $e1$ which can be reduced to a single marker if
$e2 = e1$.

Therefore function \code{transGlobal} is defined as
\begin{verbatim}
  transGlobal :: DeclEvent -> FTrav [GenToplv]
\end{verbatim}
It translates every struct definition together with all nested struct/union/enum definitions and returns the sequence of 
the corresponding Cogent toplevel definitions, wrapped by the origin markers as described.

Since the nested definitions still occur seperately in the list of \code{DeclEvent}s processed by \code{transGlobals}, their processing
must be suppressed. This is implemented by sorting the list according to the position of the first character in the source file
(which will put all nested definitions after their surrounding definitions), marking processed nested definitions in the 
monadic user state, and testing every struct/union/enum definition found by \code{transGlobal} in the \code{DeclEvent} list 
whether it is marked as already processed.

A struct/union definition is translated to a Cogent type definition where the type name is constructed as described 
in Section~\ref{design-names}. A struct is translated to a corresponding record type, a union is translated to an 
abstract type, as described in Section~\ref{design-types}.
In both cases the type name names the boxed type, i.e., it corresponds to the C type of a pointer to the struct/union.

An enum definition with a tag is translated to a Cogent type definition where the type name is constructed as described 
in Section~\ref{design-names}. The name is always defined for type \code{U32}, as described in Section~\ref{design-types}.

An enum constant definition is translated to a Cogent constant definition where the name is constructed as described 
in Section~\ref{design-names} and the type is always \code{U32}, as described in Section~\ref{design-types}.

A function definition is translated to a Cogent function definition, as described in Section~\ref{design-fundefs}.

A type definition is translated to a Cogent type definition as described in Section~\ref{design-decls-typedefs}.

\subsubsection{Translating Type References}

A type reference is translated by the function
\begin{verbatim}
  transType :: Bool -> ItemAssocType -> FTrav GenType
\end{verbatim}
It must be paired with the associated item identifier as an \code{ItemAssocType} as described in 
Section~\ref{impl-itemprops-types}. The identifier is used to retrieve the declared item properties required for
the translation.

The first parameter specifies whether typedef names should be resolved when translating the type. 
The reason why typedef names are resolved as part of the type translation function is that item properties may be 
associated with typedef names or their sub-items. If typedef names would
be resolved in a separate function the item identifier could not be associated anymore
afterwards. This would cause the declared item properties for typedef names to be ignored when translating the 
resolved type.

A type reference may be a direct type, a derived type, or a typedef name. For every typedef name a Cogent type
name is defined, as described in Sections~\ref{design-names} and~\ref{design-types-typedef}. An exception are 
externally defined typedef names not used directly in the Cogent compilation unit, they are resolved. A direct type is either
the type \code{void}, a primitive C type, which is mapped to the name of a primitive Cogent type, or it is a 
struct/union/enum type reference for which Gencot also introduces a Cogent type name or maps it to the 
primitive Cogent type \code{U32} (tagless enums). Hence, both direct types and typedef names can always be mapped
to Cogent type names, with the exception of type \code{void}, which is mapped to the Cogent unit type \code{()}.
For struct and union types the unbox operator must be applied to the Cogent type name.
If a typedef name references (directly or indirectly) a struct or union type, the corresponding Cogent
type name references the boxed type. Therefore, it must also be modified by applying the unbox operator. 

Primitive types or typedef names referencing a primitive type are semantically unboxed. However, if marked as unboxed, 
the Cogent prettyprint function will add an explicit unbox operator. For better readability Gencot specifies these types
as boxed, to suppress the redundant unbox operator.

A derived type is either a pointer type, an array type, or a function type. It is derived from a base type
which in case of a function type is the type of the function result. The base type may again be a derived
type, ultimately it is a direct type or a typedef name.

For a pointer type the translation depends on the base type. If it is a struct, union, or array type or a typedef
name referencing such a type, the pointer type is translated to the translation of the base type in its
boxed form. If it is a function type (possibly after resolving type names), the function type
is encoded as described in Section~\ref{design-types-pointer} using the function \code{encodeType}, then the function
pointer type is constructed from the encoding as described in Section~\ref{design-types-pointer}. The monadic operation
\begin{verbatim}
  encodeType :: Bool -> ItemAssocType -> FTrav String
\end{verbatim}
works similar to \code{transType} but returns the encoding as a string instead of the translated 
Cogent type. In particular, it optionally resolves typedef names and respects item properties.
If the base type is \code{void} the special type name \code{CVoidPtr} is used.

In all other cases, as described in Section~\ref{design-types}, 
the pointer type is translated to the parametric type \code{CPtr} with the translated base type as type argument.

For an array type, the translation is the parameterized type for an array according to the number of elements. 
The translated base type is used as type argument. In case that the base type is again an array type (multidimensional
array), an explicit unbox operator is applied to the type argument to differentiate the multidimensional array 
from an array of array pointers, as described in Section~\ref{design-types-array}.

A function type is always translated to the corresponding Cogent function type, where a tuple type is
used as parameter type if there is more than one parameter, and the unit type is used if there is
no parameter. Declared Read-Only and Add-Result properties are used to determine whether a parameter type
should be translated as readonly, or whether it should be returned as part of a result tuple. 

The translation functions do not respect the \code{const} qualifiers in C type specifications, they only use the 
Read-Only property to determine whether a type should be translated as readonly. The \code{const} qualifiers are 
used by \code{items-gen} to generate the default properties as described in Section~\ref{impl-ccomps-itemprops}. 

\subsubsection{Filter \code{gencot-translate}}

The filter \code{gencot-translate} is invoked as
\begin{verbatim}
  gencot-translate <source name> <item property file>
\end{verbatim}
where the arguments are the name of the original C source file and a file with the item property declarations (see 
Section~\ref{impl-itemprops}).

\subsection{Used External Items}
\label{impl-ccomps-items}

External items are those C items (functions, objects, types) which are known in the sources of the Cogent compilation
unit but not defined there. 

Functions and objects are external, if they are declared in the 
Cogent compilation unit but are not defined there. They may be defined in files of the package which do not belong to
the Cogent compilation unit (then they are declared in \code{.h} files of the package), or they may be defined in 
standard runtime libraries (then they are declared in system include files). In both cases they must have external linkage.

Gencot always assumes that all package include files used by the Cogent compilation unit are translated to Cogent and
are thus a part of the Cogent compilation unit. Thus a type can only be external if it is defined in a system include
file.

If an external item is actually used (by invoking a function, accessing an object, or referencing a type) in the Cogent
compilation unit, it must also be translated to Cogent so that it is known in the Cogent program. Functions are translated
in the form of exit wrappers, objects are translated in the form of abstract access functions, and types are translated
as usual.

Usually, C programs know many external items without actually using them. To keep the resulting Cogent code as small 
as possible, Gencot performs an automatic analysis to determine the external toplevel items which are actually used
by the Cogent compilation unit. The result of this analysis can be manually overridden by explicitly specifying additional
external items which also should be processed.

\subsubsection{The processor \code{items-used}}

The processor \code{items-used} generates a list of the external toplevel items used by the Cogent compilation unit.
It is intended as a filtering list for the processors \code{gencot-externs}, \code{gencot-exttypes}, \code{gencot-dvdtypes},
\code{items-externs} and \code{parmod-gen} which process external items, to reduce them to the relevant ones.

The processor is invoked as
\begin{verbatim}
  items-used <additional items file>
\end{verbatim}
where the argument is a list of additional external toplevel items specified manually.

A toplevel item is a struct, union, or enum, a typedef name, a function or a global variable. 

The processor writes a textual list of the used external items to its standard output. Each item is output in a separate line 
identified by its item identifer as described in Section~\ref{impl-itemprops-ids}. The argument file content must have the 
same form, it is added to the output. These additional items cannot be added afterwards, since they may use other items
which can only be detected by \code{items-used} if the additional items are known to it.

As conveniency for the developer, the argument file may contain empty lines and may contain comment lines which
start with a hash sign \code{\#}.

\subsubsection{Determining Used Functions}

A function is used by the Cogent compilation unit if it is invoked by a function which is part of the Cogent compilation unit. 
Gencot determines the invoked external functions with the help of the call graph as described in 
Section~\ref{impl-ccode-callgraph}. 

The processor parses and analyses
all C source files specified in the input file name list, resulting in the list of corresponding symbol tables, as
described in Section~\ref{impl-ccode-package}. Then
it determines for every table the invoked functions using the call graph. This must
be done before combining the tables, since invocations may also occur in the bodies of functions with internal
linkage. These functions are removed during table combination.

After determining the invoked functions, the tables are combined as described in Section~\ref{impl-ccode-package}
and the invocations are reduced to those for
which a declarations is present in the combined table. These are the invocations of functions (including function pointers)
external to all read C sources. 

Although \code{gencot-externs} only processes complete function declarations, 
\code{items-used} also lists identifiers for incompletely declared functions and for variadic functions,
if they are invoked in the Cogent compilation unit. This is intended as an 
information for the developer who has to translate invocations of such functions manually. Additionally, these functions
may be needed by \code{gencot-exttypes} and \code{gencot-dvdtypes} to process their result types.

An identifier seen by Gencot in a function invocation may also be a macro call which shall be translated by Gencot. 
In this case a dummy function declaration must be manually provided for it as part of the Gencot macro call conversion
(see Section~\ref{design-preprocessor-macros}), so that the language-c parser can parse the invocations. 
Therefore \code{items-used} will detect the macro as a used external item.

There are cases where an external function is used without being seen by the call graph. This is the case when
the function is assigned to a function pointer or when the function invocation is created by a macro call and
is thus not visible for the call graph. For these cases the function item id must be provided manually in the argument file.

\subsubsection{Determining Used Variables}

Gencot determines usage of an external variable if it or a sub-item of it is invoked as a function, using the 
call graph. Invocations of sub-items which are the result of another function are currently not detected. 
Variables accessed in other ways (e.g., read access, assignment) must be added manually in the argument file. This
provides more control about the processed external variables for the developer.

\subsubsection{Collecting Used Types}

As described in Section~\ref{design-types-itemprops}, type items are all typedef names, tag names, generated names 
for tagless struct/unions, and some derived pointer types. Typedef names and tag names can be external, if they are 
declared in a system include file. Tagless struct/unions and derived types cannot be extern in this sense, because 
for them the declaration is identical with their single use. However, the following situation can lead to external 
used items of this kind: a typedef name is defined for them in a system include file. This typedef name is used 
indirectly by the Cogent compilation unit and is thus resolved. Then the result of resolving the name becomes the 
used external type item. 

Properties declared for a derived pointer type item are always applied to the individual items having that type. 
Thus it is always possible to declare the properties directly for the individual items and Gencot does not support 
external pointer type items for this reason. For a tagless struct/union the situation is different, it is translated
to a single defined type and declared properties must be used in this type definition, they cannot be replaced by
properties declared for the individual typed items. Therefore Gencot also supports tagless struct/unions as used
external items, identified by their item identifier using a generated name as described in Section~\ref{impl-itemprops-ids}.

A type item can be used by code in a translated C soure file or by a declaration of an external function or variable.
Therefore, \code{items-used} first determines the external invoked functions and variables as described above, including
the additional items in the argument file.

Type items are reduced by an analysis of type usage by reference. Every item can be
associated with the type items referenced in the types of all sub-items. This usage relation is transitive, because 
in a definition of a composite type or typedef name, other type items can be referenced. To determine the relevant 
type items defined in system include files, \code{items-used} starts with all initial items belonging
to the Cogent compilation unit, together with all external invoked functions and the processed external variables. 
It then determines all transitively used type items (which all must be defined in system include files).

To further reduce external items, external type name which are not directly used in the Cogent compilation unit are
resolved, i.e.~always replaced by their definition (which may again reference external type items).

For the resulting set of type items the item identifiers are added to the output list. 

The initial toplevel items are collected with the help of the callback handler during analysis, as described in 
Section~\ref{impl-ccode-read}. All toplevel items can be wrapped as a \code{DeclEvent}. The callback handler 
is automatically invoked for all toplevel items in the parsed
C source files, including local variables (which are not present anymore in the symbol table after the analysis phase).

Gencot uses the callback handler
\begin{verbatim}
  collectTypeCarriers :: DeclEvent -> Trav [DeclEvent] ()
\end{verbatim}
in module \code{Gencot.Util.Types}. It adds all \code{DeclEvent}s with relevant toplevel items to the list in the 
user state of the \code{Trav} monad. The handler is used for every \code{.c} file separately, the collected
items must be combined afterwards to get all items in the Cogent compilation unit. This is
done by first combining the symbol tables as described in Section~\ref{impl-ccode-package}, and then using the union of all 
items which are local, have internal linkage, or are still contained in the resulting combined table. 
This will avoid duplicate occurrences of items.

All parameter declarations are ignored by the callback handler, since for every parameter declaration there must be 
a \code{DeclEvent} for the containing function, which is processed by the handler and has the parameters as sub-items. 
Function and global variable declarations 
are also ignored, since they represent external functions and variables which are determined separately. Also,
a function or variable declared in one source file may be defined in another source file of the Cogent compilation unit and thus 
be not external. 

Type items defined in system include files are omitted, since they will be determined using the type usage relation. 
To determine whether
an item is defined in a system include file, the simple heuristics is used that the source file name is specified
as an absolute pathname in the \code{nodeinfo}. System includes are typically accessed using absolute pathnames of the 
form \code{/usr/include/...} (after being searched by cpp). If include paths are specified explicitly for Gencot, 
system include paths must be 
specified as absolute paths whereas include paths belonging to the <package> must be specified as relative paths
to make the heuristics work.

After the language-c analysis phase the declarations of all invoked external functions are determined 
as described above and added to the combined items determined by the callback handler. No duplicate items
will result here since the combined symbol table contains every external declaration only once.
External invoked composite type members are not included, because they are sub-items of the composite type. To invoke 
a composite type member an object must exist which is an item with the composite type, so the composite is always 
a used type item.

The resulting set of items is transitively completed by adding all type items used by referencing them,
as described above. This will add all required type items defined in system include files.
Duplicates could occur if a type item is referenced
by several different types, they must be detected and avoided. Since all toplevel items correspond to 
syntactic entities in a source file, type items can be identified and compared by their position and source file.

Finally, the type items are reduced to those defined in a system include file using the same
heuristics as above. Together with the item identifiers of the external invoked functions and variables their items
identifiers are written to the output.

\subsection{External Functions and Variables}
\label{impl-ccomps-externs}

Used external functions and variables (``objects'') are translated so that they are known and usable
in the Cogent program.

\subsubsection{Translating External Functions and Variables}

Normally, a used external function is translated as an exit wrapper.
For the exit wrapper functions Gencot generates the Cogent abstract function definitions and the implementations
in antiquoted C. 
The information required for each of these functions is contained in the C function declaration.

A C function which is only declared may be incomplete, i.e., its parameters (number and types) are not known. 
We could translate such function declarations to Cogent abstract functions with Unit as argument type, however,
that would not be related to the invocations and thus be useless. Therefore Gencot does not generate exit 
wrappers for incompletely declared functions.

As described in Section~\ref{impl-ccomps-items} a used external function seen by Gencot may also be a macro call 
which shall be translated to Cogent. In this case a dummy function declaration must have been manually provided 
for it. By using an incomplete declaration the generation of an exit wrapper can be suppressed for macros 
which shall be translated to a macro in Cogent.

If an external function is accessed through a function pointer no exit wrapper is generated for it.
Function pointers are invoked as described in Section~\ref{design-operations-function}. Since in C all functions
which are sub-items of other items must be function pointers, exit wrappers are only generated for toplevel
items, sub-items are not considered here.

For a used external object (including invoked function pointers) Gencot generates a Cogent abstract function 
definition for an access function without providing an implementation in antiquoted C. Again, this is only 
done for toplevel items, sub-items are not considered.

\subsubsection{Processor \code{gencot-externs}}

The processor \code{gencot-externs} generates the Cogent abstract function definitions. It is invoked as
\begin{verbatim}
  gencot-externs <item properties file> <used external items file>
\end{verbatim}
where the arguments are a file with the item property declarations (see 
Section~\ref{impl-itemprops}) and a file with the identifiers of the used external items as created 
by \code{items-used} (see Section~\ref{impl-ccomps-items}).

The processor parses and analyses
all C source files specified in the input file name list, resulting in the list of corresponding symbol tables, as
described in Section~\ref{impl-ccode-package}. The tables are combined and from the resulting table all 
declarations of toplevel items are retrieved which are listed in the file of used items.

The processor translates these declarations using the monadic action \code{transGlobals} of
module \code{Gencot.Cogent.Translate}. It translates C function and variable declarations to abstract function 
definitions in Cogent. It uses the item properties to fine tune the translation.
The result is output using \code{prettyTopLevels} from module \code{Gencot.Cogent.Output}.

\subsubsection{Processor \code{gencot-externsac}}

The processor \code{gencot-externsac} generates the wrapper implementations in antiquoted C code.

\subsection{External Cogent Types}
\label{impl-ccomps-exttypes}

The processor \code{gencot-exttypes} generates all Cogent type definitions which origin from a C system include file.
It is invoked as
\begin{verbatim}
  gencot-exttypes <item properties file> <used external items file>
\end{verbatim}
where the arguments are a file with the item property declarations (see 
Section~\ref{impl-itemprops}) and a file with the identifiers of the used external items as created 
by \code{items-used} (see Section~\ref{impl-ccomps-itemprops}).

The task of \code{gencot-exttypes} is to translate all type item definitions listed in the 
file of used external items. It parses and analyses
all C source files specified in the input file name list and combines the resulting symbol tables as
described in Section~\ref{impl-ccode-package}. The type item (struct/union, tagged enum, and typedef name) definitions 
in the resulting table listed in the argument file are then translated.

\subsubsection{Translating Type Definitions}

For all external enum types, composite types and typedef names in the resulting set \code{gencot-exttypes} translates the
definition to Cogent as described in Sections~\ref{design-types}.
All typedef names not used directly are resolved in these definitions, as described in Section~\ref{impl-ccomps-items}.
The same monadic action \code{transGlobals} is used as for \code{gencot-translate}.

\subsection{Derived Types}
\label{impl-ccomps-dvdtypes}

The processor \code{gencot-dvdtypes} generates all Cogent type definitions required for the derived types
used in the C program. It is invoked as
\begin{verbatim}
  gencot-dvdtypes <item properties file> <used external items file>
\end{verbatim}
where the arguments are a file with the item property declarations (see 
Section~\ref{impl-itemprops}) and a file with the identifiers of the used external items as created 
by \code{items-used} (see Section~\ref{impl-ccomps-itemprops}).

A C derived type, is a pointer type, array type or function type. As described in 
Section~\ref{design-types}, Gencot usually maps these types to parameterized Gencot types. The generic types \code{CPtr},
\code{CFunPtr} and \code{CFunInc} used for pointer and function types are predefined by Gencot. However, function
pointer types use abstract types for encoding the base function type and the generic 
types of the form \code{CArr<size>} used for array types syntactically contain the array size, both
depend on the translated C program. For these types Cogent definitions must be provided.

Derived types are used in C in the form of type expressions, i.e., base types, result and parameter types are 
syntactically included in the type specification. With some exceptions, derived types may occur wherever a type
may be used. Therefore, to determine all used derived types, Gencot determines all item associated types as
described in Section~\ref{impl-itemprops-types}. These include all types used 
in translated C sources, in declarations of external used functions and variables and those defined in system
include files.

Together, \code{gencot-dvdtypes} processes
\textit{all} item associated types for all sub-item types which are a derived array or function pointer type.

\subsubsection{Generating Type Definitions}

For all derived array types two abstract generic type definitions must be generated
(see Section~\ref{design-types-array}). However, only one such pair may be generated for every array \code{<size>}, 
although array types with a specific size may be associated with several different items and element types.

For all derived function pointer types an abstract type definition or a type synonym for the fully resolved type 
must be generated for encoding the function type used as its base type (see Section~\ref{design-types-pointer}).
For fully resolved function types an additional type synonym definition must be generated for the translated function type.

Gencot first generates the translated Cogent type names for all derived array and function pointer types associated 
with an item. Then it generates the corresponding definitions for every type name
used in all these derived types. If a translated array type uses the generic name \code{CArrXX} no definition
is generated since this type name is predefined by Gencot.

Generating the type definitions for derived array and function pointer types is implemented by the monadic action
\begin{verbatim}
  genTypeDefs :: [DeclEvent] -> FTrav [GenToplv]
\end{verbatim}
in module \code{Gencot.Cogent.Translate}, where the list of \code{DeclEvent}s corresponds to the items
to be processed.

\subsubsection{Processor \code{gencot-dvdtypesah}}

\subsection{Declarations}
\label{impl-ccomps-decls}

The filter \code{gencot-deccomments} is used to provide the information about the positions of all declarations with
external linkage in
a translated file, as described in Section~\ref{impl-comments-decl}. This information is used to move declaration
comments to the corresponding definitions, as described in Section~\ref{design-comments-decl}.

Only global declarations are processed by this filter. Therefore the filter processes the list of \code{DeclEvents}
returned by \code{getDeclEvents} (see Section~\ref{impl-ccode-read}). The predicate for filtering the list
selects all declarations with external linkage (i.e. removes all object/function/enumeration/type definitions and 
all declarations with internal or no linkage). 
The main application case are functions declared as external, but moving the comments may also be useful for 
objects.

If a function or object with external linkage is declared and defined in the same file, the language-c analysis step
replaces the declaration by the corresponding definitions in the semantic map, hence it will not be found and processed
by \code{gencot-deccomments}. However, it is assumed that in this case the documentation is assocatied with the definition
and need not be moved from the declaration to the definition.

For processing the declarations we do not need a monad since every declaration can be processed on its own. This is implemented
by the function
\begin{verbatim}
  transDecl :: DeclEvent -> String
\end{verbatim}
in module \code{Gencot.Text.Decls}. It is mapped over the list of \code{DeclEvents} and the resulting string list
is output one string per line.

\subsection{Item Properties}
\label{impl-ccomps-itemprops}

To support an easy manual declaration of item properties Gencot generates default declarations from the C sources for all 
items with an identifier. The declarations are either empty or declare the properties Read-Only and/or Add-Result, if they
can be derived from the item's type. 

\subsubsection{The filter \code{items-gen}}

The filter \code{items-gen} is used to generate default item property declarations from a single C source file (\code{.h} or \code{.c}).
It is invoked as:
\begin{verbatim}
  items-gen <source file name>
\end{verbatim}

The filter processes as individual items all \code{DeclEvent}s in the source file which are function
definitions or definitions for objects. The filter processes every definition of a composite type 
(with or without a tag) and every type name definition in the source file as collective item. (The filter does not
process any derived pointer types as collective items.) The processing is done by the monadic action
\begin{verbatim}
  transGlobals :: [DeclEvent] -> FTrav ItemProperties
\end{verbatim}
defined in module \code{Gencot.Items.Translate}.

For every item, if its declared type (in the case of a typedef name item its defining type) is
a derived function, array, or pointer type, its sub-items according to the 
type are processed recursively. For a composite type item its members are processed recursively as sub-items.

An item is processed as follows. If its declared or defined type is primitive or an enum or a function pointer no output 
is generated since properties are not 
supported or always ignored for it. Otherwise, if it is a parameter item or its declared type is not composite, a property
declaration is generated for it (parameters of composite type may be affected by the Add-Result property, items of function
type may be affected by the Heap-Use property, items of pointer or array type may be affected by the Read-Only property).
For non-parameter items of composite type no output is generated since properties are not 
supported or always ignored for them.

If the non-function-pointer type of an item is intrinsically readonly, the Read-Only property is declared for it. If an 
item has an intrinsically readonly array type and is not an element of another array or a member of a composite type (in both
cases it is translated with an unboxed type and the Read-Only property would be ignored), the Read-Only property
is declared for it. Otherwise, if it is a parameter and its type is linear (is a pointer or array or contains any),
the Add-Result property is declared for it. Otherwise the declaration is empty.

All generated declarations are written in textual form to the standard output.

Whether a type is intrinsically readonly depends on its \code{const} qualifiers. For a direct type the \code{const}
qualifier is ignored, since in Cogent values of unboxed and regular types are always immutable. For
a function type the qualifier is also ignored since function types are regular in Cogent. All other array types 
and all pointer types are translated to linear types which can be mutable in
Cogent. To be intrinsically immutable, the base types of these types and all transitively contained array and pointer 
types must be const qualified. The immutability is determined by function\code{isReadOnlyType}. It is implemented
by a monadic action since it must access the symbol table to resolve references to tagged C structs.

Default Add-Result properties are generated whenever the parameter type is linear. In cases where the parameter 
shall be discarded because it is deallocated in the function or it is already returned as the function's result
the default property must be removed manually or with the help of the parameter modification descriptions before
using the properties for translating the function.

\subsubsection{The processor \code{items-externs}}

The processor \code{items-externs} is used to generate the default property declarations for items intended
to be used by \code{gencot-externs} (see Section~\ref{impl-ccomps-externs}) and \code{gencot-exttypes} 
(see Section~\ref{impl-ccomps-exttypes}). It is invoked as
\begin{verbatim}
  items-externs <used external items file>
\end{verbatim}
where the argument is the list of external toplevel items used by the Cogent compilation unit, as it is created 
by \code{items-used}.

The processor reads the source files of the Cogent compilation unit as described in Section~\ref{impl-ccode-package}
and processes all external toplevel items specified in the list of used items in the same way as \code{items-gen}
to generate default property declarations for them on standard output. The same monadic action \code{transGlobals}
is used as by \code{items-gen}.

The type of a processed item may involve an externally defined type name which is not in the list of used external items.
This means that the type name is not used directly in the Cogent code and is resolved in the translation. In this case it is 
also resolved by \code{items-externs} so that collective sub-items of the type name become individual sub-items of 
the processed item, as described in Section~\ref{design-types-itemprops}.

\subsubsection{The processor \code{items-extfuns}}

The processor \code{items-extfuns} is used to generate the list of identifiers of all used external items which 
have a derived function type as declared type. It is invoked as
\begin{verbatim}
  items-extfuns <used external items file>
\end{verbatim}
where the argument is the list of external toplevel items used by the Cogent compilation unit, as it is created 
by \code{items-used}.

The processor reads the source files of the Cogent compilation unit as described in Section~\ref{impl-ccode-package}
and processes all external toplevel items specified in the list of used items in the same way as \code{items-externs}.
Instead of generating default property declarations for them, it lists the identifiers of all function items on 
standard output. This list is intended for selecting parameter modification descriptions for these functions. It
is produced by the monadic action
\begin{verbatim}
  functionsInGlobals :: [DeclEvent] -> FTrav [String]
\end{verbatim}
defined in module \code{Gencot.Items.Translate}.

It is not possible to directly use the list of identifiers of the used external toplevel items for this purpose, since 
functions can be sub-items of toplevel items and the resolved external typedef names have to be taken into account.

The result must include all functions processed by \code{gencot-externs} and all function types and composite type members
processed by \code{gencot-exttypes}. 

Although \code{gencot-externs} only processes complete function declarations of non-variadic external functions, 
\code{items-extfuns} also lists identifiers for incompletely declared functions and for variadic functions.
This is intended as an information for the developer who has to translate invocations of such functions manually.

For declared functions and for all kinds of function pointers the function type can be specified in C using a typedef name.
Such function items are not listed, instead, the typedef name is listed as the collective item for which the parameter
modification description is required. If, however, the external typedef name is not used diectly by the Cogent compilation
unit, it is resolved, and the item for which the typedef name has been declared is listed, as described in 
Section~\ref{design-parmod-entities}.

Manually specified additional external functions are also listed, since they have been added by \code{items-used}
to the \code{<used external items file>}.

\subsection{Parameter Modification}
\label{impl-ccomps-parmod}

The goal of working with the parameter modification descriptions (see Section~\ref{design-parmod}) is to analyse 
transitive function invocation chains which only stop at functions external to the C <package>, for which no sources
are available. Therefore the approach used for item property declarations in Section~\ref{impl-ccomps-itemprops}
cannot be used here, since it treats all items as external which do not belong to the Cogent compilation unit.

Instead, we use a single filter \code{parmod-gen} which is always applied to a single C source file in the 
<package>. It may be run in two modes: in normal mode it processes all functions \textit{defined} in the source
file and outputs descriptions for them, in ``closing mode'' it processes all functions only \textit{declared} in 
the source file or in a file included by it and outputs a description template. This covers all functions which 
may be invoked by the functions defined in the source, including functions external to the <package>. 

The invocation chains can then be followed using the descriptions generated in normal mode, until a function is invoked
which is external to the <package>. Only for these functions the templates generated in closing mode are used
to ``close'' the invocation chains, all other templates (in particular, for functions defined in the <package>)
are ignored.

The filter still needs the information about the used items outside the Cogent compilation unit to correctly handle
typedef names which are used only indirectly and are resolved during the translation to Cogent. If the type of an
invoked function is specified by a type name, the invocation either refers to the type name or to individual function items,
depending on whether the type name is resolved, as described in Section~\ref{design-parmod}.

\subsubsection{\code{parmod-gen}}

The filter \code{parmod-gen} is used to generate the Json parameter modification description from a C source file.
It may be invoked in two forms:
\begin{verbatim}
  parmod-gen <source file name> <used external items file>
  parmod-gen <source file name> <used external items file> close
\end{verbatim}
The second form runs the filter in ``closing mode''.

In normal mode the filter processes all \code{DeclEvent}s in the source file which are function
definitions or definitions for objects with a type syntactically including a function type (called a ``SIFT'' 
type in the following). Every such definition is translated to an
entry in the parameter modification description. A function definition is translated to the sequence of its
own description entry and the entries for all invoked local items with a SIFT type and all parameters with 
a SIFT type.

The filter processes every definition of a composite type in the source file and translates all
its members with a SIFT type to the corresponding
description template. The filter processes every type name definition in the source file where the defining type
is a SIFT type, and translates it to the corresponding description template.

The descriptions of functions, their function parameters, composite type members, and typedef names are 
intended to be converted to item property declarations for being used by \code{gencot-translate}
when it translates the function and type definitions. The descriptions of invoked local items 
are intended for evaluating dependencies as described in Section~\ref{impl-parmod}.

In closing mode the filter processes all \code{DeclEvent}s in the source file and in all included files which
are declarations for functions or items with a SIFT type. Every declaration is translated to an entry in the parameter
modification template. The filter also processes all composite type definitions and type name definitions which
are specified in system include files in the same way as described above. 
These templates are only intended for ``closing'' the dependencies for the evaluation.

Here, resolved external type names are taken into account, when the type of a declaration is tested for 
being a SIFT type: all such type names are resolved before testing whether a derived function type directly
occurs in the type.

After analysing the C code as described in Section~\ref{impl-ccode} the call graph is generated as described
in Section~\ref{impl-ccode-callgraph}. Then another traversal is performed using the \code{CTrav} monad and
\code{runWithCallGraph} with action
\begin{verbatim}
  transGlobals :: [DeclEvent] -> CTrav Parmods
\end{verbatim}
defined in module \code{Gencot.Json.Translate}. The resulting list of JSON objects is output as described 
in Section~\ref{impl-parmod-json}.



\section{Other Components}
\label{impl-ocomps}
The following auxiliary Gencot components exist which do not process C source code:
\begin{description}
\item[\code{parmod-proc}] processes parameter modification descriptions in JSON format (see Section~\ref{impl-parmod}).
\item[\code{items-proc}] processes item property declarations in text format (see Section~\ref{impl-itemprops}).
\item[\code{auxcog-macros}] processes Cogent source code to select macro definitions.
\item[\code{auxcog-mapback}] processes Cogent source code to map Cogent constants back to preprocessor constants.
\item[\code{gencot-prclist}] processes list files to remove comments.
\end{description}

\subsection{Parameter Modification Descriptions}
\label{impl-ocomps-parmod}

Processing parameter modification descriptions is implemented by the filter \code{parmod-proc}. It reads a parameter
modification description from standard input as described in Section~\ref{impl-parmod-json}. The first command line 
argument acts as a command how to process the parameter modification description. The filter implements the following
commands with the help of the functions from module \code{Gencot.Json.Process}  (see Section~\ref{impl-parmod-modules}):
\begin{description}

\item[\code{check}]
Verify the structure of the parameter modification description 
according to Section~\ref{impl-parmod-struct} and lists all errors found (not yet implemented).

\item[\code{unconfirmed}] 
List all unconfirmed parameter descriptions using function \code{showRemainingPars}.

\item[\code{required}]
List all required invocations by their function identifiers, using function \code{showRequired}.

\item[\code{funids}]
List the function identifiers of all functions described in the parameter modification description.

\item[\code{sort}]
Takes an additional file name as command line argument. The file contains a list of function identifiers.
The input is sorted using function \code{sortParmods}. This command
is intended to be applied after the \code{merge} command to (re)establish a certain ordering.

\item[\code{filter}]
Takes an additional file name as command line argument. The file contains a list of function identifiers.
The input is filtered using function \code{filterParmods}.

\item[\code{merge}]
Takes an additional file name as command line argument. The file contains a parameter modification description 
in JSON format. Both descriptions are merged using function \code{mergeParmods}
where the first parameter is the description read from standard input.
After merging, function \code{addParsFromInvokes} is applied.
Thus, if the merged information contains
an invocation with more arguments than before, the function description is automatically extended.

\item[\code{eval}]
Using the functions \code{showRemainingPars} and \code{showRequired} it is verified that the parameter modification 
description contains no
unconfirmed parameter descriptions and no required dependencies. Then it is evaluated using function \code{evalParmods}.
The resulting parameter modification description contains no parameter dependencies and
no invocation descriptions. It is intended to be read by the filters which translate C function types and function 
definitions.

\item[\code{out}]
Using the function \code{convertParmods} the parameter modification description is converted to an item property 
map. The map is written to the standard output in the format described in Section~\ref{impl-itemprops-decl}.
\end{description}

All lists mentioned above are structured as a sequence of text lines.

If the result is a parameter modification description in JSON format it is written to the output as described in 
Section~\ref{impl-parmod-json}.

The first three commands are intended as a support for the developer when filling the description manually. The goal is that
for all three the output is empty. If there are unconfirmed parameters they must be inspected and confirmed. This usually 
modifies the list of required invocations. They can be reduced by generating and merging corresponding descriptions
from other source files.

\subsection{Item Property Declarations}
\label{impl-ocomps-items}

Processing item property declarations is implemented by the filter \code{items-proc}. It reads a item property declarations
from standard input as a sequence of text lines in the format described in Section~\ref{impl-itemprops-decl}. The first command line 
argument acts as a command how to process the item property declarations. The filter implements the following
commands:
\begin{description}

\item[\code{merge}]
Takes an additional file name as command line argument. The file contains additional item property declarations.
Both declarations are merged. If both contain properties for the same item the union of properties is declared for it.

\end{description}

\subsection{Processing Cogent Code}
\label{impl-ocomps-cogent}

There are some Gencot components which process a Cogent source to generate auxiliary files in antiquoted C or normal C code
or other output.
All these components are invoked with the name of a single Cogent source file as argument. Additionally, directories for
searching files included by cpphs can be specified with the help of \code{-I} options. The result is always written to 
standard output.

\subsubsection{\code{auxcog-remcomments}}

This component is a filter which removes comments from Cogent sources. Note that the Cogent preprocessor \code{cpphs}
does not remove comments, so it cannot be used for this task.

The filter is implemented as an awk script.

\subsubsection{\code{auxcog-macros}}

Gencot array types may have the form \code{CArrX<id>X} where \code{<id>} is a preprocessor constant (parameterless macro) 
identifier specifying the array size (see Section~\ref{design-types-array}). These types are generic in the element type 
and are thus implemented in antiquoted C. The implementation has the form 
\begin{verbatim}
typedef struct $id:(UArrX<id>X el) {
  $ty:el arr[<id>];
} $id:(UArrX<id>X el);
\end{verbatim}
where the identifier \code{<id>} occurs as size specification in the C array type. Cogent translates this antiquoted 
C definition to normal C definitions for all instances of the generic type used in the Cogent program. These C definitions
are included in the C program generated by Cogent.
Thus the definition of \code{<id>} must be available when the C program is processed by the C compiler or by Isabelle.

The definition of \code{<id>} may depend on other preprocessor macros, it may even depend on macros with parameters.
Therefore all macro definitions must be extracted from the Cogent source and made available as auxiliary C source.
This is implemented by \code{auxcog-macros} with the help of \code{cpp} and an \code{awk} script.

The macro definitions are extracted from the Cogent source \code{file.cogent} by executing the C preprocessor as
\begin{verbatim}
  cpp -P -dD -I<dir1> -I<dir2> ... -imacros file.cogent /dev/null 2> /dev/null
\end{verbatim}
The option \code{-imacros} has the effect that \code{cpp} only reads the macro definitions from \code{file.cogent}.
The actual input file is \code{/dev/null} so that no other input is read. The option \code{-dD} has the effect
to generate as output macro definitions for all known macros. This will also generate line directives, these are
suppressed by the option \code{-P}. The \code{-I} options must specify all directories where Cogent source files
are included by the preprocessor. All uses of the unbox operator \code{\#} in the Cogent code are signaled as an 
error by \code{cpp}, but the output is not affected by these errors. The error messages are suppressed by redirecting
the error output to \code{/dev/null}.

Note that \code{cpp} interpretes all include directives and conditional directives in the Cogent source. If for a macro
different definitions occur in different branches of a conditional directive, depending on a configuration flag,
\code{cpp} will select the definition according to the actual flag value.

The C preprocessor also adds definitions of internal macros. Some of them (those starting with \code{\_\_STDC}) are 
signaled as redefined when the output is processed again by \code{cpp}. This is handled
by postprocessing the output with an \code{awk} script. It removes all definitions of macros starting with \code{\_\_STDC}.

Since \code{cpp} does not recognize the comments in Cogent format, it will not eliminate them from the macro definitions 
in the output. Therefore it should be postprocessed by \code{auxcog-remcomments}. For a better result it should also 
be postprocessed by \code{auxcog-numexpr}.

\subsubsection{\code{auxcog-numexpr}}

The macro definitions selected by \code{auxcog-macros} are normally written as part of the Cogent program to expand 
to Cogent code. However, to be included in the C code they must expand to valid C code. This must basically be done
by the developer of the Cogent program. 

The filter \code{auxcog-numexpr} supports this by performing a simple conversion of numerical expressions from Cogent
to C. Numerical expressions are mostly already valid C code. Currently, the filter only removes all occurrences
of the \code{upcast} operator.

The filter is implemented as a simple \code{sed} script.

\subsubsection{\code{auxcog-mapback}}

The \code{auxcog-mapback} component supports this by converting specific kinds of Cogent code back to C. It handles 
the use of Cogent constants in integer value expressions. This occurs in code generated by Gencot: every reference 
to another preprocessor constant in the definition of a preprocessor constant is translated to a reference of the 
corresponding Cogent constant.

This is handled by including another C source file before the file generated by \code{auxcog-macros}. This
file is generated by \code{auxcog-mapback}. It contains a preprocessor macro definition for every Cogent constant
where the constant name is a mapped preprocessor constant name, which maps the Cogent constant name back to 
the preprocessor constant name. For example, if the preprocessor constant name \code{VAL1} has been mapped by
Gencot to the Cogent constant name \code{cogent\_VAL1}, the generated macro definition is
\begin{verbatim}
  #define cogent_VAL1 VAL1
\end{verbatim}

The component \code{auxcog-mapback} is implemented in Haskell and uses the Cogent parser to read the Cogent code.
It then processes every Cogent constant definition.

\subsection{Processing List Files}
\label{impl-ocomps-prclist}

Gencot uses some files containing simple lists of text strings, one in each line: the item property declaration list (see 
Section~\ref{impl-ccomps-itemprops}), and the list of used external items to be processed by \code{gencot-externs}, 
\code{gencot-exttypes}, and \code{gencot-dvdtypes}.

To be able to specify these lists in a more flexible way, Gencot allows comments of the form
\begin{verbatim}
  # ...
\end{verbatim}
where the hash sign must be at the beginning of the line.

The filter \code{gencot-prclist} is used to remove these comment lines (and also all empty lines). It is implemented
as an awk script.

\subsection{Generating Main Files}
\label{impl-ocomps-main}

The main files as described in Section~\ref{design-files} are \code{<package>.cogent} and \code{<package>.c}. They
contain include directives for all parts of the Cogent unit sources, the former for the Cogent sources, the latter 
for the C sources resulting from translating the Cogent sources with the Cogent compiler.

The content of these main files depends on the set of \code{.c} files which comprise the Cogent compilation unit.
A list of these files, as described in Section~\ref{impl-ccode-package}, is used as input for the components generating
the main files. In contrast to the components reading the package C code, these components only process the file 
names and not their content. The file name \code{additional\_includes.c} is ignored in the list, it can be used
to specify additional \code{.h} files for Gencot components reading the C code, which are not used in the Cogent 
code or after compiling Cogent back to C. For an application see Section~\ref{app-transfunction-pointer}.

\subsubsection{Generating the Cogent Main File}

The component \code{gencot-mainfile} generates the content of the Cogent main file \code{<package>.cogent}. 
It is implemented as a shell script. It reads the list of
\code{.c} files comprising the Cogent compilation unit as its input and it takes the package name \code{<package>}
as additional argument. The generated content is written to the standard output. 

The component adds include directives for the following files:
\begin{itemize}
\item The file \code{gencot.cogent}, if it exists in the current directory. This file can be used to include standard 
definitions provided by Gencot such as \code{gencot/Memory.cogent}.
\item The files \code{<package>-externs.cogent}, \code{<package>-exttypes.cogent}, \code{<package>-dvdtypes.cogent}.
\item The file \code{<package>-manabstr.cogent}, if it exists in the current directory. This file can be used to
manually define additional abstract types and functions used in the translated program.
\item The file \code{x.cogent} for every source file \code{x.c} specified as input.
\end{itemize}

\subsubsection{Generating the C Main File}

The component \code{auxcog-mainfile} generates the content of the C main file \code{<package>.c}. 
It is implemented as a shell script. It reads the list of
\code{.c} files comprising the Cogent compilation unit as its input and it takes the package name \code{<package>}
as additional argument. The generated content is written to the standard output. 

The component adds include directives for the following files:
\begin{itemize}
\item The file \code{gencot.h} for non-generic abstract standard types used by Gencot.
\item The files \code{<package>-mapback.h} and \code{<package>-macros.h}, if they exist in the current directory.
They are used to make preprocessor constants from the Cogent program available in C.
\item The files \code{<package>-dvdtypes.h} and \code{<package>-exttypes.h}, if they exist in the current directory. 
They contain definitions for the non-generic abstract types defined in \code{<package>-dvdtypes.cogent} and 
\code{<package>-dvdtypes.cogent}, respectively.
\item The file \code{<package>-manabstr.h}, if it exists in the current directory. This file can be used to
manually provide definitions for the non-generic abstract types defined in \code{<package>-manabstr.cogent}.
\item The file \code{<package>-gen.c} which must be the result of the translation of \code{<package>.cogent}
with the Cogent compiler.
\item The file \code{<package>-externs.c} with the implementations of all exit wrappers.
\item The file \code{<package>-gencot.c}, if it exists in the current directory. It contains the 
implementations of all generic abstract functions the program uses from the Gencot standard library.
\item The file \code{<package>-manabstr.c}, if it exists in the current directory. This file can be used to
manually provide implementations of the abstract functions defined in \code{<package>-manabstr.cogent}.
\item The file \code{cogent-common.c} with implementations of functions defined in \code{gum/common/common.cogent}
but not implemented by Cogent.
\item The file \code{x-entry.c} for every source file \code{x.c} specified as input. These files contain
the implementations of the entry wrappers.
\end{itemize}

The files \code{<package>-*.c} and \code{*-entry.c} result from postprocessing the corresponding 
\code{\_pp\_inferred.c} files by \code{auxcog}.


\section{Putting the Parts Together}
\label{impl-all}
The single filters and processors of Gencot are combined for the typical use cases in the shell scripts
\code{gencot}, \code{items}, \code{parmod}, and \code{auxcog}. The script \code{items} is intended for handling all 
aspects of working with the item property declarations (see Section~\ref{impl-itemprops}), the script \code{parmod} 
is intended for handling all aspects
of working with the parameter modification descriptions (see Section~\ref{impl-parmod}), the script
\code{gencot} is intended for handling all other aspects of translating from C to Cogent. The script \code{auxcog}
is intended for handling all aspects of additional processing after or in addition to processing the Cogent program by the 
Cogent compiler.

\subsection{The \code{gencot} Script}
\label{impl-all-gencot}

\subsubsection{Usage}

The overall synopsis of the \code{gencot} command is
\begin{verbatim}
  gencot <options> <subcommand> [<file>]
\end{verbatim}

The \code{gencot} command supports the following subcommands:
\begin{description}
\item[\code{check}] Test the specified C source file or include file for parsability by Gencot (see Section~\ref{app-prep} for how to 
make a C source parsable, if necessary).

\item[\code{cfile}] Translate the specified C source file to Cogent and generate the entry wrappers. If the source file name 
is \code{x.c} the results are written to files \code{x.cogent} and \code{x-entry.ac}.

\item[\code{hfile}] Translate the specified C include file to Cogent. If the source file name is \code{x.h} the result is
written to file \code{x-incl.cogent}

\item[\code{config}] Translate the specified C include file to Cogent with specific support for configuration files as described
in Section~\ref{impl-preprocessor-config}. If the source file name is \code{x.h} the result is
written to file \code{x-incl.cogent}

\item[\code{unit}] Generate additional Cogent files for the Cogent compilation unit.

\item[\code{cgraph}] Print the call graph for the Cogent compilation unit.

\end{description}

The input file \code{<file>} must be specified for the first four subcommands and it must be omitted for the latter two.
The subcommands \code{unit} and \code{cgraph} use the ``unit name'' to determine the files to be processed.
If the unit name is \code{x} the subcommand \code{unit} generates the following additional files (see Section~\ref{design-files}):
\begin{itemize}
\item \code{x.cogent}
\item \code{x-externs.cogent}
\item \code{x-exttypes.cogent}
\item \code{x-dvdtypes.cogent}
\end{itemize}

The \code{gencot} command supports the following options:
\begin{description}
\item[\code{-I}] used to specify directories where included files are searched, like for \code{cpp}. The 
option can be repeated to specify several directories, they are searched in the order in which the options
are specified.

\item[\code{-G}] Directory for searching Gencot auxiliary files. Only one can be given, default is \code{"."}.

\item[\code{-C}] Directory for retrieving stored declaration comments. Default is \code{"./comments"}.

\item[\code{-u}] The unit name. Default is \code{"all"}.

\item[\code{-k}] Keep directory with intermediate files. This is only intended for debugging.

\end{description}

As described in Section~\ref{impl-comments-decl}, Gencot moves comments from C function declarations to translated function
definitions. Since a declaration may be specified in a different C source file (often an include file) than the definition, 
the subcommands \code{cfile}, \code{hfile}, and \code{config} write all declaration comments in the processed file to
the directory specified by the \code{-C} option and use all comments found there when translating function definitions.

\subsubsection{Auxiliary Files}

Gencot uses an approach, where all manual annotations added to the C program for configuring the translation to Cogent 
are stored in auxiliary files separate from the original C program sources. This way it is possible to update the C 
sources to a newer version without the need to re-insert the annotations.

There are different auxiliary files for different purposes. The \code{gencot} script determines the auxiliary files using
a predefined naming scheme. These files are searched in the directory specified by the \code{-G} option. All auxiliary files
are optional, if a file is not found it is assumed that no corresponding annotations are needed.

When processing a file \code{x.<ext>} where \code{<ext>} is \code{c} or \code{h}, the \code{gencot} command uses the 
following auxiliary files, if they exist:
\begin{description}
\item[\code{x.gencot-addincl}] The content of this file is prepended to the processed file before processing.
\item[\code{x.gencot-noincl}] List of ignored include files, as described in Section~\ref{impl-ccode-include}.
\item[\code{x.gencot-omitincl}] The Gencot include omission list (see Section~\ref{design-preprocessor-incl}).
\item[\code{x.gencot-ppretain}] The Gencot directive retainment list (see Section~\ref{design-preprocessor-config}).
\item[\code{x.gencot-chsystem}] The content of this file is inserted immediately after the last system include directive.
\item[\code{x.gencot-manmacros}] The Gencot manual macro list (see Section~\ref{design-preprocessor-macros}).
\item[\code{x.gencot-macroconv}] The Gencot macro call conversion (see Section~\ref{design-preprocessor-macros}).
\item[\code{x.gencot-macrodefs}] The Gencot macro translation for the processed file (see Section~\ref{design-preprocessor-macros}).
\end{description}

For all cases with the exception of the last two, additionally a file with name \code{common} instead of \code{x} ist used, when it
is present. If both files are present the concatenation of them is used. In this way it is possible to specify common annotations 
for all C sources and add specific annotations in the source file specific files.

The subcommands \code{unit} and \code{cgraph} use corresponding auxiliary files for every file \code{x.c} processed by them.

The subcommand \code{unit} additionally uses the auxiliary file \code{common.gencot-std} which must contain the Gencot
standard components list, as described in Section~\ref{impl-ocomps-main}.

The subcommands \code{cfile}, \code{hfile}, \code{config}, and \code{unit} additionally use the auxiliary file named
\code{<file>-itemprops} which must contain all relevant item property declarations (see Section~\ref{impl-itemprops-decl}).

These subcommands also use the auxiliary file \code{common.gencot-namap} which must contain the name prefix map, 
as described in Section~\ref{impl-ccode-names}. No source file specific versions of the name prefix map are supported,
since the map is used globally for all names translated to Cogent.

Another kind of auxiliary files describe the Cogent compilation unit. These files are required to be present 
in the directory specified by the \code{-G} option. If they are not found, an error or a warning is signaled.

If the unit name is \code{<uname>} the auxiliary files for the Compilation unit are
\begin{description}
\item[\code{<uname>.unit}] The ``unit file''. It contains the list of C source file names
(one per line) which together constitute the Cogent compilation unit. Only the original source files \code{x.c} must 
be listed, include files \code{x.h} must not be listed.
\item[\code{<uname>-external.items}] The list of used external items, as generated by the processor \code{items-used}
(see Section~\ref{impl-ccomps-items}).
\end{description}

The first file is only used by the subcommands \code{unit} and \code{cgraph}. They process all files listed in the unit file
together with all included files. The second file is used by all \code{gencot} subcommands with the exception of \code{check}.

The subcommand \code{unit} also processes the following optional auxiliary file by generating an include directive for it in the 
main Cogent source \code{<uname>.cogent}, if the file exists:
\begin{description}
\item[\code{<uname>-manabstr.cogent}] This file is intended for manually specifying additional abstract functions used 
in the Cogent program.
\end{description}

\subsubsection{Implementation}

**todo**

The intended use of filter \code{gencot-remcomments} is for removing all comments from input to the language-c parser.
This input always consists of the actual source code file and the content of all included files. The simplest approach
would be to use the language-c preprocessor for it, immediately before parsing. 

However, it is easier for the filter \code{gencot-rempp} to remove the preprocessor directives when the comments are 
not present anymore. Therefore, Gencot applies the filter \code{gencot-remcomments} in a separate step before applying
\code{gencot-rempp}, immediately after processing the quoted include directives by \code{gencot-include}.
 
The filters \code{gencot-selcomments} and \code{gencot-selpp} for selecting comments and preprocessor directives, however, are
still applied to the single original source files, since they do not require additional information from the included files.


\subsection{The \code{items} Script}
\label{impl-all-items}

\subsubsection{Usage}

The overall synopsis of the \code{items} command is
\begin{verbatim}
  items <options> <subcommand> [<file>] [<file2>]
\end{verbatim}

The \code{items} command supports the following subcommands:
\begin{description}
\item[\code{file}] Create default property declarations for all items defined in the specified C source file
or include file \code{<file>}. 

\item[\code{unit}] Create default property declarations for all external items used in the Cogent compilation 
unit. 

\item[\code{used}] List external toplevel items used in the Cogent compilation unit. 

\item[\code{merge}] Merge the declarations in \code{<file>} and \code{<file2>}. Two declarations for the 
same item are combined by uniting the properties and removing negative properties.

\item[\code{mergeto}] Add properties from \code{<file2>} to items in \code{<file>}. Properties are only added/removed
if their toplevel item is already present in \code{<file>}.

\end{description}

If the unit name is \code{<uname>}, subcommand \code{used} writes its output to the file \code{<uname>-external.items}
The presence of this file is expected by most other Gencot command scripts. All other subcommands write their result to 
standard output.

For the subcommands \code{unit} and \code{used} no \code{<file>} must be specified, instead, they use the unit name
in the same way as the subcommand \code{unit} of the \code{gencot} command.

The \code{items} command supports the options \code{-I}, \code{-G}, \code{-u}, \code{-k} with the same meaning as
for the \code{gencot} command.

\subsubsection{Auxiliary Files}

The subcommand \code{file} uses the same auxiliary files as the subcommands \code{cfile} and \code{hfile} of \code{gencot} with
the exception of \code{common.gencot-namap} and \code{<file>-itemprops}.
The subcommand \code{unit} uses the same auxiliary files as the subcommand \code{unit} of \code{gencot} with the exception
of \code{common.gencot-namap} and \code{<file>-itemprops}.
The subcommand \code{used} uses the unit file \code{<uname>.unit} like subcommand \code{unit} and additionally
the optional auxiliary file 
\begin{description}
\item[\code{<uname>.unit-manitems}] Manually specfied external items, as described in Section~\ref{impl-ccomps-items} as
input to the processor \code{items-used}.
\end{description}

\subsubsection{Implementation}

** todo **

The \code{unit} command is implemented by the processor \code{items-externs} (see Section~\ref{impl-ccomps-itemprops}). 
The postprocessing step for removing declarations of non-external subitems is implemented by the filter
\code{items-proc} with command \code{filter} (see Section~\ref{impl-ocomps-items}).

\subsection{The \code{parmod} Script}
\label{impl-all-parmod}

\subsubsection{Usage}

The overall synopsis of the \code{parmod} command is
\begin{verbatim}
  parmod <options> <subcommand> <file> [<file2>]
\end{verbatim}

The \code{parmod} command supports the following subcommands:
\begin{description}
\item[\code{file}] Create parameter modification descriptions for all functions defined in C source file 
or include file \code{<file>}.
The result is written to standard output.

\item[\code{close}] Create parameter modification descriptions for all functions declared for the C source file
\code{<file>} in the file itself or in a file included by it.
The result is written to standard output.

\item[\code{unit}] Select entries from \code{<file>} (a file in JSON format containing parameter modification
descriptions) for all external functions used in the Cogent compilation unit. 
The result is written to standard output.

\item[\code{show}] Display on standard output information about the parameter modification description in \code{<file>}.

\item[\code{idlist}] List on standard output the item identifiers of all functions described in the 
paramer modification description in \code{<file>}.

\item[\code{diff}] Compare the parameter modification descriptions in \code{<file>} and \code{<file2>}. The output
has the same form as the Unix \code{diff} command, however, entries of functions occurring in both files are 
directly compared.

\item[\code{iddiff}] Compare the item identifiers of all functions described in \code{<file>} and \code{<file2>}.
The output has the same form as the Unix \code{diff} command.

\item[\code{addto}] Add to \code{<file>} all entries for required dependencies found in \code{<file2>}. Both files must contain 
parameter modification descriptions in JSON format. The result is written to \code{<file>}.

\item[\code{mergin}] Merge the parameter modification description entries in \code{<file>} and \code{<file2>} by building 
the union of the described functions. If a function is described in both files the entry with more confirmed parameter 
descriptions is used. The result is written to \code{<file>}.

\item[\code{replin}] Replace in \code{<file>} all function entries by an entry for the same function in \code{<file2>}
if it is present and has not less confirmed parameters. Both files must contain 
parameter modification descriptions in JSON format. The result is written to \code{<file>}.

\item[\code{eval}] Evaluate the parameter modification description in \code{<file>} as described in 
Section~\ref{impl-parmod-eval}.  The resulting parameter modification description is written to standard output.

\item[\code{out}] Convert the evaluated parameter modification description in \code{<file>} to an property declarations.
The result is written to the file \code{<file>-itemprops}.

\end{description}

The subcommand \code{unit} expects no argument \code{<file>}, instead, it uses the unit name in the same way as the subcommand \code{unit} 
of the \code{gencot} command.

The \code{parmod} command supports the options \code{-I}, \code{-G}, \code{-u}, \code{-k} with the same meaning as
for the \code{gencot} command. They are only used for the subcommands \code{file}, \code{close}, and \code{unit}.

\subsubsection{Auxiliary Files}

The subcommands \code{file} and \code{close} use the same auxiliary files as the subcommands \code{cfile} and \code{hfile} of \code{gencot} with 
the exception of \code{common.gencot-namap} and \code{<file>-itemprops}.
The subcommand \code{unit} uses the same auxiliary files as the subcommand \code{unit} of \code{gencot} with the exception
of \code{common.gencot-namap} and \code{<file>-itemprops}.

\subsubsection{Implementation}

\paragraph{The subcommand \code{unit}} 
First, all C source files are prepared for parsing, as in command \code{gencot unit}. Then the list of used external toplevel 
items \code{<uname>-external.items} is passed to processor \code{items-extfuns} to determine the list of 
used external functions. This list is then used to filter the descriptions in \code{<file2>} with \code{parmod-proc filter}.

Additionally the result is tested, whether all external functions determined by \code{items-extfuns} have been found in 
\code{<file2>}. This is done by calculating the function ids of all descriptions in the result with \code{parmod-proc funids}, 
sorting both lists and comparing them with \code{diff}. If the result is not empty, a warning is displayed on standard error.

\subsection{The \code{auxcog} Script}
\label{impl-all-auxcog}

\subsubsection{Usage}

The overall synopsis of the \code{auxcog} command is
\begin{verbatim}
  auxcog <options> <subcommand> [<file>]
\end{verbatim}

The \code{auxcog} command supports the following subcommands:
\begin{description}
\item[\code{unit}] Generate additional C files for the Cogent compilation unit.

\item[\code{shallow}] Extend the shallow embedding generated by Cogent.

\item[\code{refine}] Extend the refinement proof generated by Cogent.

\item[\code{comments}] Remove comments from the Cogent source \code{<file>} and write the result to standard output.

\end{description}

The subcommand \code{unit} expects no argument \code{<file>}, instead, it uses the unit name in the same way as the subcommand \code{unit} 
of the \code{gencot} command. It expects all C files generated by Cogent in the current directory. In particular, if the unit name 
is \code{x}, it expects files \code{<f>\_pp\_inferred.c} for the following cases of \code{<f>}:
\begin{itemize}
\item \code{x-externs}
\item \code{y-entry} for all \code{y.c} listed in the unit file \code{x.unit}
\item \code{std-<name>} for all entries \code{"anti: <name>"} in file \code{common.gencot-std}
\item optionally: \code{x-gencot}
\item optionally: \code{x-manabstr}
\end{itemize}

If the unit name is \code{x} the subcommand \code{unit} generates the following additional files in the output 
directory (see Section~\ref{design-files}):
\begin{itemize}
\item \code{x.c}
\item \code{x-externs.c}
\item \code{x-gencot.c} (only if \code{x-gencot\_pp\_inferred.c} is present)
\item \code{x-manabstr.c} (only if \code{x-manabstr\_pp\_inferred.c} is present)
\end{itemize}
and for all files \code{y.c} listed in the unit file \code{x.unit}:
\begin{itemize}
\item \code{y-entry.c}
\end{itemize}
and for all entries \code{"anti: <name>"} in file \code{common.gencot-std} the file
\begin{itemize}
\item \code{std-<name>.c}
\end{itemize}

The subcommands \code{shallow} and \code{refine} expect as \code{<file>} a file named \code{X\_ShallowShared\_Tuples.thy} 
which has been generated
by Cogent for a ``proof name'' \code{X}. Subcommand \code{shallow} expects the file \code{X\_Shallow\_Desugar\_Tuples.thy} in the same directory 
as specified for \code{<file>}. In the output directory the following files are provided (see Section~\ref{impl-isabelle-shallow}):
\begin{itemize}
\item \code{X\_ShallowShared\_Tuples.thy} (modified)
\item \code{X\_Shallow\_Desugar\_Tuples.thy} (copied from the directory of \code{<file>} and modified)
\item \code{ShallowShared\_Tuples.thy}
\item \code{X\_Shallow\_Gencot\_Tuples.thy}
\item file \code{GencotTypes.thy} and all files \code{Y\_Tuples.thy} in directory \code{isa} in the Gencot distribution
\item \code{X\_Shallow\_Gencot\_Lemmas.thy}
\item all files \code{Y\_Lemmas.thy} in directory \code{isa} in the Gencot distribution
\end{itemize}

The subcommand \code{refine} additionally expects the file \code{X\_Shallow\_Desugar.thy} in the same directory 
as specified for \code{<file>}. In the output directory the following files are provided (see Section~\ref{impl-isabelle-reftuples}):
\begin{itemize}
\item all files provided by \code{shallow} but the \code{...Lemmas.thy} files
\item \code{X\_ShallowShared.thy} (modified)
\item \code{X\_Shallow\_Desugar.thy} (copied from the directory of \code{<file>} and modified)
\item \code{ShallowShared.thy}
\item \code{X\_Shallow\_Gencot.thy}
\item file \code{GencotTypes.thy} and all files \code{Gencot\_TTT.thy} and \code{CogentCommon\_ttt.thy} 
in directory \code{isa} in the Gencot distribution
\item file \code{X\_ShallowTuplesProof.thy} (copied from the directory of \code{<file>} and modified)
\item ... more to come
\end{itemize}

The subcommand \code{comments} is mainly intended for testing the comment structure in a Cogent source generated by
\code{gencot}. In rare cases it may happen that a closing comment delimiter and a subsequent opening comment delimiter
are omitted, so that the Cogent compiler will not detect the wrong comment structure.

The \code{auxcog} command supports the options \code{-I}, \code{-G}, \code{-u} with the same meaning as
the \code{gencot} command. Additionally it supports the option \code{-O} for specifying the output directory (default \code{"."})
for the subcommands \code{unit}, \code{shallow} and \code{refine}.

\subsubsection{Auxiliary Files}

The subcommands \code{unit}, \code{shallow}, and \code{refine} use the auxiliary file \code{common.gencot-std} which must contain the Gencot
standard components list, as described in Section~\ref{impl-ocomps-main}. The subcommand \code{unit} additionally uses the 
unit file \code{<uname>.unit} as auxiliary file. Both files are expected in the directory specified 
by the \code{-G} option.

Additionally the subcommands \code{unit}, \code{shallow}, and \code{refine} also read 
the file \code{<uname>-gen.h} which must have been created by Cogent when compiling file \code{<uname>.cogent} with 
option \code{"-o <uname>-gen"}. This file is always expected in the current directory. Files included by \code{<uname>-gen.h} 
are searched in the directories specified by the \code{-I} options.

\subsubsection{Implementation}

**todo**

The subcommand \code{shallow} is implemented by applying the filter \code{auxcog-shalshared} to 
the content of file \code{X\_ShallowShared\_Tuples.thy} to generate the modified file \code{X\_ShallowShared\_Tuples.thy}.
Then the filters \code{auxcog-shalgencot} and \code{auxcog-lemmgencot} are applied to this modified file to generate
the files \code{X\_Shallow\_Gencot\_Tuples.thy} and \code{X\_Shallow\_Gencot\_Lemmas.thy}. 
The file \code{X\_Shallow\_Desugar\_Tuples.thy} is directly modified using
\code{sed} (only replacing the theory import). The remaining files are simply copied from the Gencot distribution.

The subcommand \code{refine} is implemented by applying the filter \code{auxcog-shalshared} to 
the content of file \code{X\_ShallowShared.thy} to generate the modified file \code{X\_ShallowShared.thy}. Then
the filter \code{auxcog-shalgencot} is applied to this modified file to generate
the file \code{X\_Shallow\_Gencot.thy}. The file \code{X\_Shallow\_Desugar.thy} is directly modified using
\code{sed} (only replacing the theory import). The remaining files are simply copied from the Gencot distribution.



\chapter{Application}

The goal of applying Gencot to a C <package> is to translate a subset of the C sources to Cogent,
resulting in a Cogent compilation unit which can be separately compiled and linked together with
the rest of the <package> to yield a working system. The Cogent compilation unit is always a
combination of one or more complete C compilation units, represented by the corresponding \code{.c} 
files.

We call this subset of \code{.c} files the ``translation base''. Additionally, all \code{.h} files
included directly or indirectly by the translation base must be translated. Together, we call these
source files the ``translation set''. Every file in the translation set will be translated to a 
separate Cogent source file, as described in Section~\ref{design-files}. Additional Cogent sources
and other files will be generated from the translation set to complete the Cogent Compilation Unit.

Since there is only one Cogent compilation unit in the package, we sometimes use the term ``package''
to refer to the Cogent compilation unit. However, Gencot supports working with alternative 
translation bases at the same time, which result in different Cogent compilation units.

A translation base (and its resulting Cogent compilation unit) may be named by a ``unit name'' and
is defined by a ``unit file'' \code{<uname>.unit} where \code{<uname>} is the unit name. It contains 
the names of all files comprising the translation base, every name on a separate line. 

\section{Preparing to Read the Sources}
\label{app-prep}
%\input{app-prep}

\section{Determining Used External Items}
\label{app-items}
Most other commands expect the presence of the auxiliary file \code{<uname>-external.items}
(see Sections~\ref{impl-all-gencot} and~\ref{impl-ccomps-items}).

Therefore, after all sources in the translation set can be read by Gencot, the command
\begin{verbatim}
  items used
\end{verbatim}
must be used to create the list of used external items (see Section~\ref{impl-all-items}).

If there are external items which must be processed, but are not detected automatically by Gencot,
the auxiliary file \code{<uname>.unit-manitems} must be provided in advance. If present, this file is 
automatically read by \code{items used} and the items listed in it (and all items transitively used 
from them) are added to \code{<uname>-external.items}.

If additionally required
external items become known after C code has already been translated to Cogent, they must be 
added to this file and the command \code{items used} and all following steps must be re-executed.



\section{Building Parmod Descriptions}
\label{app-parmod}
The goal of this step is to create the parameter modification descriptions (see Section~\ref{design-parmod})
for the C sources. This step is optional, if it is omitted Gencot will translate all functions using the 
assumption that all parameters of linear type are discarded by their function. This will usually lead to
invalid Cogent code if the parameter value is still used after a function invocation.

Basically, the descriptions must be determined for all functions defined in the translation
set. Usually function definitions only reside in \code{.c} files, but there are C <packages> which also put 
some function definitions in \code{.h} files.

Additionally, parmod descriptions are required for all functions invoked but not defined in
the translation set (``external functions''). Since a parmod description is always derived from the function 
definition, this implies that a superset of the translation set must be processed in this step. The strategy 
described here tries to keep this superset minimal. 

If for a function no definition is available (which is the case for function pointers and for functions
defined outside the C <package>), Gencot only generates a description template which must be filled by the
developer. In the other cases Gencot creates a description in a best effort approach, which must be confirmed
be the developer.

The automatic parts of this step are executed with the help of the script command \code{parmod} (see 
Section~\ref{impl-all-parmod}).

\subsection{Describing Defined Functions}
\label{app-parmod-defined}

The descriptions for the functions defined in the translation set are determined iteratively in a first substep. 
The result is a single parmod description file in json format. It is extended in every iteration and finally
evaluated.

To start, the command
\begin{verbatim}
  items used
\end{verbatim}
must be used to create the list of used external items, which is needed by the commands \code{parmod file}
and \code{parmod close} in the following.

Then for every file in the translation base and for every other file in the translation set which contains
function definitions, the command
\begin{verbatim}
  parmod file
\end{verbatim}
is used to create a parmod description file. These files are then merged using the command
\begin{verbatim}
  parmod mergin
\end{verbatim}
to yield the initial working file for the iterations. 

In each iteration the descriptions must be manually confirmed (using a text editor) until the command
\begin{verbatim}
  parmod show
\end{verbatim}
does not signal any remaining unconfirmed descriptions. If it then does not signal any required 
invocations which are defined in the <package>, the iterations end. Otherwise, the developer must
search for all files in the <package> where required invocations are defined (these files will
not belong to the translation set) and generate parmod descriptions for them using \code{parmod file}.
These descriptions are then added to the working file using the command
\begin{verbatim}
  parmod addto
\end{verbatim}
and the next iteration is performed.

When the iterations end, there may still be remaining required invocations. These are invocations of 
functions defined outside the <package>. To generate the description templates for them the command
\begin{verbatim}
  parmod close
\end{verbatim}
is applied to all C sources to which the command \code{parmod file} has been applied previosly. The resulting 
parmod description template files are merged using \code{parmod mergin} and the resulting description template file
is added to the working file using \code{parmod addto}. The command \code{parmod close} generates templates for
all functions and function pointers for which a declaration is visible in the C source. Since for every function
invoked in a C source a declaration must be visible, the working file will not have required invocations after
this step. Since description templates introduce no additional required invocations, after a final confirmation
step the working file is completed.

Instead of merging the parmod description template files generated by \code{parmod close}, they can also be added
separately to the working file. Often, a C source has much more visible declarations than it needs. Therefore it may
cover required invocations of other files. This way it may be the case that not all single description template
files must be generated to complete the working file.

Evaluating the completed working file using the command
\begin{verbatim}
  parmod eval
\end{verbatim}
will eliminate all transitive dependencies and yield the final parmod description file. It contains at least 
the descriptions for all functions defined in the translation set.

\subsection{Describing External Functions}
\label{app-parmod-extern}

The descriptions for the invoked external functions are determined in a second substep. It results in 
a separate parmod description file. 

The invoked external functions must be determined from the translation set, however, their descriptions must
be generated from their definitions, which are outside the translation set. This is done by the command
\begin{verbatim}
  parmod unit
\end{verbatim}
applied to the unit file and a parmod description file which must contain the descriptions for a superset
of the invoked external functions. This parmod description file must be prepared in advance.

One possibility to prepare the file is to process all C source files which do \textit{not} belong to the 
translation set with \code{parmod file} and merge the results. Since the functions defined outside the <package>
are not contained, additionally the C source files which \textit{do} belong to the translation set must be
processed with \code{parmod close} and the resulting description templates merged to the previous results. 

However, this implies that all files in the
<package> must be prepared for parsing as described in Section~\ref{app-prep}. To avoid this, the file can
be build by trying \code{parmod unit} with an empty parmod description (which is an empty JSON list code{[]}).
It will signal all missing invoked functions not defined in the translation set. From this list the defining 
files (or declaring files in the case of a function defined outside the <package>) can be identified and used 
to build the description file passed to \code{parmod unit}.

Instead of starting with an empty description, all files created by \code{parmod file} in the first substep 
(see Section~\ref{app-parmod-defined}) can be merged and used as a starting point.  

When the descriptions resulting from \code{parmod file} and the description templates resulting from \code{parmod close}
are merged, care must be taken that a description is not replaced by a template for the same function. This may 
happen, since a function may be defined in one \code{.c} file but only declared in another, then both the description
and the template will be present. The command \code{parmod mergin} always selects the description or template with less 
unconfirmed parameters. If both have the same number of unconfirmed parameters, it selects the description or template from
its first argument file. 

A fresh template generated by \code{parmod close} never has more confirmed parameter descriptions than a fresh 
description generated by \code{parmod file} for the same function. Thus, descriptions are selected over templates
if first all descriptions are merged, then the templates are merged into the result, with the file containing the
templates being the second argument to \code{parmod mergin}.

When \code{parmod unit} has successfully been executed, the result must be iteratively completed in the same way 
as in Section~\ref{app-parmod-defined}. 

Initially, however, the file \code{parmod-system.json} should be added using command
\begin{verbatim}
  parmod replin <json-file> parmod-system.json
\end{verbatim}
This file is included in the Gencot distribution and contains completed descriptions for a number of 
standard C system functions. In particular, it covers the five memory management functions, which are the 
only functions for which the heap is always used and from which heap usage is then derived for other functions. 
The command \code{parmod replin} replaces all templates in \code{<json-file>} by the description for the same function in
\code{parmod-system.json}, if present there.

Afterwards, the result of the first substep should be added 
using \code{parmod addto}, since it may contain additional descriptions not needed in the first substep.
All descriptions added this way are already fully confirmed and evaluated. Only if afterwards there are remaining 
required invocations, the file must be iteratively completed as in the first substep. 

When the file has been completed, it must be evaluated using \code{parmod eval}.


\section{Building ItemProperty Declarations}
\label{app-itemprops}
The goal of this step is to create the item property declarations (see Section~\ref{impl-itemprops})
to be used in the translation from C to Cogent. This step is optional, Gencot can translate a C source 
without them. However, the assumptions it uses for the translation will in many cases lead to 
invalid Cogent code (which is syntactically correct but does not pass the semantics checks).

Instead of specifying the additional information used for the translation as annotations in the C source files,
Gencot uses the item property declarations which reside in separate files and reference the affected C 
items by name (the ``item identifiers'' described in Section~\ref{impl-itemprops-ids}. This makes it 
easier to replace the C sources by a newer version and still reuse (most of) the item properties.

Item properties could be specified fully manually by the developer, if she knows the item naming scheme
and searches the C source files for items which require properties to be declared. However, Gencot provides
support by generating default item property declarations. This has the following advantages:
\begin{itemize}
\item The developer gets a list of all items with their correct names.
\item Gencot can use simple heuristics to populate the declarations with useful default properties.
\end{itemize}

The default item property declarations are intended to be manually modified by the developer. Additionally, the 
results of a parameter modification analysis described in Section~\ref{app-parmod} are intended to be transferred
to the item property declarations.

Property declarations are used by Gencot, if present, for all items processed in the translation. These are all internal items 
which are defined in the translation set and all external items which are declared and used in the translation set.

The automatic parts of this step are executed with the help of the script command \code{items} (see 
Section~\ref{impl-all-items}).

\subsection{Grouping Item Property Declarations in Files}
\label{app-items-files}

C items for which Gencot supports property declarations are types (struct/unions and named types), functions and objects (``variables'').
Items may have sub-items (function parameters and results, array elements, pointer targets). 

For type items there is always a single definition (which also defines sub-items). For functions and objects there 
can be declarations
in addition to the single definition. The definition or declaration introduces a name for the item and it specifies its type.
Afterwards, an item can be used any number of times (e.g. type items can be used in other definitions and declarations, functions
can be invoked and objects can be read and written).

The current version of Gencot uses the properties of such items only when translating the item's definition or declaration, not when 
it translates the item's use. 

When Gencot translates a single C source file it ignores all declarations and only processes item definitions and uses. This implies
that it only uses property declarations for exactly the items which are defined in the translated source file, and suggests to
use a seperate item property declaration file for every C source file, containing the declarations for the items defined in the 
C source file.

Gencot expects the item properties to be used when translating a single C source file \code{x.<e>} (where \code{<e>} 
is \code{c} or \code{h}) in an auxiliary file named \code{x.<e>-itemprops} (see Section~\ref{impl-all-gencot}).

When Gencot processes all C source files of the translation set together to generate the additional Cogent sources for the
Cogent compilation unit (command \code{gencot unit}, see Section~\ref{impl-all-gencot}), it processes items defined in the
translation set and also external items declared and used in the translation set. Therefore it needs property declarations 
for all these items. They can be retrieved by uniting all item property declaration files for the single C sources and adding
property declarations for the external used items. If the Cogent compilation unit is described by the unit file \code{y.<e>} 
(where the convention for the file extension \code{<e>} is \code{unit}), Gencot expects all these item property declarations
in an auxiliary file named \code{y.<e>-itemprops} (see Section~\ref{impl-all-gencot}).

Since the properties in the \code{-itemprops} files may originate from several sources (default properties, manual specification, 
parmod descriptions, merging other property declarations) it is recommended to keep these sources separate and recreate
the \code{-itemprops} files from these sources whenever they are used for a translation. 

It is recommended to put the item property declarations for all external used items of a Cogent compilation unit described by
the unit file \code{y.<e>} in a file \code{y.ext-itemprops} and always recreate the file \code{y.<e>-itemprops} by merging all
files \code{x.<e>-itemprops} and the file \code{y.ext-itemprops}.

For the sources of a file \code{x.<e>-itemprops} the following file names are recommended :
\begin{description}
\item[\code{x.<e>-dfltprops}] The default item property declarations generated by Gencot.
\item[\code{x.<e>-manprops}] Manually specified item properties to be added. 
\end{description}
Then the file \code{x.<e>-itemprops} for a single C source file is generated by first modifying \code{x.<e>-dfltprops} 
according to the parameter modification descriptions and then adding the properties from \code{x.<e>-manprops}. In the same
way the file \code{y.ext-itemprops} should be generated.

Gencot also supports item property declarations for a restricted form of type expressions (derived pointer types). These items
are not explicitly defined, they are used by simply denoting the type expression. For this kind of type items the properties are 
needed by Gencot for translating the item uses. 

Gencot does not provide default item property declarations for these items, they must always be specified manually. Since they 
are not related to a specific C source file or a specific translation set, they should be put in a single separate file. The 
recommended name for this file is \code{common.types-manprops}. The properties in this file must be used for the translation 
of all single C source files and for the generation of the additional Cogent sources for the compilation unit. This is 
achieved by merging the properties in this file to all generated \code{-itemprops} files.

\subsection{Generating Default Item Property Declarations}
\label{app-items-default}

The first step for using item property declarations is to generate the default property declarations. For a single C source 
file or include file \code{<file>} the command
\begin{verbatim}
  items file <file>
\end{verbatim}
is used. The result should be stored in file \code{<file>-dfltprops}.

To generate the default property declarations for the external items used by the Cogent compilation unit described in the 
unit file \code{y.<e>} the command
\begin{verbatim}
  items unit y.<e>
\end{verbatim}
is used. The result should be stored in file \code{y.ext-dfltprops}.

\subsection{Transferring Parameter Modification Descriptions}
\label{app-items-parmod}

If parameter modification descriptions are used as described in Section~\ref{app-parmod}, the result must be converted to item
property declarations to be used by Gencot. The result consists of a single file \code{<jfile>} in JSON format containing 
parameter modification descriptions. All entries have been confirmed and the file has been evaluated as described in 
Section~\ref{app-parmod-extern}. The file contains entries for all functions defined in the translation set, for all external 
functions used in the translation set, and maybe additional functions which have been used to follow dependencies.

Every function is an item, and every parameter is a sub-item of its function. Thus the information in the parameter modification
description can be converted to property declarations for function and parameter items. This is done using the command
\begin{verbatim}
  parmod out <jfile>
\end{verbatim}
It will create the two files \code{<jfile>-itemprops} and \code{<jfile>-omitprops}. 

The file \code{<jfile>-itemprops} contains properties to be added to the default properties. For every file \code{<file>-dfltprops}
the command 
\begin{verbatim}
  items mergeto <file>-dfltprops <jfile>-itemprops
\end{verbatim}
is used. Its output should be stored in an intermediate file \code{<file>-ip1}. The command only selects from 
\code{<jfile>-itemprops} declarations for items which are already present in
\code{<file>-dfltprops}. In this way the collective information in \code{<jfile>-itemprops} is distributed to the different
specific property declaration files for the single C sources and the file \code{y.ext-dfltprops}. Information about functions
only used for following dependencies is not transferred, it is not used by the translation to Cogent.

The file \code{<jfile>-omitprops} contains properties to be removed from the default properties. For every intermediate file
\code{<file>-ip1} created by \code{items mergeto} the command 
\begin{verbatim}
  items omitfrom <file>-ip1 <jfile>-omitprops
\end{verbatim}
is used. Its output should again be stored in an intermediate file \code{<file>-ip2}. If no manual modifications are intended
the result can directly be written to file \code{<file>-itemprops} and used for the translation by Gencot.

\subsection{Adding Manually Specified Properties}
\label{app-items-manual}

It is recommended to put manually specified item properties in separate files, grouped in the same way as the default 
properties. For every file \code{<file>-dfltprops} a file \code{<file>-manprops} should be used. To create this file,
you can make a copy of \code{<file>-dfltprops}, manually remove all declared properties and then populate the empty
declarations with manually specified properties. Remaining empty declarations can be removed.

The manual properties are then merged to the default properties. This makes it possible to re-apply the manual properties 
whenever the default properties change (e.g., because the C source is modified). To avoid removing manually specified 
properties by transmitting parameter modification description results, it is recommended to merge the manual properties 
afterwards into the intermediate file \code{<file>-ip2} using the command
\begin{verbatim}
  items mergeto <file>-ip2 <file>-manprops
\end{verbatim}
and storing the result in another intermediate file \code{<file>-ip3}.

If properties shall be specified for type expression items, they should be collected in the file \code{common.types-manprops}.
No default property declarations exist for these items. Since the type expressions may be used in all C source files, the 
file should be merged into \textit{all} item property declaration files additionally to the file specific manual properties.
This must be done using the command
\begin{verbatim}
  items merge <file>-ip3 common.types-manprops
\end{verbatim}
because here the added items shall not be reduced to those already present in \code{<file>-ip3}. The result is ready to be used
by Gencot and should be written to file \code{<file>-itemprops}.

Normally, it should not be necessary to manually remove properties. Should that be needed, the properties to be removed can
be specified in a separate file which can then be applied to remove the properties using the command \code{items omitfrom} in 
an additional step.

For convenience for the developer Gencot supports empty lines and comment lines starting with a hash sign \code{\#} in the 
\code{-manprops} files. These lines are removed by the commands \code{items merge}, \code{items mergeto}, and \code{items omitfrom}.

\subsection{Property Declarations for a Cogent Compilation Unit}
\label{app-items-unit}

When processing a Cogent compilation unit described by the unit file \code{y.<e>} the item property declarations in file
\code{y.<e>-itemprops} are used (see Section~\ref{app-items-files}). This file is built by simply merging all
generated files \code{<file>-itemprops} and the file \code{y.ext-itemprops} iteratively using the command
\begin{verbatim}
  items merge
\end{verbatim}
This command removes all duplicate item declarations which may originate from \code{common.types-manprops}.

\subsection{Adding External Items}
\label{app-items-extern}

The command \code{items unit} automatically reduces the known (declared) external items to those actually used in the
translation set, as described in Section~\ref{impl-ccomps-items}, and generates default item property declarations only
for them and their sub-items. It may be necessary to process additional external items which are not used or are not recognized
to be used. The set of external items recognized by Gencot as used can be displayed using the command
\begin{verbatim}
  items used y.<e>
\end{verbatim}
where \code{y.<e>} is the unit file describing the translation set.

To generate default properties for additional external items, a list of the additional (toplevel) items can be 
specified in the auxiliary file \code{y.<e>-manitems}. If present, this file is automatically read by \code{items unit}
and default property declarations are generated for all items listed in it (and all items transitively used from them).


\section{Automatic Translation to Cogent}
\label{app-transauto}
The goal of this step is to translate a single C source file which is part of the translation set to Cogent.
This is done using Gencot to translate the file to Cogent with embedded partially translated C code.
Afterwards the embedded C code must be translated manually, as described in Section~\ref{app-transfunction}.

The translated file is either a \code{.c} file which can be seperately compiled by the C compiler as a compilation unit,
or it is a \code{.h} file which is a part of one or more compilation units by being included by other files. In both cases
the file can include \code{.h} files, both in the package and standard C include files such as \code{stdio.h}.

To be translated by Gencot the C source file must first be prepared for being read by Gencot as described in
Section~\ref{app-prep}. Then the parameter modification descriptions for all functions and function types defined 
in the source file must be created and evaluated as described in Section~\ref{app-parmod-defined}. 

\subsection{Translating Normal C Sources}
\label{app-transauto-normal}

A \code{.c} file is translated using the command
\begin{verbatim}
  gencot [options] cfile file.c [<parmod>]
\end{verbatim}
where \code{file.c} is the file to be translated and \code{<parmod>} is the file containing the parameter modification 
descriptions. If it is not provided, Gencot assumes for all functions and function types, that linear parameters 
may be modified. The result of the translation is stored as \code{file.cogent} in the current directory.

An \code{.h} file is translated using the command
\begin{verbatim}
  gencot [options] hfile file.h [<parmod>]
\end{verbatim}
where \code{file.h} is the file to be translated and \code{<parmod>} is as above. The result of the translation is 
stored as \code{file-incl.cogent} in the current directory.

\subsection{Translating Configuration Files}
\label{app-transauto-config}

Special support is provided for translating configuration files. A configuration file is a \code{.h} file mainly
containing preprocessor directives for defining flags and macros, some of which are deactivated by being ``commented
out'', i.e., they are preceded by \code{//}. If such a file is translated using the command
\begin{verbatim}
  gencot [options] config file.h [<parmod>]
\end{verbatim}
it is translated like a normal \code{.h} file, but before, all \code{//} comment starts at line beginnings are 
removed, and afterwards the corresponding Cogent comment starts \code{-\relax-} are re-inserted before the definitions.



\section{Manual Adaptation of Data Types}
\label{app-transtype}
In some cases the mapped data type definitions generated by Gencot according to Section~\ref{design-types} must be manually
adapted by editing the corresponding \code{x-incl.cogent} files. Here we describe typical cases where an adaptation is
necessary.

\subsection{NULL Pointers}
\label{app-transtype-null}

If a linear value is used in the program at a place where in the C program it is not known to be non-null, it's type
must be changed from \code{T} to \code{(MayNull T)} (see Section~\ref{design-operations-null}).

A good hint for the property that a value may be null is if it is tested for being null in the C program. If not, however,
this case may have been overlooked. This will show up if the value's type is not changed and during the manual translation
of the function bodies an assignment of \code{NULL} must be translated for it.

A linear value may occur in the Cogent program as a function parameter, a function result, or a field in a record. In each case
all occurrences of the value must be analyzed to find out whether the type must be changed or not.

If the value is a field in a record the null pointer may be represented by the field being taken. This is possible when 
it is statically known for each occurrence of the record in the Cogent program, whether the field value is null or not. 
Then the type of the record can be changed to the type with the field taken for all cases where the field value is null.
A typical application case is when the field is initialized at a specific point in the program and never set to null
afterwards.

If the value is the parameter or result of a function pointer or the element of an array, or it is referenced by a pointer,
a type name must be introduced 
for the \code{MayNull} type instance and then used to consistently rename the type name generated by Gencot for the 
corresponding derived type. For example, if a C function pointer has type
\begin{verbatim}
  int (*)(some_struct *p)
\end{verbatim}
it is mapped by Gencot to have the abstract type
\begin{verbatim}
  F_XStruct_Cogent_some_structX_U32
\end{verbatim}
If parameter \code{p} may be null, its type should be changed to a new abstract type introduced by:
\begin{verbatim}
  type F_XMayNull_Cogent_some_structX_U32
  type MayNull_Cogent_some_struct = 
    MayNull Struct_Cogent_some_struct
\end{verbatim}

\subsection{Grouping Fields in a Record}
\label{app-transtype-group}

Sometimes in a C struct type several members work closely together by pointing to memory
shared among them. Then in Cogent it would be possible to take the corresponding record fields separately,
resulting in shared values of linear type. Instead, the fields should only be operated on by specific functions
for which the correct handling of the fields can be proven. To avoid taking the fields separately, these
functions should be defined on the record as a whole. Then it can be statically checked that the fields
are never accessed or taken/put outside these functions.

To make this more explicit, the corresponding record fields can be grouped into an embedded unboxed
record in Cogent. This is binary compatible if the members of the original struct are consecutive and in
the same order and are no bitfields. For example in the record type \code{R} as defined in 
\begin{verbatim}
  type A
  type R = {f1: A, f2: A, f3: A, f4: A, f5:A}
\end{verbatim}
the fields \code{f2} and \code{f3} can be grouped by introducing the new record type \code{E}:
\begin{verbatim}
  type A
  type E = {f2: A, f3: A}
  type R = {f1: A, embedded: #E, f4: A, f5:A}
\end{verbatim}
This is translated by Cogent to an embedded struct.

Now operations working on some or all of the grouped fields can be defined on the type \code{E} instead
of \code{R}, guaranteeing that the functions cannot access the other fields of \code{R}.

Note that it is still possible to take the fields separately by first taking \code{embedded} from \code{R}
and then taking the fields from the taken value. This can now be prevented by checking that the field
\code{embedded} is never accessed or taken/put directly in a value of type \code{R}.

Instead, using the conceptual operations \code{getref} and \code{modref} as defined in 
Section~\ref{design-operations-parts}
it is possible to apply the functions defined on \code{E} in-place without copying the field values.
This requires to manually define abstract functions
\begin{verbatim}
  getrefEmbeddedInR: R! -> E!
  modrefEmbeddedInR: all(arg,out). ModPartFun R E arg out
\end{verbatim}
together with type \code{E}. They are implemented in C using the address operator \code{\&} and should
be the only ways how to access the field \code{embedded}.

To provide even more shielding, the type \code{E} can be defined as abstract, providing the definition
as a struct in C:
\begin{verbatim}
  typedef struct {A f2; A f3; } E;
\end{verbatim}
Then it is guaranteed that the fields can never be accessed in the Cogent program. However, then also 
all function working on type \code{E} must be defined as abstract functions which are implemented in C.

\subsection{Using Pointers for Array Access}
\label{app-transtype-arrpoint}

A common pattern in C programs is to explicitly use a pointer type instead of an array type for referencing an array,
in particular if the array is allocated on the heap. This is typically done if the number of elements in the
array is not statically known at compile time. The C concept of variable length array types can sometimes be
used for a similar purpose, but is restricted to function parameters and local variables and cannot be used for
structure members.

In C the array subscription operator can be applied to terms of pointer type. The semantics is to access an element 
in memory at the specified offset after the element referenced by the pointer. This makes it possible to use 
a pointer as struct member which references an array as in
\begin{verbatim}
  struct ip {... int *p, ...} s;
\end{verbatim}
and access the array elements using the subscription operator as in \code{s.p[i]}.

Gencot translates the struct type to a record type of the form
\begin{verbatim}
  type Struct_Cogent_ip = {... p: P_U32, ...}
\end{verbatim}
and does not generate abstract functions which allow to treat \code{p} as an array, this must be done manually.

The main problem here is the unknown array size. In the C program, however, for working with the array it must be
possible to determine the array size in some way at runtime. Here we distinguish two approaches how this is done.

\subsubsection{Self-Descriptive Array}

In the first case the array size can always be determined from the array content. Then the pointer to the first 
element is always sufficient for working with the array. The two typical patterns of this kind either use a 
stop element to mark the array end, such as the zero character ending C strings, or store the array size in a
specific element or elements, such as in the header part of a network package represented as a byte array.

Working with the array is supported by manually defining and implementing an abstract data type for the array as follows.
Let \code{tt} be a unique name for the specific kind of array (how its size is determined) and let \code{El} be the
name of the array element type. We use \code{CArray\_tt\_El} as type name for the array, thus the 
type of \code{p} must be manually changed as follows:
\begin{verbatim}
  type Struct_Cogent_ip = {... p: CArray_tt_El, ...}
\end{verbatim}

The type \code{CArray\_tt\_El} is defined as
\begin{verbatim}
  type CArray_tt_El = { ttArr: #El }
\end{verbatim}
which has the same form as the Gencot pointer type for the element type.

For the type \code{CArray\_tt\_El} the polymorphic function \code{create} cannot be used since no instance has been
generated by Gencot for it. Gencot cannot do this because the array size is unknown to it. When an array 
of this type is created, the size must be specified as an argument. Hence an abstract function of the form
\begin{verbatim}
  create_CArray_tt_El : UNN -> EVT(CArray_tt_El)
\end{verbatim}
must be defined and implemented manually. It takes the actual array size (number of elements) as argument and is
implemented in C using \code{calloc} with the array size and the size of the element type. The type \code{UNN} must be
chosen by the developer so that it can represent all sizes which are used for the array. It is also used for the index
values below. The type \code{EVT(CArray\_tt\_El)} denotes, as usual, the type of the uninitialized array, realized
by marking its single field \code{ttArr} as taken. Corresponding abstract functions 
for initializing and freeing the array elements must be provided manually.

For disposing values of the type \code{CArray\_tt\_El} the usual polymorphic function \code{dispose} can be used, 
Gencot is able to automatically provide a C correct implementation for all types for which it is used.

The element access functions for single elements should be defined as instances of the operations for accessing parts 
of structured values described in Section~\ref{design-operations-parts} in the specific form for arrays as described
in Section~\ref{design-operations-array}:
\begin{verbatim}
  getCArray_tt_El : (CArray_tt_El!,UNN) -> Option El!
  setCArray_tt_El : (CArray_tt_El,UNN,El) -> (CArray_tt_El,())
  exchngCArray_tt_El : (CArray_tt_El,UNN,El) -> (CArray_tt_El,El)
  modifyCArray_tt_El : all(arg:<D,out). 
    ModFun CArray_tt_El (UNN, ModFun El arg out, arg) out
  getrefCArray_tt_El : (CArray_tt_El!,UNN) -> Option P_El!
  modrefCArray_tt_El : all(arg:<D,out). 
    ModFun CArray_tt_El (UNN, ModFun P_El arg out, arg) out
\end{verbatim}
where \code{setCArray\_tt\_El} can only be defined if the element type \code{El} is discardable.

\subsubsection{Externally Described Arrays}

In the second case the array size is determined by additional information separate from the pointer to the array.
Either the size is specified as an integer value as in 
\begin{verbatim}
  struct ip {... int *p, int psize; ...} s;
\end{verbatim}
or it is specified by a second pointer, e.g., pointing to the last element as in
\begin{verbatim}
  struct ip {... int *p, int *pend; ...} s;
\end{verbatim}
In general there may be additional information, such as pointers into the array which are used to reference
``current'' positions in the array. We assume that all this information is provided by a sequence of members
in the surrounding struct:
\begin{verbatim}
  struct ip {...; t1 m1;...tn mn; ...} s;
\end{verbatim}

These members can be grouped into an embedded struct as described in Section~\ref{app-transtype-group}. If they
are consecutive and are grouped in the same order the modified record type should be binary compatible.
We propose to name the embedded record type \code{SArray\_tt\_El} (``structured array'') in analogy to the
array type name for self-descriptive arrays. It can either be defined as a record type in Cogent:
\begin{verbatim}
  type SArray_tt_El = { m1: T1, ... Mn: Tn }
\end{verbatim}
or, providing additional shielding as a wrapped abstract type 
\begin{verbatim}
  type SArray_tt_El = { ttSArr: #USArray_tt_El }
\end{verbatim}
with a C definition:
\begin{verbatim}
  typedef struct { t1 m1;...tn mn; } USArray_tt_El;
\end{verbatim}
Note that in both cases \code{EVT(SArray\_tt\_El)} yields a usable empty-value type.

Now the single fields in the original structure can be replaced by a field of the embedded structured array type:
\begin{verbatim}
  type Struct_Cogent_ip = {... a: #SArray_tt_El, ...}
\end{verbatim}

To access and modify the group in the struct the abstract functions
\begin{verbatim}
  getrefAInStruct_Cogent_ip: 
    Struct_Cogent_ip! -> SArray_tt_El!
  modrefAInStruct_Cogent_ip: all(arg,out). 
    ModPartFun Struct_Cogent_ip SArray_tt_El arg out 
\end{verbatim}
must be defined and used in the same way as described in Section~\ref{app-transtype-group}.

Initialization and clearing functions for \code{SArray\_tt\_El} must be implemented manually, they always need the
heap for allocating or disposing the actual array. 

The element access functions for single elements can be defined as abstract instances of the operations for accessing parts 
in the same way as for self-descriptive arrays, although the implementations will differ because they have to 
take into account the fields of \code{SArray\_tt\_El}. Usually, additional functions will be required for working
with values of type \code{SArray\_tt\_El}, such as for moving an internal pointer to a ``current'' element.
If such a function modifies some of the fields of \code{SArray\_tt\_El} it must be defined as a modification function
which can be applied with the help of \code{modrefAInStruct\_Cogent\_ip}.


\section{Manual Translation of C Function Body Parts}
\label{app-transfunction}
Manual actions in a function body translation are required if Gencot cannot translate a C code fragment.
In this case it inserts a dummy expression, as described in Section~\ref{design-cstats-dummy}.

In the following sections we provide some rules and patterns how to translate such cases manually.
These rules and patterns are not exhaustive but try to cover most of the common cases.

\subsection{Pointers}
\label{app-transfunction-pointers}

Cogent treats C pointers in a special way as values of ``linear type'' and guarantees that no memory is shared
among different values of these types. More general, all values which may contain pointers (such as a struct with
some pointer members) have this property. All other values are of ``nonlinear type'' and never have common parts 
in C.

If a C program uses sharing between values of linear type, it cannot be translated directly to Cogent. Gencot always 
assumes that named C objects never share parts with the help of pointers. If this is not the case in the original 
C program, it must be manually modified. The following approach can be used for a translation of such cases. 

If several parameters of a function may share common memory, they are grouped together to a Cogent record or
abstract data type and treated as a single parameter. All functions which operate on one of the values are
changed to operate on the group value. Then no sharing occurs between the remaining function parameters.

If several local variables share common memory they can be treated in the same way. 

If a variable shares memory with a parameter this solution is not applicable. In this case the variable must be
eliminated. This is easy if the variable is only used as a shortcut to a part of the parameter value, then it
can be replaced by explicit access to the part of the parameter. For example, in the C function
\begin{verbatim}
  void f (struct{int i; x *p;} p1) {
    x *v = p1.p; ...
  }
\end{verbatim}
the occurrences of variable \code{v} can be replaced by accesses to \code{p1.p}.

Alternatively, in Cogent a \code{take} operation can be used to bind the value of \code{p} to a variable \code{v}.
This prevents accessing the value through \code{p1}. Before the end of the scope of \code{v} the value must
be put back into \code{p1} using a \code{put} operation:
\begin{verbatim}
  let p1 { p = v }
  and ...
  in p1 { p = v }
\end{verbatim}

In other cases individual solutions must be found. Note that parameters and variables of nonlinear type never
cause such problems.

\subsection{Function Pointer Invocation}
\label{app-transfunction-pointer}

When a function pointer \code{fptr} shall be invoked in Cogent, it must be converted to the corresponding Cogent function
using \code{fromFunPtr[...](fptr)} or the macro \code{FROMFUNPTR(<enc>)(fptr)} where \code{<enc>} is the encoding of the 
corresponding function type (see Section~\ref{design-operations-function}).

In C a function pointer may be \code{NULL}, therefore it is typically tested for being valid before the
referenced function is invoked, such as in 
\begin{verbatim}
  if( fptr == NULL ) ...
  else fptr( params );
\end{verbatim}
When applied to \code{NULL} the conversion function \code{fromFunPtr} returns a result which can be invoked in Cogent.
To treat the \code{NULL} case separately it must be tested using the function \code{nullFunPtr}:
\begin{verbatim}
  if nullFunPtr(fptr) then ...
  else fromFunPtr(fptr)(params)
\end{verbatim}

In Cogent the \code{then} case only covers the case where the function pointer is \code{NULL}, it does not cover
the cases where the function referenced by the pointer is not known by Cogent. A function is only known by Cogent
if there is a Cogent definition for it, either as a Cogent function or as an abstract function.

Function pointers are not needed for Cogent functions which are only invoked from Cogent, then the Cogent function can 
be assigned and stored directly. Function pointers are only relevant if they are also used in C code external to
the Cogent compilation unit. So there are two application cases:
\begin{itemize}
\item A function which is defined in C and is passed as a function pointer to Cogent for invocation there. As
described in Section~\ref{design-modular}, this invocation requires an exit wrapper. The function \code{fromFunPtr}
converts the function pointer to the exit wrapper function. To be consistent, function \code{toFunPtr} converts
every exit wrapper to a pointer to the wrapped external function.
\item A function which is defined in Cogent and is passed as a function pointer to C for invocation there. As
described in Section~\ref{design-modular}, this invocation requires an entry wrapper. Entry wrappers are automatically
generated for all translated functions which had external linkage in C. For all these functions \code{toFunPtr}
converts them to a pointer to the entry wrapper. To be consistent, function \code{fromFunPtr} converts pointers
to entry wrappers to the corresponding Cogent functions. For translations of functions with internal linkage
\code{toFunPtr} converts to a pointer to the Cogent function. If such a pointer is invoked from C the invocation
will normally fail, because of different parameter and result types. In the current version of Gencot this must
be handled manually.
\end{itemize}

Exit wrappers are automatically generated by Gencot for functions which are invoked from the Cogent compilation
unit (see Section~\ref{impl-ccomps-externs}), determined with the help of the call graph (see 
Section~\ref{impl-ccode-callgraph}). However, functions which are \textit{only} invoked through function pointers
are not detected by the call graph (the invoked function pointer is detected but not which actual functions
may have been assigned to the function pointer), so no exit wrapper will be generated and \code{fromFunPtr}
will return an invalid result. Moreover, if there are no other Cogent functions
of the same function type, the Cogent program will be translated to inconsistent C code: the invocation of the 
converted function pointer will be translated using a dispatcher function which does not exist.

Gencot supports forcing exit wrapper generation for functions not detected by the call graph, by specifying them
explicitly in the file \code{<package>.gencot-externs} when generating the unit files by executing \code{gencot unit}
(see Section~\ref{impl-all-gencot}). The following additional steps are
required when forcing an exit wrapper for an external C function \code{f} which is not detected by the call graph:
\begin{itemize}
\item To generate the correct type for the exit wrapper a parameter modification description is required for 
\code{f}. Since the required descriptions are also determined using the call graph, \code{f} must additionally
be specified explicitly when generating descriptions for external functions using \code{parmod externs}. The
\code{parmod} script reads the same file \code{<package>.gencot-externs} (see Section~\ref{impl-all-parmod}),
so this happens automatically. However, \code{parmod externs} does not generate the descriptions, it only selects 
them from a file given as input. Therefore the file containing the definition of \code{f} must be processed 
by \code{parmod init} in addition to the files processed as described in Section~\ref{app-parmod-extern} and the 
resulting descriptions must be added to the file input to \code{parmod externs}.
\item Even if explicitly specified, \code{gencot unit} generates exit wrappers only for functions which are declared
in one of the source files belonging to the Cogent compilation unit (usually the declaration will be in a \code{.h}
file included by a source file). If \code{f} is only invoked through a function pointer it may be the case that
the declaration of \code{f} is not present in either Cogent compilation unit source file. In this case the ``pseudo
source'' \code{additional\_externs.c} must be added to the Cogent compilation unit which only includes the \code{.h} 
file where \code{f} is declared. More generally, it should include all \code{.h} files with declarations of functions 
for which an exit wrapper shall be forced and which are not included by regular sources in the Cogent compilation unit.
As described in Section~\ref{impl-ocomps-main}, the name \code{additional\_externs.c} is treated in a special
way by Gencot, it is ignored when the main Cogent and C source files are generated.
\end{itemize}

\subsection{The Null Pointer}
\label{app-transfunction-null}

Translating C code which uses the null pointer is supported by the abstract data type \code{MayNull} defined
in \code{include/gencot/MayNull.cogent} (see Section~\ref{design-operations-null}).

A typical pattern in C is a guarded access to a member of a struct referenced by a pointer:
\begin{verbatim}
  if (p != NULL) res = p->m;
\end{verbatim}
In Cogent the value \code{p} has type \code{MayNull R} where \code{R} is the record type with field \code{m}. 
Then a translation to Cogent is
\begin{verbatim}
  let res = roNotNull p 
            | None -> dflt
            | Some s -> s.m
      !p
\end{verbatim}
Note that a value \code{dflt} must be selected here to bind it to res if the pointer is \code{NULL}. Also note that the access is
done in a banged context for p. Therefore it is only possible if the type of \code{m} is not linear, since otherwise the result
cannot escape from the context.

Another typical pattern in C is a guarded modification of the referenced structure:
\begin{verbatim}
  if (p != NULL) p->m = v;
\end{verbatim}
A translation to Cogent is
\begin{verbatim}
  let p = notNull p 
          | None -> null ()
          | Some s -> mayNull s{m = v}
\end{verbatim}
where the reference to the modified structure is bound to a new variable with the same name \code{p}. 
Note that in the None-case the result cannot be specified as 
\code{p} since this would be a second use of the linear value \code{p} which is prevented by Cogent.

Alternatively this can be translated using the function \code{modifyNullDflt}:
\begin{verbatim}
  let p = fst (modifyNullDflt (p, (setFld_m, v)))
\end{verbatim}
using a function \code{setFld\_m} for modifying the structure.

\subsection{The Address Operator \code{\&}}
\label{app-transfunction-addrop}

The address operator \code{\&} is used in C to determine a pointer to data which is not yet accessed through a 
pointer. The main use cases are
\begin{itemize}
\item determine a pointer to a local or global variable as in the example
\begin{verbatim}
  int i = 5;
  int *ptr = &i;
\end{verbatim}

\item determine a pointer to a member in a struct as in the example
\begin{verbatim}
  struct ii { int i1; int i2; } s = {17,4};
  int *ptr = &(s.i2);
\end{verbatim}

\item determine a pointer to an array elemnt as in the example
\begin{verbatim}
  int arr[20];
  int *ptr = &(arr[5]);
\end{verbatim}

determine a pointer to a function as in the example
\begin{verbatim}
  int f(int p) { return p+1; }
  ...
  int *ptr = &f;
\end{verbatim}
\end{itemize}

In all these cases the pointer is typically used as reference to pass it to other functions or store it
in a data structure.

The binary compatible Cogent equivalent of the pointer is a value of a linear type. However, there is
no Cogent functionality to create such values. Hence it must be implemented by an abstract function.

\subsubsection{Address of Variable}

In the first use case there are several problems with this approach. First, there is no true equivalent
for C variables in Cogent. Second, it is not possible to pass the variable to the implementation of the
abstract function, without first determining its address using the address operator. Third, if the address
operator is applied to the variable in the implementation of the abstract function, the Isabelle c-parser
will not be able to process the abstract function if the variable is local, since it only supports the address
operator when the result is a heap address or a global address.

All these problems can be solved by allocating the variable on the heap instead. Then the variable definition
must be replaced by a call to the polymorphic function \code{create} (see Section~\ref{design-operations-pointer}) and 
at the end of its scope a call to the polymorphic function \code{dispose} must be added. Then the address
operator need not be translated, since \code{create} already returns the linear value which can be 
used for the same purposes. The resulting Cogent code for a variable of type \code{int} initialized to \code{5}
would be
\begin{verbatim}
  create[EVT(CPtr U32)] heap
  | Success (ptr,heap) ->
    fst INIT(Full,CPtr U32) (ptr,#{cont=5})
    | Success ptr ->
      let  ... use ptr ...
      in dispose (fst CLEAR(Simp,CPtr U32) (ptr,()),heap)
    | Error eptr -> dispose (eptr,heap)
  | Error heap -> heap
\end{verbatim}
where \code{INIT(Full,CPtr U32)} is used to initialize the referenced value. Here, the result of the expression is
only the heap, in more realistic cases it would be a tuple with additional result values.

The drawback of this solution is that it is less efficient to allocate the variable on the heap, than to use a 
stack allocated variable. If it is only used for a short time, a better solution should be created manually.

\subsubsection{Address of Struct Member}

The second use case cannot be translated in this way, since the referenced data is a part of a larger structure.
If it is separated from the structure and allocated on the heap, the structure is not binary compatible any more.

If the overall structure is allocated on the stack, the same three problems apply as in the first use case.
This can be solved in the same way, by moving the overall structure to the heap. Then it can be represented
by the linear Cogent type
\begin{verbatim}
  type Struct_Cogent_ii = { i1: U32, i2: U32 }
\end{verbatim}

If the overall structure in Cogent is readonly an abstract polymorphic function \code{getrefFld\_i2}
can be used to access the Cogent field through a pointer. This is an instance of the general \code{getref} operation
for records as described in Section~\ref{design-operations-record}.

If the overall structure is modifyable, determining a pointer to the field would introduce sharing for the
field, since it can be modified through the pointer or by modifying the structure. A safe solution is to use 
an abstract polymorphic function \code{modrefFld\_i2} which is an instance of the general \code{modref} operation
for records as described in Section~\ref{design-operations-record}.

\subsubsection{Address of Array Element}

The third use case is similar to the second. It often occurs if the array itself is represented by a pointer to its first
element (see Section~\ref{app-transtype-arrpoint}). This case could simply be replaced by using the element index instead of a 
pointer to the element. The array index is nonlinear and thus easier to work with. However, this solution is not binary
compatible, if the element pointer is also accessed outside the Cogent compilation unit.

A binary compatible solution can be achieved in a similar way as for struct members. The first prerequisite for it is
to allocate the array on the heap.

Then the polymorphic functions \code{getrefArr}, \code{getrefArrChk}, and \code{modrefArr}
(see Section~\ref{design-operations-array}) can be used for working with pointers to elements.

\subsubsection{Address of Function}

The last use case is translated in a specific way for function pointers, using the translation function
\code{toFunPtr} (see Section~\ref{design-operations-function}), which is generated by Gencot for all function
pointer types. The resulting Cogent code has the form
\begin{verbatim}
  Cogent_f : U32 -> U32
  Cogent_f p = p+1
  ...
  let ptr = TOFUNPTR(FXU32X_U32) Cogent_f
  in ...
\end{verbatim}
In this case \code{ptr} has the nonlinear type \code{\#CFunPtr\_FXU32X\_U32} in Cogent and may freely be copied and
discarded in its scope.



\section{Completing the Cogent Compilation Unit}
\label{app-unit}
%\input{app-unit}

\chapter{Verification}

Additionally to running the translated Cogent program, the goal is to verify it by formally proving properties
about it, such as correctness or safety properties.

The Cogent compiler generates a formal specification and it generates a refinement proof that the specification
is formally valid for the C code generated by the Cogent compiler. The specification consists of definitions for
all types and functions defined in the Cogent program. The definitions are formulated in the HOL logic language
of the Isabelle proof assistant and reflect the definitions in the Cogent program very closely. In particular,
they follow a pure functional style and are thus a good basis to prove properties about them.

Cogent does not provide any support for working with the formal specification. This chapter describes methods
and support for this, assuming that the formal specification has been generated for a Cogent program which results
from translating a C program with Gencot. The support consists of three main levels. The first level supports
specifying alternative ``semantic'' definitions for the Cogent functions. The second level supports replacing the 
the data structures used in the Cogent program by more abstract data types and lifting the function semantics
accordingly. The third level supports using these lifted semantics to prove application specific properties, such
as functional or security properties.

\input{session}

%\section{Function Semantics}
%\label{verif-sem}
%\input{verif-sem}

%\section{Datatype Abstraction}
%\label{verif-type}
%\input{verif-type}

%\section{Proving Properties}
%\label{verif-prop}
%\input{verif-prop}

\end{document}
