
After comments and preprocessor directives have been removed from a C source file, it is parsed and
the C language constructs are processed to yield Cogent language constructs. 

\subsection{Including Files for C Code Processing}

When Gencot processes the C code in a source file, it may need access to information about non-local name references, 
i.e. about names which are used in the source file but declared in an included file. 
An example is a non-local type name reference. To treat the type in Cogent in the correct way, it must be known whether
it is mapped to a linear or non-liner type. To decide this, the definition for the type name must be inspected.
Hence for C code processing Gencot always reads the source file content together with that of all included files.

The easiest way to do so would be to use the integrated preprocessor of the language-c parser. It is invoked by
language-c to preprocess the input to the parser and it would expand all include directives as usual, thus providing
access to all information in the included files.

However, the preprocessor would also \textit{process} all directives in all included files. In particular, it would
remove all C code in condition groups which do not belong to the current configuration. This is not intended by Gencot,
its approach is to remove the directives which belong to the <package> and re-insert them in the target code.

There are two possible approaches how this can be done.

The first approach is to remove the preprocessor directives in advance, before feeding the source to language-c and 
its preprocessor. All include directives are retained and processed by the language-c preprocessor to include the
required content. For this approach to work, the preprocessor directives must be removed in advance from \textit{all} 
include files in the <package>. Additionally, the include directives must be modified to include the resulting
files instead of the original include files. 

The second approach is to first include all include files belonging to the <package> in the source, then removing the
directives in this file, and finally feeding the result to the language-c preprocessor. This can be done for every 
single source file when it is processed by the language-c parser, no processing of other files is necessary in advance.

Gencot uses the second approach, since this way it can process every source file independently from previous steps for 
other source files and it needs no intermediate files which must be added to the include file path of the language-c
preprocessor.

For simplicity, Gencot assumes that all files included by a quoted include directive belong to the <package>. Hence,
the first include step is to simply process all quoted include directives and retain all system include directives in
the code. The language-c preprocessor will expand the system includes as usual, thus providing the complete 
information needed for parsing and processing the C code.

If this is not adequate, Gencot could be extended by the possibility to specify file path patterns for the files to
be included in advance for removing preprocessor directives.

\subsection{Processing the C Code}



 
