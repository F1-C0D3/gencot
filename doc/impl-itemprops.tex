As described in Section~\ref{design-types-itemprops} Gencot supports declaring properties for typed 
items in the C program. The corresponding declarations are specified in textual form as a sequence
of text lines where each line may declare several properties for a single item. The item is
named by a unique identifier, every property is encoded by a short identifier.

\subsection{Item Identifiers}
\label{impl-itemprops-ids}

Item identifiers are text strings which contain no whitespace. 

Global individual items are functions or objects.
They can always be identified by their name. However, for items with internal linkage the name is only unique
per compilation unit, whereas the item declarations span the whole Cogent compilation unit which may consist of several C 
compilation units. Therefore, items with 
external linkage are identified by their plain name (which is an unstructured C identifier), items with
internal linkage are identified in the form
\begin{verbatim}
    <source file name> : <item name>
\end{verbatim}
If the source file name has extension \code{".c"} the extension is omitted, otherwise (typically in the case of \code{".h"})
it is not omitted, resulting in identifiers such as \code{"app.h:myitem"}. This approach differs from that for generating the 
Cogent name, as described in Section~\ref{design-names}, where the extension is always omitted. The reason for preserving
the extension here is to better inform the developer in which file the item has been defined, not for preventing name collision.

As source file name only the base name is used. C packages with source files of the same name in different directories are
not supported by the current Gencot version.

Local individual items are variables (``objects'') locally defined in a function body. They are identified in the form
\begin{verbatim}
    ? <item name>
\end{verbatim}
which is not unique. The current version of Gencot does not try to identify local variables in a unique way, since the 
language-c parser does not directly provide the name of the surrounding function for them. Even if that name would be
used for disambiguation, this would not help for the case of different variables with the same name in the same function.
Therefore in current Gencot version it is not feasible to declare item properties for individual local variables. 

Collective items are identified by certain type expressions. For a typedef name the identifier has the form
\begin{verbatim}
    typedef | <typedef name>
\end{verbatim}
For a tag name it has one of the forms
\begin{verbatim}
    struct | <tag name>
    union  | <tag name>
\end{verbatim}
For a tagless struct or union a pseudo tag name is used which has either of the forms
\begin{verbatim}
    <lnr>_x_h
    <lnr>_x_c
\end{verbatim}
where \code{<lnr>} is the line number in the source file where the tagless struct or union begins syntactically 
and the suffix is constructed from the name \code{x.h} or \code{x.c} of the source file. In \code{x} all minus signs
and dots are substituted by an underscore.

For a derived pointer type where the base type can be used as a collective item it has the form
\begin{verbatim}
    <base item identifier> *
\end{verbatim}

Sub-items of an item are identified by appending a sub-item specification to the item's identifier.
The members of a struct or union are specified in the form
\begin{verbatim}
    <item identifier> . <member name>
\end{verbatim}
The elements of an array are collectively specified in the form
\begin{verbatim}
    <item identifier> / []
\end{verbatim}
The data referenced by a pointer is specified in the form
\begin{verbatim}
    <item identifier> / *
\end{verbatim}
The result of a function is specified in the form
\begin{verbatim}
    <item identifier> / ()
\end{verbatim}
The parameters of a function are specified in either of the forms
\begin{verbatim}
    <item identifier> / <parameter name>
    <item identifier> / <position number>
\end{verbatim}
where the position number of the first parameter is 1. A virtual parameter with the Global-State property
declared must always be specified in the former form with the parameter name. The position is automatically
determined by Gencot when the parameter is added to the Cogent function.

As an example, if \code{f\_t} is a typedef name for a function type with a parameter \code{par} 
which is a pointer to an array of function pointers, the results of the function pointers in the
array can be identified by 
\begin{verbatim}
    typedef|f_t/par/*/[]/*/()
\end{verbatim}

All these syntactic rules for item identifiers are defined in Haskell module \code{Gencot.Items.Identifier}.

\subsection{Item Associated Types}
\label{impl-itemprops-types}

Item properties affect how C types are translated to Cogent types. Thus during translation for every C
type the corresponding item identifier must be known to retrieve the item properties for it. Module
\code{Gencot.Items.Types} defines the type 
\begin{verbatim}
    type ItemAssocType = (String,Type)
\end{verbatim}
for pairs consisting of an item identifier and a C type. Whenever a C type is processed in a declaration
it is first paired with the identifier of the declared item to an \code{ItemAssocType} and then processed in
this form. 

C Types are used in a C source in the following cases:
\begin{itemize}
\item As type of a global or local variable.
\item As result or parameter type of a derived function type.
\item As base type of a derived pointer or array type.
\item As type of a field in a composite type (struct or union).
\item As defining type for a typedef name.
\end{itemize}
In all these cases the type is associated with an item and can be paired with the corresponding item identifier.
There are other uses of types in C, such as in \code{sizeof} expressions, these cases are ignored by Gencot and 
must be handled manually.

The language-c
parser constructs for every defined or declared function the derived function type, therefore variables and functions
can be treated in the same way, they only differ in their type: functions have a derived function type, whereas
variables may have all other types, including types derived from function types such as function pointer types. Fields 
always belong to a composite type, therefore we can reach them from their composite type.
Together we can reach all item associated types used in a C program by determining all toplevel items (variables/functions, 
composite types, typedef names), associate them with their types and then transitively determining all their sub-items 
paired with their types (field types, parameter and result types, base types of pointer and array types). 

We also associate enum types with an item id, although they are not affected by properties. However, Gencot must determine 
and translate enum types as described in Section~\ref{design-types-enum}.

Generally, Gencot translates item associated types to Cogent. It translates functions as described in Section~\ref{design-fundefs}
and it translates enum types, composite types and type definitions as described in Section~\ref{design-types}.
Gencot translates global variables with Const-Val property to an abstract access function, using the translated 
type of the variable or a pointer to the variable.
Gencot translates local variables to Cogent variables with associated bindings, as described in Section~\ref{impl-ccode-cstats}.
In its current version Gencot does not translate other local declarations, such as for types and functions.
Also note, that in the current Gencot version local variables have no unique item identifier, thus no properties
can be declared for them.

Union types are translated without referencing the field types
(see Section~\ref{design-types-struct}). To support a manual translation Gencot treats them in 
the same way as struct types and also processes the types of their fields.

Gencot usually translates derived types to a parameterized Cogent type with the base type as type argument. Hence the base
type is always required as type argument. Therefore Gencot also processes the sub-items corresponding to the base types 
of derived types.

For functions, the associated type is either a typedef name (resolving to a derived function type) 
or an expression for a derived function type. In the latter case no Cogent type definition is needed for it, since Gencot
directly translates it to a Cogent type expression (Section~\ref{design-fundefs}) using the parameter and result types. 

\subsection{Property Identifiers}
\label{impl-itemprops-property}

Each property is identified by a two-letter code as follows:
\begin{description}
\item[Read-Only:] ro
\item[Not-Null:] nn
\item[No-String:] ns
\item[Heap-Use:] hu
\item[Add-Result:] ar
\item[Global-State:] gs
\end{description}

The numerical argument of Global-State properties is directly appended, such as in \code{gs5}. As an exception,
the numerical argument 0 is omitted and the plain property identifier \code{gs} is used.

\subsection{Item Property Declarations}
\label{impl-itemprops-decl}

An item property declaration consists of a single text line which starts with an item identifier, followed
by an arbitrary number of property identifiers in arbitrary order, separated by space sequences.
For better readability a colon may be appended to the item identifier.

As an example, to declare properties for the parameters \code{p1, par2, param3} of a function pointer \code{myfunp},
introduce an additional parameter for a global variable, and declare the Heap-Use property for the function itself, 
five declarations are needed:
\begin{verbatim}
    myfunp/*/p1:     nn ro
    myfunp/*/par2:   ar mf
    myfunp/*/param3: ns
    myfunp/*/gstate: gs3
    myfunp/*:        hu
\end{verbatim}

When item property declarations are merged (see Section~\ref{impl-ocomps-items}) property specfication can be negative
be prepending a minus sign \code{-} (without separating spaces):
\begin{verbatim}
    mystruct.m1:     nn -ro
\end{verbatim}
Upon merging a negative property suppresses the property for the item, independent where it is specified. Thus negative
propertiy specifications have higher priority than normal ones.

\subsection{Internal Representation}
\label{impl-itemprops-internal}

When used the textual form is converted to an internal Haskell representation of type 
\begin{verbatim}
  type ItemProperties = Map String [String]
\end{verbatim}
defined in module \code{Gencot.Util.Properties}. It maps item property identifiers to the list of identifiers of 
declared properties (including the minus sign for negative properties.
If several declarations for the same item are present in the textual form they are combined by uniting the 
declared properties. Repeated property specifications are removed, the order of the properties in the list is
arbitrary. Negative properties and corresponding normal properties are not normalized, they can coexist in the
same list.

The \code{ItemProperties} map is added as component to the user state of the \code{FTrav} monad. The monadic actions
\begin{verbatim}
  getItems :: (String -> [String] -> Bool) -> FTrav [String]
  getProperties :: String -> FTrav [String]
  hasProperty :: String -> String -> FTrav Bool
\end{verbatim}
are defined in \code{Gencot.Traversal} for accessing the item properties. The actions \code{getProperties} and 
\code{hasProperty} treat negative properties literally and do not interprete them.
