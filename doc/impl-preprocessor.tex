\subsection{Filters for Processing Steps}

Directive processing is done for the output of \code{gencot-remcomments}. All comments have been removed. 
However, there may be line directives present. 

The filter \code{gencot-selpp} selects all preprocessor directives and
copies them to the output without changes. All other lines are replaced by empty lines, so that the original
line numbers for all directives can still be determined. 

The filter \code{gencot-rempp <file>} removes all preprocessor directives from its input, replacing them by empty
lines. All other lines are copied to the output without modification. If \code{<file>} is specified it must 
contain a list of regular expressions for directives which shall be retained.

The filter \code{gencot-unline} is a utility filter. It expects line directives in its input. It removes all
included content (detected by the file specification in the line directives) and expands line directives
in the content from \code{<stdin>} to sequences of empty lines.

How the directives are processed depends on the kind of directives (see Section~\ref{design-preprocessor}).
Gencot provides the processing filters \code{gencot-prcppconst, gencot-prcppmacro, gencot-prcppincl}. 
Conditional directives are not processed, they are inserted
without changes. However, they are merged into the target code in a different way, therefore Gencot provides 
the specific merging filter \code{gencot-mrgppcond <file>} for them. All other directives are merged into the 
target code by filter \code{gencot-mrgpp <file>}. Both filters merge the content in \code{<file>} 
into the input and write the merged code to the output. \code{<file>} must contain the directives to be merged.

Constant definitions must be preprocessed to map constants names in the replacement bodies (see below). This
is done by filter \code{gencot-preppconst}.

\subsection{Separating Directives}

Gencot supports to keep some directives in the output of \code{gencot-rempp} to handle cases where
the C code of different groups in a section causes conflicts. These conditional directives are still selected
by \code{gencot-selpp} and re-inserted by \code{gencot-mrgcond}. 

\subsubsection{Filter \code{gencot-selpp}}

The filter for selecting preprocessor directives from the input for separate processing and insertion into
the generated target code is implemented as an awk script.

It detects all kinds of preprocessor directives, which always begin at the beginning of a separate line.
A directive always ends at the next newline which is not preceded by a backslash \code{\\}. All corresponding
lines are copied to the output without modifications.

Copied directives are normalized in the following sense: All whitespace before and after the leading hash
sign are removed. This is done to simplify further directive processing.

Every other input line is replaced by an empty line in the output.

\subsubsection{Filter \code{gencot-rempp}}

The filter for removing preprocessor directives from its input is implemented as an awk script.
Basically, it replaces lines which are a part of a directive by empty lines. However, there are the following
exceptions:
\begin{itemize}
\item line directives are never removed, they are required to identify the position in the original source
during code processing.
\item system include directives are never removed, they are intended to be interpreted by the language-c
preprocessor to make the corresponding information available during code processing. Since it is assumed that
all quoted include directives have already been processed in an initial step, simply all include
directives are retained.
\item directives which match a regular expression from a specified list are not removed, they are intended
to be interpreted by the language-c preprocessor to suppress information which causes conflicts during code
processing.
\end{itemize}

For conditional directives always all directives belonging to the same section are treated in the same way.
To retain them the first directive (\code{\#if}, \code{\#ifdef}, \code{\#ifndef}) must match a regular expression
in the list. For all other directives of a section (\code{\#else}, \code{\#elif}, \code{\#endif}) the 
regular expressions are ignored.

The regular expressions are specified in the argument file line by line. Before a directive is matched with 
a regular expression, it is normalized by removing all whitespace around the leading hash sign, hence the
regular expressions can be written without considering such whitespace. An example file content is
\begin{verbatim}
  ^#if[[:blank:]]+!?[[:blank:]]*defined\(SUPPORT_X\)
  ^#define[[:blank:]]+SUPPORT_X
  ^#undef[[:blank:]]+SUPPORT_X
\end{verbatim}
It retains all directives which define the macro \code{SUPPORT\_X} or depend on its definition.

\subsubsection{Filter \code{gencot-unline}}

This filter is implemented as an awk script.

Line directives in the input are expanded to the required number of empty lines
which have the same effect. This is done to simplify reading the input for all subsequent filters.

All content which does not origin in file \code{<stdin>} is removed together with its line directives.

\subsection{Processing Directives}

\subsubsection{Processing Constants Defined as Preprocessor Macros}

Roughly, preprocessor constant definitions are macro definitions without macro parameters but with a 
nonempty replacement body (i.e., not a flag). However, parameterless macros can be used for defining
and inserting all other kinds of C code fragments, such as statements or declarations. So, not all
parameterless macro definitions can be translated to Cogent value definitions.

Parameterless macro definitions can be recognized syntactically, since for a macro with parameters 
an opening parenthesis must follow the macro name without separating whitespace. 

Parameterless macro definitions to be translated to Cogent value definitions can in general not be 
recognized without parsing the replacement body as C code, and even that would not be safe since 
it can consist of arbitrary fragments with other macro calls embedded. Hence Gencot supports a manual
specification for cases which cannot be recognized automatically. Since usually these complex cases
are seldom, the manual specification consists of a file specifying names of parameterless macros which
should \textit{not} be translated to a Cogent value definition. The name of this file is passed to the filter
\code{gencot-preppconst} as an additional argument.

The replacement body of a preprocessor constant definition may reference the names of other preprocessor
constant definitions. They must be replaced by the mapped Cogent name which corresponds to the name of
the Cogent constant. In general, that would also require at least a lexical analysis of the replacement
bodies to find the preprocessor constant names. Gencot uses a simpler approach and implements this as follows.

A file is generated which contains for every preprocessor constant definition a preprocessor definition 
which maps the original name to the Cogent name. This file is prepended to the original file where all
constant definitions are masked in a way that they are not recognized by the C preprocessor.
The resulting file is processed by the C preprocessor, this will substitute all names in the replacement 
bodies by their mapped form. Together, these steps are implemented by the preprocessing filter 
\code{gencot-preppconst}.

Since constant names used in replacement bodies may have been defined in included files, the filter
\code{gencot-preppconst} is intended to be applied to the file after all include directives have been 
expanded, as described for C code in Section~\ref{impl-ccode-include}. In the input line directives
are expected which specify the origin file for all its content. They are transferred to the output.

The output of \code{gencot-preppconst} is then processed by
filter \code{gencot-prcppconst} to translate the constant definitions to Cogent value definitions.

\paragraph{Filter \code{gencot-preppconst}}

The preprocessing filter is implemented as an awk script. It selects all parameterless macro definitions,
omitting those where the name is listed in the argument file. To omit flags it only selects macro 
definitions whith text after the macro name or with a continuation line. Gencot assumes, that continuation
lines always contain a nonempty replacement body.

The selected macro definitions are processed twice. In the first phase for every macro definition of the
form
\begin{verbatim}
  #define CONST1 replacement-body
\end{verbatim}
a definition of the following form is generated:
\begin{verbatim}
  #define CONST1 cogent_CONST1
\end{verbatim}
where \code{cogent\_CONST1} is the result of mapping the name \code{CONST1} to a Cogent variable name
according to Section~\ref{design-names}.

In the second phase a masked definition of the following form is generated:
\begin{verbatim}
  _#define XCONST1 replacement-body
\end{verbatim}
All backslashes marking continuation lines are duplicated, because the C preprocessor removes them.

The sequence of all definitions from the first phase is prepended to the sequence of definitions from the second 
phase. Then the resulting file is piped through the C preprocessor
which applies the definitions from the first phase to the replacement-bodies in the definitions from the
second phase.

The definitions from the first phase are preceded by a line directive for the dummy file name \code{<mappings>}
so that their lines do not count to the content of \code{<stdin>}.

\paragraph{Filter \code{gencot-prcppconst}}

The processing filter is implemented as an awk script. It is intended to be applied to the result of 
\code{gencot-preppconst}. It contains the masked definitions to be processed, both from the original source file
and from all included files. The definitions from included files are not translated to Cogent value
definitions, but they are needed to determine the type and value of externally defined constants.

The type and value needed for the Cogent value definition are determined from the replacement body as follows.

If the body is in the definition line and does not contain any whitespace it may be a C literal. First, 
enclosing parentheses are removed to also recognize replacement bodies of the form \code{(127)}. Then 
the following cases are recognized.
\begin{itemize}
\item if the body starts with a decimal digit and only contains digits and letters it is assumed to be an 
integer literal in decimal, octal (leading zero), or hexadecimal (leading \code{0x}) notation. Its type
is determined to be \code{U8, U16, U32,} or \code{U64} depending on the integer value.
\item if the body starts with a minus sign, followed by an integer literal as above, its type is
determined as \code{U32} and the unsigned value is constructed by calculating the 2-complement.
\item if the body is enclosed in single quotes it is assumed to be a character constant. Its type is
determined as \code{U8}.
\item if the body is enclosed in double quotes it is assumed to be a string constant, its type is determined
as \code{String}.
\item if the body is a single identifier it is resolved using the previous constant definitions. If successful,
the type is determined from that definition. As value the identifier is used.
\end{itemize}

If the body is in the definition line and contains whitespace, it is cecked whether it consists of a sequence
of words which are either string literals or constant names which resolves to type string. In this case it 
is assumed that the value is a sequence of string literals to be concatenated. Its type is determined as 
\code{String}, its value is constructed by concatenating all string values.

In all other cases (also for all bodies using continuation lines) Gencot assumes an expression of type int
and determines the type as \code{U32}. As value the unmodified C expression is used.

Cogent value definitions are only generated from constant definitions belonging to the content of \code{<stdin>}
Every generated Cogent value definition is wrapped in \code{\#ORIGIN} markers according to the line offset
of the original constant definition in \code{<stdin>}.

\subsubsection{Processing Other Preprocessor Macros}

\subsubsection{Processing Include Directives}

\subsection{Merging Directive Processing Results}

When the conditional directives are merged into the target code, the other 
directives must have already been merged in, since the conditional directives are inserted depending on the content 
of groups and their positions. Therefore, first all other directives must be merged using the filter \code{gencot-mrgpp},
then the conditional directives must be merged using the filter \code{gencot-mrgppcond}.

\subsubsection{Filter \code{gencot-mrgppcond}}

The filter for merging the conditional directives into the target code is implemented as an awk script. As argument
it takes the name of a file containing the directives to be merged. Since conditional directives need not be
processed it is intended that this file contains the output of filter \code{gencot-selpp}, i.e., all directives
selected from the source, processed by \code{gencot-unline} to remove included input and line directives. 
\code{gencot-mrgppcond} selects only the conditional directives and merges them into
the filter input, additionally generating origin markers for every merged directive. 

The filter input must contain the generated Cogent target code and the other preprocessor directives. 
All content in the input must have been marked by origin markers. 

In its BEGIN rule the filter reads the
conditional directives from the argument \code{<file>} and associates them with their line numbers, building the
list of all sections, ordered according to the line number of their first directive (\code{\#if, \#ifdef, \#ifndef}).
Every section is represented by its list of directives.

While processing the input the program maintains a stack of active sections and for every section in the stack the 
active group. A section is active if some of its directives have been output but not all. The active group is that
corresponding to the last directive which has been output for the section, i.e. it is the group which will contain
all target code which is currently output.

The filter only uses the \code{\#ORIGIN} markers to position conditional directives. Gencot assumes that for every 
\code{\#ENDORIG} marker a previous \code{\#ORIGIN} marker exists for the same line number. Since a condition group
always contains a sequence of complete lines, the information about the origin lines in the input is fully specified
by the \code{\#ORIGIN} markers, the \code{\#ENDORIG} markers are not relevant for placing the conditional directives.

For every \code{\#ORIGIN} marker in the input the following steps are performed:
\begin{itemize}
\item if the line belongs to a group of an active section which is after the active group, all directives of the active
section are output until the group containing the line is reached, this group is the new active group of the section.
\item if for an active section the line does not belong to the active group or any of its following groups, 
the \code{\#endif} directive for the 
section is output and the section becomes inactive (is removed from the stack of active sections).
\item if the line belongs to a group of a section which is (after the previous step) not active, the section is set 
active (put on the stack) and all its directives are output until the group containing the line is reached.
\item finally the \code{\#ORIGIN} directive and all following lines which are not an \code{\#ORIGIN} marker
are output without any changes.
\end{itemize}

Note that due to the semantics of the \code{\#elif} and \code{\#else} directives, for every group in a section all
directives of preceding groups are relevant and must be output when the active group changes, even if this produces 
empty groups in between.

Due to the nesting structure, when newly active sections are pushed on the stack in the order of the position of their
first directive, the stack reflects the section nesting and the sections will be inactivated in the reverse order and
can be removed from the stack accordingly.

If an \code{\#ORIGIN} marker idicates, that the next line belongs to a group \textit{before} the active group, the
steps described above imply that the current section is ended and restarted from the beginning.

If the line before an inserted conditional directive contains an \code{\#ENDORIG} marker the newline after it belongs
to the marker. If the marker is not used to insert a comment it will be removed completely and the conditional
directive will be appended to the previous code line. In this case an additional newline is inserted before 
the conditional directive.

\subsubsection{Filter \code{gencot-mrgpp}}

The filter for merging other directives into the target code is implemented as an awk script. It can be used to 
merge arbitrary code with origin markers into target code with origin markers, if every code part should be
merged in exactly once.

The filter only interpretes \code{\#ORIGIN} markers. Whenever an \code{\#ORIGIN} marker is reached in the target
code, all content is merged in up to the first \code{\#ORIGIN} marker with a line number not before that in the target
code.

If the line before an inserted code contains an \code{\#ENDORIG} marker an additional newline is inserted before 
the code for the same reason as for conditional directives.
