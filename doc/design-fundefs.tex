
A C function definition is translated by Gencot to a Cogent function definition. Old-style C function definitions
where the parameter types are specified by separate declarations between the parameter list and the function body
are not supported by Gencot because of the additional complexity of comment association.

The Cogent function name is generated from the C function name as described in Section~\ref{design-names}.

The Cogent function type is generated from the C function result type and from all C parameter types as described
in Section~\ref{design-types-function}. In a C
function definition the types for all parameters must be specified in the parameter list, if old-style function
definitions are ignored.

\subsection{Function Bodies}
\label{design-fundefs-body}

In C the function body consists of a compound statement which is specified in imperative programming style. In Cogent
the function body consists of an expression which is specified in functional programming style with additional 
restrictions which are crucial for proving properties of the Cogent program. Therefore Gencot does not try
to translate function bodies, this must be done by a human programmer.

It would be possible, however, to translate C declarations embedded in the body. These may be type definitions
and definitions for local variables. However, there are no good choices for the generated target code. Type
definitions cannot be local in an expression in Cogent, they must be moved to the toplevel where they may cause
conflicts. Local variable definitions could be translated to Cogent variable bindings in let-expressions, however,
C assignments cannot be translated for them. Also, the resulting mixture of C code and Cogent code is expected 
to be quite confusing to the programmer who has to do the manual translation. Therefore, no declarations in 
function bodies are processed by Gencot.

The only processing done for function bodies is the substitution of names occurring free in the body. These may
be names with global scope (for types, functions, tags, global variables, enum constants or preprocessor constants)
or parameter names. For all names with global scope Gencot has generated a Cogent definition using a mapped name.
These names are substituted in the C code of the function body by the corresponding mapped names so that the 
mapping need not be done manually by the programmer.

This does not include the mapping of derived types. Type derivation in C is done in declarators which refer
a common type specification in a declaration. In Cogent there is no similar concept, every declarator must be 
translated to a separate declaration. This is not done due to the reasons described above. As result, a translated
local declaration may have the form
\begin{verbatim}
  Struct_s1 a, *b, c[5];
\end{verbatim}
although the Cogent types for \code{a, b, c} would be \code{\#Struct\_s1}, \code{Struct\_s1}, and 
\code{\#CarrayOf \#Struct\_s1}, respectively. This translation for the derived types must be done manually.

Additionally, the function parameter names usually occur free in the function body. To make them apparent to
the programmer, Gencot generates a Cogent pattern for the (single) parameter of the Cogent function which 
consists of a tuple of variables with the names generated from the C parameter names. As described in 
Section~\ref{design-names} the C parameter names are only mapped if they are uppercase, otherwise they are
translated to Cogent unmodified. If they are mapped they are substituted in the body. Since it is very unusual
to use uppercase parameter names in C, the Cogent function will normally use the original C parameter names.

The generated Cogent function definition has the form
\begin{verbatim}
<name> :: (<ptype1>, ..., <ptypen>) -> <restype>
<name> (<pname1>, ..., <pnamen>) =
<compound statement>
\end{verbatim}
where the \code{<compound statement>} is plain C code with substituted names.

Of course this code cannot be syntax checked by the Cogent compiler. 
It would be possible to include the C function body in a comment to improve that. However, then
still a dummy result expression would be needed for the Cogent function to be valid Cogent code.
It is not trivial to generate that for arbitrary result types, hence Gencot does not try
to separate the C code from the Cogent code other than putting it in a new line after the 
Cogent function header.

\subsection{Comments in Function Definitions}

A C function definition which is not old-style syntactically consists of a declaration with a single declarator
and the compound statement for the body.
It is treated by Gencot as a structured source code part with the declaration and the body as subparts
without any main part code. According to the structures of declarations the declaration has the single declarator as subpart
and optionally a full specifier, if present. The declarator has the parameter declarations as subparts.

\subsubsection{Function Header}

The target code part for the declaration and for its single declarator is the header of the Cogent function definition
(first two lines in the schema in the previous section). The target code part for the full specifiers with tags in
the declaration (which may be present for the result type and for each parameter) is a sequence of corresponding 
type definitions, as described for declarations in Section~\ref{design-decls-tags}, which is placed 
after the Cogent function definition. The target code part for full specifiers without tags is the generated type
expression embedded in the Cogent type for the corresponding parameter or the result.

All parameter declarations consist of a single declarator and the optional full specifier. The target code part for
a parameter declaration and its declarator is the corresponding parameter type in the Cogent function type expression.
Hence, comments associated with parameter declarations in C are moved to the parameter type expression in Cogent.

\subsubsection{Function Body}

To preserve comments embedded in the C function body it is also considered as a structured source code part. Its 
subpart structure corresponds to the syntactic structure of the C AST. Since in the target code only identifiers 
are substituted, the target code
structure is the same as that of the source code. The structure is only used for identifying and re-inserting
the transferrable comments and preprocessor directives. Note that this works only if the conditional directive 
structure is compatible with the syntactic structure, i.e., a group must always contain a complete syntactical
unit such as a statements, expression or declaration, which is the usual case in C code in practice.

An alternative approach would be to treat all nonempty source code lines as subparts of a function body, resulting
in a flat sequence structure of single lines. The advantage is that it is always compatible to the conditional 
directive structure and
all comment units would be transferred. However, generating the corresponding origin markers in an abstract syntax
tree is much more complex than generating them for syntactical units for which the origin information is present
in the syntax tree. Since the Gencot implementation generates the
target code as an abstract syntax tree, the syntactical statement structure is preferred. 
